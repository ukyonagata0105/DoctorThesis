% -----------------------------------------------------------------------------------------------------------------------------------------
% 第3章 舞台としての公共交通政策:現状と課題
% -----------------------------------------------------------------------------------------------------------------------------------------

\chapter{舞台としての公共交通政策:現状と課題}
\label{chap:public_transport_policy}

% 本章は、土木学会論文集(2025年7月)に掲載された「『日本版MaaS』は『交通まちづくり』の一類型として捉えられるか?目標とガバナンスについての一考察」を基に、博士論文の構成に合わせて再構成したものである。

% -----------------------------------------------------------------------------------------------------------------------------------------
\section{はじめに}
\label{sec:ch3_intro}

% -----------------------------------------------------------------------------------------------------------------------------------------
\subsection{本章の目的}

本章では、本研究の「舞台」となる公共交通政策の現状と課題を整理する。まず日本の公共交通政策の変遷を概観し、次に制度設計の現状を分析し、最後にJapan MaaSプロジェクトの実証分析を通じて実装ギャップの実態を明らかにする。

本章における分析モデルとして、政策過程論を採用する。政策過程は、目的を設定した上で現状を理解し、そのギャップを課題として認識した上でそれを解決するための方策を列挙し、実施する政策を決定するという一連の流れである。この中で、どのように目標を立て、またどのように目標を設定しているかが、本章における大まかな分析対象である。

本章でいう「交通まちづくり」の概念は、太田(2008)の述べるような「交通に関連する地域の課題への対応をベースにして、市民と行政が協働して進めるまちづくり」である。原田編(2015)の述べるところの定義である「まちづくりの目的に貢献する交通計画」とは異なり、「市民の参加と協働により発展的に進める活動プロセス」である。この中では、「交通計画における住民、市民の参加」「都市計画、都市づくりとの連携」の2点が重要視されている。

% -----------------------------------------------------------------------------------------------------------------------------------------
\subsection{MaaSの勃興}

近年、日本において「Mobility as a Service」の勃興が著しい。2019年度より始まった「日本版MaaS推進事業」は、瞬く間に全国各地に広がり、現在までに100事業以上のプロジェクトを送り出している。本稿ではこうしたプロジェクトがどのような体制の下に進められ、またどのような検証体制が敷かれているかについて、国土交通省の公開している資料を基として実証的に考察する。

地域公共交通政策のうち、Mobility as a Service(MaaS)と呼ばれる分野は、この数年で急速に勃興してきた。それ以前から、「交通に関連する地域の課題への対応をベースにして、市民と行政が協働して進めるまちづくり」として、「交通まちづくり」という思想体系が存在、発展してきた。

% -----------------------------------------------------------------------------------------------------------------------------------------
\subsection{先行研究}

日本版MaaSについて、その特徴を述べた文献はいくつかある。

まず注目されるのは、特に過疎地域や地方中核市における交通が抱える課題を解決するためのアプローチとしての見立てである。魏(2022)は、日本のMaaSの注目点が「地方型MaaS」である点に言及したうえで、日本のMaaSの特徴について以下のように述べている。

\begin{quote}
「地方型」MaaSは、地域の資源や特性を最大限活かしながらより満足度の高いモビリティサービスを提供することであり、地域にある交通モード間の連携強化により、その場その場で発生した交通需要(生活交通需要、観光交通需要)を上手く汲み上げ、集約することで実現できるものである。
\end{quote}

また藤本(2021)では、欧州におけるMaaSが「都市型MaaS」と「地方型MaaS」に分化されており、また日本版MaaSが「地方型」を含む5類型に分割されている点に触れたうえで、地方型MaaSを検討するポイントを5つに整理している。

MaaS自体の実装にかかわる研究も進んでいる。後藤(2021)は、MaaSを旅行業の一類型ととらえ、企画旅行業務を適用する可能性やデータ供給の課題について論じている。

既存の公共交通運営体制に関する議論は、多くの場合「どのような取り組み」があったか、また「どのような評価が可能か」という2つに大別される。前者の場合、生活バスよっかいちについての事例報告を行った福本・加藤(2012)や、多くの日本モビリティマネジメント会議の報告に代表される。また「どのような評価が可能か」については、アクセシビリティ評価の地域への適用について実例を分析した喜多(2022)などがあげられる。公共交通に関する責任分担や意思決定組織についての検討も複数ある。

本稿で前提とした「交通まちづくり」の考えでは、交通政策における市民参加と地域戦略との連携が重視されていると考えられる。また後述するものであるが、地域公共交通に関する法体系においても、交通政策の戦略性を重視している。ただ少なくとも「日本版MaaS」において、法体系からの課題意識を導入した包括的な研究、また交通事業者としてではない形で計画の要件定義から携わる市民参画の有無、に関する実証的な研究は見当たらなかった。

% -----------------------------------------------------------------------------------------------------------------------------------------
\subsection{リサーチクエスチョンと仮説}

本章は、いわゆる「日本版MaaS」を「交通まちづくり」実践の一類型と捉え、以下の2つの問いについて検討し、結果としてこの捉え方が適切かについて確認するものである。第一に、「日本版MaaS」の政策進捗において目標設定がどこに置かれているか。第二に、立場を問わない市民による批判回路がどのように整備されているのかである。

本分析における仮説としてはそれぞれ、目標設定はMaaS事業の事業性に関して設定されるがそれ以外の地域に関する効果については設定されないこと、そして市民による批判回路は協議会に閉じずパブリックコメント・市民代表参加・座談会・ワークショップなどの機会を通じて設定されていること、と設定している。

% -----------------------------------------------------------------------------------------------------------------------------------------
\section{MaaSの概念と展開}
\label{sec:ch3_maas_concept}

% -----------------------------------------------------------------------------------------------------------------------------------------
\subsection{フィンランドにおけるMaaSの勃興}

まずMaaSは、Hekkilä(2014)にて初めて学術的な表舞台に立った、公共交通の新しい提供概念である。氏が所属する大学へ依頼されたヘルシンキ市の交通課題解決のために、修士論文で書き上げたものである。これによると、Nokiaを主軸としたフィンランドの通信政策やその事業に対する理解の蓄積が、ほぼそのまま交通政策に導入されていることがわかる。

この論文では、論文冒頭のサマリーにおいてその目的を明確に示している。具体的には、公共交通の利用促進が叫ばれ、都市構造への影響が避けられない中で、既存の交通サービス目標は「課題に十分にこたえられていない」と批判している。そのうえで、「公共交通セクターの構造転換が必要である」として「多様で魅力的な交通サービスの便利な提供」としてMaaSを定義している。

具体的なMaaSの構造として、「交通サービスプロバイダーから交通サービスを買い、それをユーザーへ提供する」主体を「モビリティオペレーター」と定義している。結果として、交通サービスを従来提供していた事業者と利用者の間に、もう一つの主体が挟まることが特徴である。

% -----------------------------------------------------------------------------------------------------------------------------------------
\subsection{MaaSに必要な統治構造変革}

このMaaSコンセプトを実現するうえで、政策上で何が必要であるのかについて、Heikkiläは7つの政策目標を提示している。企業・当局・機関・ユーザーなどすべてのステークホルダーの協力体制の調整、サービス・エコシステムの前提条件を満たすための法律・規制の改正、共通のルールと適切な規制の構築およびその遵守状況の監視、モビリティサービス提供の再編成、変革されたオペレーションの確立、購入・補助金手続きの見直し、そしてパイロットとテストエリアの設置である。

このような政策目標を実現するために、フィンランドでは多様な法改正が行われている。「モビリティ・ローミング」の実現を目標として、既存サービス事業者及び参入する交通事業者が交通に関する情報提供を義務付けられるとともに、チケット発行と決済に関するシステムへの相互運用が可能となる法律が成立、2017年に施行された。

% -----------------------------------------------------------------------------------------------------------------------------------------
\subsection{MaaSとインターネットの類似性}

そもそもフィンランドでは、通信産業と運輸産業に関する規制を、同一の機関が担当している。インターネットの特徴は、「いかなる情報であっても運搬可能とすること」、そのために「デジタル化」を挟むことである。

フィンランドにて定義されたMaaSに存在するインターネットエコシステムと類似の概念は3点ある。第一にモジュール化であり、様々な機能についてルールを決め目的ごとに分化して整備させることで、システム上でのメタボリズムと堅牢性を確保する。第二にネットワークであり、前述のモジュールを接続することで一つの目的を完結させることを支援する。第三にインターフェースの統合であり、様々なサービスがあったとしてもユーザーとの接点については統合して管理できるようにする。

% -----------------------------------------------------------------------------------------------------------------------------------------
\subsection{日本版MaaSの定義と展開}

日本における、中央官庁によるMaaS推進事業は、主に国土交通省と経済産業省が所管して展開されてきた。2019年には経済産業省と共同で「スマートモビリティチャレンジ」が開始され、全国28事業が指定されることとなった。さらに国土交通省単独でも、2019年より「日本版MaaS推進事業」が推進され、初年度においては19事業、2020年度には36事業が選定された。

「日本版MaaS」の定義については、国土交通省により以下のように述べられている:

\begin{quote}
日本版MaaSは、地域住民や旅行者一人一人のトリップ単位での移動ニーズに対応して、複数の公共交通やそれ以外の移動サービスを最適に組み合わせて検索・予約・決済等を一括で行うサービスであり、観光や医療等の目的地における交通以外のサービス等との連携により、移動の利便性向上や地域の課題解決にも資する重要な手段となるものです。
\end{quote}

この定義を参照すると、日本におけるMaaSは構造的なものではなく、地域課題解決ベースのサービス提供について重きを置いている。これに基づけば日本版MaaSは、交通サービスの対価支払い・生産・提供の一連の消費活動を組み替える意図は持たない。つまり、国土交通省が考えるMaaSは、制度やシステムの変革というよりむしろ、交通事業者が提供する公共交通サービスについての改善アイデアコンテスト並びに実証実験、ととらえた方が説明はつきやすい。

% -----------------------------------------------------------------------------------------------------------------------------------------
\section{法体系と交通まちづくり}
\label{sec:ch3_legal_framework}

% -----------------------------------------------------------------------------------------------------------------------------------------
\subsection{交通政策基本法}

交通政策基本法は、2013年に可決・成立した、日本における交通政策の理念を定めた法律である。本法では、2条から7条までを理念として、国には「国民の理解を得るよう努める」こと、地方自治体には「その地方公共団体の区域の自然的経済的社会的諸条件に応じた施策を策定し、及び実施する責務」、交通事業者は「その業務を適切に行うよう努めるとともに、国又は地方公共団体が実施する交通に関する施策に協力するよう努める」ことをそれぞれ求め、またこの3者ならびに住民その他の関係者に対して、「基本理念の実現に向けて、相互に連携を図りながら協力するよう努める」としている。

公共交通に関わる点の理念について確認すると、「交通の機能を管理する」という点に重点が置かれている。まず第2条にて「(交通の)機能が十分に発揮されることにより、国民その他の者の交通に対する基本的な需要が適切に充足されることが重要であるという基本的認識の下に行われなければならない」として、交通政策の意義が定義されている。

そして第3条にて、目的志向型の交通政策目標設定義務、つまり「交通が、豊かな国民生活の実現に寄与するとともに、我が国の産業、観光等の国際競争力の強化並びに地域経済の活性化、地域社会の維持及び発展その他地域の活力の向上に寄与するものとなるよう、その機能の確保及び向上が図られることを旨として行われなければならない」が記載されている。

% -----------------------------------------------------------------------------------------------------------------------------------------
\subsection{交通政策基本計画}

交通政策基本計画は、交通政策基本法に基づいて策定される計画である。第1次の計画は2015年に策定され、2021年に更新されている。

計画の構成としては、「交通に関する施策の『基本的方針』、計画期間内に目指すべき『目標』、目標の各々について取り組むべき『施策』の三層構造となっており、関係者の責務・役割や連携・協働等についても、施策の推進に当たって『留意すべき事項』として整理している」。

横ぐしの法律を背景としているが、地域公共交通に関連して目標となっている数値はほとんどの場合、事業がどの程度導入されているかが指標となっている。

% -----------------------------------------------------------------------------------------------------------------------------------------
\subsection{地域公共交通活性化・再生に関する法律}

地域公共交通活性化・再生に関する法律は、2007年より施行された公共交通における「総合計画法」である。この法律は、元々は「施策支援法」であった。「頑張る地域を応援する」、「LRT」をはじめとした道路運送法だけでは対応できなかったバス事業以外の交通事業について包括的な連携を図るとともに、交通システムごとの支援メニューを取りそろえたシステムであった。

しかし、2010年近傍のいわゆる「事業仕分け」を経て、地域の生活交通を確保する「総合計画法」に転化し、取り組み支援型の事業支援システムを残したまま強制力を持ち始めた。この法律に基づいて策定される「地域公共交通計画」「地域公共交通利便増進計画」を所管する地方運輸局には、路線届出の最終的な受理権限もある。

% -----------------------------------------------------------------------------------------------------------------------------------------
\subsection{交通まちづくりの定義とその親和性}

主に交通政策基本法を基底とした理念体系を確認すると、この内容が驚くほど「交通まちづくり」のそれと類似していることが分かる。

交通まちづくりは、都市の課題解決のために、公共交通を中心とした交通を動員する交通政策の概念である。既存の交通需要を満たすための交通政策とは異なり、地域の課題解決に貢献する交通を、目的の設定から市民参加を加える形で設計、提供する計画手法で、2005年ごろから主に札幌、横浜、岡山、熊本など政令市における市民団体より勃興してきた。

% -----------------------------------------------------------------------------------------------------------------------------------------
\section{市民参加の捉え方}
\label{sec:ch3_citizen_participation}

% -----------------------------------------------------------------------------------------------------------------------------------------
\subsection{公共交通政策における市民参加に関する議論}

公共交通政策における市民参加に関する規範的な議論は、帰結主義に立って市民参加が情報提供と労力提供の役割を持つと説く太田(2009)、非帰結主義に立って6つの要件を示す太田(2000)、実証的に7つの要件を提示し、協働を通じて「新たな公共、地球の公益」を創造すべきとした森栗(2009)などがある。

近年の交通政策における市民参加に関する実証的な議論では、その市民参加形態は、主に不足する財源や人的資源を確保するため、市民が交通運営者として参画する物に集中している。ただ、こうしたスタンスではあくまで、どのように交通事業を成立させるか、また「目の前の交通に対する課題をどのように解決するか」という観点に基づいている。

こうしたスタンスは、こと「横ぐし」の領域を扱う地域政策においては、課題の相対化を妨げ、重要度の高い解決すべき課題を見落とす懸念がある。さらに交通事業者などの公益的な企業は、地域政策の構成員として「選ばれる」というプロセスを経ないことが多いため、概して一方的になりやすい可能性がある。

% -----------------------------------------------------------------------------------------------------------------------------------------
\subsection{公共サービスとしての公共交通}

一方で、地域公共交通は自治体の計画によって定義され、非競合性を持たないため、インフラとしてではなく公共サービスとして定義されるべきである。通信サービスや道路などの非競合性をある程度持つ財とは違い、利用できる時間や容量が、物理的に決定されているからである。つまり、一旦整備したところで供給量は簡単に変化させられない即時材であるとともに、また提供に必要な労力や資源には限りがある。市民参加を通じてそうした資源を導入できるとしても限界があるから、得られる効果とコストとのバランスについて検討すべきである。

% -----------------------------------------------------------------------------------------------------------------------------------------
\subsection{望ましい市民参加のスタンス}

以上を踏まえ、市民参加の類型としては、地域の目標のうち公共交通が貢献すべき領域を設定する、政策目標設定から関わる形態が望ましい。一方で交通サービスに関しては、他行政サービスと同列に比較され、その地域への貢献によって選択されるべきであろう。

現状の市民参加は、あくまで目の前にある課題の解決を旨としており、実証主義に基づいた主張通りに情報提供と労力提供の役割を持っている。しかしそれだけでは、全ての計画段階における批判はなし得ないものであり、少なくとも交通の「要件定義」段階である目標立案レベルへの市民参加は、公共サービスの一方性を排するためには重要な要素である。

こうした目標への市民参加は、それ自体がハンス・ケルゼンの指摘する民主主義の本質と深く結びついている。政策の目標、手段、あらゆる点に対して市民に批判するための回路を残すことで、武力によるクーデターがなくとも政権の構造転換をなしえる、というのがその主張である。

% -----------------------------------------------------------------------------------------------------------------------------------------
\section{実装ギャップの実証分析:Japan MaaS}
\label{sec:ch3_maaS_analysis}

% -----------------------------------------------------------------------------------------------------------------------------------------
\subsection{分析手法}

分析手法について、「計画における計画指標(Key Performance Indicator, 以下KPI)に事業利用状況以外の指標がある」「会議体において、交通事業者、システム供給者、学識者、自治体以外の市民参加がある」の二つの点を検討し、交通政策を地域政策として運営できているかの評価とする。

第1の分析項目については、MaaSが「事業の手段」としてではなく、「政策の手段」として機能しているかを確認するものである。この点から、評価指標の状況について観察対象とすることとした。MaaS事業自体の事業性や交通事業の利用状況だけではなく、それ以外の評価指標を計画に置いているかを確認する。

第2の分析項目については、地域における交通事業者など公益的な企業の特性を鑑み、市民参加が行われており目標や手段への批判プロセスがあるかを確認する。市民の行政への批判回路について、議会を通じた参加等はあるが、交通計画では協議会参加、グループヒアリング、アンケート調査、パブリックコメントが推奨手法として取り上げられている。これらのうち、合議への参加を通じて直接の批判とできるのは協議会参加のみである。

以上の検討より、会議体への市民の参加を指標としてとらえることとした。

% -----------------------------------------------------------------------------------------------------------------------------------------
\subsection{分析対象}

日本版MaaSは、その事業概要を一定のフォーマットに落とし込んで申請する仕組みとなっており、国土交通省は採択のたびにそれらをまとめた文書を公開している。これらには協議会の構成員や目標数値が記載されており、これを分析対象とする。2020年の日本版MaaS推進事業では、全国36の事業が認定されており、提出された計画については全件採択となっている。

ここで、本事業によって実施された事業の特性を説明するために、日本版MaaSとして展開された各事業について、要素技術の利用状況について確認し、類型化を試みる。日本版MaaSの主な要素技術として、サブスク・定期など企画乗車券、オンラインで交通情報が見られる、あるいは決済できるアプリケーション、病院、商業クーポンなど外部情報の提供、デマンド交通、シェアサイクル等の新モビリティ、さらにデータ連携基盤の構築や利用が認められた。

特にアプリケーションの提供が31事例、外部情報との連携が32事例と多かったものの、企画乗車券の導入と新モビリティの導入比率は低く、またデータ連携基盤の利用や構築は12事例と少なかった。なお、参入の高度化に必要なAPIをはじめとしたデータ連携のオープン化に関して言及しているのは2事例のみであった(加賀、しんゆり)。

% -----------------------------------------------------------------------------------------------------------------------------------------
\subsection{分析結果}

分析の結果、双方の判断を満たし、地域政策として日本版MaaSを運用している事例は存在しなかった。

第1の分析項目について、交通事業やMaaS事業の指標を導入している事業は32/36であったが、それ以外の地域目標を設定している事業は10のみであった。さらに第2の分析項目については、自治体が参加している事業に限定すると、パブリックコメントや協議会への市民参加を確認できた事業は2/32にとどまった。

\begin{table}[htbp]
\centering
\caption{Japan MaaS 36プロジェクトの評価指標分析}
\label{tab:maas_analysis}
\begin{tabular}{lcc}
\toprule
指標タイプ & プロジェクト数 & 割合 \\
\midrule
事業指標のみ & 32 & 89\% \\
非事業指標を含む & 10 & 28\% \\
市民参加の仕組み & 2 & 6\% \\
\bottomrule
\end{tabular}
\end{table}

% -----------------------------------------------------------------------------------------------------------------------------------------
\subsection{事例分析}

分析の射程を示すため、2つの事例を取り上げる。

\subsubsection{地方版MaaSの広域連携基盤構築モデル事業(ひたち圏域)}

日立市を中心とする茨城県北部の圏域におけるMaaS事業展開について、3市1村の協力の下で行われた日本版MaaS事業である。

本事業の目標は3つに分かれている。一つ目は「利用者関連指標」であり、取組ページへのアクセス数、アプリDL数、チケット販売数などが設定されている。二つ目は「交通事業者関連指標」であり、参加する事業者数により達成される。三つ目は「MaaS事業者関連指標」であり、接続事業者数を目標と据えている。

市民参加状況については、市民団体らしき仕組みは確認できず、市民代表者の参加形跡も確認できなかった。また、市民参加、パブリックコメントなどの情報を検索したが、見当たらなかった。

\subsubsection{鞆の浦MaaS}

広島県福山市の景勝地、鞆の浦におけるMaaS事業展開である。本事業の目標は、デジタルチケット発行数、アンケート指標(満足度70\%以上など)、電動レンタサイクル利用者の回遊エリア拡大状況に分かれている。

市民参加状況については、市民団体らしき仕組みは確認できなかったが、観光関連団体の代表として「公益社団法人福山観光コンベンション協会」が出席しており、これは交通事業者外の参加と認められる。

% -----------------------------------------------------------------------------------------------------------------------------------------
\subsection{考察}

分析対象とした日本版MaaS推進事業へ認定されたそれぞれは、その目標を考えると、ほとんどの場合交通事業の成立、また日本版MaaS事業の成立を旨としている。実際のところ、本事業を担当する国土交通省新モビリティサービス推進課の担当者は、「申請前の1年で事業検討して、申請後の1年で事業として自走できるようにしてほしい」と述べており、事業としての成立を企図している。一方で、こうしたスタンスは実際のところ、海外ではすでに行き詰まり感を見せている。日本におけるMaaSは、こうした国際的な流れを反映していない。

また交通政策の理念として、交通の機能について「地域の活力に貢献」するように施策を運営するためには、少なくとも地域がどのようになるべきかの指標を入れるべきである。交通事業、MaaS事業の指標のみでは、こうした運営には不十分である。それは交通事業の、利益を産む/費用を抑える事業としての価値こそ管理できるが、地域交通事業はこうした効率化が難しい現状を鑑みると、地域における機能については管理できないからである。

さらに、こうした行政機関が手がける交通サービスに対する批判回路についても、十分に構築されていない状況が見えてきた。公共交通政策は多くの場合、土木工学を主軸とした専門性を問われることから、こうした回路が機能しにくいと推察される。

% -----------------------------------------------------------------------------------------------------------------------------------------
\subsection{日本版MaaSは交通まちづくりの一類型として捉えられるか}

以上の議論を踏まえて、日本版MaaSを交通まちづくりとして捉えられるかについて議論する。交通まちづくりの基本線は、初期構想段階では市民参加を含んでいた以外は一貫しており、地域の課題を解決する方針のもと交通を計画し、提供するものであった。

これを踏まえると、日本版MaaSはその構想段階では交通まちづくりの要件に合うものの、実装段階においてはその要件に合致しないと理解できる。前者については、日本版MaaSの特徴として地域課題を解決する交通サービスの提供を支援する政策である点が該当する。一方で後者については、日本版MaaSの計画において交通サービスの利用状況が評価の中心にあり、必ずしも地域課題解決の状況をモニタリングしていない点が該当する。

このような状況になった要因として、いくつかの仮説を挙げておきたい。第1に、新たなガバナンスシステムの導入へのコストが高いことがあげられる。一般に、デジタル時代の政府活動への意向にはそれに即した評価、批判のシステムへ移行する必要があるが、それには取引費用がかかりすぎることが分かっている。第2に、これは第1の指摘と関連するが、自治体の持つ政策評価システムが業績評価に偏っている点があげられる。三重県の「さわやか運動」から始まった政策評価では、担当者が直接事業内容を説明する「事務事業評価」が核であり、その中で簡略化された評価として業績評価が利用されることが多い。これは一種の日本の評価システムの文化であり、日本版MaaSもその流れを汲んでいる可能性は十分にある。

% -----------------------------------------------------------------------------------------------------------------------------------------
\section{小括:公共交通政策を「実験場」として位置づける理由}
\label{sec:ch3_summary}

公共交通政策は、いくつかの理由から本研究の「実験場」として適している。第一に、複雑なステークホルダー関係である。国、自治体、交通事業者、市民など多様な主体が関与している。第二に、明確な政策目標である。持続可能な公共交通の維持・発展という目標が明確である。第三に、実装ギャップの可視化である。制度的意図と実践の乖離が観察可能である。第四に、データの入手可能性である。政策文書、統計データへのアクセスが比較的容易である。

本章の分析から、以下の示唆が得られた:

第一に、日本版MaaSは構想段階では「交通まちづくり」と捉えられるが、実装段階ではそうではない。政策的意図として「連携・共創」が謳われていても、実践レベルでは事業効率性の追求に偏向してしまう構造的課題が存在する。

第二に、交通政策の理念として、交通の機能について「地域の活力に貢献」するように施策を運営するためには、少なくとも地域がどのようになるべきかの指標を入れる必要がある。交通事業やMaaS事業の指標のみでは、こうした運営には不十分である。

第三に、行政機関が手がける交通サービスに対する批判回路についても、十分に構築されていない状況が見えてきた。公共交通政策は多くの場合、土木工学を主軸とした専門性を問われることから、こうした回路が機能しにくいと推察される。

「利用者視点」を導入しているとうたう交通政策が各地で展開されているが、少なくとも政策において相手にするのは市民であり、この国の主権者である。専門性が必ずしも正しさではなく、目指す方向性はイデオロギーであり批判されうることを前提として、民主主義国家の計画のあり方を交通の面から探る作業が求められる。

次章では、この実装ギャップの認知的要因を計算論的に分析する。

% -----------------------------------------------------------------------------------------------------------------------------------------
