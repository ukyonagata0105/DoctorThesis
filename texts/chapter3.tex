% -----------------------------------------------------------------------------------------------------------------------------------------
% 第3章 舞台としての公共交通政策:現状と課題
% -----------------------------------------------------------------------------------------------------------------------------------------

\chapter{舞台としての公共交通政策:現状と課題}
\label{chap:public_transport_policy}

% -----------------------------------------------------------------------------------------------------------------------------------------
\section{はじめに}
\label{sec:ch3_intro}

本章では、本研究の「舞台」となる公共交通政策の現状と課題を整理する。まず日本の公共交通政策の変遷を概観し、次に制度設計の現状を分析し、最後にJapan MaaSプロジェクトの実証分析を通じて実装ギャップの実態を明らかにする。

% -----------------------------------------------------------------------------------------------------------------------------------------
\section{日本の公共交通政策の変遷}
\label{sec:ch3_transition}

% -----------------------------------------------------------------------------------------------------------------------------------------
\subsection{5つの時期区分}

日本の地域公共交通政策は、2002年の規制緩和以降、大きく5つの時期に区分できる\cite{nagata2024maas}:

\begin{description}
    \item[第1期:規制緩和期(1997-2002)] 市場原理の導入、路線許可制から届出制への移行
    \item[第2期:地域協議導入期(2003-2008)] 地域公共交通会議制度の創設
    \item[第3期:民主党政権期(2009-2012)] 地方分権の最大化
    \item[第4期:網形成計画期(2013-2017)] ネットワーク再構築の重視
    \item[第5期:効率重視期(2018-現在)] 生産性向上・効率化の強調
\end{description}

% -----------------------------------------------------------------------------------------------------------------------------------------
\subsection{STOフレームワークによる分析}

Van de Velde (1999) \cite{vandevelde1999organisational} が提唱したSTOフレームワークは、公共交通政策を3つの意思決定階層で分析する手法である:

\begin{table}[htbp]
\centering
\caption{STOフレームワークによる政策分析}
\label{tab:sto_framework}
\begin{tabular}{llll}
\toprule
階層 & 主要な意思決定事項 & 想定期間 & 主な責任主体 \\
\midrule
S (Strategy) & 政策目標、長期ビジョン & 10年〜 & 国、広域自治体 \\
T1 (上位戦術) & サービス水準目標、路線網骨格 & 1-5年 & 国、自治体 \\
T2 (下位戦術) & 具体的ダイヤ・ルート & 半年-2年 & 交通事業者 \\
O (Operations) & 日常運行、維持管理 & 日々〜月 & 交通事業者 \\
\bottomrule
\end{tabular}
\end{table}

% -----------------------------------------------------------------------------------------------------------------------------------------
\section{制度設計の現状}
\label{sec:ch3_institutional_design}

% -----------------------------------------------------------------------------------------------------------------------------------------
\subsection{官民連携(PPP)と官官連携(PuP)}

公共交通政策における連携形態は、大きく分けて官民連携(Public-Private Partnership, PPP)と官官連携(Public-Public Partnership, PuP)がある。

\subsubsection{官民連携(PPP)の事例}
\begin{itemize}
    \item 空港コンセッション(北海道・関西エアポート)
    \item 駅無人化への地域対応(滝沢市とJR東日本)
    \item デマンド交通の導入(雲南市だんだんタクシー)
\end{itemize}

\subsubsection{官官連携(PuP)の事例}
\begin{itemize}
    \item 横の連携:盛岡都市圏地域公共交通計画
    \item 縦の連携:地域公共交通会議制度
    \item 階層型連携:前橋都市圏の自治体バス広域連携
\end{itemize}

% -----------------------------------------------------------------------------------------------------------------------------------------
\subsection{地域公共交通計画の義務化}

2014年の法改正により、地域公共交通網形成計画の策定が自治体に義務化された。この制度は、国がSレベルの枠組みを提供し、自治体がT1レベルで具体化する役割分担を想定している。

しかし、自治体の資源制約により、形式的な計画策定に留まるケースも指摘されている。

% -----------------------------------------------------------------------------------------------------------------------------------------
\section{実装ギャップの実証分析:Japan MaaS}
\label{sec:ch3_maaS_analysis}

% -----------------------------------------------------------------------------------------------------------------------------------------
\subsection{Japan MaaSの概要}

Japan MaaSは、MaaS(Mobility as a Service)の概念を日本の地域課題解決に応用する公的政策イニシアティブである。2020年度に認定された38プロジェクトを対象に、計画書の内容分析を行った。

% -----------------------------------------------------------------------------------------------------------------------------------------
\subsection{分析結果}

分析の結果、以下の実装ギャップが明らかになった:

\begin{table}[htbp]
\centering
\caption{Japan MaaS 38プロジェクトの評価指標分析}
\label{tab:maas_analysis}
\begin{tabular}{lcc}
\toprule
指標タイプ & プロジェクト数 & 割合 \\
\midrule
事業指標のみ & 35 & 92\% \\
非事業指標を含む & 11 & 29\% \\
市民参加の仕組み & 2 & 5\% \\
\bottomrule
\end{tabular}
\end{table}

この結果は、「連携・共創」の政策的意図が、実践レベルでは事業効率性の追求に偏向していることを示している。

% -----------------------------------------------------------------------------------------------------------------------------------------
\subsection{構造的要因}

実装ギャップの背景には、以下の構造的要因が存在する:

\begin{enumerate}
    \item \textbf{自治体の資源制約}:「逆三角形」負担構造(地方分権改革推進有識者会議, 2023)
    \item \textbf{技術的複雑性}:MaaS導入に必要な専門知識の不足
    \item \textbf{ステークホルダー間の利害対立}:交通事業者、自治体、市民の間での優先順位の相違
    \item \textbf{認知的要因}:意思決定における認知バイアスの影響(第4章で詳述)
\end{enumerate}

% -----------------------------------------------------------------------------------------------------------------------------------------
\section{小括:公共交通政策を「実験場」として位置づける理由}
\label{sec:ch3_summary}

公共交通政策は、以下の理由から本研究の「実験場」として適している:

\begin{enumerate}
    \item \textbf{複雑なステークホルダー関係}:国、自治体、交通事業者、市民など多様な主体が関与
    \item \textbf{明確な政策目標}:持続可能な公共交通の維持・発展
    \item \textbf{実装ギャップの可視化}:制度的意図と実践の乖離が観察可能
    \item \textbf{データの入手可能性}:政策文書、統計データへのアクセスが比較的容易
\end{enumerate}

次章では、この実装ギャップの認知的要因を計算論的に分析する。

% -----------------------------------------------------------------------------------------------------------------------------------------
