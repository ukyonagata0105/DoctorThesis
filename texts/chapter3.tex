% -----------------------------------------------------------------------------------------------------------------------------------------
% 第3章 舞台としての公共交通政策:現状と課題
% -----------------------------------------------------------------------------------------------------------------------------------------

\chapter{舞台としての公共交通政策:現状と課題}
\label{chap:public_transport_policy}

% -----------------------------------------------------------------------------------------------------------------------------------------
\section{はじめに}
\label{sec:ch3_intro}

本章では、本研究の「舞台」となる公共交通政策の現状と課題を整理する。まず日本の公共交通政策の変遷を概観し、次に制度設計の現状を分析し、最後にJapan MaaSプロジェクトの実証分析を通じて実装ギャップの実態を明らかにする。

本章における分析モデルとして、政策過程論を採用する。政策過程は、目的を設定した上で現状を理解し、そのギャップを課題として認識した上でそれを解決するための方策を列挙し、実施する政策を決定するという一連の流れである。この中で、どのように目標を立て、またどのように目標を設定しているかが、本章における大まかな分析対象である。

本章でいう「交通まちづくり」の概念は、太田(2008)の述べるような「交通に関連する地域の課題への対応をベースにして、市民と行政が協働して進めるまちづくり」である。原田編(2015)の述べるところの定義である「まちづくりの目的に貢献する交通計画」とは異なり、「市民の参加と協働により発展的に進める活動プロセス」である。この中では、「交通計画における住民、市民の参加」「都市計画、都市づくりとの連携」の2点が重要視されている。

% -----------------------------------------------------------------------------------------------------------------------------------------
\section{日本の公共交通政策の変遷}
\label{sec:ch3_transition}

% -----------------------------------------------------------------------------------------------------------------------------------------
\subsection{5つの時期区分}

日本の地域公共交通政策は、2002年の規制緩和以降、大きく5つの時期に区分できる\cite{nagata2024maas}:

\begin{description}
    \item[第1期:規制緩和期(1997-2002)] 市場原理の導入、路線許可制から届出制への移行
    \item[第2期:地域協議導入期(2003-2008)] 地域公共交通会議制度の創設
    \item[第3期:民主党政権期(2009-2012)] 地方分権の最大化
    \item[第4期:網形成計画期(2013-2017)] ネットワーク再構築の重視
    \item[第5期:効率重視期(2018-現在)] 生産性向上・効率化の強調
\end{description}

% -----------------------------------------------------------------------------------------------------------------------------------------
\subsection{STOフレームワークによる分析}

Van de Velde (1999) \cite{vandevelde1999organisational} が提唱したSTOフレームワークは、公共交通政策を3つの意思決定階層で分析する手法である:

\begin{table}[htbp]
\centering
\caption{STOフレームワークによる政策分析}
\label{tab:sto_framework}
\begin{tabular}{llll}
\toprule
階層 & 主要な意思決定事項 & 想定期間 & 主な責任主体 \\
\midrule
S (Strategy) & 政策目標、長期ビジョン & 10年〜 & 国、広域自治体 \\
T1 (上位戦術) & サービス水準目標、路線網骨格 & 1-5年 & 国、自治体 \\
T2 (下位戦術) & 具体的ダイヤ・ルート & 半年-2年 & 交通事業者 \\
O (Operations) & 日常運行、維持管理 & 日々〜月 & 交通事業者 \\
\bottomrule
\end{tabular}
\end{table}

% -----------------------------------------------------------------------------------------------------------------------------------------
\section{制度設計の現状}
\label{sec:ch3_institutional_design}

% -----------------------------------------------------------------------------------------------------------------------------------------
\subsection{官民連携(PPP)と官官連携(PuP)}

公共交通政策における連携形態は、大きく分けて官民連携(Public-Private Partnership, PPP)と官官連携(Public-Public Partnership, PuP)がある。

\subsubsection{官民連携(PPP)の事例}
\begin{itemize}
    \item 空港コンセッション(北海道・関西エアポート)
    \item 駅無人化への地域対応(滝沢市とJR東日本)
    \item デマンド交通の導入(雲南市だんだんタクシー)
\end{itemize}

\subsubsection{官官連携(PuP)の事例}
\begin{itemize}
    \item 横の連携:盛岡都市圏地域公共交通計画
    \item 縦の連携:地域公共交通会議制度
    \item 階層型連携:前橋都市圏の自治体バス広域連携
\end{itemize}

% -----------------------------------------------------------------------------------------------------------------------------------------
\subsection{地域公共交通計画の義務化}

2014年の法改正により、地域公共交通網形成計画の策定が自治体に義務化された。この制度は、国がSレベルの枠組みを提供し、自治体がT1レベルで具体化する役割分担を想定している。

しかし、自治体の資源制約により、形式的な計画策定に留まるケースも指摘されている。

% -----------------------------------------------------------------------------------------------------------------------------------------
\section{実装ギャップの実証分析:Japan MaaS}
\label{sec:ch3_maaS_analysis}

% -----------------------------------------------------------------------------------------------------------------------------------------
\subsection{分析の枠組みとリサーチクエスチョン}

本章では、いわゆる「日本版MaaS」を「交通まちづくり」実践の一類型として捉え、以下の2つの問いについて検討する:

\begin{enumerate}
    \item 「日本版MaaS」の政策進捗において目標設定がどこに置かれているか
    \item 立場を問わない市民による批判回路がどのように整備されているのか
\end{enumerate}

本研究は、日本版MaaSを「MaaS」としてではなく、「交通まちづくり」の一類型として分析し、その目的意識と計画における責任体系の構築状況、すなわちガバナンス状況を探るものである。

分析手法について、「計画における計画指標(Key Performance Indicator, KPI)に事業利用状況以外の指標がある」「会議体において、交通事業者、システム供給者、学識者、自治体以外の市民参加がある」の二つの点を検討し、交通政策を地域政策として運営できているかの評価とする。

% -----------------------------------------------------------------------------------------------------------------------------------------
\subsection{分析対象}

日本版MaaSは、その事業概要を一定のフォーマットに落とし込んで申請する仕組みとなっており、国土交通省は採択のたびにそれらをまとめた文書を公開している。これらには協議会の構成員や目標数値が記載されており、これを分析対象とする。

2020年の日本版MaaS推進事業では、全国36の事業が認定されており、提出された計画については全件採択となっている。本事業によって実施された事業の特性を説明するために、日本版MaaSとして展開された各事業について、要素技術の利用状況を確認し、類型化を試みる。

日本版MaaSの主な要素技術として、サブスク・定期など企画乗車券、オンラインで交通情報が見られる、あるいは決済できるアプリケーション、病院、商業クーポンなど外部情報の提供、デマンド交通、シェアサイクル等の新モビリティ、さらにデータ連携基盤の構築や利用が認められた。特にアプリケーションの提供が31事例、外部情報との連携が32事例と多かったものの、企画乗車券の導入と新モビリティの導入比率は低く、またデータ連携基盤の利用や構築は12事例と少なかった。

% -----------------------------------------------------------------------------------------------------------------------------------------
\subsection{分析結果}

分析の結果、双方の判断を満たし、地域政策として日本版MaaSを運用している事例は存在しなかった。

第一に、交通事業やMaaS事業の指標を導入している事業は32/36であったが、それ以外の地域目標を設定している事業は10のみであった。第二に、自治体が参加している事業に限定すると、パブリックコメントや協議会への市民参加を確認できた事業は2/32にとどまった。

\begin{table}[htbp]
\centering
\caption{Japan MaaS 38プロジェクトの評価指標分析}
\label{tab:maas_analysis}
\begin{tabular}{lcc}
\toprule
指標タイプ & プロジェクト数 & 割合 \\
\midrule
事業指標のみ & 32 & 89\% \\
非事業指標を含む & 10 & 28\% \\
市民参加の仕組み & 2 & 6\% \\
\bottomrule
\end{tabular}
\end{table}

この結果は、「連携・共創」の政策的意図が、実践レベルでは事業効率性の追求に偏向していることを示している。

% -----------------------------------------------------------------------------------------------------------------------------------------
\subsection{構造的要因}

実装ギャップの背景には、以下の構造的要因が存在する:

\begin{enumerate}
    \item \textbf{自治体の資源制約}:「逆三角形」負担構造(地方分権改革推進有識者会議, 2023)
    \item \textbf{技術的複雑性}:MaaS導入に必要な専門知識の不足
    \item \textbf{ステークホルダー間の利害対立}:交通事業者、自治体、市民の間での優先順位の相違
    \item \textbf{認知的要因}:意思決定における認知バイアスの影響(第4章で詳述)
\end{enumerate}

新たなガバナンスシステムの導入へのコストが高いことも要因として挙げられる。一般に、デジタル時代の政府活動への移行には、それに即した評価・批判のシステムへの移行が必要であるが、それには取引費用がかかりすぎることが分かっている。また、自治体の持つ政策評価システムが業績評価に偏っている点も指摘されている。三重県の「さわやか運動」から始まった政策評価では、担当者が直接事業内容を説明する「事務事業評価」が核であり、その中で簡略化された評価として業績評価が利用されることが多い。これは一種の日本の評価システムの文化であり、日本版MaaSもその流れを汲んでいる可能性がある。

% -----------------------------------------------------------------------------------------------------------------------------------------
\subsection{事例分析}

分析の射程を示すため、2つの事例を取り上げる。

\subsubsection{地方版MaaSの広域連携基盤構築モデル事業(ひたち圏域)}

日立市を中心とする茨城県北部の圏域におけるMaaS事業展開について、3市1村の協力の下で行われた日本版MaaS事業である。

本事業の目標は3つに分かれている。一つ目は「利用者関連指標」であり、取組ページへのアクセス数、アプリDL数、チケット販売数などが設定されている。二つ目は「交通事業者関連指標」であり、参加する事業者数により達成される。三つ目は「MaaS事業者関連指標」であり、接続事業者数を目標と据えている。

市民参加状況については、市民団体らしき仕組みは確認できず、市民代表者の参加形跡も確認できなかった。また、市民参加、パブリックコメントなどの情報を検索したが、見当たらなかった。

\subsubsection{鞆の浦MaaS}

広島県福山市の景勝地、鞆の浦におけるMaaS事業展開である。本事業の目標は、デジタルチケット発行数、アンケート指標(満足度70\%以上など)、電動レンタサイクル利用者の回遊エリア拡大状況に分かれている。

市民参加状況については、市民団体らしき仕組みは確認できなかったが、観光関連団体の代表として「公益社団法人福山観光コンベンション協会」が出席しており、これは交通事業者外の参加と認められる。

% -----------------------------------------------------------------------------------------------------------------------------------------
\subsection{日本版MaaSは交通まちづくりの一類型として捉えられるか}

以上の議論を踏まえて、日本版MaaSを交通まちづくりとして捉えられるかについて議論する。

交通まちづくりの基本線は、初期構想段階では市民参加を含んでいた以外は一貫しており、地域の課題を解決する方針のもと交通を計画し、提供するものであった。

これを踏まえると、日本版MaaSはその構想段階では交通まちづくりの要件に合うものの、実装段階においてはその要件に合致しないと理解できる。前者については、日本版MaaSの特徴として地域課題を解決する交通サービスの提供を支援する政策である点が該当する。一方で後者については、日本版MaaSの計画において交通サービスの利用状況が評価の中心にあり、必ずしも地域課題解決の状況をモニタリングしていない点が該当する。

% -----------------------------------------------------------------------------------------------------------------------------------------
\section{小括:公共交通政策を「実験場」として位置づける理由}
\label{sec:ch3_summary}

公共交通政策は、以下の理由から本研究の「実験場」として適している:

\begin{enumerate}
    \item \textbf{複雑なステークホルダー関係}:国、自治体、交通事業者、市民など多様な主体が関与
    \item \textbf{明確な政策目標}:持続可能な公共交通の維持・発展
    \item \textbf{実装ギャップの可視化}:制度的意図と実践の乖離が観察可能
    \item \textbf{データの入手可能性}:政策文書、統計データへのアクセスが比較的容易
\end{enumerate}

本章の分析から、以下の示唆が得られた:

第一に、日本版MaaSは構想段階では「交通まちづくり」と捉えられるが、実装段階ではそうではない。政策的意図として「連携・共創」が謳われていても、実践レベルでは事業効率性の追求に偏向してしまう構造的課題が存在する。

第二に、交通政策の理念として、交通の機能について「地域の活力に貢献」するように施策を運営するためには、少なくとも地域がどのようになるべきかの指標を入れる必要がある。交通事業やMaaS事業の指標のみでは、こうした運営には不十分である。

第三に、行政機関が手がける交通サービスに対する批判回路についても、十分に構築されていない状況が見えてきた。公共交通政策は多くの場合、土木工学を主軸とした専門性を問われることから、こうした回路が機能しにくいと推察される。

「利用者視点」を導入しているとうたう交通政策が各地で展開されているが、少なくとも政策において相手にするのは市民であり、この国の主権者である。専門性が必ずしも正しさではなく、目指す方向性はイデオロギーであり批判されうることを前提として、民主主義国家の計画のあり方を交通の面から探る作業が求められる。

次章では、この実装ギャップの認知的要因を計算論的に分析する。

% -----------------------------------------------------------------------------------------------------------------------------------------
