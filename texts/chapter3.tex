% -----------------------------------------------------------------------------------------------------------------------------------------
% 第3章 舞台としての公共交通政策:現状と課題
% -----------------------------------------------------------------------------------------------------------------------------------------

\chapter{舞台としての公共交通政策:現状と課題}
\label{chap:public_transport}

% -----------------------------------------------------------------------------------------------------------------------------------------
\section{はじめに}
\label{sec:ch3_introduction}

% -----------------------------------------------------------------------------------------------------------------------------------------
\subsection{本稿の前提}

近年,日本において「Mobility as a Service」の勃興が著しい.2019年度より始まった「日本版MaaS推進事業」は,瞬く間に全国各地に広がり,現在までに100事業以上のプロジェクトを送り出している.本稿ではこうしたプロジェクトがどのような体制の下に進められ,またどのような検証体制が敷かれているかについて,国土交通省の公開している資料を基として実証的に考察する.なお,本稿における「日本版MaaS」は,国土交通省による定義に基づく.

前提として,本稿のスタンスを明らかにする.本稿でいう「公共交通」とは,他人の運転の提供によって,移動できるようになるサービスであり,これはタクシー,ライドシェアなどを含む.

また本稿における分析モデルとして,政策過程論を採用する.政策過程は,目的を設定した上で現状を理解し,そのギャップを課題として認識した上でそれを解決するための方策を列挙し,実施する政策を決定するという一連の流れである.この中で,どのように目標をたて,またどのように目標を設定しているかが,本稿における大まかな分析対象である.

本稿で言う「交通まちづくり」の概念は,太田\textsuperscript{1)}の述べるような「交通に関連する地域の課題への対応をベースにして,市民と行政が協働して進めるまちづくり」である.原田編\textsuperscript{2)}の述べるところの定義である「まちづくりの目的に貢献する交通計画」とは異なり,「市民の参加と協働により発展的に進める活動プロセス」である.この中では,「交通計画における住民,市民の参加」「都市計画,都市づくりとの連携」の2点が重要視されている.

% -----------------------------------------------------------------------------------------------------------------------------------------
\subsection{先行研究}

日本版MaaSについて,その特徴を述べた文献はいくつかある.

まず注目されるのは,特に過疎地域や地方中核市における交通が抱える課題を解決するためのアプローチとしての見立てである.魏\textsuperscript{3)}は,日本のMaaSの注目点が「地方型MaaS」である点に言及したうえで,日本のMaaSの特徴について以下のように述べている.

\begin{quote}
「地方型」MaaS は,地域の資源や特性を最大限活かしながらより満足度の高いモビリティサービスを提供することであり,地域にある交通モード間の連携強化により,その場その場で発生した交通需要(生活交通需要,観光交通需要)を上手く汲み上げ,集約することで実現できるものである.
\end{quote}

また藤本\textsuperscript{4)}では,欧州におけるMaaSが「都市型MaaS」と「地方型MaaS」に分化されており,また日本版MaaSが「地方型」を含む5類型に分割されている点に触れたうえで,地方型MaaSを検討するポイントを5つに整理している.

MaaS自体の実装にかかわる研究も進んでいる.後藤\textsuperscript{5)}は,MaaSを旅行業の一類型ととらえ,企画旅行業務を適用する可能性やデータ供給の課題について論じている.

既存の公共交通運営体制に関する議論は,多くの場合「どのような取り組み」があったか,また「どのような評価が可能か」という2つに大別される.前者の場合,生活バスよっかいちについての事例報告を行った福本・加藤\textsuperscript{6)}や,多くの日本モビリティマネジメント会議の報告に代表される.また「どのような評価が可能か」については,アクセシビリティ評価の地域への適用について実例を分析した喜多\textsuperscript{7)}などがあげられる.公共交通に関する責任分担や意思決定組織についての検討も複数ある.

本稿で前提とした「交通まちづくり」の考えでは,交通政策における市民参加と地域戦略との連携が重視されていると考えられる.また後述するものであるが,地域公共交通に関する法体系においても,交通政策の戦略性を重視している.ただ少なくとも「日本版MaaS」において,法体系からの課題意識を導入した包括的な研究,また交通事業者としてではない形で計画の要件定義から携わる市民参画の有無,に関する実証的な研究は見当たらなかった.

このように,上記に示したような実証的な研究を実施し,結果として「交通まちづくり」の思想を援用できないか検討する作業には新規性があり,結果によっては思想によるバックアップをもとに日本におけるMaaSの広がりを支援できる可能性がある.

本稿のクエスチョンと仮説}\label{ux672cux7a3fux306eux30afux30a8ux30b9ux30c1ux30e7ux30f3ux3068ux4eeeux8aac}

本稿は,いわゆる「日本版MaaS」を「交通まちづくり」実践の一類型と捉え,以下の2つの問いについて検討し,結果としてこの捉え方が適切かについて確認するものである.

・「日本版MaaS」の政策進捗において目標設定がどこに置かれているか

・立場を問わない市民による批判回路がどのように整備されているのか

本分析における仮説としてはそれぞれ,

・目標設定は,MaaS事業の事業性に関して設定されるが,それ以外の地域に関する効果については設定されない

・市民による批判回路は,協議会に閉じず,パブリックコメント,市民代表参加,座談会,ワークショップなどの機会を通じて設定されている

と設定している.

本稿の流れ}\label{ux672cux7a3fux306eux6d41ux308c}

日本版MaaSについて議論する前にまず,2章及び3章では,日本および勃興場所であるフィンランドにおけるMaaSの展開について簡単にまとめる.次に4章では,公共交通に関わる法体系についてまとめ,前提となる理論について記述する.上記2点については前者の問いに対応する.5章にて「市民参加」に関する理論体系を確認する.これは後者の問いに対応する.これを踏まえて,6章では事業2年目で件数が一番多く,「地域特有の課題の解決に寄与することが見込まれる,地域特性に応じたMaaSのモデルとなり得る」と紹介され,本研究の方向性と合致する2020年の「日本版MaaS」において行われた事業について総攬し,全ての事業に対して前述の問いを確認する.最後に,現在の政策に提供される示唆をまとめる.

\section{フィンランドにおけるMaaSの勃興:提案初期論文の確認と法改正}\label{ux30d5ux30a3ux30f3ux30e9ux30f3ux30c9ux306bux304aux3051ux308bmaasux306eux52c3ux8208ux63d0ux6848ux521dux671fux8ad6ux6587ux306eux78baux8a8dux3068ux6cd5ux6539ux6b63}

初期に提案されたMaaSの特徴}\label{ux521dux671fux306bux63d0ux6848ux3055ux308cux305fmaasux306eux7279ux5fb4}

まずMaaSは,Hekkilä\textsuperscript{8)}にて初めて学術的な表舞台に立った,公共交通の新しい提供概念である.氏が所属する大学へ依頼されたヘルシンキ市の交通課題解決のために,修士論文で書き上げたものである.これによると,Nokiaを主軸としたフィンランドの通信政策やその事業に対する理解の蓄積が,ほぼそのまま交通政策に導入されていることがわかる.

この論文では,論文冒頭のサマリーにおいてその目的を明確に示している.具体的には,公共交通の利用促進が叫ばれ,都市構造への影響が避けられない中で,既存の交通サービス目標は「課題に十分にこたえられていない」と批判している.そのうえで,「公共交通セクターの構造転換が必要である」として「多様で魅力的な交通サービスの便利な提供」としてMaaSを定義している.

具体的なMaaSの構造として,「交通サービスプロバイダーから交通サービスを買い,それをユーザーへ提供する」主体を「モビリティオペレーター」と定義している.結果として,交通サービスを従来提供していた事業者と利用者の間に,もう一つの主体が挟まることが特徴である.

公共政策上のMaaSに必要な統治構造変革}\label{ux516cux5171ux653fux7b56ux4e0aux306emaasux306bux5fc5ux8981ux306aux7d71ux6cbbux69cbux9020ux5909ux9769}

このMaaSコンセプトを実現するうえで,政策上で何が必要であるのかについて,Heikkiläは表-1のように,7つの政策目標を提示している.

表-1 MaaSを進めるのに必要な要件

(Hekkilä\textsuperscript{8)}より筆者翻訳・作成)

{\def\LTcaptype{none} % do not increment counter
\begin{longtable}[]{@{}
  >{\raggedright\arraybackslash}p{(\linewidth - 4\tabcolsep) * \real{0.0408}}
  >{\raggedright\arraybackslash}p{(\linewidth - 4\tabcolsep) * \real{0.1916}}
  >{\raggedright\arraybackslash}p{(\linewidth - 4\tabcolsep) * \real{0.2496}}@{}}
\toprule\noalign{}
\endhead
\bottomrule\noalign{}
\endlastfoot
& 項目名 & 具体的な内容 \\
1 & 企業,当局,機関,ユーザーなど,すべてのステークホルダーの協力体制の調整 & エコシステムの充足のためだけでなく,すべてのステークホルダーのモチベーションを高めるために,すべてのステークホルダーの要求を開発プロセスに取り入れる協力関係を構築せねばならない. \\
2 & サービス・エコシステムの前提条件を満たすための法律・規制の改正 & 法律,規制改正の必要性がある.現行の法律において,サービス・エコシステムの開発や新企業の市場参入を妨げている法規制を取り除く必要がある.自治体は,利害関係者の提案を集めて上位のガバナンスに伝達することで,仲介者として機能できる. \\
3 & 共通のルールと適切な規制の構築,およびその遵守状況の監視 & 新興市場が機能し,すべてのステークホルダーの利益を確保するためには,適切な規制が必要である.規制は,真の健全な競争と企業の成長を保証するものでなければならない. \\
4 & モビリティサービス提供の再編成 & 交通サービスの提供を変革し,HSL(ヘルシンキ交通局)による公共交通機関のチケット販売の独占を撤廃し,他の企業も公共交通機関のチケットを販売できるようにする必要がある. \\
5 & 変革されたオペレーションの確立 & 変革された事業を確立するためには,現在のビジネスモデルや構造を変更し,モビリティサービス事業者を生み出す必要がある.そのため,サービス・エコシステムの規制をクリアした事業者の市場参入をサポートする必要がある. \\
6 & 購入・補助金手続きの見直し & 現在,自治体や国はHSLを通じて公共交通機関を補助し,交通サービスを交通サービス事業者から直接購入している.組織の変革では,購入と補助金はMaaSオペレーター(≒販売者)に直結し,そこから他のレベルに資金が分配されることになる.このような資金の配分は,(サービス提供者がどのように投資するかを決定できるため,)サービス事業者がモビリティのニーズを革新的に満たすための余地を提供する. \\
7 & パイロットとテストエリア & 新しいサービスなどのオペレーションをまず小規模にテストした後,プロトタイプを開発し,再び市場に投入するフィードバックシステムを機能させことができる. \\
\end{longtable}
}

このような政策目標を実現するために,フィンランドでは多様な法改正が行われている.「モビリティ・ローミング」の実現を目標として,既存サービス事業者及び参入する交通事業者が交通に関する情報提供を義務付けられるとともに,チケット発行と決済に関するシステムへの相互運用が可能となる法律が成立, 2017年に施行されたという\footnote{石神(2020)}.

MaaSとインターネットの類似性}\label{maasux3068ux30a4ux30f3ux30bfux30fcux30cdux30c3ux30c8ux306eux985eux4f3cux6027}

次に,そもそものインターネット産業における構造を見て,その構造の類似点を探ってみる.そもそもフィンランドでは,こうした通信産業と運輸産業に関する規制を,同一の機関が担当している.

インターネットの特徴は,「いかなる情報であっても運搬可能とすること」,そのために「デジタル化」を挟むことである.デジタル化とは,従来連続分布であったものの「断片化」,つまり離散分布化である.これをシステムに当てはめるのであれば,今まであやふやであった役割を定義し,分割し,その接続要件を定義し,以てシステム全体の安全性や機能を担保することである.\footnote{企業の水平統合への議論にも繋がる.機能の分化,統合を行わずシステム全体が瓦解あるいは不安定になった事例として,日本航空(日本航空,日本エアシステム)やみずほ銀行(第一勧業銀行,富士銀行,日本興業銀行)があげられる.}

先述の論文で述べられているのは,交通に関わる様々な機能を分化させ,接続して全体システムを機能させる構造図である。Heikkilä(2014)は「Reorganized Mobility Scheme」として,交通サービスプロバイダーから交通サービスを買い,それをユーザーへ提供する「モビリティオペレーター」という主体を定義し,交通事業者と利用者の間に新たな仲介者を挟むモデルを提唱している。

\begin{figure}[htbp]
\centering
\includegraphics[width=0.7\textwidth]{figure/ch3/image1.png}
\caption{Hekkilä(2014)で提唱されるMaaSの主体関係モデル(Reorganized Mobility Scheme、筆者翻訳・作成)}
\label{fig:ch3_hekkila_model}
\end{figure}

そもそもフィンランドはIT企業ノキアを擁しており,IT産業については同様の産業蓄積がある.Heikkila\textsuperscript{8)}では,Caseyの議論を引用して,元々断片的であったシステム群を組み合わせ,さまざまなプレーヤー,サービス,技術を参入させることで,クオリティをある程度犠牲にしつつも分散的かつオープンな市場を形成した「インターネット革命」について言及している.また,次いで引用される,将来的な運輸連合の在り方について議論したドイツ運輸事業者連合会(VDV)の資料には,MaaSに近い議論がある.具体的には既存の運輸連合とは異なり,公共交通事業者と消費者の間に「一つのシンプルなインターフェース」である組織を噛ませることで,より消費者に合った商品開発,情報提供,パッケージングを通じてサービスを魅力的にできるだろう,としている.

以上を纏めると,フィンランドにて定義されたMaaSに存在するインターネットエコシステムと類似の概念は3点ある.1点目はモジュール化であり,様々な機能についてルールを決め,目的ごとに分化して整備させることで,システム上でのメタボリズムと堅牢性を確保する.2点目はネットワークであり,前述のモジュールを接続することで,一つの目的を完結させることを支援する.もう1点目はインターフェースの統合であり,様々なサービスがあったとしても,ユーザーとの接点については統合して管理できるようにする.

欧米におけるMaaSの学術上の定義と分析}\label{ux6b27ux7c73ux306bux304aux3051ux308bmaasux306eux5b66ux8853ux4e0aux306eux5b9aux7fa9ux3068ux5206ux6790}

特に欧米においては,MaaSの内容について,エコシステムやモデル面から学術面の分析が進んでいる.

Kamargianni,Matyas\textsuperscript{12)}は,MaaSのエコシステムとしてコアにデータ提供者,交通事業者,ユーザーを置き,ビジネス環境の構成要員に大学や労働組合,政策立案の主体まで含めたモデルを提唱した.

さらに, Arby,Karlsson,Sarasini\textsuperscript{13)}は,MaaSの段階として,ビジネスの観点から,以下の4レベルに分けたMaaSのモデルを紹介した.具体的には, レベル0:統合なし レベル1:情報の統合 レベル2:予約・支払いの統合 レベル3:提供するサービスの統合 レベル4:社会全体目標の統合 である.

一方で,MaaSというモデル自体の限界も指摘されている.そもそも交通事業者から受け取る手数料が少額であり,また交通がMaaSで提供される統合アプリケーションで購入されているケースも,例えばフランダースでは3\%程度に過ぎないという\footnote{Zipper(2020)}.

\section{3.日本版MaaSの展開と捉え方}\label{ux65e5ux672cux7248maasux306eux5c55ux958bux3068ux6349ux3048ux65b9}

翻って,日本での展開は如何様に進んだのだろうか.ここでは,公開されている関連通達の歴史を総攬し,どのような公式の主張がなされたのかを確認する.そして,これの個別事業について,どのような分析の方法をあてはめるのが適切かについて議論する.

日本におけるMaaS関連事業の基礎研究と展開}\label{ux65e5ux672cux306bux304aux3051ux308bmaasux95a2ux9023ux4e8bux696dux306eux57faux790eux7814ux7a76ux3068ux5c55ux958b}

日本における,中央官庁によるMaaS推進事業は,主に国土交通省と経済産業省が所管して展開されてきた.2019年には経済産業省と共同で「スマートモビリティチャレンジ」が開始され,全国28事業が指定されることとなった\footnote{事業目的によりガバナンスが行われており,その目的となる5つは以下のとおりである:①他の移動と重ね掛けして効率化②モビリティでのサービス提供③需要側の変容を促す仕掛け④異業種との連携による収益活用,付加価値の創出⑤モビリティ関連データ取得,交通・都市政策との連携.なお,このプロジェクトは2020年に52事業へ拡大することとなる.}.さらに国土交通省単独でも,2019年より「日本版MaaS推進事業」が推進され,初年度においては19事業,2020年度には36事業が選定された\footnote{国土交通省.}.

2020年3月に策定,その後バージョン2にアップデートされた「MaaS関連データの連携に関するガイドライン」では,「Society5.0リファレンスアーキテクチャ」に基づき,MaaSエコシステムも併せて検討が行われている.利用されるデータを「MaaS関連データのうち,各MaaSにおいて設定された最低限のルール等に基づき,当該MaaSプラットフォームを利用する全てのデータ利用者が利用可能なものとして,当該プラットフォームに提供等が行われるデータ」としての「協調的データ」と,「MaaS関連データのうち,当該データの提供者との契約等により個別に共有が行われるものとして,各MaaSプラットフォームに提供等が行われるデータ」としての「競争的データ」に分割して定義される.各交通事業者は,公開されるデータの共通化(フォーマット,言葉の意味など)について共通のものを整備することが推奨されている.また,バス及びフェリーについては,前者はGTFS-JP,後者は「標準的なフェリー・旅客船航路情報フォーマット仕様書(ver2.0)」は標準仕様として設定されている.

日本版MaaSの定義}\label{ux65e5ux672cux7248maasux306eux5b9aux7fa9}

このような国土交通省の動きは,国土交通省の定義する「日本版MaaS」の考えに基づいている.「日本版MaaS」の定義についてはこのように述べられている.

\begin{quote}
日本版MaaSは,地域住民や旅行者一人一人のトリップ単位での移動ニーズに対応して,複数の公共交通やそれ以外の移動サービスを最適に組み合わせて検索・予約・決済等を一括で行うサービスであり,観光や医療等の目的地における交通以外のサービス等との連携により,移動の利便性向上や地域の課題解決にも資する重要な手段となるものです.\footnote{国土交通省.}
\end{quote}

この定義を参照すると,日本におけるMaaSは構造的なものではなく,地域課題解決ベースのサービス提供について重きを置いている.これに基づけば日本版MaaSは,いかに交通サービスの対価支払い・生産・提供の一連の消費活動を組み替える意図は持たない.つまり,国土交通省が考えるMaaSは,制度やシステムの変革というよりむしろ,交通事業者が提供する公共交通サービスについての改善アイデアコンテスト並びに実証実験,ととらえた方が説明はつきやすい.

国土交通省の引用するMaaSの定義は,基本的に「都市と地方の新たなモビリティサービス懇談会」に依拠したものになる.具体的には,「出発地から目的地までの移動ニーズに対して最適な移動手段をシームレスに一つのアプリで提供するなど,移動を単なる手段としてではなく,利用者にとっての一元的なサービスとして捉える概念」である.

日本政府の発行する文書におけるMaaSという単語の初出は,2018年3月7日の第24回「未来投資会議」であり,議事での言及はないものの,国土交通省がすでに検討段階に入っていることがうかがえる.具体的には,①交通関連データ連携②運賃設計③まちづくりとの連携,の3点において,MaaSを活用する意向であることを示している.同年10月に先述の「都市と地方の新たなモビリティサービス懇談会」が設置され,MaaSに関わる規則体系が整備されていくことになる.

このような考えは,いわゆる「地方創生」を念頭に置いていると考えることもできる.実際,日本版MaaSではサービスの提供目的を「地域課題の解決」に置き―最近では新型コロナ対応も追加された―,モデル変革を目的としておらず,結果として公共交通政策全体に入るメスはそれほど大きくない.つまり,サービス提供に係るデータプラットフォーム,サービスにおけるアイデアを競って開発している状態で,公共交通へのガバナンスそのものにはあまり変化が見られない.

日本版MaaSの捉え方}\label{ux65e5ux672cux7248maasux306eux6349ux3048ux65b9}

以上から,日本における政策としての「日本版MaaS」は,源流にあるMaaSとは分けて考えるべきである.つまり,本事業にて解決されるべきとされる課題は,公共交通政策のより効率的,効果的な推進のための仕組みを新たに作るというよりは,むしろ既存の仕組みを活用して地域課題解決のために交通を動員しようとしている.

以上より,日本版MaaSのそれぞれの事業については,既存の地域公共交通政策の一類型として捉えた分析が妥当である.つまり,責任体系の分配システムは変わっておらず,計画を中心とした分析が有効である.

日本におけるMaaSの定義}\label{ux65e5ux672cux306bux304aux3051ux308bmaasux306eux5b9aux7fa9}

なお,日本におけるMaaSの定義については,議論が乱立しているが,これらは主に2つのアプローチに分類される.第一にコンセプトとしてのアプローチ,第2にチケット統合サービスとしてのアプローチであり,これが混合している.

前者の代表的な論者として中村文彦,牧村和彦,加藤博和の3者がある,中村はMaaSについて, 一つのように公共交通サービスを統合して運用する必要性を強調する.

\begin{quote}
最大のポイントは『as a Service』の部分にあると思います.つまり公共交通を『まるで一つのサービスであるかのように』使えなければいけないのです.\footnote{AndE(2023).}
\end{quote}

また牧村は,MaaSについて,アプリケーションへ注目が集まる状況に警鐘を鳴らし,MaaSの本質としてライフスタイルへ注目する\footnote{牧村(2021) p18.}.

\begin{quote}
MaaS は自動車という伝統的な交通手段に加えて,新たな選択肢を提供し,自家用車という魅力的な移動手段と同等かそれ以上に魅力的な移動サービスにより, 持続可能な社会を構築していこうというまったく新しい価値観やライフスタイルを創出していく概念だ.
\end{quote}

さらに加藤は,MaaSについて,新たなコンセプトの訳語としての役割を与えている.つまり,「M:もっと a:あなたらしく a:あんしんして S:せいかつできるために」であり,「少なくとも,S=Smartphoneではない」「移動利便性(mobility)向上と地域活性化によって『乗って楽しい』『降りても楽しい』おでかけの選択が促進され『健幸』に資する」とする\footnote{加藤(2019) p6.}.こうしたコンセプト,牧原)の指摘するイギリス行政学的に言えばドクトリンとしてMaaSを活用している状況にある\footnote{MaaSを日本に導入した書籍として,日高,牧村,井上,井上(2018)への言及は欠かせないが,論者として書きぬけないためここでは除外した.この書籍ではMaaSについてコンセプト的に紹介しており,記述としては以下である;「あらゆる交通手段を統合し,その最適化を図ったうえで,マイカーと同等か,それ以上に快適な移動サービスを提供する新しい概念」.}.

後者の代表的な論者として石田東生がある.石田は2020年の記事の中で,最大のMaaSシステムとして「全日本空輸(ANA)や,日本航空(JAL)の既存の予約ネットワークシステム」を挙げており,MaaSが統一して予約できるチケットサービスとして捉えられている\textsuperscript{5)}.

\section{現在の法体系と「交通まちづくり」}\label{ux73feux5728ux306eux6cd5ux4f53ux7cfbux3068ux4ea4ux901aux307eux3061ux3065ux304fux308a}

次に,地域公共交通に関する法律の概要から見ていく.日本版MaaSは現状,一つの理念法,一つの計画,一つの行政法に基づき遂行されている.

交通政策基本法}\label{ux4ea4ux901aux653fux7b56ux57faux672cux6cd5}

交通政策基本法は,2013年に可決・成立した,日本における交通政策の理念を定めた法律である.当時の民主党政権が掲げた公約の一つでもあり,政権交代後も議論が続き,成立した.本法では,2条から7条までを理念として,国には「国民の理解を得るよう努める」こと,地方自治体には「その地方公共団体の区域の自然的経済的社会的諸条件に応じた施策を策定し,及び実施する責務」,交通事業者は「その業務を適切に行うよう努めるとともに,国又は地方公共団体が実施する交通に関する施策に協力するよう努める」ことをそれぞれ求め,またこの3者ならびに住民その他の関係者に対して,「基本理念の実現に向けて,相互に連携を図りながら協力するよう努める」としている.また「交通政策基本計画」の策定を国に義務付け,環境大臣との協議を義務付けるなど,所謂「横ぐし」の法体系となっている.

公共交通に関わる点の理念について確認すると,「交通の機能を管理する」という点に重点が置かれている.まず第2条にて「(交通の)機能が十分に発揮されることにより,国民その他の者の交通に対する基本的な需要が適切に充足されることが重要であるという基本的認識の下に行われなければならない」として,交通政策の意義が定義されている.そして第3条にて,目的志向型の交通政策目標設定義務,つまり「交通が,豊かな国民生活の実現に寄与するとともに,我が国の産業,観光等の国際競争力の強化並びに地域経済の活性化,地域社会の維持及び発展その他地域の活力の向上に寄与するものとなるよう,その機能の確保及び向上が図られることを旨として行われなければならない」が記載されている.さらに,その機能を各交通手段が「特性に応じて適切に役割を分担し,かつ,有機的かつ効率的に連携する」ことを規定し(第5条),また「関係者が連携し,及び協働しつつ,行われなければならない」とする(第6条).

交通政策基本計画}\label{ux4ea4ux901aux653fux7b56ux57faux672cux8a08ux753b}

交通政策基本計画は,前述した交通政策基本法に基づいて策定される計画である.第1次の計画は2015年に策定され,2021年に更新されている.

計画の位置づけは2015年発行の第1次計画に記載されており,「交通政策基本法が提示するこれらの交通政策の長期的な方向性を踏まえつつ,政府が今後講ずべき交通に関する施策について定めるものである」とされている.また構成としては,「交通に関する施策の「基本的方針」,計画期間内に目指すべき「目標」,目標の各々について取り組むべき「施策」の三層構造となっており,関係者の責務・役割や連携・協働等についても,施策の推進に当たって「留意すべき事項」として整理している」(第1次計画「はじめに」より).また新型コロナウイルス感染症拡大に伴う利用客数現象の段になって,第2次計画にはこうした時代背景を踏まえる形で,「交通が直面する「危機」と,それを乗り越える決意」という章が追加され,交通事業に対する危機感がつづられている.

横ぐしの法律を背景としているが,地域公共交通に関連して目標となっている数値はほとんどの場合,事業がどの程度導入されているか,が指標となっている.

表-2 交通政策基本計画における目標値設定

{\def\LTcaptype{none} % do not increment counter
\begin{longtable}[]{@{}
  >{\raggedright\arraybackslash}p{(\linewidth - 4\tabcolsep) * \real{0.1579}}
  >{\raggedright\arraybackslash}p{(\linewidth - 4\tabcolsep) * \real{0.1473}}
  >{\raggedright\arraybackslash}p{(\linewidth - 4\tabcolsep) * \real{0.1790}}@{}}
\toprule\noalign{}
\endhead
\bottomrule\noalign{}
\endlastfoot
& 第1次計画 & 第2次計画 \\
施策項目記号 & 基本目標A & 基本目標A \\
地域公共交通に関する

施策項目名 & ・自治体中心に,コンパクトシティ化等まちづくり施策と連携し,地域交通ネットワークを再構築する

・地域の実情を踏まえた多様な交通サービスの展開を後押しする & 地域が自らデザインする,持続可能で,多様かつ質の高いモビリティの実現 \\
目標数値

(交通事業/計画) & 地域公共交通網形成計画の策定総数

鉄道事業再構築実施計画(鉄道の上下分離等)の認定件数

デマンド交通の導入数

LRT の導入割合

コミュニティサイクルの導入数 & 地域公共交通計画の策定件数

地域公共交通特定事業の実施計画の認定総数

鉄道事業再構築実施計画(鉄道の上下分離等)の認定件数

新たなモビリティサービスに係る取組が行われている地方公共団体の数 \\
目標数値

(非交通事業/計画) & 航路,航空路が確保されている有人離島の割合 & 航路,航空路が確保されている有人離島の割合 \\
\end{longtable}
}

地域公共交通活性化・再生に関する法律}\label{ux5730ux57dfux516cux5171ux4ea4ux901aux6d3bux6027ux5316ux518dux751fux306bux95a2ux3059ux308bux6cd5ux5f8b}

地域公共交通活性化・再生に関する法律は,2007年より施行された公共交通における「総合計画法」である.この法律は,元々は「施策支援法」であった.「頑張る地域を応援する」,「LRT」をはじめとした道路運送法だけでは対応できなかったバス事業以外の交通事業について包括的な連携を図るとともに,交通システムごとの支援メニューを取りそろえたシステムであったのである.さらに,「地域公共交通総合事業」が設定され,関連する実証実験やその調査に関しての補助も設定された.

しかし,2010年近傍のいわゆる「事業仕分け」を経て,地域の生活交通を確保する「総合計画法」に転化し,取り組み支援型の事業支援システムを残したまま強制力を持ち始めた.この法律に基づいて策定される「地域公共交通計画」「地域公共交通利便増進計画」を所管する地方運輸局には,路線届出の最終的な受理権限もある.結果としてこのシステムには,地域ごとに,その地域が考える「あるべき交通システム」を設計するためのこうした計画に,「どのような交通手段を導入すべきか」「どのように交通を維持確保していくべきか」という点で,メニューの採用に圧力がかかりやすい特性がある\footnote{永田(2022)第2章.}.

交通まちづくりの定義とその親和性}\label{ux4ea4ux901aux307eux3061ux3065ux304fux308aux306eux5b9aux7fa9ux3068ux305dux306eux89aaux548cux6027}

主に交通政策基本法を基底とした理念体系を確認すると,この内容が驚くほど「交通まちづくり」のそれと類似していることが分かる.

交通まちづくりは前述のように,都市の課題解決のために,公共交通を中心とした交通を動員する交通政策の概念である.既存の交通需要を満たすための交通政策とは異なり,地域の課題解決に貢献する交通を,目的の設定から市民参加を加える形で設計,提供する計画手法で,2005年ごろから主に札幌,横浜,岡山,熊本など政令市における市民団体より勃興してきた.

土木学会にて行われた交通まちづくりの議論では,議論が進むにつれて市民参加の重要性について序列を下げる傾向はみられた\footnote{もちろん,こうした議論が政令市から立ち上がった点に関して,財源の自由度や市民全体の理解レベルの高さが寄与した点は容易に想像がつく.このような高い教育レベルと自由度を交通まちづくり初期の前提と考えるならば,日本全国へ交通まちづくりを広げるにあたって,多様な市民の参加を諦めた可能性は大いにある.}ものの,方向性は変わらない.すなわち,公共交通を手段とする方向性は,交通まちづくりの根幹であると言える.政策の直接の接続は確認できていないものの,思想の文脈はここに接続している.

\section{5.市民参加の捉え方}\label{ux5e02ux6c11ux53c2ux52a0ux306eux6349ux3048ux65b9}

本稿は市民参加プロセスについて,手続主義,非帰結主義のスタンスを取るものの,公共交通それ自体に対しては帰結主義のアプローチを取る.

公共交通政策における市民参加に関する議論}\label{ux516cux5171ux4ea4ux901aux653fux7b56ux306bux304aux3051ux308bux5e02ux6c11ux53c2ux52a0ux306bux95a2ux3059ux308bux8b70ux8ad6}

公共交通政策における市民参加に関する規範的な議論は,帰結主義に立って市民参加が情報提供と労力提供の役割を持つと説く太田\textsuperscript{8)},非帰結主義に立って6つの要件\footnote{1:民意の反映 2:民意による利害調整 3:専門知の活用と専門家の専横の排除 4:政治プロセスの失敗回避のための国からの各種規制 5:効率的な交通政策が導出可能 6:地方自治体から独立.4に関しては,今日の地方自治法と当時のそれは違うから注意が必要である.また6に関しては,都市圏域や生活圏行きが自治体とは違う点から指摘されているのであるので,これらが同一と認められるのであれば独立していなくても問題ないと思量される.}を示す太田\textsuperscript{9)},実証的に7つの要件を提示し,協働を通じて「新たな公共,地球の公益」を創造すべきとした森栗\textsuperscript{10)}などがある.

近年の交通政策における市民参加に関する実証的な議論では,その市民参加形態は,主に不足する財源や人的資源を確保するため,市民が交通運営者として参画する物に集中している\footnote{代表的なものだと,山下(2009),福本・加藤(2012),加藤(2012)などが挙げられる.}.ただ,こうしたスタンスではあくまで,どのように交通事業を成立させるか,また「目の前の交通に対する課題をどのように解決するか」という観点に基づいている.市民参加や住民参加は,目の前にある課題を炙り出し埋め合わせるのに有用であると評価され,計画立案プロセスに組み込まれている.

ただこうしたスタンスは,こと「横ぐし」の領域を扱う地域政策においては,課題の相対化を妨げ,重要度の高い解決すべき課題を見落とす懸念がある.たとえば高齢者の健康増進に公共交通を貢献させるという議論をしても,実のところ効果は他の政策の方が高い,ということは検討されにくい.さらに交通事業者などの公益的な企業は,地域政策の構成員として「選ばれる」というプロセスを経ないことが多いため\footnote{市場原理の導入も,実際のところ選ばれた状態を確立できているかは不明である.この認識は規制緩和後でも,市場の入退出が活発でない現状に起因する.寺田(2004)によれば,乗合バス市場における新規参入は活発ではなく,規制緩和から1年後までの新規参入者の市場シェアは 1\%を下回る.規制緩和による退出ルールの明確化が休廃止を増加させた傾向はなく,乗合バスの市場構造は規制緩和後にもあまり変化していない」という.},概して一方的になりやすい可能性がある\footnote{公益法人のガバナンスの更なる強化等に関する有識者会議(2021).}.こうした状況は,何らかのシステムの導入を通じて是正されるべきと考えられる.

公共サービスとしての公共交通}\label{ux516cux5171ux30b5ux30fcux30d3ux30b9ux3068ux3057ux3066ux306eux516cux5171ux4ea4ux901a}

一方で,地域公共交通は自治体の計画によって定義され,非競合性を持たないため,インフラとしてではなく公共サービスとして定義されるべきである.通信サービスや道路などの非競合性をある程度持つ財とは違い,利用できる時間や容量が,物理的に決定されているからである.つまり,一旦整備したところで供給量は簡単に変化させられない即時材であるとともに,また提供に必要な労力や資源には限りがある.市民参加を通じてそうした資源を導入できるとしても限界があるから,得られる効果とコストとのバランスについて検討すべきである.

望ましい帰結/手続きのスタンスの提案}\label{ux671bux307eux3057ux3044ux5e30ux7d50ux624bux7d9aux304dux306eux30b9ux30bfux30f3ux30b9ux306eux63d0ux6848}

以上を踏まえ,市民参加の類型としては,地域の目標のうち公共交通が貢献すべき領域を設定する,政策目標設定から関わる形態が望ましい.一方で交通サービスに関しては,他行政サービスと同列に比較され,その地域への貢献によって選択されるべきであろう.

現状の市民参加は,あくまで目の前にある課題の解決を旨としており,実証主義に基づいた太田の主張通りに情報提供と労力提供の役割を持っている.しかしそれだけでは,全ての計画段階における批判はなし得ないものであり,少なくとも交通の「要件定義」段階である目標立案レベルへの市民参加は,公共サービスの一方性を排するためには重要な要素である.

具体的にはまず,初めに地域において,交通事業以外の数値によって語られる公共交通がもたらすべき影響,つまり「効果」の定義を行う.結果として,交通をあくまで手段として,その「効果」への貢献を推定して議論できる.ここには基本的に,政策手段としての事業を推進する立場となる点から,交通事業者を意思決定主体として組み込んではならない.

こうした目標への市民参加は,それ自体がハンス・ケルゼン\textsuperscript{11)}の指摘する民主主義の本質と深く結びついている.政策の目標,手段,あらゆる点に対して市民に批判するための回路を残すことで,武力によるクーデターがなくとも政権の構造転換をなしえる,というのがその主張である.

\section{日本における公共政策上のMaaSの分析}\label{ux65e5ux672cux306bux304aux3051ux308bux516cux5171ux653fux7b56ux4e0aux306emaasux306eux5206ux6790}

\section{2020年の日本版MaaS推進36事業の総攬}\label{ux5e74ux306eux65e5ux672cux7248maasux63a8ux903236ux4e8bux696dux306eux7dcfux652c}

以上を踏まえた上で,日本においてMaaSを推進する「日本版MaaS」について確認したい.

リサーチクエスチョンの設定}\label{ux30eaux30b5ux30fcux30c1ux30afux30a8ux30b9ux30c1ux30e7ux30f3ux306eux8a2dux5b9a}

以下の2点をリサーチクエスチョンとして置く.

①その政策進捗において目標設定がどこに置かれているか

②立場を問わない市民による批判回路がどのように整備されているのか

本研究は,日本版MaaSを「MaaS」としてではなく,「交通まちづくり」の一類型として分析し,その目的意識と計画における責任体系の構築状況,すなわちガバナンス状況を探るものである.

分析項目・手法}\label{ux5206ux6790ux9805ux76eeux624bux6cd5}

分析手法について,「計画における計画指標(Key Performance Indicator,以下KPI)に事業利用状況以外の指標がある」「会議体において,交通事業者,システム供給者,学識者,自治体以外の市民参加がある」の二つの点を検討し,交通政策を地域政策として運営できているかの評価とする.

①については,MaaSが「事業の手段」としてではなく,「政策の手段」として機能しているかを確認するものである.これを分析する手法としては議事録を分析する,議会を分析するなどの手法が考えられるが,そもそもすべての事例でMaaSについて議会で論点となっているわけではない点,また議事録を公開している協議会自体が少ない点から,こうした議事録ベースのアプローチを行うのは困難である.

この点から,評価指標の状況について観察対象とすることとした.MaaS事業自体の事業性や交通事業の利用状況だけではなく,それ以外の評価指標を計画に置いているかを確認する.

②については,地域における交通事業者など公益的な企業の特性を鑑み,市民参加が行われており目標や手段への批判プロセスがあるかを確認する.市民の行政への批判回路について,議会を通じた参加等はあるが,交通計画では協議会参加,グループヒアリング,アンケート調査,パブリックコメントが推奨手法として取り上げられている\footnote{国土交通省(2020).}.これらのうち,合議への参加を通じて直接の批判とできるのは協議会参加のみである.

以上の検討より,会議体への市民の参加を指標としてとらえることとした.座談会,協議会を代表とする場における,交通事業者,システム供給者,学識者,自治体以外の市民参加の存在を,国土交通省が発行する先述のドキュメントと検索エンジン「Google」「Bing」「Yahoo!Japan」を利用した文書探索を通じて確認した.

分析対象}\label{ux5206ux6790ux5bfeux8c61}

日本版MaaSは,その事業概要を一定のフォーマットに落とし込んで申請する仕組みとなっており,国土交通省は採択のたびにそれらをまとめた文書を公開している\footnote{国土交通省(2020b).}.これらには協議会の構成員や目標数値が記載されており,これを分析対象とする.2020年の日本版MaaS推進事業では,全国36の事業が認定されており,提出された計画については全件採択となっている.

ここで,本事業によって実施された事業の特性を説明するために,日本版MaaSとして展開された各事業について,要素技術の利用状況について確認し,類型化を試みる.日本版MaaSの主な要素技術として,サブスク・定期など企画乗車券,オンラインで交通情報が見られる,あるいは決済できるアプリケーション,病院,商業クーポンなど外部情報の提供,デマンド交通,シェアサイクル等の新モビリティ,さらにデータ連携基盤の構築や利用が認められた.特にアプリケーションの提供が31事例,外部情報との連携が32事例と多かったものの,企画乗車券の導入と新モビリティの導入比率は低く,またデータ連携基盤の利用や構築は12事例と少なかった.なお,参入の高度化に必要,なAPIをはじめとしたデータ連携のオープン化に関して言及しているのは2事例のみであった(加賀,しんゆり).

分析結果}\label{ux5206ux6790ux7d50ux679c}

分析の結果,双方の判断を満たし,地域政策として日本版MaaSを運用している事例は存在しなかった.

①について,交通事業やMaaS事業の指標を導入している事業は32/36であったが,それ以外の地域目標を設定している事業は10のみであった.さらに②については,自治体が参加している事業に限定すると,パブリックコメントや協議会への市民参加を確認できた事業は2/32にとどまった.

\begin{figure}[htbp]
\centering
\includegraphics[width=0.9\textwidth]{figure/ch3/image2.png}
\caption{日本版MaaSにおける要素技術の状況}
\label{fig:ch3_table3}
\end{figure}

\subsection{事例}\label{ux4e8bux4f8b}

分析の射程を示すため,今回の分析の結果を2つほど挙げる.

\subsection{1.地方版MaaSの広域連携基盤構築モデル事業(ひたち圏域)茨城交通株式会社}\label{ux5730ux65b9ux7248maasux306eux5e83ux57dfux9023ux643aux57faux76e4ux69cbux7bc9ux30e2ux30c7ux30ebux4e8bux696dux3072ux305fux3061ux570fux57dfux8328ux57ceux4ea4ux901aux682aux5f0fux4f1aux793e}

日立市を中心とする茨城県北部の圏域におけるMaaS事業展開について,3市1村の協力の下で行われた日本版MaaS事業である.

\subsection{Ⅰ.指標}\label{ux6307ux6a19}

①について,本事業の目標は3つに分かれている.一つ目は「利用者関連指標」であり,内容は以下の通り利用者数が参照されている.

・取組ページへのアクセス数:20,000imp以上

・地域内の取組認知度:50\%以上

・アプリDL数(合計):2,500以上

・チケット販売数:10,000枚以上

・周遊券の販売枚数:100枚以上

・セット券・企画乗車券の販売数:100枚以上

・通勤型デマンドの利用者数:300人以上

・ラストワンマイルデマンド利用者数:100人以上

2つ目は「交通事業者関連指標」であり,指標は参加する事業者数により達成されるものになっている.いずれもMaaSによってカバーされる人口はじめとしたMaaSの広がりを目標としている.

・基盤に参加する事業者数:3社以上

・利用者数ベースのカバー数:70\%以上

・新しく造成した商品数:5つ以上

・MaaS基盤へのデータ提供社数:4社以上

3つ目は「MaaS事業者関連指標」であり,以下のように接続事業者を目標と据えていて,2つ目と同様の性格を帯びる.

・システム基盤へ対応事業者数:3社以上

・チケット発券する事業者数:3社以上

\subsection{Ⅱ.市民参加状況}\label{ux5e02ux6c11ux53c2ux52a0ux72b6ux6cc1}

②について,参加する主体について確認したところ,市民団体らしき仕組みは確認できず,市民代表者の参加形跡も確認できなかった.また,市民参加,パブリックコメントなどの情報をGoogle検索上で探したが,見当たらなかった点から市民参加システムなしとした.

\subsection{2.鞆の浦MaaS}\label{ux9786ux306eux6d66maas}

広島県福山市の景勝地,鞆の浦におけるMaaS事業展開について,同市の協力の下で行われた日本版MaaS事業である.

\subsection{Ⅰ.指標}\label{ux6307ux6a19-1}

①について,本事業の目標は,そのデータ取得方法に合わせて3つに分かれている.

一つ目は「デジタルチケット発行数」であり,実証実験チケットの利用枚数200枚が目標になっている.

二つ目は「アンケート指標」であり,以下の通りアプリケーションの満足度や性能へのフィードバック,回遊個所の向上等をアンケートで確認している.

\hl{・}満足度

目標 満足度70\%以上

・setowaに対する評価検証

目標:今後の課題点抽出10件以上

・実証実験を通じた来訪誘発・回遊性向上効果

目標:setowa利用により立寄り場所の増えた者の割合10%

三つ目は電動レンタサイクル利用者の回遊エリア拡大状況であり,電動レンタサイクルのデータをもとに平均利用距離15km以上を目指している.

以上より本事業では,関連する交通事業の指標というよりはむしろ,MaaS事業に特化した指標が並んでいると判断した.

\subsection{Ⅱ.市民参加状況}\label{ux5e02ux6c11ux53c2ux52a0ux72b6ux6cc1-1}

②について,参加する主体について確認したところ,市民団体らしき仕組みは確認できなかったが,観光関連団体の代表として「公益社団法人福山観光コンベンション協会」が出席しており,これは交通事業者外の参加と認められることから,市民参加システムありとした.

考察}\label{ux8003ux5bdf}

分析対象とした日本版MaaS推進事業へ認定されたそれぞれは,その目標を考えると,ほとんどの場合交通事業の成立,また日本版MaaS事業の成立を旨としている.実際のところ,本事業を担当する国土交通省新モビリティサービス推進課の担当者は,「申請前の1年で事業検討して,申請後の1年で事業として自走できるようにしてほしい」と述べており\footnote{2022年5月のヒアリングによる.},事業としての成立を企図している.一方で,こうしたスタンスは実際のところ,海外ではすでに行き詰まり感を見せている\footnote{海外でのMaaSの行き詰まり感に関しては,2024年3月に破産,事業承継を実施したほか,技術とデータフォーマットが障壁として挙げられているA. Gerber and K. Hinkelmann (eds.): Analysing Barriers in the Business Ecosystem of European MaaS Providers: An Actor-Network Approach, Society 5.0 2023 (EPiC Series in Computing, vol. 93), pp. 68--81や,輸送事業者が細分化され,データ共有の枠組みが不十分な地域でオープンデータ不足を障壁とするMarc, Hasselwander., João, F., Bigotte.: Transport Authorities and Innovation: Understanding Barriers for MaaS Implementation in the Global South., Transportation research procedia, 2022 , また特に通常の交通アプリとの差異を見分けられない状況を原因と示したAndrea, L., Hauslbauer., B., Verse., E., Guenther., T., Petzoldt.: Access over ownership: Barriers and psychological motives for adopting mobility as a service (MaaS) from the perspective of users and non-users. Transportation research interdisciplinary perspectives, 2023.などが挙げられる.}.日本におけるMaaSは,こうした国際的な流れを反映していない.

また交通政策の理念として,交通の機能について「地域の活力に貢献」するように施策を運営するためには,少なくとも地域がどのようになるべきかの指標を入れるべきである.交通事業,MaaS事業の指標のみでは,こうした運営には不十分である.それは交通事業の,利益を産む/費用を抑える事業としての価値こそ管理できるが,地域交通事業はこうした効率化が難しい現状を鑑みると,地域における機能については管理できないからである.

\begin{figure}[htbp]
\centering
\includegraphics[width=0.9\textwidth]{figure/ch3/image3.png}
\caption{日本版MaaSにおける目標値決定状況}
\label{fig:ch3_table4}
\end{figure}

さらに,こうした行政機関が手がける交通サービスに対する批判回路についても,十分に構築されていない状況が見えてきた.もちろん「現状のサービスはユーザーヒアリングを経ている」という反論もあるだろうし,「議会制民主主義が批判回路となりうる」という意見もあるだろう.しかし,公共交通政策は多くの場合,土木工学を主軸とした専門性を問われることから\footnote{土木計画のうち,交通計画の費用便益分析においては,どの程度の需要が発生するかが重要なファクタであり,その算出には一般的に「四段階推定法」と呼ばれる手法を用いるのが主流である.屋井(2021)参照.},こうした回路が機能しにくいと推察される.以上より,専門家が知識面,手法面でサポートする形で批判回路を設定すべき,と思量される点を前提とすると,現状の批判システムは不十分である可能性が示唆される.

日本版MaaSは交通まちづくりの一類型としてとらえられるか?}\label{ux65e5ux672cux7248maasux306fux4ea4ux901aux307eux3061ux3065ux304fux308aux306eux4e00ux985eux578bux3068ux3057ux3066ux3068ux3089ux3048ux3089ux308cux308bux304b}

ここまでの議論を踏まえて,日本版MaaSを交通まちづくりとして捉えられるかについて議論したい. 交通まちづくりの基本線は,初期構想段階では市民参加を含んでいた以外は一貫しており,地域の課題を解決する方針のもと交通を計画し,提供するものであった.

これを踏まえると,日本版MaaSはその構想段階では交通まちづくりの要件に会うものの,実装段階においてはその要件に合致しないと理解できる.前者については,日本版MaaSの特徴として地域課題を解決する交通サービスの提供を支援する政策である点が該当する.一方で後者については,日本版MaaSの計画において交通サービスの利用状況が評価の中心にあり,必ずしも地域課題解決の状況をモニタリングしていない点が該当する.

このような状況になった要因として,いくつかの仮説を挙げておきたい.第1に,新たなガバナンスシステムの導入へのコストが高いことがあげられる.一般に,デジタル時代の政府活動への意向にはそれに即した評価,批判のシステムへ移行する必要があるが,それには取引費用が掛かりすぎることが分かっている\footnote{Duleavy et al (2006)}.第2に,これは第1の指摘と関連するが,杉谷\textsuperscript{15)}の述べるようにそもそも自治体の持つ政策評価システムが業績評価に偏っている点があげられる.三重県の「さわやか運動」から始まった政策評価では,担当者が直接事業内容を説明する「事務事業評価」が核であり,その中で簡略化された評価として業績評価が利用されることが多い.これは一種の日本の評価システムの文化であり,日本版MaaSもその流れを汲んでいる可能性は十分にある.

一方で,市民参加の文脈については,そもそも日本版MaaSの概念において組み込まれていない状況で,情報公開や批判回路についても明確ではない.公共政策学の蓄積を踏まえると,地域交通政策においてこの点は後退基調にあると考えられ,早急な改善が求められる点といえる.

本分析の限界}\label{ux672cux5206ux6790ux306eux9650ux754c}

本稿の限界の一つは,そもそも調査対象が日本版MaaS事業のみである点である.地域公共交通に関する事業はMaaS事業だけではなく,LRT事業や再編事業など多様である.また批判プロセスの調査についても,Googleの機能に制約を受けるものであって,これの限界を越えるには個別インタビューなどの質的調査を施し,分析を行う必要がある.さらに,1998年の行政改革会議を経て実施された地域公共交通の規制緩和について,その総決算を行えていない点もまた課題である.

\section{7.終わりに}\label{ux7d42ux308fux308aux306b}

本稿では,MaaSを推進する政策において,交通に関わる理念法である「交通政策基本法」を参照した上で,「交通まちづくり」の概念を引用し,その思想を日本版MaaSの政策分析に援用できるかを確認した.

まず,日本版MaaSとフィンランドにおいて発案された時点のそれとを比較し,前者が地域課題解決のための仕組みとして変容したことを導き出した.この点は「交通まちづくり」の思想と一致している.次に,日本版MaaS推進事業において採択されたすべての事業内容を確認,分析した.その結果,これらの事業は「交通まちづくり」の思想と完全に合致するものではないと明らかになった.

本稿の結論は,日本版MaaSに関する文献で上げた魏\textsuperscript{3)}の結論を支持するものであった.つまり,地域の課題解決というよりはむしろ,日本版MaaSでとらえられた課題意識は地域交通事業の課題解決であった.ただ,本稿ではそういった決定やプログラム運営に至った経緯については立ち入らなかった.

地域公共交通政策として進行中の各政策を概観すると,こうした指摘は未だ色褪せていないと考える.

有識者会議の下で2022年8月27日にまとまった「アフターコロナにおける地域交通のリ・デザイン」において,政府は「3つの共創」を利用して,地域交通の確保に取り組むとしている.加えて,2023年10月1日に改正された「地域公共交通の活性化及び改正に関する法律」に基づく政策展開においては,特に公的資金投入の拡大と「鉄道再構築協議会」によるさらなる事業者自治体間協議の活性化が取りざたされている.ただ,責任分担をどのように建てつけるのかの議論は見られず\textsuperscript{注}\footnote{例えば今回の法改正で,JR各社の運営する鉄道に関する「再構築協議会」こそ設置されこれを「十分な説明」の舞台とすることとなったが,JRと自治体間の関係性については基本的には変わらない.本制度の成立は基本的に性善説によっている.これが依拠できないものとなった場合,国の関与が行われるよう変化する本制度において,JRによる同意を取れない状況での路線廃止は路線がいかなる状況でも行われる上に国地方係争処理委員会の対象外になることが想定される.結果として,国は種々の路線に関する責任を持たず,JRの撤退行動を許容することになる.

   また今回の法改正にて,ある地域の交通を統合的に受託するエリア一括運行事業が制定された.本事業は既存のバス交通に関する補助を合計して自治体へ渡し,それを利用して一括運行事業者を選ぶ仕組みである.これの機能は自治体の能力水準によっている.自治体は補助額削減インセンティブを持ち,事業者に費用削減を要求するが,適切な競争や契約形態を選択しなければそれは実現されない.既存のバス交通関連補助を利用している点で,地域公共交通会議をはじめとした協議会経由の補助システムと同様となり,責任体系は既存のシステムと変わりない.}\textsuperscript{)},また公共交通の意義や事業への批判システム,事業者の一方性を排するシステムの構築は検討されていない.公共交通の存在意義を見つめなおし,従来の政策を批評し更新し続ける人的・物的・経済的資源の動員体制を取ってこそ健全な地域交通が達成されると考える初期の「交通まちづくり」の思想文脈からすれば,この状況は好ましくない.

「利用者視点」を導入しているとうたう交通政策が各地で展開されているが,少なくとも政策において相手にするのは市民であり,この国の主権者である.もちろん主権者の意見が全て正しいわけでもないが,主権の所在に目を背け,専門性で固め,「利用者のため」と謳って実施される交通政策はまさに,善意で舗装された地獄への道となり得るのである.専門性が必ずしも正しさではなく,目指す方向性はイデオロギーであり批判されうること,これらを前提として,民主主義国家の計画のあり方を交通の面から探る作業は,今後の課題としたい.加えて,パブリック・インボルブメントと呼ばれる,サービスではなくインフラ整備にかかる市民参加プロセスについても,今後の研究における検討課題としたい.

\section{\texorpdfstring{\textbf{Notes}}{Notes}}\label{notes}

\begin{enumerate}
\def\labelenumi{\arabic{enumi}.}
\item
  石神孝裕:MaaS先進国フィンランドの視察, IBS Annual Report, 2020 より.
\item
  企業の水平統合への議論にも繋がる.機能の分化,統合を行わずシステム全体が瓦解あるいは不安定になった事例として,日本航空(日本航空,日本エアシステム)やみずほ銀行(第一勧業銀行,富士銀行,日本興業銀行)があげられる.
\item
  David Zipper: The Problem With `Mobility as a Service', Bloomberg.
\end{enumerate}

\url{https://www.bloomberg.com/news/articles/2020-08-05/the-struggle-to-make-mobility-as-a-service-make-money},2020,2022/10/15閲覧

\begin{enumerate}
\def\labelenumi{\arabic{enumi}.}
\setcounter{enumi}{3}
\item
  事業目的によりガバナンスが行われており,その目的となる5つは以下のとおりである:①他の移動と重ね掛けして効率化②モビリティでのサービス提供③需要側の変容を促す仕掛け④異業種との連携による収益活用,付加価値の創出⑤モビリティ関連データ取得,交通・都市政策との連携.なお,このプロジェクトは2020年に52事業へ拡大することとなる.
\item
  国土交通省:日本版MaaSの推進,
\end{enumerate}

\begin{quote}
https://www.mlit.go.jp/sogoseisaku/japanmaas/promotion/index.html , 2022/10/18閲覧
\end{quote}

\begin{enumerate}
\def\labelenumi{\arabic{enumi}.}
\setcounter{enumi}{5}
\item
  国土交通省:MaaSのモデル形成,
\end{enumerate}

https://www.mlit.go.jp/sogoseisaku/japanmaas/promotion/model/index.html,2022/10/18閲覧

\begin{enumerate}
\def\labelenumi{\arabic{enumi}.}
\setcounter{enumi}{6}
\item
  AndE : 新しい移動の概念「MaaS」の現在MaaS社会実現のカギは「つなげる」こと 競争と協調を見極める, 2023 より.
\end{enumerate}

\begin{quote}
https://www.andemagazine.jp/2023/05/25/maas-interview.html 2023/10/06閲覧.
\end{quote}

\begin{enumerate}
\def\labelenumi{\arabic{enumi}.}
\setcounter{enumi}{7}
\item
  牧村和彦:MaaSが都市を変える 移動×都市DXの最前線,学芸出版社,2021
\item
  加藤博和; 地域公共交通改革のためいまこそ殻を破ろう!~地域の,そして日本の将来を救うために~, 令和元年度 第2回 (第16回)国土交通省交通政策審議会交通体系分科会 地域公共交通部会, 2019 p6.
\item
  MaaSを日本に導入した書籍として,日高洋佑,牧村和彦,井上佳三,井上岳一:MaaS モビリティ革命の先にある全産業のゲームチェンジ,日経BP社,2018 p18. への言及は欠かせないが. 論者として書きぬけないためここでは除外した.この書籍ではMaaSについてコンセプト的に紹介しており,記述としては以下である:「あらゆる交通手段を統合し,その最適化を図ったうえで,マイカーと同等か,それ以上に快適な移動サービスを提供する新しい概念」
\item
  永田右京:公共交通の構造転換,慶應義塾大学総合政策学部卒業論文, 2022. 第2章.
\item
  もちろん,こうした議論が政令市から立ち上がった点に関して,財源の自由度や市民全体の理解レベルの高さが寄与した点は容易に想像がつく.このような高い教育レベルと自由度を交通まちづくり初期の前提と考えるならば,日本全国へ交通まちづくりを広げるにあたって,多様な市民の参加を諦めた可能性は大いにある.
\item
  1:民意の反映 2:民意による利害調整 3:専門知の活用と専門家の専横の排除 4:政治プロセスの失敗回避のための国からの各種規制 5:効率的な交通政策が導出可能 6:地方自治体から独立.
\end{enumerate}

4に関しては,今日の地方自治法と当時のそれは違うから注意が必要である.また6に関しては,都市圏域や生活圏行きが自治体とは違う点から指摘されているのであるので,これらが同一と認められるのであれば独立していなくても問題ないと思量される.

\begin{enumerate}
\def\labelenumi{\arabic{enumi}.}
\setcounter{enumi}{13}
\item
  代表的なものだと, 山下祐介:地域公共交通をめぐる社会実験と住民参加,運輸と経済第69巻12号,2009., 福本雅之,加藤博和: 地域公共交通への住民参画の促進方策に関する検討,第45回土木計画学研究発表会(春大会),2012.などが挙げられる.
\item
  市場原理の導入も,実際のところ選ばれた状態を確立できているかは不明である.この認識は規制緩和後でも,市場の入退出が活発でない現状に起因する.寺田一薫:バス産業の規制緩和,日本評論社,2004.によれば,乗合バス市場における新規参入は活発ではなく,規制緩和から1年後までの新規参入者の市場シェアは 1\%を下回る.規制緩和による退出ルールの明確化が休廃止を増加させた傾向はなく,乗合バスの市場構造は規制緩和後にもあまり変化していない」という.
\item
  公益法人のガバナンスの更なる強化等に関する有識者会議: 公益法人のガバナンスの更なる強化等のために(中間とりまとめ), 2021
\item
  国土交通省:日本版MaaSの推進,
\end{enumerate}

\begin{quote}
\href{https://www.mlit.go.jp/sogoseisaku/japanmaas/promotion/index.html,\%20}{https://www.mlit.go.jp/sogoseisaku/japanmaas/promotion/index.html,} 2022/10/18閲覧
\end{quote}

\begin{enumerate}
\def\labelenumi{\arabic{enumi}.}
\setcounter{enumi}{17}
\item
  国土交通省:MaaSのモデル形成,
\end{enumerate}

https://www.mlit.go.jp/sogoseisaku/japanmaas/promotion/model/index.html,2022/10/18閲覧

\begin{enumerate}
\def\labelenumi{\arabic{enumi}.}
\setcounter{enumi}{18}
\item
  2022年5月のヒアリングによる.
\item
  海外でのMaaSの行き詰まり感に関しては,2024年3月に破産,事業承継を実施したほか,技術とデータフォーマットが障壁として挙げられているA. Gerber and K. Hinkelmann (eds.): Analysing Barriers in the Business Ecosystem of European MaaS Providers: An Actor-Network Approach, Society 5.0 2023 (EPiC Series in Computing, vol. 93), pp. 68--81や,輸送事業者が細分化され,データ共有の枠組みが不十分な地域でオープンデータ不足を障壁とするMarc, Hasselwander., João, F., Bigotte.: Transport Authorities and Innovation: Understanding Barriers for MaaS Implementation in the Global South., Transportation research procedia, 2022 , また特に通常の交通アプリとの差異を見分けられない状況を原因と示したAndrea, L., Hauslbauer., B., Verse., E., Guenther., T., Petzoldt.: Access over ownership: Barriers and psychological motives for adopting mobility as a service (MaaS) from the perspective of users and non-users. Transportation research interdisciplinary perspectives, 2023.などが挙げられる.
\item
  土木計画のうち,交通計画の費用便益分析においては,どの程度の需要が発生するかが重要なファクタであり,その算出には一般的に「四段階推定法」と呼ばれる手法を用いるのが主流である. 屋井鉄雄: 土木と環境の計画理論-3つの並行プロセスによる計画づくり-,株式会社数理工学社, 2021. 参照.
\item
  Dunleavy, Patrick, et al.: New public management is dead---long live digital-era governance. Journal of Public Administration Research and Theory 16(3), pp.467-494, 2006.
\item
  例えば2023年の法改正にて,JR各社の運営する鉄道に関する「再構築協議会」こそ設置されこれを「十分な説明」の舞台とすることとなった(「十分な説明」については国土交通省:新会社がその事業を営むに際し当分の間配慮すべき事項に関する指針, 国土交通省告示第千六百二十二号,2001.参照)が,JRと自治体間の関係性については基本的には変わらない.本制度の成立は基本的に性善説によっている.これが依拠できないものとなった場合,国の関与が行われるよう変化する本制度において,JRによる同意を取れない状況での路線廃止は路線がいかなる状況でも行われる上に国地方係争処理委員会の対象外になることが想定される.結果として,国は種々の路線に関する責任を持たず,JRの撤退行動を許容することになる.
\end{enumerate}

また今回の法改正にて,ある地域の交通を統合的に受託するエリア一括運行事業が制定された.本事業は既存のバス交通に関する補助を合計して自治体へ渡し,それを利用して一括運行事業者を選ぶ仕組みである.これの機能は自治体の能力水準によっている.自治体は補助額削減インセンティブを持ち,事業者に費用削減を要求するが,適切な競争や契約形態を選択しなければそれは実現されない.既存のバス交通関連補助を利用している点で,地域公共交通会議をはじめとした協議会経由の補助システムと同様となり,責任体系は既存のシステムと変わりない.

\section{}\label{section}

\section{\texorpdfstring{\textbf{References}}{References}}\label{references}

\begin{enumerate}
\def\labelenumi{\arabic{enumi})}
\item
  太田勝敏編著:『交通まちづくり』の展開と課題,方向性, IATSSReview(国際交通安全学会誌) 33(2), 2008
\item
  原田昇,羽藤英二,高見淳史: 交通まちづくり ―地方都市からの挑戦―, 鹿島出版社,2015
\item
  魏蜀楠: 佐世保市 『地方型』 MaaS の導入可能性に関する政策研究,2022.
\item
  藤本 直樹:~わが国における「地方型MaaS」の推進に向けた政策の方向性と課題についての考察,~情報経営,~2021,~81 巻,~p. 113-116
\item
  後藤 大:~MaaSの推進と法規制,~保険学雑誌,~2021, 2021巻,~653 号,~p. 653\_67-653\_75
\item
  加藤博和,福本雅之: 地域参画型公共交通サービス供給の成立可能性と持続可能性に関する実証分析-「生活バスよっかいち」を対象として-,土木学会論文集D/65 巻 4 号, 2009
\item
  喜多秀行: 社会的共通資本としての地域公共交通サービスの計画方法論,土木計画学研究発表会(春大会) 2022
\item
  Sonja Hekkilä: Mobility as a Service -A Proposal for Action for the Public Administration, Case Helsinki, 2014
\item
  太田和博:地域交通政策の意思決定における住民参画の意義と課題,運輸と経済第69巻12号,2009
\item
  太田和博:公平性にかなう地域交通政策の策定システム,三田商学研究 Vol.43 No.3 P.187-, 2000
\item
  森栗茂一:交通計画における住民協働の有効性と展開手法,運輸と経済第69巻12号,2009
\item
  ハンス・ケルゼン: 民主主義の本質と価値,長尾隆一・植田俊太郎,岩波文庫, 2015, 1929
\item
  M. Kamargianni, Melinda Matyas: The Business Ecosystem of Mobility-as-a-Service,Computer Science, 2017
\item
  Sochor,J.,Arby,H.,Karlsson,M.,Sarasini,S.: A topological approach to Mobility as a Service: A proposed tool for understanding requirements and effects,and for aiding the integration of societal goals. Proc. ICoMaaS -- 1st International Conference on Mobility as a Service,2017.
\item
  牧原出: 行政改革と調整のシステム,行政学叢書,東京大学出版会,2007
\item
  杉谷和哉:政策にエビデンスは必要なのか-EBPMと政治のあいだ-,ミネルヴァ書房,2022
\end{enumerate}

% -----------------------------------------------------------------------------------------------------------------------------------------
