% -----------------------------------------------------------------------------------------------------------------------------------------
% 第4章 認知バイアスの政策協調への影響:計算論的分析
% (既存論文から統合)
% -----------------------------------------------------------------------------------------------------------------------------------------

\chapter{認知バイアスの政策協調への影響:計算論的分析}
\label{chap:cognitive_bias_analysis}

% -----------------------------------------------------------------------------------------------------------------------------------------
% 以下、既存論文の内容を統合
% 元論文:Re-designing "Collaboration" and "Co-creation" in Regional Public Transport Policy:
%         Integrated Approach to Cognitive Biases and Institutional Coordination
% -----------------------------------------------------------------------------------------------------------------------------------------

\section{研究背景と問題設定}
\label{sec:ch4_background}

日本の地域公共交通政策は、2002年の規制緩和以来、「連携」と「共創」を強調するアプローチへと大きく変容してきた。この変化は、多様なアクター—政府機関、交通事業者、市民、他産業—の専門知識とリソースを活用し、複雑な政策課題に取り組む協調的アプローチへの、より広範な公共ガバナンスの潮流を反映している。

2021年の国土交通省による「ポストコロナ時代の地域交通の共創に関する検討会」は、特にこの政策動向を象徴している。この枠組みは、「共創的交通」を三つの重要な次元で明示的に求めている:地域コミュニティの活性化に貢献する人の流れの生成、交通事業者と他産業間のクロスセクター協働、そして交通を共有責任として捉えるコミュニティ参画である。

しかし、協調と共創への政策的強調にもかかわらず、実証的証拠は政策意図と実装成果の間に大きなギャップがあることを示唆している。協調的枠組みの拡大は、必ずしもより効果的または持続可能な交通ソリューションに直結しておらず、公共交通ガバナンスにおいて協調と共創が有効に機能する条件について根本的な問いを提起している。

\section{Japan MaaS政策分析:実装ギャップの実証}
\label{sec:ch4_maas_analysis}

Japan MaaSイニシアティブの全38プロジェクトの分析は、協調的ガバナンスにおける重要な実装ギャップを明らかにした。2020年度Japan MaaSイニシアティブで認定された全プロジェクトの分析結果:

\begin{itemize}
    \item \textbf{事業指標}:92\%のプロジェクト(35/38)が、利用率、収益生成、運行効率に焦点を当てた指標を含んでいた
    \item \textbf{非事業指標}:わずか29\%のプロジェクト(11/38)が、社会的影響、アクセシビリティ改善、環境成果に対処する指標を組み込んでいた
    \item \textbf{市民参加}:わずか5\%のプロジェクト(2/38)が、プロジェクトガバナンスと評価における市民参加の意味あるメカニズムを確立していた
\end{itemize}

これらの発見は、協調的枠組みがより広範な公共目的に奉仕するのではなく、交通事業者の利益によって捕捉されている可能性を示しており、真の民主的参加の最小限の達成にとどまっている。

\section{計算論的モデリング・フレームワーク}
\label{sec:ch4_modeling}

\subsection{協調制御モデルの構造}
\label{subsec:ch4_model_structure}

本研究は、Yoshihara et al. (2009) の協調ロボット制御理論を応用し、政策実装におけるステークホルダー間の相互作用をモデル化する。N関節ロボットアームは政策ネットワークを表現し、各関節$i$は以下の状態変数を持つ政策ステークホルダーに対応する:

\begin{itemize}
    \item 位置ベクトル: $\mathbf{x}_i \in \mathbb{R}^2$
    \item 速度ベクトル: $\mathbf{v}_i \in \mathbb{R}^2$
    \item 関節角度: $\theta_i \in \mathbb{R}$
    \item リンク長: $a_i \in \mathbb{R}^+$
\end{itemize}

このマッピングにより、特定の能力(リンク長)と政策立場(関節角度)を持つ自律エージェントとしての政策ステークホルダーを表現し、集合的な政策目標を達成するために協調しなければならない関係をモデル化できる。

\subsection{シンボル-概念対応表}
\label{subsec:ch4_symbol_table}

\begin{table}[htbp]
\centering
\caption{シンボル-概念対応表}
\label{tab:symbol_concept}
\begin{tabular}{llll}
\toprule
記号 & 数学的意味 & 制度的意味 & パラメータ範囲 \\
\midrule
$\theta_i$ & 関節角度 & ステークホルダーの政策立場 & $(-\infty, +\infty)$ \\
$a_i$ & リンク長 & ステークホルダーの影響力・能力 & $\mathbb{R}^+$ \\
$b_{sq,i}$ & 現状維持バイアス係数 & 変化への抵抗 & $[0, 1]$ \\
$b_{cf,i}$ & 確証バイアス係数 & 自己判断への過信 & $[-1, 1]$ \\
$b_{nf,i}$ & 狭い視野係数 & 局所最適化への焦点 & $[0, 1]$ \\
$k_i$ & 協調係数 & 他者との協調度 & $[0, 1]$ \\
\bottomrule
\end{tabular}
\end{table}

\subsection{運動学モデル}
\label{subsec:ch4_kinematics}

各関節の位置は、前の関節の位置と累積関節角度に依存する:

\begin{equation}
\mathbf{x}_i = \mathbf{x}_{i-1} + a_i \begin{bmatrix} \cos(\sum_{j=0}^{i-1} \theta_j) \\ \sin(\sum_{j=0}^{i-1} \theta_j) \end{bmatrix}
\end{equation}

ここで $\mathbf{x}_0 = \mathbf{0}$ である。これは、政策成果に対するステークホルダーの決定の累積効果を表現している。

\section{認知バイアスの統合}
\label{sec:ch4_bias_integration}

\subsection{現状維持バイアス}
\label{subsec:ch4_status_quo}

現状維持バイアス $b_{sq,i} \in [0,1]$ は、関節角度変化への抵抗として組み込まれる:

\begin{equation}
\frac{d\theta_i}{dt} = (1 - b_{sq,i}) \cdot \omega_i
\end{equation}

ここで $\omega_i$ はバイアスなしの角速度である。

\subsection{確証バイアス}
\label{subsec:ch4_confirmation}

確証バイアス $b_{cf,i} \in [-1,1]$ は、協調係数の修正として組み込まれる:

\begin{equation}
k_i' = k_i \cdot (1 + b_{cf,i} \cdot 0.5)
\end{equation}

\subsection{狭い視野}
\label{subsec:ch4_narrow_framing}

狭い視野 $b_{nf,i} \in [0,1]$ は、全体目標ベクトルの削減として組み込まれる:

\begin{equation}
\mathbf{v}_{tilda,i}' = (1 - b_{nf,i}) \cdot \mathbf{v}_{tilda,i}
\end{equation}

\section{シミュレーション実験}
\label{sec:ch4_simulation}

\subsection{実験条件}
\label{subsec:ch4_conditions}

8関節ロボットアームを用い、以下の条件下でシミュレーションを実施した:
\begin{itemize}
    \item 制御ゲイン $V = 1.0$
    \item 目標速度ゲイン $G_t = 0.5$
    \item シミュレーション時間:30秒
    \item 各バイアス値:0.0〜1.0の範囲で段階的に変化
\end{itemize}

\subsection{評価指標}
\label{subsec:ch4_metrics}

以下の評価指標を用いた:
\begin{description}
    \item[精度×距離 ($A_t$)] 政策目標への到達精度と移動距離の積
    \item[エネルギー効率 ($E_t$)] 関節動作の効率性
    \item[関節活動度 ($J_{act,i}$)] ステークホルダーの活動レベル
\end{description}

\section{結果と考察}
\label{sec:ch4_results}

\subsection{現状維持バイアスの閾値効果}
\label{subsec:ch4_threshold}

現状維持バイアスには、顕著な閾値効果が観察された。バイアス値が約0.25を超えると、協調効率が急激に低下する。

この閾値は、bounded confidenceモデルにおける合意形成の閾値 $\varepsilon = 0.5$ の半分に相当し、政策変化を受け入れる「臨界点」として解釈できる。

\subsection{確証バイアスの逆説的効果}
\label{subsec:ch4_paradoxical}

興味深いことに、確証バイアスは適度な範囲($b_{cf} \approx 0.3$)では協調を促進する効果が観察された。これは、一定の自信が意思決定の迅速化に寄与するためと解釈できる。

しかし、$b_{cf} > 0.5$ では協調が阻害され、他者からのフィードバックが無視される傾向が強まった。

\subsection{狭い視野の一貫した負の影響}
\label{subsec:ch4_narrow_negative}

狭い視野は、一貫して協調効率を低下させた。特に、複数のステークホルダーが同時に狭い視野を持つ場合、全体最適から大きく乖離した局所解に収束する傾向が観察された。

\section{統計的サマリー}
\label{sec:ch4_statistics}

\begin{table}[htbp]
\centering
\caption{認知バイアス効果の統計的サマリー}
\label{tab:bias_statistics}
\begin{tabular}{lccc}
\toprule
バイアス種別 & 効果の方向 & 閾値 & 推奨対処 \\
\midrule
現状維持バイアス & 非線形閾値効果 & $\approx 0.25$ & 段階的導入 \\
確証バイアス & 逆説的促進 & $\approx 0.3$(最適) & 適度な自信の許容 \\
狭い視野 & 一貫して負 & なし & 全体目標の可視化 \\
\bottomrule
\end{tabular}
\end{table}

\section{小括:認知バイアスによる協調失敗のメカニズム}
\label{sec:ch4_summary}

シミュレーション実験から、以下の知見が得られた:

\begin{enumerate}
    \item 現状維持バイアスには\textbf{閾値効果}があり、臨界点を超えると急激に協調が阻害される
    \item 確証バイアスは適度な範囲では\textbf{逆説的に協調を促進}する可能性がある
    \item 狭い視野は\textbf{一貫して負の影響}を持ち、全体最適を損なう
\end{enumerate}

これらの知見は、第6章での制度設計において重要な示唆となる。

% -----------------------------------------------------------------------------------------------------------------------------------------
