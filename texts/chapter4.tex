% -----------------------------------------------------------------------------------------------------------------------------------------
% 第4章 認知バイアスの計算論的分析
% -----------------------------------------------------------------------------------------------------------------------------------------

\chapter{認知バイアスの計算論的分析}
\label{chap:cognitive_bias}

% -----------------------------------------------------------------------------------------------------------------------------------------
\section{研究の背景と問題設定}
\label{sec:ch4_introduction}

% -----------------------------------------------------------------------------------------------------------------------------------------
\subsection{研究の背景}

地域公共交通政策において「連携」と「共創」が強調されるようになったのは、2002年の規制緩和以降の変化である。市場メカニズムと公的介入のバランスを変えながら、多様なステークホルダーの専門知とリソースを活用して複雑な政策課題に対処するアプローチが求められている。

こうした協調的ガバナンスの制度化は、最近の政策展開によって加速している。地域公共交通計画の策定が努力義務となり、ステークホルダー間の効率的な情報交換と協議を促進するための会議体の設立が強調されている。この制度的枠組みは、連携と共創のプロセスが機能すべき正式な構造を提供している。

% -----------------------------------------------------------------------------------------------------------------------------------------
\subsection{政策の変遷}

日本の地域公共交通政策は、市場メカニズムと公的介入のバランスに関する異なるアプローチを反映して、明確な段階を経て発展してきた。

2002年の規制緩和では、路線運行を許可制から届出制に変更し、補助金配分を会社全体から路線別の支援に変更し、ターゲットを絞った補助金を通じて市場競争を導入しながら不可欠なサービスを維持した。

2006年の地域公共交通会議では、正式な協調的ガバナンスのメカニズムを確立し、コミュニティバスの運行を可能にし、交通計画へのステークホルダー参加のためのプラットフォームを提供した。

2010年の生存交通プロジェクトでは、「頑張る地域」のみが生活交通サービスを維持することを許可される競争的アプローチを制度化し、地方のイニシアティブと自立を強調した。

2021年の共創フレームワークでは、リソースの動員(「交通リソースの総動員」)、共同管理(事業者との直接的な協力)、サービス最適化(「まとめて減らす」アプローチ)を含む、共創を中心的な組織原理として明確に採用した。

% -----------------------------------------------------------------------------------------------------------------------------------------
\subsection{実装ギャップの実証的証拠}

連携と共創の原則の実際の実装を評価するため、本研究では日本のMaaSイニシアティブの包括的な分析を実施した。これは共創政策の具体的な現れを代表している。2020年度に承認された38のプロジェクトすべての分析により、政策の理想と実装の現実との間に大きな乖離が明らかになった。

事業パフォーマンス指標に関しては、プロジェクトの92%(35/38)が利用率、収益生成、運営効率に焦点を当てた指標を含んでいた。対照的に、非事業指標に関しては、プロジェクトの29%(11/38)のみが社会的影響、アクセシビリティ改善、環境成果に対処する指標を組み込んでいた。最も懸念されるのは、市民参加のメカニズムのレベルであり、プロジェクトのわずか5%(2/38)のみがプロジェクトのガバナンスと評価における市民参加の有意義なメカニズムを確立していた。

これらの結果は、協調的枠組みがより広い公共目的に奉仕するのではなく、交通事業者の利益に捕捉されている可能性を示しており、真の民主的参加の最小限の達成が懸念される。

% -----------------------------------------------------------------------------------------------------------------------------------------
\subsection{研究目的とアプローチ}

本研究は3つの主要な研究問いに取り組む。

第一に、協調的交通政策がその表明された目的をどの程度達成しているかを検討すること。ビジネス重視の指標からより広い社会的・民主的価値を組み込むへの移行という、連携政策の実装ギャップを調査する。

第二に、認知バイアスとステークホルダーの特性が政策実装における協調的調整の効果にどのように影響するかを探求すること。計算論的モデリングを用いて、ステークホルダー間の調整メカニズムを分析する。

第三に、連携の利益を最適化しながらその限界を緩和するのに役立つ制度的配置について決定すること。理論的に基礎づけられた制度設計の原則を導出する。

本研究は、実装ギャップの既存の実証的証拠と、認知バイアスがステークホルダー調整の効果にどのように影響するかに関する理論的洞察を統合する統合的アプローチを採用する。

% -----------------------------------------------------------------------------------------------------------------------------------------
\section{理論的枠組みと先行研究}
\label{sec:ch4_framework}

% -----------------------------------------------------------------------------------------------------------------------------------------
\subsection{協調的ガバナンス理論}

協調的ガバナンスは、Ansell and Gash (2008)によって「一つまたは複数の公共機関が、非政府のステークホルダーを、合意形成志向で審議的な集団的意思決定プロセスに直接関与させる統治のあり方」として定義されている。

交通政策の文脈では、協調的アプローチは従来の規制枠組みの限界に対する解決策として推進されてきた。Kato et al. (2009)は、成功したコミュニティ参加型公共交通のための4つの重要な条件を特定している。ステークホルダー間での認識と責任分担の共有、すべての参加者にとっての相互利益、主要な調整者の存在、そしてステークホルダーの努力とサービス改善成果との関連性である。

しかし、協調的ガバナンスは本質的な課題に直面している。Emerson et al. (2012)は、協調には大きな取引コストが必要であり、最小公約数的な解決策につながる可能性があり、組織された利益に捕捉される可能性があると指摘している。

% -----------------------------------------------------------------------------------------------------------------------------------------
\subsection{認知バイアスと意思決定}

意思決定における認知バイアスの影響に関する研究は、協調的調整の失敗を理解するための重要な洞察を提供する。

確証バイアス(Confirmation Bias)は、自己の信念を確認する情報の優先的選択として現れる。ステークホルダーは自分の立場を支持する証拠を求め、矛盾する証拠を無視または軽視する傾向がある。

現状維持バイアス(Status Quo Bias)は、現在の条件を維持する体系的な選好として現れる。変化への抵抗と革新的または破壊的な交通政策の実施の困難さにつながる。

狭い視野(Narrow Framing)は、問題を孤立して考え、より広い文脈や長期的な影響を考慮しない傾向である。全体最適ではなく、部分最適化への固執をもたらす。

% -----------------------------------------------------------------------------------------------------------------------------------------
\section{計算論的モデリング・フレームワーク}
\label{sec:ch4_modeling}

% -----------------------------------------------------------------------------------------------------------------------------------------
\subsection{協調制御モデルの構造}

本研究は、Yoshihara et al. (2009)の協調ロボット制御理論を適応させ、政策実装におけるステークホルダー間の相互作用をモデル化する。N関節ロボットアームは政策ネットワークを表し、各関節iは特定の能力と位置を持つ政策ステークホルダーに対応する。

位置ベクトル、速度ベクトル、関節角度、リンク長などの状態変数を用いて、ステークホルダーの状態を表現する。このマッピングにより、特定の能力(リンク長)と政策位置(関節角度)を持つ自律エージェントとして政策ステークホルダーを表現し、集合的な政策目標を達成するために調整する必要がある状況をモデル化できる。

% -----------------------------------------------------------------------------------------------------------------------------------------
\subsection{認知バイアスの統合}

現状維持バイアスは、関節角度の変化に対する抵抗として組み込まれる。バイアスのない角速度に対して、バイアス係数を乗じることで、変化への抵抗を表現する。

確証バイアスは、調整係数の修正として組み込まれる。正の確証バイアスは自己判断への過剰な自信を表し、負の値は過度の自己疑念を表す。これにより、ステークホルダーが他者との調整において自分の判断をどのように重視するかをモデル化できる。

狭い視野バイアスは、全体最適化の無視として組み込まれる。他の関節の目標速度を考慮する際、バイアス係数によって無視する割合を調整し、全体最適よりも部分最適を優先する傾向を表現する。

% -----------------------------------------------------------------------------------------------------------------------------------------
\section{実験設計と結果}
\label{sec:ch4_experiment}

% -----------------------------------------------------------------------------------------------------------------------------------------
\subsection{実験の概要}

認知バイアスが調整パフォーマンスに与える影響を定量的に評価するため、本研究では合計1,820回の実験を実施した。

ベースライン実験はバイアスなしで100回、確証バイアス実験は11条件で各40回(合計440回、強度0.0-1.0、ステップ0.1)、現状維持バイアス実験は21条件で各40回(合計840回、強度0.0-1.0、ステップ0.05)、狭い視野実験は11条件で各40回(合計440回、強度0.0-1.0、ステップ0.1)である。

決定論的結果を避けるため、初期条件をランダム化した実験設計を実装した。初期関節角度は±0.1ラジアンの範囲で変化させ、目標位置は中心から±0.01mの範囲で変化させ、制御ゲインは±0.5の範囲で変化させた。

% -----------------------------------------------------------------------------------------------------------------------------------------
\subsection{統計分析結果}

確証バイアスの分析では、調整パフォーマンスに有意な影響が見られなかった(F=0.838, p=0.602)。平均パフォーマンスは全強度レベルで0.970±0.002と安定しており、このバイアスタイプには建設的な役割がある可能性を示唆している。

現状維持バイアスの分析では、調整パフォーマンスに有意な影響が見られた(F=1.593, p=0.044)。強度0.2前後に閾値効果が見られ、このレベルを超えるとパフォーマンスが低下した。

狭い視野の分析では、調整パフォーマンスに有意な負の影響が見られた(F=1.985, p=0.028)。R²=0.204、傾き-0.00107の線形劣化が示され、狭い視野の強度が増加するにつれてシステム全体のパフォーマンスが低下することが確認された。

% -----------------------------------------------------------------------------------------------------------------------------------------
\subsection{実験結果の要約}

1,820回の実験を通じて、以下の知見が得られた。

確証バイアスは、調整パフォーマンスを維持する建設的な効果を持つ。適度な自信は協調を促進する可能性がある。

現状維持バイアスは、閾値0.2前後で急激な変化を示す。約0.25を超えると、臨界点を超えて急激に協調が阻害される。

狭い視野は、線形パフォーマンス劣化を示す。一貫して負の影響を持ち、全体最適を損なう。

% -----------------------------------------------------------------------------------------------------------------------------------------
\section{制度的設計フレームワーク}
\label{sec:ch4_design}

% -----------------------------------------------------------------------------------------------------------------------------------------
\subsection{三層制度アーキテクチャ}

Japan MaaS分析と計算論的モデリングの結果に基づき、両方の研究で特定された調整の課題に対処する包括的な制度的設計フレームワークを提案する。

第一層は政治-市民インターフェース(民主的品質設定)である。市民の参加プロセスを通じて交通サービスの品質目標を定義し、政治的監視と明確なアカウンタビリティのメカニズムを提供する。確証バイアスを建設的に活用し、共有された品質目標に対するステークホルダーの信頼を育む。

第二層は行政的調整(環境設計)である。品質目標を運用フレームワークに変換し、専門的な行政能力を構築する。現状維持バイアスを、破壊的な閾値を下回る漸進的変化アプローチで管理する。

第三層は事業-運用インターフェース(自律的実装)である。品質制約内での自律的な運用を可能にし、狭い視野を防ぐための横断的指標を導入する。全体最適化を促進するパフォーマンス監視と調整システムを確立する。

% -----------------------------------------------------------------------------------------------------------------------------------------
\subsection{制度的設計への示唆}

計算論的分析から、以下の設計原則が導かれる。

現状維持バイアスへの対処としては、臨界点(約0.25)を超えないよう、段階的な変化導入を設計することが重要である。

確証バイアスの活用としては、適度な自信は協調を促進するため、完全な中立性よりも構造的な多様性を確保することが有効である。

狭い視野への対処としては、全体目標の可視化と横断的指標の導入を必須とすることが求められる。

% -----------------------------------------------------------------------------------------------------------------------------------------
\section{小括}
\label{sec:ch4_summary}

本章では、協調ロボット制御理論を用いた計算論的モデリング・フレームワークを開発し、協調的交通政策実装における認知バイアスの影響を分析した。

1,820回のシミュレーション実験を通じて、三つの認知バイアスが政策協調に与える影響を定量的に解明した。確証バイアスは統計的に有意な調整低下を引き起こさず、適度な範囲では逆説的に協調を促進する可能性がある。現状維持バイアスは統計的に有意な負の効果を持ち、約0.25を超えると臨界点を超えて急激に協調が阻害される。狭い視野は有意な線形劣化を示し、一貫して負の影響を持つ。

これらの発見に基づき、政治-行政インターフェースと行政-事業インターフェースを分離する三層制度アーキテクチャを提案した。このフレームワークは、民主的アカウンタビリティと運用効率の両方を最適化することを目指している。

次章では、本章の計算論的分析と「執政の創造性」の理論を踏まえ、生成AIと人間の協調的関係性を具体化するZK-SNARKs型政策評価システムを提案する。

% -----------------------------------------------------------------------------------------------------------------------------------------
