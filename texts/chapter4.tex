% -----------------------------------------------------------------------------------------------------------------------------------------
% 第4章 認知バイアスの政策協調への影響:計算論的分析
% -----------------------------------------------------------------------------------------------------------------------------------------

\chapter{認知バイアスの政策協調への影響:計算論的分析}
\label{chap:cognitive_bias_analysis}

% 本章は、計画行政学会研究発表大会(2024年9月)で発表された
% "Exploring Cognitive Bias Effects on Stakeholder Coordination in Regional Public Transport Policy:
% A Computational Analysis Using Cooperative Control Theory"
% を基に、博士論文の構成に合わせて再構成したものである。

% -----------------------------------------------------------------------------------------------------------------------------------------
\section{研究背景と問題設定}
\label{sec:ch4_background}

% -----------------------------------------------------------------------------------------------------------------------------------------
\subsection{研究の背景}

日本の地域公共交通政策は、2002年の規制緩和以来、「連携」(collaboration)と「共創」(co-creation)を強調するアプローチへと大きく変容してきた。この変化は、多様なアクター—政府機関、交通事業者、市民、他産業—の専門知識とリソースを活用し、複雑な政策課題に取り組む協調的アプローチへの、より広範な公共ガバナンスの潮流を反映している。

この変革の制度化は、近年の政策展開によってさらに加速している。地域公共交通計画の策定は義務的努力となり、ステークホルダー間の効率的な情報交換と協議を促進する会議体の設立を強調する具体的ガイダンスが示されている。この制度的枠組みは、連携と共創のプロセスが機能すべき正式な構造を提供している。

% -----------------------------------------------------------------------------------------------------------------------------------------
\subsection{政策の文脈と変遷}

日本の地域公共交通政策は、市場メカニズムと公的介入のバランスに対する異なるアプローチを反映した、明確な段階を経て進化してきた:

\begin{description}
    \item[2002年規制緩和] 路線運行を許可制から届出制に移行し、補助金配分を会社全体から路線別支援に変更し、ターゲットを絞った補助金を通じて本質的なサービスを維持しながら市場競争を導入した。
    
    \item[2006年地域公共交通会議] 正式的な協調ガバナンスのメカニズムを確立し、コミュニティバスの運行を可能にし、交通計画へのステークホルダー参加のプラットフォームを提供した。
    
    \item[2010年生活交通サバイバル事業] 「努力する地域」のみが生活交通サービスを維持することを許可する競争的アプローチを制度化し、地方のイニシアティブと自立を強調した。
    
    \item[2021年共創フレームワーク] 共創を中心的組織原則として明示的に採用し、リソース動員(「交通リソースの総動員」)、共同管理(事業者間の直接協働)、サービス最適化(「束ねて削減」アプローチ)を包含した。
\end{description}

% -----------------------------------------------------------------------------------------------------------------------------------------
\subsection{実装ギャップの実証的証拠}

連携と共創の原則の実践的な実装を評価するために、共創政策の具体的な現れである日本のMaaS(Mobility as a Service)イニシアティブの実証分析を行った。2020年度Japan MaaSイニシアティブで認定された全38プロジェクトの分析は、政策の理想と実装の現実の間に大きな乖離を明らかにした。

\textbf{事業パフォーマンス指標に関して}、92\%のプロジェクト(35/38)が、利用率、収益生成、運行効率に焦点を当てた指標を含んでいた。

\textbf{非事業指標に関して}、わずか29\%のプロジェクト(11/38)が、社会的影響、アクセシビリティ改善、環境成果に対処する指標を組み込んでいた。

\textbf{市民参加メカニズムに関して}、最も懸念されるのは、わずか5\%のプロジェクト(2/38)が、プロジェクトガバナンスと評価における意味ある市民参加のメカニズムを確立していたに過ぎないことである。

\begin{table}[htbp]
\centering
\caption{Japan MaaS 38プロジェクトの実装ギャップ分析}
\label{tab:ch4_maas_gap}
\begin{tabular}{lcc}
\toprule
指標タイプ & プロジェクト数 & 割合 \\
\midrule
事業パフォーマンス指標 & 35/38 & 92\% \\
非事業指標(社会的影響等) & 11/38 & 29\% \\
市民参加メカニズム & 2/38 & 5\% \\
\bottomrule
\end{tabular}
\end{table}

これらの発見は、協調的枠組みがより広範な公共目的に奉仕するのではなく、交通事業者の利益によって捕捉されている可能性を示しており、真の民主的参加の最小限の達成にとどまっている。

% -----------------------------------------------------------------------------------------------------------------------------------------
\subsection{研究目的とアプローチ}

本研究は3つの主要な研究問いに取り組む。第一に、連携的交通政策がその表明された目的をどの程度達成しているかを検討すること。第二に、認知バイアスが協調的調整の効果にどのように影響するかを計算論的モデリングアプローチを用いて探求すること。第三に、連携の利益を最適化しながらその限界を緩和するのに役立つ制度的配置について議論することである。

本研究は、実証的政策評価、協調ロボット制御理論を用いた計算論的分析、そして両方の流れから情報を得た制度設計についての議論を組み合わせた統合的アプローチを採用する。本研究は本質的に\textit{探索的}であり、計算論的モデルは認知バイアス効果に関する仮説を生成する類推的枠組みとして機能し、さらなる実証的検証を必要とする。

% -----------------------------------------------------------------------------------------------------------------------------------------
\section{理論的枠組み}
\label{sec:ch4_theory}

% -----------------------------------------------------------------------------------------------------------------------------------------
\subsection{協調ガバナンス理論}

協調ガバナンスは公共行政において支配的なパラダイムとして登場し、Ansell and Gash(2008)によって「1つ以上の公共機関が、非政府ステークホルダーを、形式的で、合意志向で、審議的な集団的意思決定プロセスに直接関与させ、公共政策を立案または実施し、または公共プログラムまたは資産を管理することを目的とする統治配置」と定義されている。

交通政策の文脈において、協調的アプローチは伝統的な規制フレームワークの限界に対する解決策として推進されてきた。Kato et al.(2009)は、成功する地域住民参加型公共交通のための4つの重要な条件を特定している:ステークホルダー間の共有認識と責任分担、すべての参加者にとっての相互利益、主要な調整者の存在、そしてステークホルダーの努力とサービス改善成果の間のつながりである。

しかし、協調ガバナンスは固有の課題に直面している。Emerson et al.(2012)は、協調には多大な取引コストが必要であり、最低共通分母の解につながる可能性があり、組織化された利益によって捕捉される可能性があると指摘している。これらの課題は、技術的複雑さ、規制上の制約、商業的利益が追加的な調整の困難を生み出す交通政策において特に深刻である。

% -----------------------------------------------------------------------------------------------------------------------------------------
\subsection{公共サービス提供における共創}

共創は伝統的な協調を超える進化を表し、公共機関とステークホルダー間の共同価値創造を強調する(Voorberg et al., 2015)。交通政策において、共創はいくつかの形態で現れる:非伝統的な交通リソースを活用するリソース動員、事業者間の正式なパートナーシップを通じた共同運営、そして他のサービスと交通を束ねて包括的なモビリティソリューションを創出するサービス統合である。

九州運輸局(2023)は、共創が以前は不可能だった路線再編を可能にしサービス効率を改善する一方で、事業パフォーマンスの改善は限定的であり、必要な信頼関係の構築にはかなりの時間投資が必要であることを発見した。

% -----------------------------------------------------------------------------------------------------------------------------------------
\subsection{協調的政策実装における認知バイアス}

協調的交通政策実装に大きな影響を与えるいくつかの認知バイアスがある。

\textbf{現状維持バイアス(Status quo bias)}は、現在の条件を維持することへの体系的な好みを反映し、革新的な交通政策の実装を困難にする(Watkins \& Musselwhite, 2025; Samuelson \& Zeckhauser, 1988)。

\textbf{確証バイアス(Confirmation bias)}は、ステークホルダーが既存の信念を確認するような方法で情報を探したり解釈したりするときに現れ、代替的な視点への開放性を低下させる(Rastogi et al., 2020; Acciarini et al., 2020)。

\textbf{狭い視野(Narrow framing bias)}は、ステークホルダーがシステム全体の最適化よりも局所的な最適化に焦点を当てるときに発生する(Kahneman, 2011)。

これらのバイアスは、政策設定において異なるメカニズムを通じて作用する。それらは情報処理に影響を与え(ステークホルダーがデータをどのように知覚し優先順位を付けるかに影響することによって)、ステークホルダー間の相互作用を複雑にし(マルチステークホルダー設定における認知的コンセンサスの達成を妨げることによって)、そして意思決定のダイナミクスを形成する(政策革新を遅らせたりブロックしたりすることによって)。

重要なことに、認知バイアスは有害な効果と潜在的に建設的な効果の両方を示すことができる。ほとんどのバイアスは調整を妨げるが、構造化されたステークホルダーエンゲージメントと知識交換は、根本的な仮定を明らかにし、時には優先順位をシフトさせ意思決定の質を向上させることができる(Knoppen et al., 2021; Watkins \& Musselwhite, 2025)。

% -----------------------------------------------------------------------------------------------------------------------------------------
\subsection{協調制御理論とマルチエージェント調整}

協調制御理論は、複数の自律エージェントが共有目標を達成するために調整するための枠組みを提供する。この理論は、コンセンサスメカニズム、フォーメーション問題、およびリソース配分を、分散的かつ適応的な方法を通じて扱う(Gulzar et al., 2018)。社会システムへのその拡張は、複雑で分散化された環境を管理するために、分散的意思決定と適応的調整の原則を活用する。

本研究は、Yoshihara et al.(2009)の協調ロボット制御フレームワークを適用する。これは、N関節ロボットアームが局所情報交換を通じてどのように協調動作を達成するかをモデル化するものである。類推的マッピングを可能にする重要な洞察は、ロボットアームの関節と政策ステークホルダーの両方が、局所情報制約と個別の行動傾向の下で運用しながら、共有目標に向けて行動を調整しなければならないということである。

% -----------------------------------------------------------------------------------------------------------------------------------------
\section{計算論的モデリング・フレームワーク}
\label{sec:ch4_modeling}

% -----------------------------------------------------------------------------------------------------------------------------------------
\subsection{類推的マッピング:正当化と限界}

本研究は、N関節ロボットアームを階層的政策ネットワークの\textit{類推的モデル}として使用する。このマッピングでは、各関節$i$は特定のガバナンスレベルでの政策ステークホルダーに対応し、システムの集合的挙動は政策ネットワークの調整ダイナミクスを表現する。

\textbf{直列チェーントポロジーの正当化}:日本の交通ガバナンスは階層構造を通じて運用される。国が政策フレームワークを設定し、都道府県が地域計画を調整し、市町村が地域交通施策を実施し、事業者がサービスを提供する。このガバナンス階層は直列依存性を示す—市町村レベルの決定は都道府県および国のフレームワークによって実質的に制約され、運用上の決定は市町村計画によって制約される。直列運動連鎖は、各レベルの「位置」(政策成果)が先行レベルの累積決定に依存するこの階層的依存構造を捉えている。

\textbf{認められた限界}:この類推的マッピングのいくつかの重要な限界を認める必要がある。トポロジーについては、実際のステークホルダーネットワークは純粋に直列ではなく、並列およびメッシュの相互作用を示す。直列チェーンは階層的依存性を捉えるが、同じガバナンスレベルでのステークホルダー間の水平的調整は捉えない。固定リンク長については、ステークホルダーの影響力($a_i$)は定数として扱われるが、実際には影響力は動的で文脈依存である。この単純化は、調整期間にわたって安定した制度的役割を仮定している。2次元出力空間については、モデルは$\mathbb{R}^2$で動作し、これは2つの主要な政策次元(例えば効率性とアクセシビリティ)を表現すると解釈できる。実際の政策成果はより高次元であり、この削減は2つの優先された目標間のトレードオフを捉えている。単一ターゲットについては、モデルは共有政策目標$\bm{x}_d$を仮定する。ステークホルダーはしばしば対立する目標を持つが、この単純化は、合意ターゲットが民主的プロセスを通じて確立されたシナリオをモデル化する。

% -----------------------------------------------------------------------------------------------------------------------------------------
\subsection{モデル構造}

各関節$i$は状態変数を持つ:位置ベクトル$\bm{x}_i \in \mathbb{R}^2$、速度ベクトル$\bm{v}_i \in \mathbb{R}^2$、関節角度$\theta_i \in \mathbb{R}$、およびリンク長$a_i \in \mathbb{R}^+$。

\begin{table}[htbp]
\centering
\caption{シンボル-概念対応表}
\label{tab:ch4_symbols}
\begin{tabular}{p{1.5cm}p{2.5cm}p{1.2cm}p{3cm}p{4cm}}
\toprule
記号 & 数学的意味 & 単位 & 制度的意味 & 理論的根拠 \\
\midrule
$\theta_i$ & 関節角度 & rad & ステークホルダーの政策立場 & Yoshihara et al. \\
$a_i$ & リンク長 & m & ステークホルダーの制度的影響力(固定) & Yoshihara et al. \\
$b_{sq,i}$ & 現状維持バイアス係数 & -- & 変化への抵抗 & Samuelson \& Zeckhauser; bounded confidence \\
$b_{cf,i}$ & 確証バイアス係数 & -- & 自己判断への過信 & DeGrootモデルの非対称重み \\
$b_{nf,i}$ & 狭い視野係数 & -- & 局所最適化への焦点 & 限定合理性モデル \\
$k_i$ & 協調係数 & -- & 他者との協調度 & 合意形成モデル \\
\bottomrule
\end{tabular}
\end{table}

% -----------------------------------------------------------------------------------------------------------------------------------------
\subsection{運動学モデル}

各関節の位置は、前の関節の位置と累積関節角度に依存する:

\begin{equation}
\bm{x}_i = \bm{x}_{i-1} + a_i \begin{bmatrix} \cos(\sum_{j=0}^{i-1} \theta_j) \\ \sin(\sum_{j=0}^{i-1} \theta_j) \end{bmatrix}
\end{equation}

ここで$\bm{x}_0 = \bm{0}$である。これは、政策成果に対するステークホルダーの決定の累積効果を表現している。

速度は位置の時間微分として計算される:

\begin{equation}
\bm{v}_i = \frac{d\bm{x}_i}{dt}
\end{equation}

% -----------------------------------------------------------------------------------------------------------------------------------------
\subsection{制御目標と調整メカニズム}

システムの目標は、政策目標を表現する目標位置$\bm{x}_d$を追跡するようにエンドエフェクタ(最終ステークホルダー)を導くことである。各関節$i$は協調制御アルゴリズムに従う。

\textbf{目標方向ベクトル計算}:目標からエンドエフェクタへの方向ベクトル:

\begin{equation}
\tilde{\bm{v}}_i = \bm{x}_d - \bm{x}_N
\end{equation}

\textbf{協調係数による重み付け}:各関節は協調係数$k_i$に基づいて調整に参加する:

\begin{equation}
\bm{v}_{target,i} = k_i \cdot \tilde{\bm{v}}_i
\end{equation}

% -----------------------------------------------------------------------------------------------------------------------------------------
\section{認知バイアスの統合}
\label{sec:ch4_bias_integration}

% -----------------------------------------------------------------------------------------------------------------------------------------
\subsection{現状維持バイアスのモデル化}

現状維持バイアス$b_{sq,i} \in [0,1]$は、関節角度変化への抵抗として組み込まれる。Friedkin-Johnsenモデルのアンカリングパラメータに基づき、より小さい感受性値はより大きな変化への抵抗をもたらす:

\begin{equation}
\frac{d\theta_i}{dt} = (1 - b_{sq,i}) \cdot \omega_i
\end{equation}

ここで$\omega_i$はバイアスなしの角速度である。$b_{sq,i} = 0$は完全な感受性を表し、$b_{sq,i} = 1$は完全な固定を表す。

% -----------------------------------------------------------------------------------------------------------------------------------------
\subsection{確証バイアスのモデル化}

確証バイアス$b_{cf,i} \in [-1,1]$は、DeGrootスタイルの意見ダイナミクスにおける類似性依存重み付けに基づき、協調係数の修正として組み込まれる:

\begin{equation}
k_i' = k_i \cdot (1 + b_{cf,i} \cdot 0.5)
\end{equation}

正の値は自己判断への過信を表し、他者からの影響を減らす。負の値は他者の判断への過度の感受性を表す。

% -----------------------------------------------------------------------------------------------------------------------------------------
\subsection{狭い視野のモデル化}

狭い視野$b_{nf,i} \in [0,1]$は、限定合理性モデルに基づき、全体目標ベクトルの削減として組み込まれる。エージェントがグローバルパフォーマンス指標ではなく局所報酬信号に制限される状況をモデル化する:

\begin{equation}
\tilde{\bm{v}}_i' = (1 - b_{nf,i}) \cdot \tilde{\bm{v}}_i
\end{equation}

$b_{nf,i} = 0$は完全なグローバル視野を表し、$b_{nf,i} = 1$は完全な局所視野(目標ベクトルがゼロになる)を表す。

% -----------------------------------------------------------------------------------------------------------------------------------------
\section{シミュレーション実験}
\label{sec:ch4_simulation}

% -----------------------------------------------------------------------------------------------------------------------------------------
\subsection{実験設計}

8関節ロボットアームを用い、日本の階層的ガバナンス構造(国→都道府県→市町村→事業者→...)をモデル化した。1,820回のシミュレーション実験をランダム化された初期条件で実施した。

\textbf{実験パラメータ}:制御ゲイン$V = 1.0$、目標速度ゲイン$G_t = 0.5$、シミュレーション時間:30秒、リンク長$a_i$:0.5から2.0の間でランダムに設定、各バイアス値:0.0から1.0の範囲で段階的に変化させた。

% -----------------------------------------------------------------------------------------------------------------------------------------
\subsection{評価指標}

以下の評価指標を用いた:

\begin{description}
    \item[精度($A_t$)] 最終エンドエフェクタ位置と目標位置の間のユークリッド距離
    \item[エネルギー効率($E_t$)] シミュレーション期間中の総関節動作
    \item[関節活動度($J_{act,i}$)] 各関節の角度変化の合計
\end{description}

% -----------------------------------------------------------------------------------------------------------------------------------------
\section{結果}
\label{sec:ch4_results}

% -----------------------------------------------------------------------------------------------------------------------------------------
\subsection{確証バイアスの効果}

確証バイアスは、統計的に有意な調整低下を引き起こさなかった($F=0.838$, $p=0.602$)。これは直感に反する結果に見えるかもしれないが、構造化された協調設定では、一定レベルの自己確信が実際には意思決定プロセスを加速し、集団パフォーマンスを損なうことなく集団コンセンサスへの到達を容易にする可能性があることを示唆している。

重要なことに、確証バイアスは適度な範囲($b_{cf} \approx 0.3$)では協調を促進する効果が観察された。これは、一定の自信が意思決定の迅速化に寄与するためと解釈できる。しかし、$b_{cf} > 0.5$では協調が阻害され、他者からのフィードバックが無視される傾向が強まった。

% -----------------------------------------------------------------------------------------------------------------------------------------
\subsection{現状維持バイアスの閾値効果}

現状維持バイアスは、統計的に有意な負の効果をもたらした($F=1.593$, $p=0.044$; $\eta^2=0.039$)。より詳細な分析により、顕著な\textbf{閾値効果}が明らかになった。

バイアス値が約0.25を超えると、協調効率が急激に低下する。この閾値は、bounded confidenceモデルにおける合意形成の閾値$\varepsilon = 0.5$の半分に相当し、政策変化を受け入れる「臨界点」として解釈できる。

\begin{table}[htbp]
\centering
\caption{現状維持バイアスの閾値効果}
\label{tab:ch4_threshold}
\begin{tabular}{ccc}
\toprule
バイアス値 & 精度低下 & 解釈 \\
\midrule
0.0--0.25 & 5\%以下 & 順調な調整 \\
0.25--0.50 & 15--30\% & 調整困難 \\
0.50--1.00 & 50\%以上 & 調整失敗 \\
\bottomrule
\end{tabular}
\end{table}

% -----------------------------------------------------------------------------------------------------------------------------------------
\subsection{狭い視野の一貫した負の影響}

狭い視野は、有意な線形劣化を示した($F=1.985$, $p=0.028$; $R^2=0.204$)。狭い視野は、一貫して協調効率を低下させた。

特に、複数のステークホルダーが同時に狭い視野を持つ場合、全体最適から大きく乖離した局所解に収束する傾向が観察された。これは、各ステークホルダーが自分の局所的な最適化を追求し、システム全体のパフォーマンスを考慮しないためである。

% -----------------------------------------------------------------------------------------------------------------------------------------
\section{統計的サマリー}
\label{sec:ch4_statistics}

\begin{table}[htbp]
\centering
\caption{認知バイアス効果の統計的サマリー}
\label{tab:ch4_bias_statistics}
\begin{tabular}{lcccc}
\toprule
バイアス種別 & F値 & p値 & 効果 size & 推奨対処 \\
\midrule
確証バイアス & 0.838 & 0.602 & -- & 適度な自信の許容 \\
現状維持バイアス & 1.593 & 0.044* & $\eta^2=0.039$ & 段階的導入 \\
狭い視野 & 1.985 & 0.028* & $R^2=0.204$ & 全体目標の可視化 \\
\bottomrule
\end{tabular}
\footnotesize
* $p < 0.05$
\end{table}

% -----------------------------------------------------------------------------------------------------------------------------------------
\section{制度的設計への示唆}
\label{sec:ch4_institutional}

計算論的分析と実装ギャップの実証分析に基づき、本研究は3層の制度的アーキテクチャを提案する:

\begin{description}
    \item[層1:民主的品質設定] 政治-行政インターフェースにおいて、政策目標と品質基準を民主的に設定する。この層では、合意ターゲット$\bm{x}_d$が確立される。
    
    \item[層2:行政的調整] 行政-事業インターフェースにおいて、ステークホルダー間の調整を管理する。現状維持バイアスの閾値を考慮した段階的導入を実施する。
    
    \item[層3:運用実装] 事業レベルにおいて、サービスを提供する。狭い視野を防ぐため全体目標の可視化を行う。
\end{description}

この3層アーキテクチャは、民主的アカウンタビリティと運用効率の両方を最適化するために、政治-行政インターフェースと行政-事業インターフェースを分離する。

% -----------------------------------------------------------------------------------------------------------------------------------------
\section{小括:認知バイアスによる協調失敗のメカニズム}
\label{sec:ch4_summary}

シミュレーション実験から、以下の知見が得られた。第一に、確証バイアスは統計的に有意な調整低下を引き起こさなかった($F=0.838$, $p=0.602$)。適度な範囲では逆説的に協調を促進する可能性がある。第二に、現状維持バイアスは統計的に有意な負の効果をもたらした($F=1.593$, $p=0.044$)。閾値効果があり、約0.25を超えると臨界点を超えて急激に協調が阻害される。第三に、狭い視野は有意な線形劣化を示した($F=1.985$, $p=0.028$)。一貫して負の影響を持ち、全体最適を損なう。

これらの知見は、第6章での制度設計において重要な示唆となる。特に、現状維持バイアスの閾値効果は、政策変革を段階的に導入する必要性を示唆しており、狭い視野の一貫した負の影響は、全体目標の可視化と共有の重要性を強調している。

次章では、生成AIと人間の関係性について、ZK-SNARKs型政策評価システムを通じて考察する。

% -----------------------------------------------------------------------------------------------------------------------------------------
