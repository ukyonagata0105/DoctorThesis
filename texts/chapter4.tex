% -----------------------------------------------------------------------------------------------------------------------------------------
% 第4章 認知バイアスの政策協調への影響:計算論的分析
% -----------------------------------------------------------------------------------------------------------------------------------------

\chapter{認知バイアスの政策協調への影響:計算論的分析}
\label{chap:cognitive_bias_analysis}

% -----------------------------------------------------------------------------------------------------------------------------------------
\section{研究設計}
\label{sec:ch4_design}

% -----------------------------------------------------------------------------------------------------------------------------------------
\subsection{協調ロボット制御モデルによる政策ネットワーク表現}

本研究では、Yoshihara et al. (2009) \cite{yoshihara2009cooperative} の協調ロボット制御理論を応用し、政策ネットワークにおけるステークホルダー間の協調をモデル化する。

N関節ロボットアームは、政策ネットワークとして以下のように解釈される:
\begin{itemize}
    \item 各関節 $i$ は政策ステークホルダーに対応
    \item 関節角度 $\theta_i$ はステークホルダーの政策立場を表す
    \item リンク長 $a_i$ はステークホルダーの影響力・能力を表す
    \item エンドエフェクタ位置は政策目標の達成状態を表す
\end{itemize}

% -----------------------------------------------------------------------------------------------------------------------------------------
\subsection{三つの認知バイアスのモデル化}

% -----------------------------------------------------------------------------------------------------------------------------------------
\subsubsection{現状維持バイアス(Status Quo Bias)}
現状維持バイアス $b_{sq,i} \in [0,1]$ は、変化への抵抗として以下のようにモデル化される:

\begin{equation}
\frac{d\theta_i}{dt} = (1 - b_{sq,i}) \cdot \omega_i
\end{equation}

ここで $\omega_i$ はバイアスなしの角速度である。

% -----------------------------------------------------------------------------------------------------------------------------------------
\subsubsection{確証バイアス(Confirmation Bias)}
確証バイアス $b_{cf,i} \in [-1,1]$ は、自己判断への過信として協調係数を修正する:

\begin{equation}
k_i' = k_i \cdot (1 + b_{cf,i} \cdot 0.5)
\end{equation}

% -----------------------------------------------------------------------------------------------------------------------------------------
\subsubsection{狭い視野(Narrow Framing)}
狭い視野 $b_{nf,i} \in [0,1]$ は、全体目標への配慮の減少として表現される:

\begin{equation}
\mathbf{v}_{tilda,i}' = (1 - b_{nf,i}) \cdot \mathbf{v}_{tilda,i}
\end{equation}

% -----------------------------------------------------------------------------------------------------------------------------------------
\section{シミュレーション実験}
\label{sec:ch4_simulation}

% -----------------------------------------------------------------------------------------------------------------------------------------
\subsection{実験条件}

8関節ロボットアームを用い、以下の条件下でシミュレーションを実施した:
\begin{itemize}
    \item 制御ゲイン $V = 1.0$
    \item 目標速度ゲイン $G_t = 0.5$
    \item シミュレーション時間:30秒
    \item 各バイアス値:0.0〜1.0の範囲で段階的に変化
\end{itemize}

% -----------------------------------------------------------------------------------------------------------------------------------------
\subsection{評価指標}

以下の評価指標を用いた:
\begin{description}
    \item[精度$\times$距離 ($A_t$)] 政策目標への到達精度と移動距離の積
    \item[エネルギー効率 ($E_t$)] 関節動作の効率性
    \item[関節活動度 ($J_{act,i}$)] ステークホルダーの活動レベル
\end{description}

% -----------------------------------------------------------------------------------------------------------------------------------------
\section{結果と考察}
\label{sec:ch4_results}

% -----------------------------------------------------------------------------------------------------------------------------------------
\subsection{現状維持バイアスの閾値効果}

現状維持バイアスには、顕著な閾値効果が観察された。バイアス値が約0.25を超えると、協調効率が急激に低下する(図\ref{fig:status_quo_threshold}を参照)。

この閾値は、bounded confidenceモデルにおける合意形成の閾値 $\varepsilon = 0.5$ の半分に相当し\cite{fortunato2005consensus}、政策変化を受け入れる「臨界点」として解釈できる。

% -----------------------------------------------------------------------------------------------------------------------------------------
\subsection{確証バイアスの逆説的効果}

興味深いことに、確証バイアスは適度な範囲($b_{cf} \approx 0.3$)では協調を促進する効果が観察された。これは、一定の自信が意思決定の迅速化に寄与するためと解釈できる。

しかし、$b_{cf} > 0.5$ では協調が阻害され、他者からのフィードバックが無視される傾向が強まった。

% -----------------------------------------------------------------------------------------------------------------------------------------
\subsection{狭い視野の一貫した負の影響}

狭い視野は、一貫して協調効率を低下させた。特に、複数のステークホルダーが同時に狭い視野を持つ場合、全体最適から大きく乖離した局所解に収束する傾向が観察された。

% -----------------------------------------------------------------------------------------------------------------------------------------
\section{小括:認知バイアスによる協調失敗のメカニズム}
\label{sec:ch4_summary}

シミュレーション実験から、以下の知見が得られた:

\begin{enumerate}
    \item 現状維持バイアスには\textbf{閾値効果}があり、臨界点を超えると急激に協調が阻害される
    \item 確証バイアスは適度な範囲では\textbf{逆説的に協調を促進}する可能性がある
    \item 狭い視野は\textbf{一貫して負の影響}を持ち、全体最適を損なう
\end{enumerate}

これらの知見は、第6章での制度設計において重要な示唆となる。

% -----------------------------------------------------------------------------------------------------------------------------------------
