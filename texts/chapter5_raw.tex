政策評価の新時代
-ZK-Snarks概念を援用した、ウィキッド・プロブレムに対応する政策評価の仕組み-
Policy Evaluation for a New Era A Policy Evaluation Framework for Wicked
Problems Using ZK-SNARKs Concepts

◯永田右京(岩手県立大学大学院総合政策研究科D2)

1.課題意識
現代社会においては、単一の専門知識や価値観では解決困難な複雑な政6策課題、いわゆるウィキッド・プロブレムが頻発している。これらの問題に対処するためには、多様な関係者がそれぞれの専門知識や経験を持ち寄り、協力して解決策を模索する必要がある。しかし、関係者間の価値観の相違や情報の非対称性、プライバシーへの懸念などが、効果的な協働を阻害する要因となっている。
従来の政策評価においては、専門家によるトップダウン的な評価や、ロジック・モデルを採用したカスケード式の目標設定・評価が主流であったが、ウィキッド・プロブレムにおいては、その複雑さから、専門家のみによる評価では十分な解決策が得られない場合がある(Head,
2022)。また、関係者間の対話を通じたボトムアップ的な解決策の模索も試みられているが、ここでも情報の非対称性や信頼性やプライバシー保護の観点から課題が残る。
 こうした問題点は、コミュニケーションにおいて正確な理解はそもそも非常に難しい、また知識の非対称性が「正しい情報」への理解を困難にしている、という二重の状況から生じている。
 前者に関しては、ルーマンの述べるコミュニケーションの3段階である情報、伝達、理解のうち理解の多様性は非常に大きく、このエントロピーを削減する作業が非常に困難であるところが挙げられる。社会を構成するコミュニケーションの連鎖を考慮に入れる1と、そこに正確性を組み込むのは非常に難しい。後者に関しては前者と関連する議論として、今まで連鎖してきたコミュニケーションとしての前提知識が合致しなければ、専門家ないし「ある人」の価値観を正確には共有できない問題がある。だからこそ、コミュニケーションの背景と切り離し、一方で共有できる価値観をベースにした評価制度こそが求められるところである。

2.ZK-Sharksの概要 2.1 ZK-SNARKsとは何か ZK-SNARKs(Zero-Knowledge
Succinct Non-interactive Arguments of
Knowledge)は、証明者が検証者に対して秘密情報を開示することなく、その情報の正しさを証明する技術である。この技術は4つの特性から成り立つ。まずZero-Knowledge(ゼロ知識)は、証明を通して元の秘密情報が一切漏洩しないことを意味する。Succinct(簡潔)は、どんなに複雑な計算であっても証明サイズが常に数百バイト程度と一定であることを示す。Non-interactive(非対話)は、証明者から検証者への1回の送信で証明が完了し、復数回のやり取りを必要としないことを表す。Arguments
of
Knowledge(知識の論証)は、証明者が本当にその知識を所有している必要があり、偽造が数学的に不可能であることを意味する。
ZK-SNARKsが実用的な理由は、従来のゼロ知識証明システムが持っていた複数の根本的な限界を同時に解決したことにある。第一に処理速度の革新であり、検証時間が数ミリ秒で完了することで従来の数時間から数日という検証時間を大幅に短縮した。第二に証明サイズの一定性であり、100万ステップの複雑な計算であっても証明サイズは288バイト程度と一定である。第三に適用範囲の幅幅さであり、あらゆるコンピュータ処理を証明対象とすることが可能である。最後に信頼性の高さであり、楕円曲線暗号の安全性に基づいて数学的な偽造が不可能である。

(1行あける) 3.ZK-SNARKsフレームワークの提案
本研究では、プライバシー保護型知識共有の実現として、関係者が機密性の高い専門知識や経験を詳細を開示することなく貢献できる仕組みを構築し、企業秘密や個人情報を保護しながら社会課題解決に必要な知識を活用することを第一の目的とする。第二に、従来の主観的信頼関係に依存しないアルゴリズムによる情報の確からしさ検証により、専門知識の妥当性をアルゴリズム的に評価し権威主義的な意思決定を回避する信頼性の技術的保証を実現する。第三に、複雑な検証手続きの自動化による意思決定の迅速化と、大規模かつ多様なステークホルダー群との効率的な協議の実現という効率的な合意形成プロセスを目指す。最後に、意思決定プロセスの透明性を保ちながら個別の機密情報を保護し、政策根拠の存在証明による行政説明責任を強化する透明性と秘匿性の両立を図る。
これらの目的により、従来不可能であった「秘密を明かすことなく専門性の確からしさを提供する」政策対話の新たなパラダイムを創出することを目指している。
3.1 ZK-SNARKs型の仕組みにおける評価枠組み
ZK-SNARKs型の仕組みにおいては、既存の評価に関する議論が適用できる。この枠組みとしてやはり入札のような、中身を開示しないけれど目的達成はわかる、という仕組みをその都度構築することが求められる。具体的には、各政策課題や意思決定局面において、関係者が以下の要素を満たす証明システムを動的に設計する必要がある。
入札型秘匿証明の構築において、課題解決に必要な専門性や資源を保有していることの能力証明(詳細な手法や情報源は秘匿)、類似課題での成功事例や経験の存在を示す実績証明(具体的な顧客名や機密情報は秘匿)、提案する解決策の実現可能性と期待成果を保証するコミット証明(実装詳細や競争優位情報は秘匿)、そして提供する成果物の品質基準達成を担保する品質保証(内部プロセスや技術仕様は秘匿)という4つの要件が求められる。このようなシステムは、従来の公開入札制度が持つ透明性の利点を維持しながら、関係者のプライバシーと競争優位性を保護し、より多様で質の高い提案を引き出すことを可能にする。
3.2 アルゴリズムとしてのLLM as a Judgeの活用可能性と限界
このような要件を満たす必要がある中で、ZK-SNARKs型システムにおけるLLM as
a
Judge(大規模言語モデル判定者)の導入は、評価プロセスの自動化と一貫性向上において大きな可能性を秘めている。
LLM as a
JudgeがZK-SNARKsフレームワークと本質的に合致する理由は、両者が「複雑な判断プロセスを単一のアルゴリズム処理に集約する」という共通の設計思想を持つことにある。ZK-SNARKsが任意の計算を多項式表現に変換して単一の証明アルゴリズムで処理するように、LLM
as a
Judgeは多様で複雑な評価基準を統一的な言語モデル処理に収束させる。この収束性により、主観的で曖昧な人間判断を客観的で再現可能なアルゴリズム判断に置き換えることが可能となり、ZK-SNARKsの「アルゴリズムによる確からしさの提供」という目標と完全に一致する。
LLM as a
Judgeの可能性として、まずスケーラブルな自動評価が挙げられる。2024-2025年の研究により、LLMは「人間のような推論と意思決定プロセスを模倣」し、従来専門家に依存していた役割の「費用効率的でスケーラブルな代替手段」を提供することが確認されている1。多基準意思決定(MCDM)フレームワーク(AHP、TOPSIS、VIKOR等)との統合により、複雑な政策評価基準を体系的に処理することが可能となる1。次に、ペアワイズ比較手法により複数の政策提案を一貫した基準で比較評価し、「陪審員システム」(複数LLMによるアンサンブル評価)を通じて判定信頼性を向上させることで、一貫性のある判定が実現される1。さらに、評価対象の詳細を開示せずに判定結果のみを提供する構造がZK-SNARKsの秘匿性要件と適合し、大量の提案や複雑な評価基準への対応が人的リソースの制約を解決することで、秘匿性と効率性の両立が図られる。
しかし、LLM as a
Judgeには重要な限界と課題が存在する。多様なバイアス問題として、評価順序による判定の偏りであるポジションバイアス、LLMが自身の生成内容を過大評価する自己強化バイアス、権威ある情報源からの内容を根拠に関わらず高評価する権威バイアスなどが確認されている1。人間判断との不整合も深刻な課題である。2025年時点でも最先端判定LLM(GPT-4シリーズ等)の人間との一致率は0.7以下であり1、多言語環境では一貫性がさらに低下し(Fleiss'
Kappa ≈
0.3)1、実用的な精度の確保が困難となる。信頼性と透明性の問題として、詳細なスコアリングにおける恣意性の増大、敵対的攻撃に対する脆弱性、評価プロセスの説明可能性不足が指摙されている1。これらは政策評価における公正性と透明性の確保を困難にする要因となる。さらに政策適用における特有の課題として、政策判断の複雑性と多面性への対応限界、社会的価値観や倫理的考慮の処理困難、ステークホルダーの多様性への配慮不足が挙げられる。これらの課題は、政策対話システムの設計において慎重な考慮を要する。
3.3 LLM as a Judge の限界を超えるための人間支援機能と改善プロセス LLM
as a
Judgeシステムにおいて人間が結果を意図的に歪めるのではなく、アルゴリズム自体の改善に寄与する複数の支援機能が確立されている。その基盤となるのが、Anthropic(2022)により提案されたConstitutional
AI手法である{[}\^{}4{]}。この革新的な手法は、人間のラベル付けを最小限に抑えながらAIの行動を制御することを可能にした。この手法は憲法的ルールセット(人間が事前に定義した行動規範・価値基準のリスト)、自己批判・自己修正機能(AIが自身の回答を憲法に照らして評価し改善)、最小限の人間監督(詳細な判定ではなく原則設計のみに人間が関与)という3つの核心原則に基づく。実装プロセスは2段階から構成され、教師あり学習段階では初期モデルからサンプルを生成し、AIが自己批判と修正を行って修正された応答で元のモデルを微調整する。強化学習段階では、微調整されたモデルからサンプルを生成し、別のAIモデルがサンプルを評価して選好モデルを訓練し、この選好モデルを報酬信号としてRL
from AI
Feedback(RLAIF)を実施する。この仕組みにより、人間は直接的な判定者ではなく、システムの設計者・調整者として機能し、複雑な価値判断を明確なアルゴリズム処理に変換することで、主観的判断の客観化が実現される。
EvalPlanner(2025年)に代表される最新手法では、計画コンポーネントと推論コンポーネントを分離し、人間が評価手順の設計に関与しながらLLMが推論実行を担当する役割分担が確立されている{[}\^{}5{]}。この分離により、人間がLLMの論理プロセスを読解・修正することが容易になり、意図しない結果が生じた際のアルゴリズム改善が可能となる。
さらに控訴プロセス型の人間介入システムでは、AIが初期判定を行い、問題のあるケースのみ人間専門家が再検討する段階的介入が実現されている{[}\^{}6{]}。このアプローチは結果の意図的歪曲を避けながら、システム改善を図る効果的な手法として評価されている。加えてハイブリッド評価システムとして複数LLMによるアンサンブル評価と人間による検証を組み合わせ、メタジャッジフレームワークにより判定信頼性を向上させる手法も確立されている。これらの研究成果は、LLM
as a
Judgeシステムにおいて人間がアルゴリズムの改善者として機能し、結果の歪曲者ではなくシステムの設計者・調整者として関与する新しいパラダイムを示している。
ZK-SNARKsフレームワークでの戦略では、これらの最新の人間支援機能を組み込んだ統合アプローチが必要である。Constitutional
AI原則に基づく評価基準の設計、計画・推論分離型のEvalPlannerシステム、控訴プロセス型の段階的人間介入、そして複数LLMによるメタジャッジフレームワークを組み合わせることで、効率性と信頼性を両立するZK-SNARKs型政策評価システムの実現が可能となる。
参考文献 1) Eylon Yogev. ``Efficient Universal Arguments.'' Proceedings
of the 2019 Annual ACM Symposium on Theory of Computing, 2019.
https://eprint.iacr.org/2017/1063.pdf 2) Jens Groth. ``On the Size of
Pairing-based Non-interactive Arguments.'' Advances in Cryptology --
EUROCRYPT 2016, 2016. https://eprint.iacr.org/2016/260.pdf 3) Haitao Shi
et al.~``A Survey on LLM-as-a-Judge.'' arXiv preprint arXiv:2411.15594,
2024. DOI: 10.48550/arXiv.2411.15594; Wang, Junpeng et
al.~``LLMs-as-Judges: A Comprehensive Survey on LLM-based Evaluation
Methods.'' arXiv:2412.05579v2, 2024-2025. Available at:
https://arxiv.org/abs/2412.05579; Gu, Yupeng et al.~``Evaluating Large
Language Models as Generalist Judges for Open-Domain Dialogue Systems.''
Proceedings of the 2025 Conference on Empirical Methods in Natural
Language Processing, 2025. 4) Yuntao Bai, Andy Jones, et
al.~``Constitutional AI: Harmlessness from AI Feedback.'' arXiv preprint
arXiv:2212.08073, 2022. Available at: https://arxiv.org/abs/2212.08073
5) Swarnadeep Saha et al.~``Learning to Plan \& Reason for Evaluation
with Thinking-LLM-as-a-Judge.'' arXiv:2501.18099v1, 2025. 6) Shankar,
Shvetank et al.~``How AI can learn from the law: putting humans in the
loop only on appeal.'' Nature Communications, 2024.
