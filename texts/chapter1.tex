% -----------------------------------------------------------------------------------------------------------------------------------------
% 第1章 序論
% -----------------------------------------------------------------------------------------------------------------------------------------

\chapter{序論}
\label{chap:introduction}

% -----------------------------------------------------------------------------------------------------------------------------------------
\section{研究の背景}
\label{sec:intro_background}

% -----------------------------------------------------------------------------------------------------------------------------------------
\subsection{生成AIと人間の関係性という問い}

近年、生成AI(Generative AI)の急速な発展により、政策形成プロセスにおけるAI活用の可能性が広く議論されている。ChatGPTをはじめとする大規模言語モデル(Large Language Models, LLM)は、テキスト生成、要約、分析などのタスクにおいて人間と同等、あるいはそれ以上の性能を示す場面も増えている。しかし、AIが「人間を代替」するのではなく、「人間を補完」する関係性をどのように設計すべきかという問いは、依然として未解決のままである。

本研究では、生成AIを人間の「執政の創造性」\footnote{価値判断、コミュニケーション、新たな規範の創造といった人間固有の能力}を補完する「杖」として位置づけ、両者の協調的関係性のあり方を探求する。

% -----------------------------------------------------------------------------------------------------------------------------------------
\subsection{公共交通政策における「連携・共創」の潮流}

日本の公共交通政策においては、2002年の規制緩和以降、「連携」と「共創」が重要な政策概念として位置づけられてきた。特に2021年の「ポストコロナ時代の地域交通の共創に関する検討会」(国土交通省)においては、交通事業者による地域活性化、異業種との協働、コミュニティ参画という三つの次元での「共創」が提唱されている。

しかし、こうした政策的意図にもかかわらず、実装段階では多くの課題が指摘されている。

% -----------------------------------------------------------------------------------------------------------------------------------------
\section{問題の所在}
\label{sec:intro_problem}

% -----------------------------------------------------------------------------------------------------------------------------------------
\subsection{協調的ガバナンスの実装ギャップ}

公共交通政策における「連携・共創」は、制度的には整備されつつあるものの、実践レベルでは大きなギャップが存在する。例えば、Japan MaaSの38プロジェクトを分析した先行研究\footnote{第3章で詳述}によれば、92\%のプロジェクトが事業指標のみを重視し、社会的影响やアクセシビリティ改善を評価指標に組み込んだのは29\%に留まる。さらに、市民参加の仕組みを設けたプロジェクトはわずか5\%であった。

この実装ギャップは、単なる制度的欠陥だけでなく、人間の認知特性に起因する可能性がある。

% -----------------------------------------------------------------------------------------------------------------------------------------
\subsection{人間の認知限界と政策形成}

人間の意思決定は、認知バイアス(cognitive biases)の影響を強く受けることが知られている\cite{kahneman2011thinking}。特に政策形成プロセスでは、以下のバイアスが重要な影響を及ぼす:

\begin{itemize}
    \item \textbf{現状維持バイアス(Status Quo Bias)}:変化への抵抗、現状の維持選好
    \item \textbf{確証バイアス(Confirmation Bias)}:自己の信念を確認する情報の優先的選択
    \item \textbf{狭い視野(Narrow Framing)}:局所的最適化への固執、全体最適の見落とし
\end{itemize}

これらのバイアスは、ステークホルダー間の協調を阻害し、政策の実装ギャップを生む一因となっている可能性がある。

% -----------------------------------------------------------------------------------------------------------------------------------------
\subsection{生成AIの可能性と限界}

生成AIは、膨大な情報の処理、パターン認識、予測を行うことで、EBPM(証拠に基づく政策形成)を支援する強力なツールとなり得る。しかし、AIには本質的な限界も存在する:

\begin{enumerate}
    \item \textbf{規範的判断・価値創造の不在}:何が社会にとって「善い」のかを判断できない
    \item \textbf{文脈理解の困難性}:学習データの範囲外の「未知の状況」への適応限界
    \item \textbf{「創造性」の源泉の欠如}:人間的な自発的な揺らぎやアナログな現実世界の機微を再現できない
\end{enumerate}

これらの限界を踏まえつつ、AIを「杖」として活用する関係性をどのように設計すべきかが問われている。

% -----------------------------------------------------------------------------------------------------------------------------------------
\section{研究目的と意義}
\label{sec:intro_purpose}

本研究の目的は、以下の三点である:

\begin{enumerate}
    \item 公共交通政策における実装ギャップの要因として、人間の認知バイアスの影響を計算論的に解明する
    \item 生成AIと人間の協調的関係性を具体化するシステムとして、ZK-SNARKs型政策評価システムの可能性を探る
    \item 認知バイアスと生成AIを考慮した制度設計への示唆を導出する
\end{enumerate}

本研究の意義は、生成AIと人間の関係性を理論的・実証的に探求し、より良い政策形成のための指針を提供することにある。

% -----------------------------------------------------------------------------------------------------------------------------------------
\section{論文の構成}
\label{sec:intro_structure}

本論文は、以下の7章から構成される:

\begin{description}
    \item[第2章] 先行研究のレビュー:Human-AI Policyの議論を中心に
    \item[第3章] 舞台としての公共交通政策:現状と課題
    \item[第4章] 認知バイアスの政策協調への影響:計算論的分析
    \item[第5章] 生成AIと人間の関係性:ZK-SNARKs型政策評価システム
    \item[第6章] 制度設計への示唆
    \item[第7章] 結論:都市計画への展開
\end{description}

% -----------------------------------------------------------------------------------------------------------------------------------------
