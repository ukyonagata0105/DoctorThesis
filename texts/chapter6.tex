% -----------------------------------------------------------------------------------------------------------------------------------------
% 第6章 制度設計への示唆
% -----------------------------------------------------------------------------------------------------------------------------------------

\chapter{制度設計への示唆}
\label{chap:institutional_design}

% -----------------------------------------------------------------------------------------------------------------------------------------
\section{理論的含意}
\label{sec:ch6_theoretical_implications}

% -----------------------------------------------------------------------------------------------------------------------------------------
\subsection{第4章のシミュレーション結果からの設計原則}

第4章の計算論的分析から、以下の設計原則が導かれる:

\begin{description}
    \item[現状維持バイアスへの対処] 臨界点($b_{sq} \approx 0.25$)を超えないよう、段階的な変化導入を設計する
    \item[確証バイアスの活用] 適度な自信は協調を促進するため、完全な中立性よりも構造的な多様性を確保する
    \item[狭い視野への対処] 全体目標の可視化と横断的指標の導入を必須とする
\end{description}

% -----------------------------------------------------------------------------------------------------------------------------------------
\subsection{第5章のZK-SNARKsシステムからの制度的含意}

ZK-SNARKs型政策評価システムは、以下の制度的含意を持つ:

\begin{enumerate}
    \item \textbf{秘匿性の制度化}:企業秘密や個人情報を保護しながら政策参加を可能にする制度
    \item \textbf{自動評価の境界}:AIによる評価と人間による判断の境界を明確化する
    \item \textbf{市民参加の質的向上}:市民討議を通じた評価基準の共同設計
\end{enumerate}

% -----------------------------------------------------------------------------------------------------------------------------------------
\section{各バイアスへの対処戦略}
\label{sec:ch6_bias_strategies}

% -----------------------------------------------------------------------------------------------------------------------------------------
\subsection{現状維持バイアス:小さな変化の積み重ね}

現状維持バイアスへの対処として、以下の戦略を提案する:

\begin{itemize}
    \item \textbf{パイロット導入}:小規模な実験から開始し、成功事例を蓄積
    \item \textbf{段階的拡大}:臨界点を超えないよう、徐々に適用範囲を拡大
    \item \textbf{デフォルト設定の変更}:現状維持の方向に働く制度上のデフォルトを見直し
\end{itemize}

% -----------------------------------------------------------------------------------------------------------------------------------------
\subsection{確証バイアス:多様な視点の構造的導入}

確証バイアスへの対処として、以下の戦略を提案する:

\begin{itemize}
    \item \textbf{「悪魔の代理人」の制度化}:AIまたは人間による批判的視点の提示を必須化
    \item \textbf{多様なステークホルダーの参画}:異なる立場・利害を持つ主体の関与を確保
    \item \textbf{反証可能性の確保}:仮説を反証する証拠を積極的に探索するプロセス
\end{itemize}

% -----------------------------------------------------------------------------------------------------------------------------------------
\subsection{狭い視野:全体目標の可視化}

狭い視野への対処として、以下の戦略を提案する:

\begin{itemize}
    \item \textbf{横断的評価指標}:単一の指標ではなく、複数の指標による評価
    \item \textbf{システムマップの作成}:政策の全体像を可視化した図の共有
    \item \textbf{長期的影響の分析}:短期的な成果だけでなく、長期的な影響も評価
\end{itemize}

% -----------------------------------------------------------------------------------------------------------------------------------------
\section{生成AIを組み込んだ制度設計}
\label{sec:ch6_ai_institutional_design}

% -----------------------------------------------------------------------------------------------------------------------------------------
\subsection{政治的-行政的インターフェース}

政治レベルと行政レベルの間で、AIは以下のように位置づけられる:

\begin{itemize}
    \item \textbf{政治的判断}:人間(政治家・議会)が最終的に決定
    \item \textbf{行政的分析}:AIが情報の整理・分析を支援
    \item \textbf{民主的アカウンタビリティ}:人間が説明責任を負う
\end{itemize}

% -----------------------------------------------------------------------------------------------------------------------------------------
\subsection{行政的-事業的インターフェース}

行政レベルと事業レベルの間で、AIは以下のように位置づけられる:

\begin{itemize}
    \item \textbf{政策の具体化}:AIが選択肢の生成を支援
    \item \textbf{運用の効率化}:AIが日常的な判断を補助
    \item \textbf{人間による監督}:重要な判断は人間が行う
\end{itemize}

% -----------------------------------------------------------------------------------------------------------------------------------------
\subsection{ZK-SNARKs型システムの制度的位置づけ}

ZK-SNARKs型政策評価システムは、以下の制度的枠組みの中で運用される:

\begin{enumerate}
    \item \textbf{評価基準の共同設計}:市民討議を通じた基準の策定
    \item \textbf{透明性の確保}:AIの評価プロセスと基準の公開
    \item \textbf{控訴の権利}:評価結果に対する異議申立ての機会
    \item \textbf{人間による最終判断}:AIの評価は参考情報として位置づけ
\end{enumerate}

% -----------------------------------------------------------------------------------------------------------------------------------------
\section{「連携・共創」の再設計}
\label{sec:ch6_collaboration_redesign}

% -----------------------------------------------------------------------------------------------------------------------------------------
\subsection{三層制度設計(S-T1-T2-O)の再考}

STOフレームワークを踏まえ、認知バイアスと生成AIを考慮した制度設計を提案する:

\begin{table}[htbp]
\centering
\caption{認知バイアスと生成AIを考慮した三層制度設計}
\label{tab:three_tier_design}
\begin{tabular}{lp{5cm}p{5cm}}
\toprule
階層 & 主な課題 & 対処戦略 \\
\midrule
S & 現状維持バイアスによる変化抵抗 & 長期ビジョンの共有、段階的目標設定 \\
T1 & 狭い視野による部分最適化 & 横断的指標、全体目標の可視化 \\
T2 & 確証バイアスによる視点の固定化 & 多様な視点の構造的導入 \\
O & 日常的な認知バイアス & AIによる対話的介入 \\
\bottomrule
\end{tabular}
\end{table}

% -----------------------------------------------------------------------------------------------------------------------------------------
\subsection{生成AIを「杖」として活用するガバナンス}

生成AIを活用するガバナンスのあり方として、以下の原則を提案する:

\begin{enumerate}
    \item \textbf{補完性の原則}:AIは人間の能力を補完し、代替しない
    \item \textbf{透明性の原則}:AIの活用方法と限界を明示する
    \item \textbf{アカウンタビリティの原則}:最終的な判断と責任は人間が負う
    \item \textbf{継続的学習の原則}:AIと人間の協調プロセスを継続的に改善する
\end{enumerate}

% -----------------------------------------------------------------------------------------------------------------------------------------
\section{小括}
\label{sec:ch6_summary}

本章では、第4章の計算論的分析と第5章のZK-SNARKs型システム設計を踏まえ、制度設計への示唆を導出した。主な貢献は:

\begin{enumerate}
    \item 認知バイアスへの具体的な対処戦略の提示
    \item 生成AIを組み込んだ制度設計の方向性の提示
    \item 「連携・共創」の再設計に向けた枠組みの提示
\end{enumerate}

次章では、本研究の総括と、都市計画への展開について論じる。

% -----------------------------------------------------------------------------------------------------------------------------------------
