% -----------------------------------------------------------------------------------------------------------------------------------------
% 第6章 制度設計への示唆
% -----------------------------------------------------------------------------------------------------------------------------------------

\chapter{制度設計への示唆}
\label{chap:institutional_design}

% -----------------------------------------------------------------------------------------------------------------------------------------
\section{理論的含意}
\label{sec:ch6_theoretical_implications}

% -----------------------------------------------------------------------------------------------------------------------------------------
\subsection{第4章のシミュレーション結果からの設計原則}

第4章の計算論的分析から、以下の設計原則が導かれる。現状維持バイアスへの対処としては、臨界点($b_{sq} \approx 0.25$)を超えないよう、段階的な変化導入を設計することが重要である。確証バイアスの活用としては、適度な自信は協調を促進するため、完全な中立性よりも構造的な多様性を確保することが有効である。狭い視野への対処としては、全体目標の可視化と横断的指標の導入を必須とすることが求められる。

% -----------------------------------------------------------------------------------------------------------------------------------------
\subsection{第5章のZK-SNARKsシステムからの制度的含意}

ZK-SNARKs型政策評価システムは、以下の制度的含意を持つ。第一に、秘匿性の制度化である。企業秘密や個人情報を保護しながら政策参加を可能にする制度が必要である。第二に、自動評価の境界である。AIによる評価と人間による判断の境界を明確化する必要がある。第三に、市民参加の質的向上である。市民討議を通じた評価基準の共同設計が求められる。

% -----------------------------------------------------------------------------------------------------------------------------------------
\section{各バイアスへの対処戦略}
\label{sec:ch6_bias_strategies}

% -----------------------------------------------------------------------------------------------------------------------------------------
\subsection{現状維持バイアス:小さな変化の積み重ね}

現状維持バイアスへの対処として、小さな変化の積み重ねという戦略を提案する。具体的には、パイロット導入として小規模な実験から開始し、成功事例を蓄積することが有効である。また、段階的拡大として、臨界点を超えないよう、徐々に適用範囲を拡大することが求められる。さらに、デフォルト設定の変更として、現状維持の方向に働く制度上のデフォルトを見直すことも重要である。

% -----------------------------------------------------------------------------------------------------------------------------------------
\subsection{確証バイアス:多様な視点の構造的導入}

確証バイアスへの対処として、多様な視点の構造的導入という戦略を提案する。「悪魔の代理人」の制度化として、AIまたは人間による批判的視点の提示を必須化することが有効である。また、多様なステークホルダーの参画として、異なる立場・利害を持つ主体の関与を確保することが求められる。さらに、反証可能性の確保として、仮説を反証する証拠を積極的に探索するプロセスを組み込むことも重要である。

% -----------------------------------------------------------------------------------------------------------------------------------------
\subsection{狭い視野:全体目標の可視化}

狭い視野への対処として、全体目標の可視化という戦略を提案する。横断的評価指標として、単一の指標ではなく、複数の指標による評価を導入することが有効である。また、システムマップの作成として、政策の全体像を可視化した図の共有が求められる。さらに、長期的影響の分析として、短期的な成果だけでなく、長期的な影響も評価することが重要である。

% -----------------------------------------------------------------------------------------------------------------------------------------
\section{生成AIを組み込んだ制度設計}
\label{sec:ch6_ai_institutional_design}

% -----------------------------------------------------------------------------------------------------------------------------------------
\subsection{政治的-行政的インターフェース}

政治レベルと行政レベルの間で、AIは以下のように位置づけられる。政治的判断については、人間(政治家・議会)が最終的に決定する。行政的分析については、AIが情報の整理・分析を支援する。民主的アカウンタビリティについては、人間が説明責任を負う。

% -----------------------------------------------------------------------------------------------------------------------------------------
\subsection{行政的-事業的インターフェース}

行政レベルと事業レベルの間で、AIは以下のように位置づけられる。政策の具体化については、AIが選択肢の生成を支援する。運用の効率化については、AIが日常的な判断を補助する。人間による監督については、重要な判断は人間が行う。

% -----------------------------------------------------------------------------------------------------------------------------------------
\subsection{ZK-SNARKs型システムの制度的位置づけ}

ZK-SNARKs型政策評価システムは、以下の制度的枠組みの中で運用される。第一に、評価基準の共同設計である。市民討議を通じた基準の策定が必要である。第二に、透明性の確保である。AIの評価プロセスと基準の公開が求められる。第三に、控訴の権利である。評価結果に対する異議申立ての機会を確保する必要がある。第四に、人間による最終判断である。AIの評価は参考情報として位置づけられる。

% -----------------------------------------------------------------------------------------------------------------------------------------
\section{「連携・共創」の再設計}
\label{sec:ch6_collaboration_redesign}

% -----------------------------------------------------------------------------------------------------------------------------------------
\subsection{三層制度設計(S-T1-T2-O)の再考}

STOフレームワークを踏まえ、認知バイアスと生成AIを考慮した制度設計を提案する。

\begin{table}[htbp]
\centering
\caption{認知バイアスと生成AIを考慮した三層制度設計}
\label{tab:three_tier_design}
\begin{tabular}{lp{5cm}p{5cm}}
\toprule
階層 & 主な課題 & 対処戦略 \\
\midrule
S & 現状維持バイアスによる変化抵抗 & 長期ビジョンの共有、段階的目標設定 \\
T1 & 狭い視野による部分最適化 & 横断的指標、全体目標の可視化 \\
T2 & 確証バイアスによる視点の固定化 & 多様な視点の構造的導入 \\
O & 日常的な認知バイアス & AIによる対話的介入 \\
\bottomrule
\end{tabular}
\end{table}

S層では現状維持バイアスによる変化抵抗が課題となるため、長期ビジョンの共有と段階的目標設定が有効である。T1層では狭い視野による部分最適化が課題となるため、横断的指標と全体目標の可視化が求められる。T2層では確証バイアスによる視点の固定化が課題となるため、多様な視点の構造的導入が必要である。O層では日常的な認知バイアスが課題となるため、AIによる対話的介入が有効である。

% -----------------------------------------------------------------------------------------------------------------------------------------
\subsection{生成AIを「杖」として活用するガバナンス}

生成AIを活用するガバナンスのあり方として、以下の原則を提案する。第一に、補完性の原則である。AIは人間の能力を補完し、代替しない。第二に、透明性の原則である。AIの活用方法と限界を明示する。第三に、アカウンタビリティの原則である。最終的な判断と責任は人間が負う。第四に、継続的学習の原則である。AIと人間の協調プロセスを継続的に改善する。

% -----------------------------------------------------------------------------------------------------------------------------------------
\section{小括}
\label{sec:ch6_summary}

本章では、第4章の計算論的分析と第5章のZK-SNARKs型システム設計を踏まえ、制度設計への示唆を導出した。主な貢献として、認知バイアスへの具体的な対処戦略の提示、生成AIを組み込んだ制度設計の方向性の提示、「連携・共創」の再設計に向けた枠組みの提示が挙げられる。

次章では、本研究の総括と、都市計画への展開について論じる。

% -----------------------------------------------------------------------------------------------------------------------------------------
