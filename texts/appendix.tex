% -----------------------------------------------------------------------------------------------------------------------------------------
% 付録
% -----------------------------------------------------------------------------------------------------------------------------------------

\chapter{付録}
\label{app:appendix}

\section{協調ロボット制御モデルの数式展開}
\label{app:robot_model}

% -----------------------------------------------------------------------------------------------------------------------------------------
\subsection{運動学モデル}

N関節ロボットアームにおいて、各関節の位置は以下の式で与えられる:

\begin{equation}
\mathbf{x}_i = \mathbf{x}_{i-1} + a_i \begin{bmatrix} \cos(\sum_{j=0}^{i-1} \theta_j) \\ \sin(\sum_{j=0}^{i-1} \theta_j) \end{bmatrix}
\end{equation}

ここで、$\mathbf{x}_0 = \mathbf{0}$ である。

% -----------------------------------------------------------------------------------------------------------------------------------------
\subsection{協調係数の計算}

各関節の協調係数 $k_i$ は、以下の式で計算される:

\begin{equation}
k_i = \exp\left(-4\ln(2) \frac{||\mathbf{v}_{l,i} - \mathbf{v}_{i+1}||^2 + \epsilon_1}{||\mathbf{v}_{l,i}||^2 + \epsilon_2}\right)
\end{equation}

ここで、$\epsilon_1, \epsilon_2$ は数値安定性のための小さな正の定数である。

% -----------------------------------------------------------------------------------------------------------------------------------------
\section{Japan MaaS プロジェクト分析の詳細}
\label{app:maas_analysis}

% -----------------------------------------------------------------------------------------------------------------------------------------
\subsection{分析対象プロジェクト一覧}

2020年度に認定された38のJapan MaaSプロジェクトを分析対象とした。プロジェクトは以下のカテゴリに分類される:

\begin{itemize}
    \item 観光型MaaS:XX件
    \item 地域課題解決型MaaS:XX件
    \item 企業主導型MaaS:XX件
\end{itemize}

% -----------------------------------------------------------------------------------------------------------------------------------------
\subsection{評価指標のコーディング基準}

評価指標は以下の基準でコーディングした:

\begin{table}[htbp]
\centering
\caption{評価指標のコーディング基準}
\begin{tabular}{ll}
\toprule
カテゴリ & コーディング基準 \\
\midrule
事業指標 & 利用者数、収益、運行効率など \\
非事業指標 & 社会影响、環境効果、アクセシビリティ改善など \\
市民参加 & ワークショップ、アンケート、協議会など \\
\bottomrule
\end{tabular}
\end{table}

% -----------------------------------------------------------------------------------------------------------------------------------------
\section{ZK-SNARKs型政策評価システムの実装詳細}
\label{app:zk_implementation}

% -----------------------------------------------------------------------------------------------------------------------------------------
\subsection{システム構成}

ZK-SNARKs型政策評価システムは、以下のコンポーネントから構成される:

\begin{itemize}
    \item 秘匿化処理モジュール
    \item LLM評価エンジン
    \item Constitutional AI基準ライブラリ
    \item 控訴処理インターフェース
\end{itemize}

% -----------------------------------------------------------------------------------------------------------------------------------------
\subsection{技術的仕様}

\begin{table}[htbp]
\centering
\caption{技術的仕様}
\begin{tabular}{ll}
\toprule
項目 & 仕様 \\
\midrule
LLM & GPT-4 / Claude \\
Temperature & 0 \\
Self-Consistency & 5回 \\
TEE & AWS Nitro Enclaves \\
\bottomrule
\end{tabular}
\end{table}

% -----------------------------------------------------------------------------------------------------------------------------------------
