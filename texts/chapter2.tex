% -----------------------------------------------------------------------------------------------------------------------------------------
% 第2章 先行研究のレビュー:Human-AI Policyの議論
% -----------------------------------------------------------------------------------------------------------------------------------------

\chapter{先行研究のレビュー:Human-AI Policyの議論}
\label{chap:literature_review}

% -----------------------------------------------------------------------------------------------------------------------------------------
\section{はじめに}
\label{sec:ch2_intro}

本章では、本研究の理論的基盤となる先行研究をレビューする。特に、生成AIと政策形成、人間-AI協調の理論、協調的ガバナンス論、認知バイアスと意思決定、ZK-SNARKsと政策評価の五つの領域を中心に整理し、理論的空白を特定する。

% -----------------------------------------------------------------------------------------------------------------------------------------
\section{Human-AI Policy:政策形成におけるAIと人間の関係性}
\label{sec:ch2_human_ai_policy}

% -----------------------------------------------------------------------------------------------------------------------------------------
\subsection{AI政策論の展開}

公共政策におけるAI活用に関する議論は、2010年代後半から急速に発展してきた。初期の議論は、AIによる行政サービスの効率化や自動化に焦点が置かれていたが、近年ではAIと人間の関係性そのものが問いの中心となっている\cite{shneiderman2020human}。

この転換の背景には、生成AI(ChatGPT、Claude、Gemini等)の登場がある。これらの技術は、従来のAI(分類・予測)とは異なり、創発的なテキスト生成能力を持つ。この能力は、政策文書の作成、選択肢の生成、市民との対話など、政策形成の核となるプロセスに直接関与しうる。

% -----------------------------------------------------------------------------------------------------------------------------------------
\subsection{Human-Centered AIの理念}

Shneiderman (2020) は、Human-Centered AI(HCAI)の理念として、以下の2軸マトリクスを提示している:

\begin{description}
    \item[高自動化・低制御] AIが自律的に判断し、人間は結果を受け入れるのみ
    \item[高自動化・高制御] AIが提案を行い、人間が最終判断を下す
    \item[低自動化・高制御] 人間が主導し、AIが補助的な役割を果たす
    \item[低自動化・低制御] 人間もAIも十分に機能しない状態
\end{description}

本研究が着目するのは「高自動化・高制御」の領域である。この領域では、AIの計算能力と人間の規範判断力が相互に補完し合う。

% -----------------------------------------------------------------------------------------------------------------------------------------
\subsection{「執政の創造性」とAIの補完性}

行政学の観点からは、「執政の創造性」という概念が重要である。これは、公務員が直面する複雑な問題に対し、価値判断、コミュニケーション、新たな規範の創造を通じて対応する能力を指す。

生成AIは、以下の領域では人間を補完し得る:
\begin{itemize}
    \item データの処理・分析
    \item 選択肢の生成・提示
    \item 文書作成の効率化
    \item 多様な視点の提示
    \item 認知バイアスの指摘(「悪魔の代理人」機能)
\end{itemize}

一方で、以下の領域では人間の役割が不可欠である:
\begin{itemize}
    \item 規範的判断(何が「善い」か)
    \item 文脈に応じた柔軟な対応
    \item 新たな価値の創造
    \item 政治的なアカウンタビリティ
    \item 最終的な責任の所在
\end{itemize}

AIが「できないこと」を明示的に理解することは、AIの効果的な活用にとって不可欠である。AIは学習データの範囲内での推論しか行えず、規範的な価値判断や、学習データに含まれない文脈への適応は本質的に困難である。

% -----------------------------------------------------------------------------------------------------------------------------------------
\subsection{AIの「杖」としての位置づけ}

生成AIを「杖(Aaron's rod)」として位置づける視点は、AIが人間を代替するのではなく、人間の能力を補完・増幅する道具として活用する考え方である。聖書においてアロンの杖がモーセの役割を補完したように、AIは政策決定者の「杖」として機能する。

この視点からは、以下の設計原則が導かれる:
\begin{enumerate}
    \item AIは人間の最終判断を前提とする
    \item AIの限界を明示的に理解する
    \item 人間-AI協調のプロセスを透明化する
    \item AI自体のバイアスに対処する
    \item 説明責任は常に人間が負う
\end{enumerate}

この「杖」としての位置づけは、AIを「執政の創造性」を支援する道具として捉え直す視点を提供する。

% -----------------------------------------------------------------------------------------------------------------------------------------
\section{協調制御理論と社会システムへの応用}
\label{sec:ch2_cooperative_control}

% -----------------------------------------------------------------------------------------------------------------------------------------
\subsection{協調制御理論の基礎}

協調制御理論(Cooperative Control Theory)は、複数の自律エージェントが協調して共通の目標を達成するための制御手法を研究する分野である\cite{yoshihara2009cooperative}。

この理論は、以下の特徴を持つ:
\begin{itemize}
    \item 分散的な意思決定
    \item 局所的な情報に基づく協調
    \item 全体的な目標の達成
\end{itemize}

% -----------------------------------------------------------------------------------------------------------------------------------------
\subsection{社会システムへの応用可能性}

協調制御理論は、社会システムの分析にも応用可能である。特に、複数のステークホルダーが関与する政策ネットワークにおいて、各主体が自律的に行動しながら全体としての政策目標を達成するプロセスをモデル化できる。

本研究では、協調ロボット制御モデルを用いて、政策ネットワークにおけるステークホルダー間の協調を分析する(第4章で詳述)。

% -----------------------------------------------------------------------------------------------------------------------------------------
\section{協調的ガバナンス論}
\label{sec:ch2_collaborative_governance}

% -----------------------------------------------------------------------------------------------------------------------------------------
\subsection{協調的ガバナンスの定義}

Ansell and Gash (2008) \cite{ansell2008collaborative} は、協調的ガバナンスを以下のように定義している:

\begin{quote}
「一つまたは複数の公共機関が、非政府のステークホルダーを、合意形成志向で審議的な集団的意思決定プロセスに直接関与させる統治のあり方」
\end{quote}

% -----------------------------------------------------------------------------------------------------------------------------------------
\subsection{公共交通における連携・共創}

日本の公共交通政策においては、「連携」と「共創」が重要な概念として位置づけられている\cite{kato2009community}。Kato et al. (2009) は、コミュニティ参加型地域公共交通の成功条件として以下を指摘している:

\begin{enumerate}
    \item 関係ステークホルダー間での認識と責任分担の共有
    \item 各ステークホルダーが参加から利益を得られること
    \item ステークホルダーを調整するキーパーソンの存在
    \item ステークホルダーの努力が利用促進・価値向上につながること
\end{enumerate}

% -----------------------------------------------------------------------------------------------------------------------------------------
\subsection{実装ギャップの指摘}

しかし、こうした理論的条件にもかかわらず、実践レベルでは多くの課題が指摘されている。Emerson et al. (2012) \cite{emerson2012integrative} は、協調的ガバナンスが直面する課題として以下を指摘している:

\begin{itemize}
    \item 高い取引コスト
    \item 最小公約数的な解決策への収束
    \item 組織された利益による捕捉
\end{itemize}

% -----------------------------------------------------------------------------------------------------------------------------------------
\section{認知バイアスと意思決定}
\label{sec:ch2_cognitive_bias}

% -----------------------------------------------------------------------------------------------------------------------------------------
\subsection{行動経済学の基礎概念}

Kahneman (2011) \cite{kahneman2011thinking} は、人間の思考を「システム1(速い思考)」と「システム2(遅い思考)」に分類し、認知バイアスがシステム1の特性に起因することを示した。

% -----------------------------------------------------------------------------------------------------------------------------------------
\subsection{政策プロセスにおける認知バイアス}

政策形成において特に重要な認知バイアスとして、以下の三つを取り上げる:

\subsubsection{現状維持バイアス(Status Quo Bias)}
Samuelson and Zeckhauser (1988) \cite{samuelson1988status} によって提唱された概念で、変化よりも現状を維持することを好む傾向を指す。

\subsubsection{確証バイアス(Confirmation Bias)}
既存の信念や仮説を支持する情報を優先的に探し、反証する情報を無視・軽視する傾向\cite{russio2015confirmation}。

\subsubsection{狭い視野(Narrow Framing)}
問題を孤立して考え、より広い文脈や長期的な影響を考慮しない傾向\cite{kahneman2011thinking}。

% -----------------------------------------------------------------------------------------------------------------------------------------
\section{ZK-SNARKsと政策評価}
\label{sec:ch2_zksnarks}

% -----------------------------------------------------------------------------------------------------------------------------------------
\subsection{ZK-SNARKsの基本概念}

ZK-SNARKs(Zero-Knowledge Succinct Non-interactive Arguments of Knowledge)は、暗号技術の一種であり、秘密情報を公開することなく、その情報の正しさを証明する技術である\cite{ben2014succinct}。

ZK-SNARKsは以下の4つの特性を持つ:
\begin{description}
    \item[Zero-Knowledge] 証明を通して元の秘密情報が一切漏洩しない
    \item[Succinct] 証明サイズが常に数百バイト程度と一定
    \item[Non-interactive] 証明者から検証者への1回の送信で証明完了
    \item[Arguments of Knowledge] 真の知識を所有している必要があり偽造不可能
\end{description}

% -----------------------------------------------------------------------------------------------------------------------------------------
\subsection{政策評価への応用可能性}

ZK-SNARKsの概念を政策評価に応用することで、「秘密を守りながら専門性を証明する」仕組みが実現可能になる。例えば、企業が自社の技術情報を公開せずに、政策課題への貢献可能性を証明できる。

本研究では、ZK-SNARKsの概念を援用した政策評価システムをLLM as a Judgeと組み合わせて提案する(第5章で詳述)。

% -----------------------------------------------------------------------------------------------------------------------------------------
\section{小括:理論的空白の特定}
\label{sec:ch2_summary}

先行研究のレビューから、以下の理論的空白が明らかになった:

\begin{enumerate}
    \item \textbf{認知バイアスと政策協調の接続}:協調的ガバナンスの失敗要因として認知バイアスに着目した研究は限定的である
    \item \textbf{計算論的分析手法の欠如}:政策協調プロセスを計算論的にモデル化した研究は少ない
    \item \textbf{生成AIの「杖」としての理論化}:AIと人間の補完的関係性を理論的に位置づけた研究は不十分である
    \item \textbf{ZK-SNARKs概念の政策評価への応用}:暗号技術の概念を政策評価に応用した試みは先駆的である
\end{enumerate}

本研究は、これらの空白を埋めることを目指す。

% -----------------------------------------------------------------------------------------------------------------------------------------
