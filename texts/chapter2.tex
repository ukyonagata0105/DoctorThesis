% -----------------------------------------------------------------------------------------------------------------------------------------
% 第2章 先行研究のレビュー:Human-AI Policyの議論
% -----------------------------------------------------------------------------------------------------------------------------------------

\chapter{先行研究のレビュー:Human-AI Policyの議論}
\label{chap:literature_review}

% -----------------------------------------------------------------------------------------------------------------------------------------
\section{はじめに}
\label{sec:ch2_intro}

本章では、本研究の理論的基盤となる先行研究をレビューする。特に、「執政の創造性」とルーマン理論、生成AIと政策形成、人間-AI協調の理論、協調的ガバナンス論、認知バイアスと意思決定、ZK-SNARKsと政策評価の六つの領域を中心に整理し、理論的空白を特定する。

% -----------------------------------------------------------------------------------------------------------------------------------------
\section{「執政の創造性」とは何か}
\label{sec:ch2_executive_creativity}

% -----------------------------------------------------------------------------------------------------------------------------------------
\subsection{定義}

「執政の創造性」とは、以下のように定義される:

\begin{quote}
\textbf{執政の創造性}:社会の構成員同士のコミュニケーションを前提に、価値判断を基盤として、新たなガバナンスの範囲を生成・配分していく能力。
\end{quote}

この定義は、三つの要素から構成される。第一に、政策は、単一の主体による合理的決定ではなく、複数の主体間のコミュニケーションを通じて形成される。第二に、政策決定は統計的平均や客観的指標への還元が不可能な、主体的な価値判断を要する。第三に、政策は固定された枠組みではなく、状況に応じて新たな範囲を創造し続ける動的なプロセスである。

この概念は、カンギレム(Canguilhem)の「規範の創造」概念\cite{canguilhem1966}および井庭の創造システム理論\cite{iba2009}を統合したものである。カンギレムは、「正常」とは統計的平均ではなく、「生の独自的な規範性」を肯定することであると論じた。健康とは客観的指標ではなく、主体によって体験される価値である。同様に、政策における「正しさ」も、統計的効率性ではなく、主体による価値判断に基づく。

井庭の創造システム理論では、コミュニケーションの「わかり合えなさ」から出発し、「発見」を要素とするシステム上で創造が起こるとされる。本研究は、この創造のプロセスを政策の文脈に適用し、「執政の創造性」として定式化する。

% -----------------------------------------------------------------------------------------------------------------------------------------
\subsection{政策過程モデルにおける価値判断}

「執政の創造性」の重要性は、既存の政策過程モデルを分析することで裏付けられる。主要な政策過程モデルを見てみよう。

\subsubsection{政策の窓モデル}

キングドン\cite{kingdon1984}\cite{kusano1997}による政策の窓モデルは、政策過程に「問題」「政策代替案」「政治」という3つの独立した流れが存在し、それらが特定の時点で合流したときに政策決定が生じることを示す。

キングドンは、この3つの流れが合流する瞬間を「政策の窓」が開くと表現する。窓は一時的にしか開かず、逃したら再び開くまで長く待たなければならない。

このモデルが示すのは、政策決定は論理的な手順ではなく、問題と解決案と政治的機会の偶然の出会いによって生じるということである。この「偶然の出会い」を見極め、窓が開いた瞬間に行動するには、人間の価値判断と政治的勘が不可欠である。AIは過去のデータから傾向を分析することはできても、「今がチャンスだ」と感じ取り、その瞬間に動くという状況認識と行動のタイミングについては、人間の判断に依存せざるを得ない。

\subsubsection{唱導連携モデル}

唱導連携モデル(Advocacy Coalition Framework)は、サバティエ\cite{sabatier1988}らによって開発されたモデルであり、特定の政策分野において、共通の信念体系を共有する主体たちが連携(コアリション)を形成し、政策を唱導するプロセスを分析する。

このモデルの特徴は、政策を単なる利害調整ではなく、信念体系の競合として捉える点にある。各連携は、(1)深層核信念(基本的価値観)、(2)政策核信念(具体的政策目標)、(3)二次的側面(手段的判断)という階層的な信念体系を持つ。

ここで重要なのは、どの信念を優先すべきか、どの連携の主張を採用すべきかという判断が、常に価値判断を伴うということである。統計的データや客観的分析だけでは、信念の競合を解決できない。どの価値を重視するかという判断が必要であり、これは人間の創造的適応能力に依存する。

\subsubsection{村松モデル}

村松\cite{muramatsu1981}モデルは、日本の政策過程を「与党・官僚・利益団体」の三者関係として分析する枠組みである。このモデルでは、政策決定はこれら三者の交渉と取引を通じて行われる。

真渕による修正版では、野党勢力や新規参入者の役割も考慮される。既存の産業組織が維持しようとする価値と、新規参入者が求める変革の価値が衝突する中で、政治家はバランスを取る必要がある。

このモデルが示すのもまた、利害の調整と価値の優先順位付けが政治的本質であり、それは計算可能な最適化問題ではないということである。三者間の「妥当な落としどころ」を見出すには、人間同士の交渉と相互調整が不可欠である。

% -----------------------------------------------------------------------------------------------------------------------------------------
\subsection{ウィキッド・プロブレムとの関連}

公共政策の多くは、Rittel \& Webber\cite{rittel1973}\cite{sugitani2021}の「ウィキッド・プロブレム(Wicked Problems)」の性質を帯びている。ウィキッド・プロブレムとは、決定的な問題定義がなく、問題そのものが何であるかについて利害関係者の間で合意がない問題である。停止ルールもなく、いつ問題が「解決」されたのかを客観的に判断できない。解は「良い/悪い」ではなく「より良い/より悪い」であり、最適解は存在せず、複数の利害のバランスを取るしかない。また、解を適用した結果、予期せぬ副作用が生じる可能性があり、社会実験はやり直しが効かない。さらに、すべてのウィキッド・プロブレムは本質的にユニークであり、過去の経験から単純に適用できない。

ウィキッド・プロブレムへの対応は、終わりのない創造的プロセスである。そして、「終わりがないからこそ、政治がその不快さを受け入れる必要がある」のである。この終わりのない創造的プロセスこそが、人間による政策の固有の意味であり、AIには代替不可能な領域である。

% -----------------------------------------------------------------------------------------------------------------------------------------
\section{ルーマン理論による基礎づけ}
\label{sec:ch2_luhmann}

% -----------------------------------------------------------------------------------------------------------------------------------------
\subsection{なぜルーマンか}

コミュニケーションを理論的基軸とする場合、ユルゲン・ハーバーマスの「コミュニケーション的行為理論」\cite{habermas1981}も選択肢として存在する。しかし、本研究はルーマン\cite{luhmann1984}の立場を採用する。その理由を説明する。

ハーバーマスは、コミュニケーションを「了解志向的」な行為として捉える。コミュニケーションの理想的状態では、参加者が互いに「了解」に到達し、合意を形成することが期待される。この立場からは、コミュニケーションの「失敗」や「誤解」は、理想的言語状況からの逸脱として問題視される。

これに対し、ルーマンはコミュニケーションを「了解の否定」を内包するプロセスとして捉える。ルーマンによれば、コミュニケーションは常に「理解」と「誤解」の双方を可能性として含んでおり、「わかり合えなさ」こそがコミュニケーションの本質的な特徴である。

本研究がルーマンの立場を採用する理由は、以下の二点にある。

第一に、「わかり合えなさ」を「創造の源泉」として位置づける視点が必要である。ハーバーマス的な「理想的了解」を前提とすれば、AIも「十分に良い」情報処理を行うことで「機能的な了解」に貢献できると主張しうる。しかし、ルーマン的な視点からは、「わかり合えなさ」から生じる価値判断と意味構成のプロセスこそが創造の核心であり、これをAIは代替できない。

第二に、政策的決定における「価値競合」の不可避性も重要である。公共政策は、ハーバーマスが想定する「理想的言語状況」において理性の力だけで解決可能な問題ではなく、複数の正当な価値が競合するウィキッド・プロブレムとしての性質を持つ。このような状況では、「了解」への到達よりも、「わかり合えなさ」を前提とした創造的適応こそが求められる。

% -----------------------------------------------------------------------------------------------------------------------------------------
\subsection{コミュニケーションの3段階:情報・伝達・理解}

ルーマン\cite{luhmann1984}\cite{kneer1993}によれば、コミュニケーションは「情報(Information)・伝達(Mitteilung)・理解(Verstehen)」という3つの選択過程から構成される。

\begin{description}
    \item[情報(Information)] 多数の可能性の地平からの一つの選択であり、何を語るかの選択である。
    \item[伝達(Mitteilung)] 多数の伝達可能性からの選択であり、いかに語るかの選択である。
    \item[理解(Verstehen)] 多数の理解可能性からの選択であり、いかに受け止めるかの選択である。
\end{description}

ルーマンは、「三つの選択のはたらきのすべてが総合されるときにはじめてコミュニケーションというものが成り立つ」と強調する。この3層構造は、コミュニケーションが単純な情報伝達ではなく、各段階で選択と解釈が介在する複雑な過程であることを示している。

% -----------------------------------------------------------------------------------------------------------------------------------------
\subsection{構造的カップリング:本研究の核心概念}

ここで重要になるのが、ルーマンにおける「構造的カップリング(strukturelle Kopplung)」\cite{luhmann1984}\cite{maturana1980}の概念である。これは、作動上は完全に独立(閉鎖)している複数のシステムが、互いに不可欠な環境として影響し合う関係を指す。

ルーマン理論において、最も重要な構造的カップリングは、「心的システム(意識)」と「社会システム(コミュニケーション)」の関係である。心的システムは思考し、社会システムはコミュニケーションする——それぞれ別の作動を行う。しかし、心的システムがなければコミュニケーションは発生しない。意識はコミュニケーションに対して、刺激や誘発を与えたり、あるいは邪魔をしたりすることができる。ただし、意識がコミュニケーションを「因果的に決定」するわけではない。意識はあくまで環境として、コミュニケーション・システムに「刺激」を与え、システム側がそれを独自の論理で処理する\footnote{ここで注意すべきは、ルーマン理論において「認知」は心的システム内部に所在し、社会的システムに「分布」しているわけではないことである。構造的カップリングは、社会的コミュニケーションが心的システムを刺激することを可能にするが、認知プロセスそのものは心的システムのオートポイエーシスとして完結する。}。

% -----------------------------------------------------------------------------------------------------------------------------------------
\subsection{「理解の創造性」の定義}

以上の理論的整理を踏まえ、本研究で論じる「理解の創造性」を以下のように定義する:

\begin{quote}
\textbf{理解の創造性}:心的システムと社会システムの構造的カップリングにおいて、心的システムから社会システムへの刺激として提供される、価値判断を伴う意味構成のプロセス。
\end{quote}

この定義は、ルーマンの厳密な意味での「理解」(コミュニケーション接続)とは区別される。本研究が着目するのは、コミュニケーション接続を駆動する「価値判断の源泉」としての心的システムの役割である。

この観点から、「わかり合えなさ」は単なる誤解ではなく、構造的カップリングにおいて各心的システムが独自の価値判断に基づいて意味を構成する結果として生じる、コミュニケーションの本質的な特徴として理解される。

この価値判断を伴う意味構成のプロセスこそが、人間固有の創造性であり、本研究が「執政の創造性」として定式化する対象である。

% -----------------------------------------------------------------------------------------------------------------------------------------
\subsection{AIの原理的限界}

ルーマンの理論において、機械はオートポイエーシス・システム(生命・意識・社会)とは区別される「非ポイエティック」な存在とされている\cite{luhmann1984}。AIシステムも基本的には計算機プログラム(機械)上で動作するシステムである。AI——その最先端の形態を含めて——は心的システムに該当しない。

\textbf{オートポイエティックでない}:たとえ自己学習能力を持つAIであっても、その「学習」は人間が設計したアルゴリズムと訓練データに依存している。外部からの入力なしに自律的に自身の「思考」を産出し続ける閉鎖的なネットワークを持たない。

\textbf{自己言及的な意味処理を行わない}:AIの確率的出力は、文脈に応じて「もっともらしい」次のトークンを選択するが、この選択プロセスは自己言及的ではない。AIは「この意味を選択したこと自体」を次の処理の地平として開くことはなく、単に統計的パターンに基づいて出力を生成する。

\textbf{価値判断を伴わない}:心的システムからの刺激は「何が重要か」「何を優先すべきか」という規範的判断を前提とするが、AIの出力は統計的パターンに基づいており、独自の規範的判断を伴わない。

以上の議論から、公共政策におけるAIの原理的限界が明らかになった。AIは「情報」の選択や「伝達」の補助には機能しうるが、「理解」の創造的プロセスには原理的に参加できない。AIは心的システムを持たないため、価値判断を伴う意味構成を行えないのである。

% -----------------------------------------------------------------------------------------------------------------------------------------
\section{Human-AI Policy:政策形成におけるAIと人間の関係性}
\label{sec:ch2_human_ai_policy}

% -----------------------------------------------------------------------------------------------------------------------------------------
\subsection{AI政策論の展開}

公共政策におけるAI活用に関する議論は、2010年代後半から急速に発展してきた。初期の議論は、AIによる行政サービスの効率化や自動化に焦点が置かれていたが、近年ではAIと人間の関係性そのものが問いの中心となっている\cite{shneiderman2022}。

この転換の背景には、生成AI(ChatGPT、Claude、Gemini等)の登場がある。これらの技術は、従来のAI(分類・予測)とは異なり、創発的なテキスト生成能力を持つ。この能力は、政策文書の作成、選択肢の生成、市民との対話など、政策形成の核となるプロセスに直接関与しうる。

% -----------------------------------------------------------------------------------------------------------------------------------------
\subsection{Human-Centered AIの理念}

Shneiderman (2022) \cite{shneiderman2022} は、Human-Centered AI(HCAI)の理念として、以下の2軸マトリクスを提示している:

\begin{description}
    \item[高自動化・低制御] AIが自律的に判断し、人間は結果を受け入れるのみ
    \item[高自動化・高制御] AIが提案を行い、人間が最終判断を下す
    \item[低自動化・高制御] 人間が主導し、AIが補助的な役割を果たす
    \item[低自動化・低制御] 人間もAIも十分に機能しない状態
\end{description}

本研究が着目するのは「高自動化・高制御」の領域である。この領域では、AIの計算能力と人間の規範判断力が相互に補完し合う。

% -----------------------------------------------------------------------------------------------------------------------------------------
\subsection{AIの「杖」としての位置づけ}

生成AIを「杖(Aaron's rod)」として位置づける視点は、AIが人間を代替するのではなく、人間の能力を補完・増幅する道具として活用する考え方である。

この視点からは、以下の設計原則が導かれる:
\begin{enumerate}
    \item AIは人間の最終判断を前提とする
    \item AIの限界を明示的に理解する
    \item 人間-AI協調のプロセスを透明化する
    \item AI自体のバイアスに対処する
    \item 説明責任は常に人間が負う
\end{enumerate}

この「杖」としての位置づけは、AIを「執政の創造性」を支援する道具として捉え直す視点を提供する。

生成AIは、以下の領域では人間を補完し得る:
\begin{itemize}
    \item データの処理・分析
    \item 選択肢の生成・提示
    \item 文書作成の効率化
    \item 多様な視点の提示
    \item 認知バイアスの指摘(「悪魔の代理人」機能)
\end{itemize}

一方で、以下の領域では人間の役割が不可欠である:
\begin{itemize}
    \item 規範的判断(何が「善い」か)
    \item 文脈に応じた柔軟な対応
    \item 新たな価値の創造
    \item 政治的なアカウンタビリティ
    \item 最終的な責任の所在
\end{itemize}

% -----------------------------------------------------------------------------------------------------------------------------------------
\section{協調制御理論と社会システムへの応用}
\label{sec:ch2_cooperative_control}

% -----------------------------------------------------------------------------------------------------------------------------------------
\subsection{協調制御理論の基礎}

協調制御理論(Cooperative Control Theory)は、複数の自律エージェントが協調して共通の目標を達成するための制御手法を研究する分野である\cite{yoshihara2009cooperative}。

この理論は、以下の特徴を持つ:
\begin{itemize}
    \item 分散的な意思決定
    \item 局所的な情報に基づく協調
    \item 全体的な目標の達成
\end{itemize}

% -----------------------------------------------------------------------------------------------------------------------------------------
\subsection{社会システムへの応用可能性}

協調制御理論は、社会システムの分析にも応用可能である。特に、複数のステークホルダーが関与する政策ネットワークにおいて、各主体が自律的に行動しながら全体としての政策目標を達成するプロセスをモデル化できる。

本研究では、協調ロボット制御モデルを用いて、政策ネットワークにおけるステークホルダー間の協調を分析する(第4章で詳述)。

% -----------------------------------------------------------------------------------------------------------------------------------------
\section{協調的ガバナンス論}
\label{sec:ch2_collaborative_governance}

% -----------------------------------------------------------------------------------------------------------------------------------------
\subsection{協調的ガバナンスの定義}

Ansell and Gash (2008) \cite{ansell2008collaborative} は、協調的ガバナンスを以下のように定義している:

\begin{quote}
「一つまたは複数の公共機関が、非政府のステークホルダーを、合意形成志向で審議的な集団的意思決定プロセスに直接関与させる統治のあり方」
\end{quote}

% -----------------------------------------------------------------------------------------------------------------------------------------
\subsection{公共交通における連携・共創}

日本の公共交通政策においては、「連携」と「共創」が重要な概念として位置づけられている\cite{kato2009community}。Kato et al. (2009) は、コミュニティ参加型地域公共交通の成功条件として以下を指摘している:

\begin{enumerate}
    \item 関係ステークホルダー間での認識と責任分担の共有
    \item 各ステークホルダーが参加から利益を得られること
    \item ステークホルダーを調整するキーパーソンの存在
    \item ステークホルダーの努力が利用促進・価値向上につながること
\end{enumerate}

% -----------------------------------------------------------------------------------------------------------------------------------------
\subsection{実装ギャップの指摘}

しかし、こうした理論的条件にもかかわらず、実践レベルでは多くの課題が指摘されている。Emerson et al. (2012) \cite{emerson2012integrative} は、協調的ガバナンスが直面する課題として以下を指摘している:

\begin{itemize}
    \item 高い取引コスト
    \item 最小公約数的な解決策への収束
    \item 組織された利益による捕捉
\end{itemize}

% -----------------------------------------------------------------------------------------------------------------------------------------
\section{認知バイアスと意思決定}
\label{sec:ch2_cognitive_bias}

% -----------------------------------------------------------------------------------------------------------------------------------------
\subsection{行動経済学の基礎概念}

Kahneman (2011) \cite{kahneman2011thinking} は、人間の思考を「システム1(速い思考)」と「システム2(遅い思考)」に分類し、認知バイアスがシステム1の特性に起因することを示した。

% -----------------------------------------------------------------------------------------------------------------------------------------
\subsection{政策プロセスにおける認知バイアス}

政策形成において特に重要な認知バイアスとして、以下の三つを取り上げる:

\subsubsection{現状維持バイアス(Status Quo Bias)}
Samuelson and Zeckhauser (1988) \cite{samuelson1988status} によって提唱された概念で、変化よりも現状を維持することを好む傾向を指す。

\subsubsection{確証バイアス(Confirmation Bias)}
既存の信念や仮説を支持する情報を優先的に探し、反証する情報を無視・軽視する傾向\cite{russio2015confirmation}。

\subsubsection{狭い視野(Narrow Framing)}
問題を孤立して考え、より広い文脈や長期的な影響を考慮しない傾向\cite{kahneman2011thinking}。

% -----------------------------------------------------------------------------------------------------------------------------------------
\section{ZK-SNARKsと政策評価}
\label{sec:ch2_zksnarks}

% -----------------------------------------------------------------------------------------------------------------------------------------
\subsection{ZK-SNARKsの基本概念}

ZK-SNARKs(Zero-Knowledge Succinct Non-interactive Arguments of Knowledge)は、暗号技術の一種であり、秘密情報を公開することなく、その情報の正しさを証明する技術である\cite{ben2014succinct}。

ZK-SNARKsは以下の4つの特性を持つ:
\begin{description}
    \item[Zero-Knowledge] 証明を通して元の秘密情報が一切漏洩しない
    \item[Succinct] 証明サイズが常に数百バイト程度と一定
    \item[Non-interactive] 証明者から検証者への1回の送信で証明完了
    \item[Arguments of Knowledge] 真の知識を所有している必要があり偽造不可能
\end{description}

% -----------------------------------------------------------------------------------------------------------------------------------------
\subsection{政策評価への応用可能性}

ZK-SNARKsの概念を政策評価に応用することで、「秘密を守りながら専門性を証明する」仕組みが実現可能になる。例えば、企業が自社の技術情報を公開せずに、政策課題への貢献可能性を証明できる。

本研究では、ZK-SNARKsの概念を援用した政策評価システムをLLM as a Judgeと組み合わせて提案する(第5章で詳述)。

% -----------------------------------------------------------------------------------------------------------------------------------------
\section{小括:理論的空白の特定}
\label{sec:ch2_summary}

先行研究のレビューから、以下の理論的空白が明らかになった:

\begin{enumerate}
    \item \textbf{「執政の創造性」の理論化}:ルーマンの社会システム理論を用いて「執政の創造性」を基礎づけ、AIと人間の役割分担を明確にした研究は限定的である
    \item \textbf{認知バイアスと政策協調の接続}:協調的ガバナンスの失敗要因として認知バイアスに着目した研究は限定的である
    \item \textbf{計算論的分析手法の欠如}:政策協調プロセスを計算論的にモデル化した研究は少ない
    \item \textbf{ZK-SNARKs概念の政策評価への応用}:暗号技術の概念を政策評価に応用した試みは先駆的である
\end{enumerate}

本研究は、これらの空白を埋めることを目指す。

% -----------------------------------------------------------------------------------------------------------------------------------------
