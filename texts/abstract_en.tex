% -----------------------------------------------------------------------------------------------------------------------------------------
% Abstract (English)
% -----------------------------------------------------------------------------------------------------------------------------------------

\chapter*{Abstract}
\addcontentsline{toc}{chapter}{Abstract}

This study explores the collaborative relationship between generative AI and humans, using public transport policy as an empirical context.

Chapter 1 discusses the research background, problem statement, and objectives. It points out that the implementation gap in "collaboration and co-creation" in public transport policy may be attributable not only to institutional deficiencies but also to human cognitive biases.

Chapter 2 reviews prior research on Human-AI Policy, collaborative governance theory, cognitive bias and decision-making, and ZK-SNARKs in policy evaluation, identifying theoretical gaps.

Chapter 3 examines the evolution of Japanese public transport policy and current institutional design, revealing the reality of implementation gaps through empirical analysis of 38 Japan MaaS projects.

Chapter 4 employs computational analysis using cooperative robot control models to uncover the threshold effect of status quo bias, the paradoxical effect of confirmation bias, and the consistently negative impact of narrow framing.

Chapter 5 positions generative AI as a "staff" that complements human "executive creativity," proposing a policy evaluation system utilizing ZK-SNARKs concepts. It designs a three-layer architecture combining Constitutional AI, citizen deliberation, and LLM as a Judge.

Chapter 6 derives strategies for addressing cognitive biases and implications for institutional design incorporating generative AI.

Chapter 7 summarizes the research and discusses future directions for urban planning applications.

This study provides theoretically and empirically grounded guidelines for better policy formation by exploring the relationship between generative AI and humans.

\vspace{1cm}
\noindent
\textbf{Keywords}: Generative AI, Policy Formation, Cognitive Bias, Public Transport Policy, ZK-SNARKs, Institutional Design

% -----------------------------------------------------------------------------------------------------------------------------------------
