% -----------------------------------------------------------------------------------------------------------------------------------------
% 要旨(日本語)
% -----------------------------------------------------------------------------------------------------------------------------------------

\chapter*{要旨}
\addcontentsline{toc}{chapter}{要旨}

本研究は、公共交通政策を舞台に、生成AIと人間の協調的関係性を探求したものである。

第1章では、研究の背景、問題の所在、研究目的を論じた。公共交通政策における「連携・共創」の実装ギャップが、制度的欠陥だけでなく、人間の認知バイアスに起因する可能性を指摘した。

第2章では、Human-AI Policy、協調的ガバナンス論、認知バイアスと意思決定、ZK-SNARKsと政策評価に関する先行研究をレビューし、理論的空白を特定した。

第3章では、日本の公共交通政策の変遷と制度設計の現状を整理し、Japan MaaS 38プロジェクトの実証分析を通じて実装ギャップの実態を明らかにした。

第4章では、協調ロボット制御モデルを用いた計算論的分析により、現状維持バイアスの閾値効果、確証バイアスの逆説的効果、狭い視野の一貫した負の影響を解明した。

第5章では、生成AIを人間の「執政の創造性」を補完する「杖」として位置づけ、ZK-SNARKsの概念を援用した政策評価システムを提案した。Constitutional AI、市民討議、LLM as a Judgeを組み合わせた三層アーキテクチャを設計した。

第6章では、認知バイアスへの対処戦略と生成AIを組み込んだ制度設計への示唆を導出した。

第7章では、研究の総括と、都市計画への展開について論じた。

本研究は、生成AIと人間の関係性を理論的・実証的に探求し、より良い政策形成のための指針を提供した。

\vspace{1cm}
\noindent
\textbf{キーワード}:生成AI、政策形成、認知バイアス、公共交通政策、ZK-SNARKs、制度設計

% -----------------------------------------------------------------------------------------------------------------------------------------
