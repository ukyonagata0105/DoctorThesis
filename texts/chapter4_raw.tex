\section{Re-designing ``Collaboration'' and ``Co-creation'' in Regional
Public Transport
Policy:}\label{re-designing-collaboration-and-co-creation-in-regional-public-transport-policy}

\subsection{Integrated Approach to Cognitive Biases and Institutional
Coordination}\label{integrated-approach-to-cognitive-biases-and-institutional-coordination}

\subsection{List of Figures and
Tables}\label{list-of-figures-and-tables}

\begin{itemize}
\tightlist
\item
  Table 1: Symbol-Concept Correspondence Table
\item
  Table 2: Comparative Analysis of Institutional Design Types
\item
  Figure 2: Constructive Effect of Confirmation Bias
\item
  Figure 3: Threshold Effect of Status Quo Bias
\item
  Figure 4: Linear Degradation of Narrow Framing Bias
\item
  Figure 5: Comprehensive Bias Effects Analysis
\item
  Figure 6: Bias Model Comparison
\item
  Figure 7: Realistic Performance Analysis with Confidence Intervals
\item
  Figure 8: Joint 0 (Base) Bias Effects
\item
  Figure 9: Joint 7 (End-Effector) Bias Effects
\item
  Figure 10: Summary of All Bias Effects
\item
  Figure 11: Three-Tier Institutional Design for Effective Collaboration
\item
  Figure 12: Robot Arm Animation Scenarios for Policy Analysis
\item
  Figure 13: Simulation Process Flow for Stakeholder Coordination
\item
  Table 3: Statistical Summary of Cognitive Bias Effects
\end{itemize}

\subsection{Abstract}\label{abstract}

This study develops a computational modeling framework to analyze
cognitive bias effects in collaborative transport policy implementation
and proposes institutional design principles for effective stakeholder
coordination. Using cooperative robot control theory, this study models
stakeholder interactions in policy networks where each agent represents
a policy stakeholder with specific capabilities and positions. Through
systematic simulation experiments, this research analyzes how three key
cognitive biases---confirmation bias, status quo bias, and narrow
framing bias---affect coordination effectiveness in collaborative
governance. The computational analysis reveals that confirmation bias
can paradoxically enhance coordination when properly leveraged, while
status quo bias creates threshold effects requiring careful
institutional management. Narrow framing bias consistently degrades
system performance through reduced global optimization. Based on these
findings and analysis of implementation gaps in existing collaborative
frameworks, this study proposes a three-tier institutional architecture
that separates political-administrative and administrative-business
interfaces to optimize both democratic accountability and operational
efficiency. This study contributes to policy science by providing
theoretically grounded institutional design principles derived from
computational analysis of cognitive mechanisms underlying stakeholder
coordination in public transport governance.

\textbf{Keywords}: Public transport policy, Collaboration, Co-creation,
Stakeholder coordination, Institutional design, Cognitive bias, Japan
MaaS

\subsection{1. Introduction}\label{introduction}

\subsubsection{1.1 Research Background and Problem
Statement}\label{research-background-and-problem-statement}

Regional public transport policy in Japan has undergone significant
transformation since the 2002 deregulation, evolving from a primarily
regulatory approach to one emphasizing ``collaboration'' (連携) and
``co-creation'' (共創) among diverse stakeholders. This shift reflects
broader trends in public governance toward collaborative approaches that
seek to harness the expertise and resources of multiple
actors---government agencies, transport operators, citizens, and other
businesses---to address complex policy challenges.

The institutionalization of this transformation has been further
accelerated by recent policy developments. The formulation of regional
public transport plans has become a mandatory effort (国土交通省2022:1
地域公共交通計画と乗合バス等の補助制度の連動化に関する解説パンフレット),
with specific guidance emphasizing the establishment of conference
bodies to facilitate efficient information exchange and consultation
among stakeholders (国土交通省2025: 5
地域公共交通計画等の作成と運用の手引き 第4版 理念編). This institutional
framework provides the formal structure within which collaboration and
co-creation processes must operate.

The policy discourse around collaboration and co-creation gained
particular prominence following the 2021 policy framework ``Research
Group on Co-creation of Regional Transport for the Post-Corona Era'' by
the Ministry of Land, Infrastructure, Transport and Tourism (MLIT). This
framework explicitly calls for ``co-creative transport'' involving three
key dimensions: transport operators actively generating human flows to
revitalize regional communities, cross-sector collaboration between
transport operators and other industries, and community engagement that
treats transport as a shared responsibility.

However, despite the policy emphasis on collaboration and co-creation,
empirical evidence suggests significant gaps between policy intentions
and implementation outcomes. The proliferation of collaborative
frameworks has not necessarily translated into more effective or
sustainable transport solutions, raising fundamental questions about the
conditions under which collaboration and co-creation can function
effectively in public transport governance.

This implementation gap is further exacerbated by the structural
resource constraints facing local governments in Japan. The 2023 report
by the Council of Experts on Decentralization Reform specifically
identified an ``inverted triangle'' burden structure where municipal
governments face disproportionately higher implementation costs compared
to prefectural and national levels. The report documented several
critical resource pressures: ``In some cases, a single staff member
handles multiple ministry operations, and multiple plans are formulated
within one department. In such cases, there is a sense of duplication in
the content of these plans and work becomes concentrated due to
overlapping planning periods''; ``Due to resource shortages, sufficient
time cannot be secured for planning-related administrative work, leading
to cases where national templates or examples from other municipalities
are almost entirely followed to secure funding tied to plans and avoid
being publicly identified as non-implementing entities''; and ``Even
when plan formulation itself is a best-effort obligation and formulation
procedures are left to local government discretion, considerable
procedural costs such as deliberation council reviews occur to obtain
regional consensus'' (Council of Experts on Decentralization Reform,
2023: 9-12). These structural constraints create a context where
collaborative and co-creative approaches, despite their theoretical
appeal, may be undermined by practical implementation realities.

\subsubsection{1.2 Policy Context and
Evolution}\label{policy-context-and-evolution}

Japanese regional public transport policy has evolved through distinct
phases, each reflecting different approaches to the balance between
market mechanisms and public intervention:

\textbf{2002 Deregulation}: Shifted route operations from
permission-based to notification-based system and changed subsidy
allocation from company-wide to route-specific support, introducing
market competition while maintaining essential services through targeted
subsidies.

\textbf{2006 Regional Public Transport Conferences}: Established formal
collaborative governance mechanisms, enabling community bus operations
and providing platforms for stakeholder participation in transport
planning.

\textbf{2010 Life Transport Survival Project}: Institutionalized a
competitive approach where only ``striving regions'' were permitted to
maintain living transport services, emphasizing local initiative and
self-reliance.

\textbf{2021 Co-creation Framework}: Explicitly embraced co-creation as
the central organizing principle, encompassing resource mobilization
(``total mobilization of transport resources''), joint management
(direct operator collaboration), and service optimization (``bundle and
reduce'' approaches).

\subsubsection{1.3 Empirical Evidence of Implementation
Gaps}\label{empirical-evidence-of-implementation-gaps}

To assess the practical implementation of collaboration and co-creation
principles, this study conducted comprehensive analysis of Japan's MaaS
(Mobility as a Service) initiative, which represents a concrete
manifestation of co-creation policy. Analysis of all 38 projects
approved under the FY2020 Japan MaaS initiative reveals significant
discrepancies between policy ideals and implementation reality.

Regarding business performance indicators, 92\% of projects (35/38)
included metrics focused on usage rates, revenue generation, or
operational efficiency. In stark contrast, for non-business indicators,
only 29\% of projects (11/38) incorporated indicators addressing social
impact, accessibility improvement, or environmental outcomes. Most
concerning was the level of citizen participation mechanisms, where
merely 5\% of projects (2/38) established meaningful mechanisms for
citizen participation in project governance and evaluation.

These findings indicate that collaborative frameworks may be captured by
transport operator interests rather than serving broader public
purposes, with minimal achievement of genuine democratic participation.

\subsubsection{1.4 Research Objectives and
Approach}\label{research-objectives-and-approach}

This study addresses three primary research questions. First, the policy
implementation gap is examined by investigating to what extent
collaborative transport policies achieve their stated objectives of
moving beyond business-focused metrics to incorporate broader social and
democratic values. Second, coordination mechanisms are analyzed by
exploring how cognitive biases and stakeholder characteristics affect
the effectiveness of collaborative coordination in transport policy
implementation. Third, institutional design is addressed by determining
what institutional arrangements can optimize the benefits of
collaboration while mitigating its inherent limitations and coordination
failures.

This research employs an integrated approach combining literature review
of empirical policy evaluation (Nagata, 2024), computational modeling
using cooperative robot control theory, and institutional design theory.
This approach contributes to policy science by synthesizing existing
empirical evidence of implementation gaps with theoretical insights into
how cognitive biases affect stakeholder coordination effectiveness.

\subsection{2. Theoretical Framework and Literature
Review}\label{theoretical-framework-and-literature-review}

\subsubsection{2.1 Collaborative Governance
Theory}\label{collaborative-governance-theory}

Collaborative governance has emerged as a dominant paradigm in public
administration, defined by Ansell and Gash (2008) as ``a governing
arrangement where one or more public agencies directly engage non-state
stakeholders in a collective decision-making process that is formal,
consensus-oriented, and deliberative and that aims to make or implement
public policy or manage public programs or assets.''

In the context of transport policy, collaborative approaches have been
promoted as solutions to the limitations of traditional regulatory
frameworks. Kato et al.~(2009) identify four critical conditions for
successful community-participatory public transport: shared recognition
and responsibility distribution among stakeholders, mutual benefit for
all participants, presence of key coordinators, and connection between
stakeholder efforts and service improvement outcomes.

However, collaborative governance faces inherent challenges. Emerson et
al.~(2012) note that collaboration requires significant transaction
costs, may lead to lowest-common-denominator solutions, and can be
captured by well-organized interests. These challenges are particularly
acute in transport policy, where technical complexity, regulatory
constraints, and commercial interests create additional coordination
difficulties.

\subsubsection{2.2 Co-creation in Public Service
Delivery}\label{co-creation-in-public-service-delivery}

Co-creation represents an evolution beyond traditional collaboration,
emphasizing joint value creation between public agencies and
stakeholders (Voorberg et al., 2015). In transport policy, co-creation
manifests in several forms. Resource mobilization involves utilizing
non-traditional transport resources such as vehicles, drivers, and
infrastructure from diverse sources. Joint operations encompass formal
partnerships between transport operators through joint ventures and
shared services that enable greater coordination. Service integration
focuses on bundling transport with other services to create
comprehensive mobility solutions that address multiple user needs
simultaneously.

The Japanese policy framework identifies three dimensions of
co-creation: transport operators actively generating community
vitalization, cross-sector collaboration for problem-solving, and
community ownership of transport services. However, empirical studies
reveal mixed results. The Kyushu Regional Transport Bureau (2023) found
that while co-creation enables previously impossible route
reorganizations and improves service efficiency, business performance
improvements remain limited, and building necessary trust relationships
requires significant time investment.

\subsubsection{2.3 Empirical Evidence on Collaboration and Co-creation
Effects}\label{empirical-evidence-on-collaboration-and-co-creation-effects}

\paragraph{2.3.1 Collaboration Effects}\label{collaboration-effects}

Kato et al.~(2009) analyzed community-participatory regional public
transport and identified four existence requirements, though they did
not address the elements needed to realize these conditions. These
requirements include ensuring that related stakeholders share
recognition and responsibility distribution, that each stakeholder can
benefit from this participation, that key persons exist to coordinate
stakeholders, and that stakeholder efforts connect to usage promotion
and value enhancement.

Kita (2006) positively evaluated the provision of collaboration and
cooperation venues in comprehensive coordination plan formulation from a
technical personnel development perspective, noting that ``providing
venues and opportunities for various transport-related stakeholders to
collaborate while receiving support from researchers and technical
experts to improve planning skills\ldots{} is precisely the policy that
the state should implement.''

\paragraph{2.3.2 Co-creation Effects}\label{co-creation-effects}

The Kyushu Regional Transport Bureau (2023) conducted comprehensive
interviews and analysis of co-creation across Kyushu regions,
identifying both effects and challenges.

Regarding co-creation effects, the analysis found that previously
unimplementable route reorganizations became feasible, and both
convenience maintenance/improvement and transport efficiency improvement
were achieved simultaneously. However, significant co-creation
challenges also emerged. Business performance improvements remain
limited, and simply appealing transport operators' difficulties cannot
enable municipal support. Building trust relationships necessary for
collaboration requires considerable time investment, and few
opportunities exist for discussing role distribution, cost burden, and
community development between transport operators and other private
businesses and administration.

Nomura (2023) analyzed regional transport security through Limited
Liability Partnership (LLP) in Ichinohe Town, where municipalities and
transport operators participate as parallel LLP members, reducing
registration burden and responsibility distribution costs, positioning
this as a co-creation example.

Yoshida (2021) implemented inter-operator collaboration in Hachinohe
City (2007), reducing 200 services while achieving equal intervals,
resulting in increased ridership and profitability.

\paragraph{2.3.3 Regional Transport Re-design and Sustainability
Dimensions}\label{regional-transport-re-design-and-sustainability-dimensions}

The Japanese regional public transport policy framework emphasizes
``regional transport re-design'' as a comprehensive approach to
rebuilding sustainable transport systems. This approach rests on three
pillars: ``three co-creation'' dimensions (public-private co-creation,
inter-operator co-creation, and cross-sector co-creation), ``transport
DX'' implementing digital technologies such as autonomous driving and
MaaS, and ``transport GX'' focusing on vehicle electrification and local
renewable energy consumption (MLIT, 2024). The objective is to enhance
convenience, sustainability, and productivity of regional public
transport through collaboration and cooperation among regional
stakeholders.

Sustainability, as articulated in the United Nations 2030 Agenda for
Sustainable Development (UN, 2015), necessitates a balanced approach
across three dimensions: economic, social, and environmental. The Agenda
explicitly commits to achieving sustainable development in all three
dimensions ``in a balanced and integrated manner,'' recognizing that
eradicating poverty in all its forms requires addressing economic
viability alongside social inclusion and environmental protection.

This tripartite framework of sustainability has critical implications
for understanding transport collaboration. Purely economic models of
collaboration, focusing solely on cost-sharing and revenue optimization,
prove insufficient to explain the full spectrum of collaborative
behaviors observed in regional transport contexts. Economic motivations
alone cannot account for collaborations undertaken primarily to maintain
social connectivity in depopulated areas, to reduce environmental impact
through electrification, or to preserve community identity through
transport services. The integration of economic rationality with social
necessity and environmental responsibility creates a more comprehensive
explanatory framework for collaborative transport governance.

\subsubsection{2.4 Cooperative Control Theory and Social
Systems}\label{cooperative-control-theory-and-social-systems}

\textbf{Introduction}

Cooperative control theory provides a framework for coordinating
multiple autonomous agents to achieve shared objectives, forming the
backbone of multi-agent systems (MAS). This theory has evolved to
address complex challenges in automation, robotics, and increasingly,
the analysis of social systems and collaborative governance.

\textbf{Core Concepts and Mechanisms}

Cooperative control theory addresses several key problems in multi-agent
coordination. Consensus mechanisms ensure all agents agree on certain
variables or states, which is fundamental for group coordination in MAS
(Gulzar et al., 2018; Ying et al., 2022). Formation and containment
problems involve arranging agents in specific patterns or ensuring they
remain within certain boundaries (Ying et al., 2022; Anand et al., 2024;
Briñón-Arranz et al., 2014). Resource allocation and coverage focus on
distributing tasks or resources efficiently among agents (He et al.,
2023; Ying et al., 2022), while flocking and connectivity problems
address the maintenance of group cohesion and communication links (Ying
et al., 2022).

\emph{Theoretical Foundations:} Cooperative control leverages graph
theory to model agent interactions, Lyapunov theory for stability, and
distributed optimization for performance (Hengster-Movrić \& Lewis,
2014; Anand et al., 2024; Briñón-Arranz et al., 2014). Algorithms are
designed to work with only local information, enabling scalability and
robustness in large systems (Hengster-Movrić \& Lewis, 2014; Yang et
al., 2023).

\textbf{Methodologies and Algorithmic Advances}

The field has developed sophisticated methodological approaches to
multi-agent coordination. Distributed control protocols enable agents to
make decisions based on local neighbor information while supporting
global objectives without centralized oversight (Hengster-Movrić \&
Lewis, 2014; Anand et al., 2024). Adaptive and learning-based control
methods increasingly utilize neural networks and reinforcement learning
to handle uncertainties and dynamic environments, allowing agents to
adapt and optimize their behavior collaboratively (Shi \& Shen, 2015;
Lan et al., 2023). Game theory integration has been applied in scenarios
such as traffic signal control, where agents must balance individual and
collective goals, thereby enhancing decision-making in dynamic,
competitive environments (Abdoos, 2020).

\textbf{Extensions to Social Systems and Collaborative Governance}

Cooperative control theory has been extended to various social and
governance applications. Urban traffic management represents one
successful application area where multi-agent cooperative control,
combined with game theory and reinforcement learning, has been applied
to optimize traffic signals across networks, reducing congestion and
improving flow through agent collaboration (Abdoos, 2020). Human-robot
and human-agent collaboration applications demonstrate how cooperative
control underpins intelligent interaction and collaboration in mixed
human-machine teams, supporting decision-making and task allocation in
complex social environments (Yang et al., 2023). In collaborative
governance contexts, the principles of consensus, distributed
decision-making, and adaptive coordination in MAS are increasingly
informing models of collaborative governance, where multiple
stakeholders must align on shared policies or actions (He et al., 2023;
Yang et al., 2023).

\textbf{Challenges and Future Directions}

Several challenges remain in the advancement of cooperative control
theory. Scalability and complexity issues arise as systems grow, making
it increasingly difficult to ensure stability, optimality, and real-time
performance (He et al., 2023; Yang et al., 2023). Uncertainty and
nonlinearity present ongoing challenges in handling unknown,
state-dependent effects and nonlinear dynamics, though adaptive and
learning-based methods show promise in addressing these issues (Shi \&
Shen, 2015; Lan et al., 2023). Integration with social systems
represents a particularly complex challenge, as translating cooperative
control principles to human-centric domains requires addressing issues
of trust, communication, and heterogeneous agent capabilities (Yang et
al., 2023; Abdoos, 2020).

\textbf{Summary} Cooperative control theory is central to the
coordination of multi-agent systems, addressing problems like consensus,
formation, and resource allocation through distributed, adaptive, and
learning-based methods. Its extension to social systems and
collaborative governance leverages these principles to manage complex,
dynamic, and decentralized environments, with ongoing research focused
on scalability, uncertainty, and integration with human factors.

\subsubsection{2.5 Cognitive Bias Modeling in Computational Social
Science}\label{cognitive-bias-modeling-in-computational-social-science}

\paragraph{2.5.1 Introduction to Computational Models of Cognitive
Bias}\label{introduction-to-computational-models-of-cognitive-bias}

Computational social science has developed sophisticated mathematical
frameworks for modeling cognitive biases in multi-agent systems. These
models provide formal representations of how cognitive limitations and
biases emerge from, influence, and persist within social networks and
decision-making processes. The study of cognitive biases through
computational models offers several advantages for understanding
collaborative governance: explicit mathematical specification of bias
mechanisms, systematic analysis of bias effects across multiple agents,
ability to conduct controlled experiments that would be impossible in
real-world settings, and formal convergence and stability analysis of
biased systems.

Understanding these computational models is essential for translating
cognitive bias concepts into the cooperative control framework used in
this study. The following sections review key modeling approaches for
three cognitive biases central to collaborative transport policy:
confirmation bias, status quo bias, and narrow framing or bounded
rationality.

\paragraph{2.5.2 Confirmation Bias
Models}\label{confirmation-bias-models}

Confirmation bias---the tendency to seek, interpret, and remember
information that confirms pre-existing beliefs---has been extensively
modeled in computational social science using several mathematical
frameworks.

\textbf{Asymmetric Learning Rate Models}

One prominent approach models confirmation bias through asymmetric
update weights in learning algorithms. In Q-learning and reinforcement
learning frameworks, agents apply different learning rates for
confirming versus disconfirming evidence. Specifically, when an agent
receives a reward \(r\) for action \(a\), the value update follows:

\[Q_i(a,t+1) = \begin{cases}
Q_i(a,t) + \alpha^+(r - Q_i(a,t)) & \text{if } r > Q_i(a,t) \text{ (confirming)} \\
Q_i(a,t) + \alpha^-(r - Q_i(a,t)) & \text{if } r \leq Q_i(a,t) \text{ (disconfirming)}
\end{cases}\]

where \(\alpha^+ > \alpha^-\) represents confirmation bias through
stronger updating from positive prediction errors (Bergerot et al.,
2024; Arehart et al., 2025). This asymmetric updating captures the
psychological mechanism that confirming information is given more weight
than disconfirming information in belief revision.

\textbf{DeGroot-Style Weighted Averaging}

The DeGroot model of opinion dynamics has been extensively adapted to
model confirmation bias through similarity-dependent weighting. In the
basic formulation, each agent \(i\) updates its opinion \(x_i\) as a
weighted average of neighbors' opinions:

\[x_i(t+1) = \sum_{j} w_{ij}(x_i(t), x_j(t)) \cdot x_j(t)\]

where the weight function \(w_{ij}\) decreases as the distance
\(|x_i - x_j|\) increases. When this dependence is strong
enough---specifically, when \(w_{ij} = 0\) for
\(|x_i - x_j| > \varepsilon\)---the model implements bounded confidence
(Banisch et al., 2024; Dong et al., 2024; Paz et al., 2024), a formal
representation of confirmation bias where agents only accept influence
from sufficiently similar others.

\textbf{Friedkin-Johnsen Extensions}

The Friedkin-Johnsen model extends opinion dynamics by incorporating
both social influence and internal anchoring, providing a rich framework
for modeling multiple biases simultaneously:

\[x_i(t+1) = \lambda_i \sum_{j} W_{ij} x_j(t) + (1 - \lambda_i) s_i\]

where \(\lambda_i \in [0,1]\) represents the agent's susceptibility to
social influence versus adherence to their internal anchor \(s_i\). The
anchoring parameter \((1 - \lambda_i)\) models status quo bias or
stubbornness, while the weight matrix \(W\) can encode confirmation bias
through distance-dependent influence. For multidimensional opinions
\(x_i \in \mathbb{R}^d\), this vector formulation allows for complex
interactions across multiple belief dimensions.

\textbf{State-Dependent Confirmation Weights}

More sophisticated models employ nonlinear, state-dependent weight
functions \(g(\|x_i - x_j\|)\) that map opinion distance to influence
strength. Both affine (linear) and nonlinear forms of \(g(\cdot)\) have
been analyzed, with convergence conditions derived for each. The bounded
confidence model emerges as a special case with a step-function kernel,
while smoothly decaying kernels represent graded confirmation bias.

\paragraph{2.5.3 Status Quo Bias Models}\label{status-quo-bias-models}

Status quo bias---the preference for maintaining current states---has
been modeled through several complementary formalisms in computational
social science.

\textbf{Bounded Confidence as Acceptance Threshold}

The bounded confidence rule can be interpreted as modeling resistance to
change: agents accept influence only if the proposed change is within a
threshold \(\varepsilon\) of their current position. Formally,
interaction occurs only when:

\[|x_i - x_j| \leq \varepsilon\]

This creates a hard resistance threshold against changes beyond
\(\varepsilon\), functioning as a direct mathematical representation of
status quo bias in social influence processes (Steiglechner et al.,
2024).

\textbf{Anchoring Parameter as Continuous Resistance}

In the Friedkin-Johnsen framework, the anchoring parameter \(\lambda_i\)
acts as a continuous measure of status quo bias. Smaller values of
\(\lambda_i\) yield greater status quo bias and resistance to change:

\[x_i(t+1) = \lambda_i \cdot \text{(social influence)} + (1 - \lambda_i) \cdot s_i\]

When \(\lambda_i \approx 0\), agents maintain their initial positions
regardless of social influence, representing extreme status quo bias.
When \(\lambda_i \approx 1\), agents are fully open to social influence
with no status quo preference.

\textbf{Hybrid Threshold Formulations}

Instead of hard cutoffs, some models use discount functions \(g(d)\)
that smoothly reduce weight with distance, representing graded
resistance where influence is attenuated beyond a soft threshold. These
hybrid formulations capture the observation that status quo resistance
is often not absolute but rather a continuous function of the magnitude
of proposed change.

\paragraph{2.5.4 Narrow Framing and Bounded Rationality
Models}\label{narrow-framing-and-bounded-rationality-models}

Narrow framing---focusing on limited subsets of available information or
outcomes---maps closely to bounded rationality models in computational
social science.

\textbf{Information Filtering}

Bounded confidence and state-dependent weighting models can be
interpreted as implementing narrow framing through selective attention
to information. Agents effectively filter out information from sources
deemed too dissimilar, reducing their consideration set to a local
neighborhood of similar opinions. This information filtering mechanism
represents a formal model of narrow framing in social learning.

\textbf{Local vs.~Global Optimization}

In cooperative control settings, narrow framing can be modeled as
restricting agents to optimize over local objectives rather than global
system performance. This maps to the distinction between local control
policies that consider only immediate neighbor states versus global
optimization that requires system-wide information. The trade-off
between local efficiency and global optimality captures key aspects of
bounded rationality in distributed systems.

\textbf{Implementation in Multi-Agent Systems}

Computational models of bounded rationality typically involve
restricting agents to: finite or limited observation of the full system
state, local reward signals rather than global performance metrics, and
heuristics or satisficing rather than full optimization (Bohren \&
Imbens, 2017; Mao \& Hovakimyan, 2021). These restrictions create
computationally tractable decision problems that mirror the cognitive
limitations underlying narrow framing in human decision-making.

\paragraph{2.5.5 Positioning Relative to Existing
Methods}\label{positioning-relative-to-existing-methods}

The computational modeling approach to cognitive biases adopted in this
study differs from several established methodologies in computational
social science.

\textbf{Comparison with Agent-Based Modeling (ABM)}

Traditional agent-based models of opinion dynamics typically focus on
the emergence of consensus, polarization, or clustering through repeated
local interactions. These models emphasize pattern formation at the
population level and often employ simple heuristic rules. In contrast,
the cooperative control framework used in this study: provides
continuous-time dynamics rather than discrete update steps, explicitly
models task-oriented coordination toward specified goals, incorporates
physical constraints and kinematic relationships between agents, and
enables direct optimization of performance metrics.

The ABM approach excels at exploring emergent phenomena in social
systems, while cooperative control provides stronger links to
engineering applications and formal stability analysis. This study
leverages cooperative control precisely because of its ability to model
both task-oriented performance and social influence processes
simultaneously.

\textbf{Comparison with System Dynamics}

System dynamics modeling emphasizes feedback loops, stocks and flows,
and aggregate behavior over time. While powerful for understanding
macro-level patterns, system dynamics typically: abstracts away from
individual agent heterogeneity, uses aggregate variables rather than
agent-level states, and focuses on generic feedback structures rather
than specific coordination mechanisms.

The cooperative control framework complements system dynamics by
maintaining explicit agent-level modeling while still enabling analysis
of system-level patterns through the emergent behavior of the
coordinated system.

\textbf{Uniqueness of Cooperative Control Approach}

This study's use of cooperative control theory to model cognitive biases
in collaborative governance offers several unique advantages: the
framework naturally accommodates heterogeneous agents with different
capabilities and positions, explicit modeling of both local and global
objectives enables analysis of narrow framing effects, continuous-time
dynamics with formal stability properties support rigorous analysis, and
the mapping between physical system parameters and cognitive bias
parameters enables precise experimental control.

The cooperative control approach is particularly well-suited to studying
collaborative transport policy because: transport systems involve
physical infrastructure and operations that can be directly modeled,
stakeholders have distinct capabilities and constraints analogous to
link lengths and joint limits, and performance can be measured both at
the system level and component level. This creates a natural mapping
between the robotic domain and the policy domain that preserves the
essential structure of collaborative coordination while enabling formal
analysis of cognitive bias effects.

\subsubsection{2.6 Empirical Evidence of Implementation Gaps: Japan MaaS
Policy
Analysis}\label{empirical-evidence-of-implementation-gaps-japan-maas-policy-analysis}

To assess the practical implementation of collaboration and co-creation
principles, this study draws on recent empirical research on Japan's
MaaS (Mobility as a Service) initiative. Japan MaaS represents a
concrete manifestation of co-creation policy, explicitly designed to
``solve regional challenges through transport improvement'' and
requiring municipal involvement in the application process.

Japan MaaS is a public policy initiative that applies the MaaS
concept---integrating transport-related services to provide them as a
unified service---to address regional challenges. Unlike purely
commercial MaaS implementations, Japan MaaS adopts a ``regional
problem-solving through transport improvement'' stance, requiring
municipal involvement in the application process. The approach
incorporates cross-industry collaboration thinking, making it closely
aligned with ``co-creation'' principles.

Previous research by Nagata (2024) analyzing all 38 Japan MaaS projects
revealed significant implementation gaps in collaborative governance.
The analysis demonstrated that 92\% of projects focused primarily on
business indicators, only 29\% incorporated non-business indicators
addressing social impact or accessibility, and merely 5\% established
meaningful citizen participation mechanisms. These findings indicate
that collaborative frameworks may be captured by business interests
rather than serving broader public purposes, with minimal achievement of
genuine democratic participation.

This pattern reflects broader challenges in collaborative governance,
including organized interest capture, technical complexity barriers to
citizen participation, and institutional path dependence that constrains
innovation in governance approaches (Nagata, 2024). These findings
motivate the need for more sophisticated understanding of how
stakeholder coordination actually functions in practice, leading to the
computational modeling approach examining the cognitive mechanisms
underlying these coordination failures.

\subsubsection{2.7 Main Cognitive Biases in Collaborative Transport
Policy
Implementation}\label{main-cognitive-biases-in-collaborative-transport-policy-implementation}

\textbf{Key Cognitive Biases}

Several cognitive biases significantly influence collaborative transport
policy implementation. Loss aversion manifests when decision-makers and
stakeholders tend to value existing assets or policies more highly than
potential gains from new initiatives, leading to resistance to change
and preference for the status quo (Watkins \& Musselwhite, 2025). Status
quo bias reflects a systematic preference for maintaining current
conditions, making it difficult to implement innovative or disruptive
transport policies (Watkins \& Musselwhite, 2025). Risk perception
errors occur when misjudgments about the likelihood or impact of risks
skew policy priorities, often leading to overly cautious or misaligned
decisions (Watkins \& Musselwhite, 2025). Confirmation bias emerges when
stakeholders seek or interpret information in ways that confirm their
pre-existing beliefs, reducing openness to alternative perspectives or
evidence (Rastogi et al., 2020). Anchoring bias occurs when initial
information or assumptions disproportionately influence subsequent
judgments, even when new, more relevant data becomes available (Rastogi
et al., 2020). Additionally, case-specific, environmental, and
human-nature biases include those arising from specific data, context or
culture, and universal cognitive tendencies, all of which shape how
transport policies are evaluated and implemented (Watkins \&
Musselwhite, 2025).

\textbf{Mechanisms of Influence}

Cognitive biases operate through several distinct mechanisms in policy
settings. Information processing is affected by cognitive biases that
influence how stakeholders perceive, interpret, and prioritize
information, often leading to selective attention or misinterpretation
of data relevant to transport policy (Acciarini et al., 2020; Rastogi et
al., 2020; Watkins \& Musselwhite, 2025). Stakeholder interaction is
complicated by biases that can hinder the achievement of cognitive
consensus in multi-stakeholder settings, as different groups may frame
problems and solutions according to their own interests and mental
models (Knoppen et al., 2021). Decision-making dynamics are influenced
by biases such as loss aversion and status quo bias that can slow down
or block policy innovation, while anchoring and confirmation bias can
entrench initial positions, making collaborative negotiation and
adaptation more difficult (Rastogi et al., 2020; Watkins \& Musselwhite,
2025; Knoppen et al., 2021).

\textbf{Constructive and Detrimental Effects}

Cognitive biases demonstrate both detrimental and potentially
constructive effects in policy settings. Most cognitive biases are
detrimental in collaborative policy settings, leading to suboptimal
decisions, resistance to necessary change, and failure to achieve
consensus or system-level optimization (Watkins \& Musselwhite, 2025;
Knoppen et al., 2021). For example, loss aversion and status quo bias
can prevent the adoption of beneficial transport reforms (Watkins \&
Musselwhite, 2025). However, constructive effects can emerge when
structured stakeholder engagement and knowledge exchange help uncover
and address underlying assumptions, sometimes shifting priorities and
improving decision quality (Knoppen et al., 2021). Awareness and
mitigation strategies, such as education, training, and iterative
decision processes, can reduce the negative impact of biases and foster
more effective collaboration (Watkins \& Musselwhite, 2025; Knoppen et
al., 2021).

\textbf{Evidence from the Literature}

Empirical research provides evidence for both the challenges and
potential solutions related to cognitive biases in policy settings.
Studies demonstrate that structured, iterative stakeholder participation
can help overcome cognitive biases by promoting knowledge exchange and
revealing hidden assumptions, leading to more balanced and
system-oriented policy outcomes (Knoppen et al., 2021). Comprehensive
reviews highlight the prevalence and impact of biases such as loss
aversion, status quo bias, and risk perception errors in transport
policy, emphasizing the need for multidisciplinary approaches to
identify and mitigate these effects (Watkins \& Musselwhite, 2025).

Cognitive biases like loss aversion, status quo bias, and risk
perception errors significantly influence collaborative transport policy
implementation, often impeding innovation and consensus. However,
structured stakeholder engagement and awareness strategies can help
mitigate these effects and improve policy outcomes.

\subsubsection{2.8 Key Principles of Institutional Design for Effective
Collaborative Governance in Public Transport
Policy}\label{key-principles-of-institutional-design-for-effective-collaborative-governance-in-public-transport-policy}

\textbf{Theoretical Frameworks}

Several theoretical frameworks inform institutional design for
collaborative governance in public transport. Network-based
collaborative governance demonstrates that effective collaborative
governance often relies on network structures that facilitate
cross-sector and interorganizational collaboration, operating at
multiple levels---collaborative, financial, and interorganizational
governance---each with distinct structures, processes, and orchestrators
to address project complexity and stakeholder diversity (Hu et al.,
2024). Inclusiveness and interdependence considerations indicate that
institutional design should ensure broad stakeholder inclusion while
managing the degree of interdependence among participants, as inclusive
arrangements where participants are not overly interdependent tend to
empower stakeholders and enhance their perceived ability to influence
policy outcomes (Fossheim \& Andersen, 2022). Accountability frameworks
reveal that accountability in collaborative institutions is shaped by
the specificity of rules and the delegation of decision-making
authority, where more detailed rules can enhance process accountability,
while greater institutional independence can support legitimacy and
flexibility (Mancheva et al., 2023). Good governance principles
establish that transparency, accountability, stakeholder engagement, and
strong leadership are foundational for effective institutional
coordination and collaborative governance in public transport (Bouraima
et al., 2023; Adji et al., 2023).

\textbf{Empirical Findings}

Empirical research provides evidence for key principles of effective
institutional design. Communication and stakeholder engagement studies
show that regular dialogue, consultation, and inclusive participation
foster cooperation, trust, and a sense of ownership among stakeholders,
which are critical for integrated planning and effective policy
implementation (Adji et al., 2023; Paulsson et al., 2018). Research on
power to influence policy indicates that institutional arrangements that
are inclusive and reduce interdependence among participants provide
greater opportunities for stakeholders to influence policymaking,
leading to more innovative and accepted solutions (Fossheim \& Andersen,
2022). Studies on balancing efficiency, accountability, and coordination
reveal multiple dimensions of institutional effectiveness. Regarding
efficiency, streamlined frameworks and clear governance structures help
reduce fragmentation and improve the implementation of transport
initiatives (Bouraima et al., 2023; Hu et al., 2024). For
accountability, transparent processes and well-defined roles enhance
trust and ensure that collaborative arrangements remain answerable to
both participants and the public (Adji et al., 2023; Mancheva et al.,
2023). Concerning coordination, multi-level governance structures and
regular stakeholder engagement enable better alignment of objectives and
resources across diverse actors (Hu et al., 2024; Paulsson et al.,
2018).

\textbf{Challenges:} Lack of political will, corruption, inadequate
participation, and poor vision are major barriers to effective
institutional coordination. Implementing good governance principles is
identified as the top strategy to overcome these challenges (Bouraima et
al., 2023).

\textbf{Summary Table: Principles and Empirical Insights}

{\def\LTcaptype{none} % do not increment counter
\begin{longtable}[]{@{}
  >{\raggedright\arraybackslash}p{(\linewidth - 4\tabcolsep) * \real{0.1849}}
  >{\raggedright\arraybackslash}p{(\linewidth - 4\tabcolsep) * \real{0.3025}}
  >{\raggedright\arraybackslash}p{(\linewidth - 4\tabcolsep) * \real{0.5126}}@{}}
\toprule\noalign{}
\begin{minipage}[b]{\linewidth}\raggedright
Principle
\end{minipage} & \begin{minipage}[b]{\linewidth}\raggedright
Mechanism/Framework
\end{minipage} & \begin{minipage}[b]{\linewidth}\raggedright
Empirical Insight (Citation)
\end{minipage} \\
\midrule\noalign{}
\endhead
\bottomrule\noalign{}
\endlastfoot
Inclusiveness & Broad stakeholder participation & Increases power to
influence policy (Fossheim \& Andersen, 2022) \\
Accountability & Clear rules, transparency, delegation & Enhances trust
and legitimacy (Adji et al., 2023; Mancheva et al., 2023) \\
Coordination & Multi-level network governance & Improves alignment and
integration (Hu et al., 2024; Paulsson et al., 2018) \\
Efficiency & Streamlined, context-specific frameworks & Reduces
fragmentation, speeds implementation (Bouraima et al., 2023; Hu et al.,
2024) \\
Stakeholder Engagement & Regular dialogue and consultation & Builds
trust, ownership, and cooperation (Adji et al., 2023; Paulsson et al.,
2018) \\
\end{longtable}
}

Effective institutional design for collaborative governance in public
transport policy requires inclusive participation, clear accountability,
multi-level coordination, and strong stakeholder engagement, supported
by transparent and context-sensitive frameworks. These principles help
balance efficiency, accountability, and coordination among diverse
stakeholders.

\subsection{3. Computational Modeling
Framework}\label{computational-modeling-framework}

\subsubsection{3.1 Cooperative Control Model
Structure}\label{cooperative-control-model-structure}

This study adapts Yoshihara et al.'s (2009) cooperative robot control
theory to model stakeholder interactions in policy implementation. An
N-joint robot arm represents a policy network where each joint \(i\)
corresponds to a policy stakeholder with state variables:

\begin{itemize}
\tightlist
\item
  Position vector: \(\mathbf{x}_i \in \mathbb{R}^2\)
\item
  Velocity vector: \(\mathbf{v}_i \in \mathbb{R}^2\)
\item
  Joint angle: \(\theta_i \in \mathbb{R}\)  
\item
  Link length: \(a_i \in \mathbb{R}^+\)
\end{itemize}

This mapping allows us to represent policy stakeholders as autonomous
agents with specific capabilities (link lengths) and policy positions
(joint angles) who must coordinate to achieve collective policy
objectives.

\subsubsection{Table 1: Symbol-Concept Correspondence
Table}\label{table-1-symbol-concept-correspondence-table}

{\def\LTcaptype{none} % do not increment counter
\begin{longtable}[]{@{}
  >{\raggedright\arraybackslash}p{(\linewidth - 10\tabcolsep) * \real{0.0921}}
  >{\raggedright\arraybackslash}p{(\linewidth - 10\tabcolsep) * \real{0.2237}}
  >{\raggedright\arraybackslash}p{(\linewidth - 10\tabcolsep) * \real{0.0855}}
  >{\raggedright\arraybackslash}p{(\linewidth - 10\tabcolsep) * \real{0.2303}}
  >{\raggedright\arraybackslash}p{(\linewidth - 10\tabcolsep) * \real{0.2105}}
  >{\raggedright\arraybackslash}p{(\linewidth - 10\tabcolsep) * \real{0.1579}}@{}}
\toprule\noalign{}
\begin{minipage}[b]{\linewidth}\raggedright
Symbol (記号)
\end{minipage} & \begin{minipage}[b]{\linewidth}\raggedright
Mathematical content (数学的内容)
\end{minipage} & \begin{minipage}[b]{\linewidth}\raggedright
Unit (単位)
\end{minipage} & \begin{minipage}[b]{\linewidth}\raggedright
Institutional meaning (制度的意味)
\end{minipage} & \begin{minipage}[b]{\linewidth}\raggedright
Parameter range (パラメータ範囲)
\end{minipage} & \begin{minipage}[b]{\linewidth}\raggedright
Evidence/Basis (根拠)
\end{minipage} \\
\midrule\noalign{}
\endhead
\bottomrule\noalign{}
\endlastfoot
\(\theta_i\) & Joint angle of joint \(i\) & radian (rad) & Stakeholder's
policy position or stance & \((-\infty, +\infty)\) & Yoshihara et
al.~(2009); robot kinematics \\
\(a_i\) & Link length of joint \(i\) & meter (m) & Stakeholder's
influence/capability or resource endowment & \(\mathbb{R}^+\) &
Yoshihara et al.~(2009); physical link property \\
\(b_{sq,i}\) & Status quo bias coefficient & dimensionless & Resistance
to change; preference for maintaining current policy & \([0, 1]\) &
Samuelson \& Zeckhauser (1988); bias modeling; Fortunato et al.~(2013)
for bifurcation thresholds \\
\(b_{cf,i}\) & Confirmation bias coefficient & dimensionless &
Overconfidence in self-judgment; selective information processing &
\([-1, 1]\) & Russo \& Schoemaker (1992); bias theory; Bergerot et
al.~(2024) for asymmetric RL effects \\
\(b_{nf,i}\) & Narrow framing bias coefficient & dimensionless & Lack of
system-wide perspective; local optimization focus & \([0, 1]\) &
Kahneman (2011); framing effects; Bohren \& Imbens (2017) for bounded
rationality \\
\(k_i\) & Coordination coefficient & dimensionless & Degree of
cooperation with other stakeholders & \([0, 1]\) & Yoshihara et
al.~(2009); cooperative control \\
\(V\) & Control gain & dimensionless & Speed of following policy goals;
policy implementation pace & \(\mathbb{R}^+\) & Control theory; gain
parameter \\
\(\mathbf{x}_i\) & Position vector of joint \(i\) & meter (m) &
Cumulative policy outcome position & \(\mathbb{R}^2\) & Robot
kinematics; state variable \\
\(\mathbf{v}_i\) & Velocity vector of joint \(i\) & m/s & Rate of policy
position change & \(\mathbb{R}^2\) & Time derivative of position \\
\(\mathbf{x}_d\) & Target position & meter (m) & Policy goal or desired
outcome & \(\mathbb{R}^2\) & Control objective; goal state \\
\(G_t\) & Target velocity gain & 1/s & Policy goal pursuit strength &
\(\mathbb{R}^+\) & Control parameter \\
\(\epsilon_1, \epsilon_2\) & Small positive constants & - & Numerical
stability parameters & \(\mathbb{R}^+\) & Numerical methods; avoids
division by zero \\
\(A_t\) & Accuracy × Distance & m & Performance metric combining
accuracy and distance traveled & \(\mathbb{R}^+\) & Performance
evaluation (Section 3.7.1) \\
\(E_t\) & Energy efficiency & m/rad² & Efficiency metric considering
joint movements & \(\mathbb{R}^+\) & Performance evaluation (Section
3.7.2) \\
\(J_{act,i}\) & Joint activity & rad/s & Degree of stakeholder
engagement/activity & \(\mathbb{R}^+\) & Performance evaluation (Section
3.7.3) \\
\(J_{smooth,i}\) & Joint smoothness & rad/s² & Stability of stakeholder
behavior & \(\mathbb{R}^+\) & Performance evaluation (Section 3.7.4) \\
\end{longtable}
}

\subsubsection{3.2 Kinematic Model}\label{kinematic-model}

Each joint's position depends on previous joints' positions and
cumulative joint angles:

\[\mathbf{x}_i = \mathbf{x}_{i-1} + a_i \begin{bmatrix} \cos(\sum_{j=0}^{i-1} \theta_j) \\ \sin(\sum_{j=0}^{i-1} \theta_j) \end{bmatrix}\]

where \(\mathbf{x}_0 = \mathbf{0}\). This represents the cumulative
effect of stakeholder decisions on policy outcomes.

Velocity is computed as the time derivative of position:

\[\mathbf{v}_i = \frac{d\mathbf{x}_i}{dt}\]

\subsubsection{3.3 Control Objectives and Coordination
Mechanisms}\label{control-objectives-and-coordination-mechanisms}

The system's objective is to guide the end-effector (final stakeholder)
to track a target position \(\mathbf{x}_d\) representing policy goals.
Each joint \(i\) follows a cooperative control algorithm:

\paragraph{3.3.1 Target Direction Vector
Calculation}\label{target-direction-vector-calculation}

Direction vector from end-effector to target:

\[\mathbf{n}_i = \mathbf{x}_d - \mathbf{x}_i\]

Normalized perpendicular vector:

\[\mathbf{e}_i = \frac{1}{||\mathbf{n}_i||} \begin{bmatrix} -n_{i,y} \\ n_{i,x} \end{bmatrix}\]

\paragraph{3.3.2 Target Velocity
Decomposition}\label{target-velocity-decomposition}

Target velocity \(\mathbf{v}_d = G_t(\mathbf{x}_d - \mathbf{x}_N)\) is
decomposed into each joint's motion direction:

\[\mathbf{v}_{l,i} = (\mathbf{e}_i \cdot \mathbf{v}_d) \mathbf{e}_i\]

\[\mathbf{v}_{r,i} = \mathbf{v}_d - \mathbf{v}_{l,i}\]

\paragraph{3.3.3 Coordination Coefficient
Calculation}\label{coordination-coefficient-calculation}

Each joint's coordination coefficient \(k_i\) is calculated based on the
difference between its target velocity and subsequent joints' actual
velocities:

\[k_i = \exp\left(-4\ln(2) \frac{||\mathbf{v}_{l,i} - \mathbf{v}_{i+1}||^2 + \epsilon_1}{||\mathbf{v}_{l,i}||^2 + \epsilon_2}\right)\]

where \(\epsilon_1, \epsilon_2\) are small positive constants for
numerical stability.

\subsubsection{3.4 Cognitive Bias
Integration}\label{cognitive-bias-integration}

\paragraph{3.4.1 Status Quo Bias}\label{status-quo-bias}

Status quo bias \(b_{sq,i} \in [0,1]\) is incorporated as resistance to
joint angle changes:

\[\frac{d\theta_i}{dt} = (1 - b_{sq,i}) \cdot \omega_i\]

where \(\omega_i\) is the angular velocity without bias. The parameter
ranges for cognitive bias effects in this model
(\(b_{cf,i} \in [-1,1]\), \(b_{sq,i} \in [0,1]\),
\(b_{nf,i} \in [0,1]\)) Status quo bias \(b_{sq,i} \in [0,1]\) is
incorporated as resistance to joint angle changes:
\[\frac{d\theta_i}{dt} = (1 - b_{sq,i}) \cdot \omega_i\]

where \(\omega_i\) is the angular velocity without bias. The parameter
ranges for cognitive bias effects in this model
(\(b_{cf,i} \in [-1,1]\), \(b_{sq,i} \in [0,1]\),
\(b_{nf,i} \in [0,1]\)) were informed by cognitive bias literature
(Samuelson \& Zeckhauser, 1988; Russo \& Schoemaker, 1992) and standard
practice in computational opinion dynamics (Arendhart et al., 2025;
Banisch et al., 2024; Paz et al., 2024). The status quo bias threshold
of 0.25 corresponds to \textbf{half of the universal consensus threshold
(ε = 0.5)} reported in bounded confidence models (Fortunato et al.,
2013).

\paragraph{3.4.2 Confirmation Bias}\label{confirmation-bias}

Confirmation bias \(b_{cf,i} \in [-1,1]\) is incorporated as
coordination coefficient modification:

\[k_i' = k_i \cdot (1 + b_{cf,i} \cdot 0.5)\]

Positive confirmation bias represents overconfidence in self-judgment,
while negative values represent excessive self-doubt. The parameter
ranges for cognitive bias effects in this model
(\(b_{cf,i} \in [-1,1]\), \(b_{sq,i} \in [0,1]\),
\(b_{nf,i} \in [0,1]\)) were informed by cognitive bias literature
(Samuelson \& Zeckhauser, 1988; Russo \& Schoemaker, 1992) and standard
practice in computational opinion dynamics (Arendhart et al., 2025;
Banisch et al., 2024; Paz et al., 2024).

\paragraph{3.4.3 Narrow Framing Bias}\label{narrow-framing-bias}

Narrow framing bias \(b_{nf,i} \in [0,1]\) is incorporated as neglect of
global optimization:

\[\tilde{\mathbf{v}}_i = \begin{cases}
(1 - b_{nf,i}) \tilde{\mathbf{v}}_i & \text{if } b_{nf,i} > 0.5 \\
\tilde{\mathbf{v}}_i & \text{otherwise}
\end{cases}\]

\paragraph{3.4.4 Complete Bias Control
Law}\label{complete-bias-control-law}

The complete control law integrates all three cognitive biases into the
cooperative velocity calculation and joint angle updates. The modified
control law with bias effects is:

\[\tilde{\mathbf{v}}_i = \prod_{j \neq i}^{N} (1 - k_j') \mathbf{v}_{l,i} + \sum_{j \neq i}^{N} k_j' \mathbf{v}_j\]

with the bias-modified coordination coefficient:

\[k_i' = k_i \cdot (1 + b_{cf,i} \cdot 0.5)\]

and the final joint angle update incorporating all biases:

\[\frac{d\theta_i}{dt} = (1 - b_{sq,i}) \cdot \left[-||\tilde{\mathbf{v}}_i|| \cdot \text{sign}(\mathbf{e}_i \cdot \mathbf{v}_d) \cdot V\right]\]

where the term in square brackets represents the narrow framing bias
effect on the velocity magnitude:

\[\text{velocity magnitude with narrow framing} = \begin{cases}
||\tilde{\mathbf{v}}_i|| \cdot (1 - b_{nf,i}) & \text{if } b_{nf,i} > 0.5 \\
||\tilde{\mathbf{v}}_i|| & \text{otherwise}
\end{cases}\]

This comprehensive formulation shows how: (1) status quo bias
\(b_{sq,i} \in [0,1]\) directly scales the angular velocity update, (2)
confirmation bias \(b_{cf,i} \in [-1,1]\) modifies the coordination
coefficient, amplifying (for positive values) or reducing (for negative
values) the cooperative tendency, and (3) narrow framing bias
\(b_{nf,i} \in [0,1]\) attenuates the effective velocity when
stakeholders focus excessively on local rather than global objectives.
The parameter ranges for cognitive bias effects in this model
(\(b_{cf,i} \in [-1,1]\), \(b_{sq,i} \in [0,1]\),
\(b_{nf,i} \in [0,1]\)) were calibrated based on empirical observations
from Japan's MaaS policy implementation (Nagata, 2024), where
stakeholder resistance patterns revealed threshold effects at
approximately 20-25\% deviation from established practices.

\subsubsection{3.5 Cooperative Velocity
Calculation}\label{cooperative-velocity-calculation}

Each joint's cooperative velocity \(\tilde{\mathbf{v}}_i\) is calculated
as:

\[\tilde{\mathbf{v}}_i = \prod_{j \neq i}^{N} (1 - k_j') \mathbf{v}_{l,i} + \sum_{j \neq i}^{N} k_j' \mathbf{v}_j\]

\subsubsection{3.6 Joint Angle Updates}\label{joint-angle-updates}

Final joint angle updates are based on the inner product of cooperative
velocity and target velocity:

\[\frac{d\theta_i}{dt} = -||\tilde{\mathbf{v}}_i|| \cdot \text{sign}(\mathbf{e}_i \cdot \mathbf{v}_d) \cdot V\]

where \(V\) is the control gain.

\subsubsection{3.7 Performance Metrics}\label{performance-metrics}

System performance is evaluated using the following metrics:

\paragraph{3.7.1 Accuracy × Distance}\label{accuracy-distance}

\[A_t = \frac{1}{1 + ||\mathbf{x}_d - \mathbf{x}_N||} \times D_t\]

where \(D_t\) is cumulative movement distance.

\paragraph{3.7.2 Energy Efficiency}\label{energy-efficiency}

\[E_t = \frac{A_t}{\sum_{i=1}^{N} \int_0^t \left(\frac{d\theta_i}{dt}\right)^2 dt}\]

\paragraph{3.7.3 Joint Activity}\label{joint-activity}

\[J_{act,i} = \left|\frac{d\theta_i}{dt}\right|\]

\paragraph{3.7.4 Joint Smoothness}\label{joint-smoothness}

\[J_{smooth,i} = \left|\frac{d^2\theta_i}{dt^2}\right|\]

\subsubsection{3.8 Simulation Implementation
Process}\label{simulation-implementation-process}

\pandocbounded{\includegraphics[keepaspectratio,alt={Figure 13: Simulation Process Flow for Stakeholder Coordination}]{../../results/figures/robot_arm_simulation_process_en_final.png}}
\emph{Figure 13: Comprehensive simulation process flow showing the
six-stage methodology for analyzing stakeholder coordination through
robot arm control. The process includes initial setup, bias application,
control execution, performance evaluation, statistical analysis, and
policy implications derivation. This framework enables systematic
analysis of 1,820 experiments with randomized conditions.}

\subsection{4. Experimental Design}\label{experimental-design}

\subsubsection{4.1 Experimental Overview}\label{experimental-overview}

To quantitatively evaluate the impact of cognitive biases on
coordination performance, this study conducted a comprehensive
experimental program totaling 1,820 runs. The experimental design
included baseline experiments with 100 runs without bias, confirmation
bias experiments comprising 11 conditions with 40 runs each (440
experiments total, intensity 0.0-1.0, step 0.1), status quo bias
experiments involving 21 conditions with 40 runs each (840 experiments
total, intensity 0.0-1.0, step 0.05), and narrow framing experiments
consisting of 11 conditions with 40 runs each (440 experiments total,
intensity 0.0-1.0, step 0.1).

\subsubsection{4.2 Bias Experiments}\label{bias-experiments}

To avoid deterministic results, this study implemented comprehensive
experimental design with randomized initial conditions:

\pandocbounded{\includegraphics[keepaspectratio,alt={Figure 12: Robot Arm Animation Scenarios for Policy Analysis}]{../../results/figures/robot_arm_animation_scenarios_en_fixed.png}}
\emph{Figure 12: Four key animation scenarios demonstrating how
different cognitive biases affect stakeholder coordination patterns.
Each scenario shows distinct joint configurations and target-reaching
behaviors: (a) normal operation without bias, (b) confirmation bias
promoting policy goal focus, (c) status quo bias maintaining existing
approaches, and (d) narrow framing limiting system-wide perspective.}

Each experiment incorporated multiple randomization parameters to ensure
statistical validity. Initial joint angles were varied within ±0.1
radians, target positions were varied around the center by ±0.01m with
radius variation of ±0.005m, control gains were varied by ±0.5, and
switching thresholds were varied by ±0.002.

\subsubsection{4.3 Statistical Analysis
Methods}\label{statistical-analysis-methods}

Statistical analysis employed multiple complementary methods to ensure
robust interpretation of results. Analysis of Variance (ANOVA) was used
to test for statistical significance of bias effects across different
experimental conditions. Confidence intervals were calculated to
estimate effect sizes and provide measures of uncertainty around mean
performance values. Regression analysis was conducted to verify linear
relationships between bias intensity and coordination performance
metrics.

\subsection{4. Experimental Results}\label{experimental-results}

\subsubsection{4.1 Statistical Analysis
Results}\label{statistical-analysis-results}

\paragraph{5.1.1 Confirmation Bias}\label{confirmation-bias-1}

Analysis of confirmation bias revealed no significant effect on
coordination performance (F=0.838, p=0.602). Average performance
remained stable at 0.970±0.002 across all intensity levels, suggesting a
constructive role for this bias type.

\paragraph{5.1.2 Status Quo Bias}\label{status-quo-bias-1}

Analysis of status quo bias demonstrated a significant effect on
coordination performance (F=1.593, p=0.044). A notable threshold effect
occurred around intensity 0.25 (half of the universal consensus
threshold ε = 0.5), with performance degradation beyond this level.
These findings have important institutional design implications,
emphasizing the importance of threshold management in policy
coordination systems.

\paragraph{5.1.3 Narrow Framing}\label{narrow-framing}

Analysis of narrow framing bias showed a significant negative effect on
coordination performance (F=1.985, p=0.028). The results demonstrated
linear degradation with R²=0.204 and a slope of -0.00107, indicating
overall system performance decline as narrow framing intensity
increased.

\subsubsection{4.2 Comprehensive Bias Effect
Analysis}\label{comprehensive-bias-effect-analysis}

\pandocbounded{\includegraphics[keepaspectratio,alt={Figure 2: Constructive Effect of Confirmation Bias}]{../../results/figures/confirmation_bias_constructive_effect_fixed.png}}
\emph{Figure 2: Constructive Effect of Confirmation Bias}

\pandocbounded{\includegraphics[keepaspectratio,alt={Figure 3: Threshold Effect of Status Quo Bias}]{../../results/figures/status_quo_threshold_effect_fixed.png}}
\emph{Figure 3: Threshold Effect of Status Quo Bias}

\pandocbounded{\includegraphics[keepaspectratio,alt={Figure 4: Linear Degradation of Narrow Framing Bias}]{../../results/figures/narrow_framing_linear_degradation_fixed.png}}
\emph{Figure 4: Linear Degradation of Narrow Framing Bias}

\subsubsection{4.3 Joint-Specific
Analysis}\label{joint-specific-analysis}

Analysis of how biases affect different stakeholder positions within the
policy network revealed:

\paragraph{5.3.1 Base Joint (Joint 0)
Analysis}\label{base-joint-joint-0-analysis}

The base joint representing political and citizen participation level
showed stabilizing effects from confirmation bias.

\paragraph{5.3.2 End-Effector Joint (Joint 7)
Analysis}\label{end-effector-joint-joint-7-analysis}

The end-effector joint representing business operator level showed the
most pronounced effects from status quo bias.

\subsubsection{5.4 Summary of Experimental
Results}\label{summary-of-experimental-results}

Through 1,820 experiments, this analysis obtained the following
findings:

\begin{enumerate}
\def\labelenumi{\arabic{enumi}.}
\tightlist
\item
  \textbf{Confirmation Bias}: Constructive effect maintaining
  coordination performance
\item
  \textbf{Status Quo Bias}: Sharp change around threshold 0.20-0.25
\item
  \textbf{Narrow Framing}: Linear performance degradation
\end{enumerate}

\subsubsection{4.4 Statistical Analysis}\label{statistical-analysis}

This study conducted comprehensive statistical analysis on 1,820
experiments (100 baseline + 440 confirmation bias + 840 status quo bias
+ 440 narrow framing bias) with randomized initial conditions to ensure
statistical validity. ANOVA tests revealed distinct patterns for each
bias type:

\textbf{Confirmation Bias}: No significant effect on performance (F =
0.838, p = 0.602), supporting the hypothesis that confirmation bias can
be constructively leveraged. Performance remained stable across all
intensity levels (0.0-1.0), with mean accuracy ranging from 0.970 to
0.970.

\textbf{Status Quo Bias}: Significant negative effect (F = 1.593, p =
0.044), confirming the threshold effect around intensity 0.20-0.25.
Performance degraded sharply beyond this threshold, validating the
institutional design recommendations for managing resistance to change.

\textbf{Narrow Framing Bias}: Significant linear degradation effect (F =
1.985, p = 0.028), demonstrating the importance of maintaining
system-wide perspectives. Linear regression analysis showed R² = 0.204
with a negative slope of -0.00107.

All experiments included appropriate confidence intervals (95\% CI) and
effect size calculations. The baseline condition showed mean accuracy of
0.969 ± 0.002, providing a stable reference point for bias comparisons.

\subsubsection{4.5 Joint-Specific
Analysis}\label{joint-specific-analysis-1}

To understand how biases affect different stakeholder positions within
the policy network, this study conducted joint-specific analysis
focusing on key positions:

\pandocbounded{\includegraphics[keepaspectratio,alt={Figure 8: Joint 0 (Base) Bias Effects}]{../../results/figures/joint_0_bias_effects.png}}
\emph{Figure 8: Bias effects on the base joint (foundational
stakeholder). Shows how fundamental policy actors respond differently to
various bias types, with confirmation bias providing stability at the
foundation level.}

\pandocbounded{\includegraphics[keepaspectratio,alt={Figure 9: Joint 7 (End-Effector) Bias Effects}]{../../results/figures/joint_7_bias_effects.png}}
\emph{Figure 9: Bias effects on the end-effector joint (final
implementation stakeholder). Demonstrates how biases at the
implementation level directly impact overall policy outcomes, with
narrow framing showing the most pronounced negative effects.}

The joint-specific analysis revealed that bias effects varied
significantly depending on stakeholder position within the policy
network. Base joints (foundational stakeholders) showed greater
resilience to confirmation bias, while end-effector joints
(implementation stakeholders) were more sensitive to narrow framing
effects.

\subsubsection{4.6 Summary of Experimental
Findings}\label{summary-of-experimental-findings}

\pandocbounded{\includegraphics[keepaspectratio,alt={Figure 10: Summary of All Bias Effects}]{../../results/figures/summary_bias_effects_with_ci.png}}
\emph{Figure 10: Comprehensive summary of bias effects across all
experimental conditions with confidence intervals. This overview
demonstrates the statistical significance and practical magnitude of
each bias type's impact on policy coordination effectiveness.}

\subsubsection{4.7 Statistical Summary for Policy
Makers}\label{statistical-summary-for-policy-makers}

This section presents key statistical findings in a format accessible to
non-expert readers, focusing on practical implications for transport
policy design and stakeholder coordination.

\textbf{Table 3: Statistical Summary of Cognitive Bias Effects}

{\def\LTcaptype{none} % do not increment counter
\begin{longtable}[]{@{}
  >{\raggedright\arraybackslash}p{(\linewidth - 12\tabcolsep) * \real{0.1209}}
  >{\raggedright\arraybackslash}p{(\linewidth - 12\tabcolsep) * \real{0.1429}}
  >{\raggedright\arraybackslash}p{(\linewidth - 12\tabcolsep) * \real{0.0989}}
  >{\raggedright\arraybackslash}p{(\linewidth - 12\tabcolsep) * \real{0.1978}}
  >{\raggedright\arraybackslash}p{(\linewidth - 12\tabcolsep) * \real{0.0879}}
  >{\raggedright\arraybackslash}p{(\linewidth - 12\tabcolsep) * \real{0.1429}}
  >{\raggedright\arraybackslash}p{(\linewidth - 12\tabcolsep) * \real{0.2088}}@{}}
\toprule\noalign{}
\begin{minipage}[b]{\linewidth}\raggedright
Bias Type
\end{minipage} & \begin{minipage}[b]{\linewidth}\raggedright
F-statistic
\end{minipage} & \begin{minipage}[b]{\linewidth}\raggedright
p-value
\end{minipage} & \begin{minipage}[b]{\linewidth}\raggedright
Mean Performance
\end{minipage} & \begin{minipage}[b]{\linewidth}\raggedright
95\% CI
\end{minipage} & \begin{minipage}[b]{\linewidth}\raggedright
Effect Size
\end{minipage} & \begin{minipage}[b]{\linewidth}\raggedright
Policy Implication
\end{minipage} \\
\midrule\noalign{}
\endhead
\bottomrule\noalign{}
\endlastfoot
Confirmation Bias & 0.838 & 0.602 & 0.970 ± 0.002 & {[}0.966, 0.974{]} &
Not significant & Can be constructively leveraged; maintaining
stakeholder confidence in shared objectives supports coordination \\
Status Quo Bias & 1.593 & 0.044* & Degrades above threshold
\textasciitilde0.25 (within bifurcation range of ε = 0.5) & Threshold
effect & Significant & Requires careful institutional management;
incremental changes below threshold maintain effectiveness \\
Narrow Framing Bias & 1.985 & 0.028* & Linear degradation & R² = 0.204 &
Slope = -0.00107 & System-wide perspectives essential; fragmented
decision-making reduces overall effectiveness \\
\end{longtable}
}

\emph{Note: p \textless{} 0.05 indicates statistical significance}

\textbf{Policy Implications for Non-Expert Readers}

The statistical analysis reveals three key patterns with direct
relevance to transport policy design:

\begin{enumerate}
\def\labelenumi{\arabic{enumi}.}
\item
  \textbf{Confirmation bias does not harm coordination}---Contrary to
  conventional wisdom that biases always impede collaboration,
  stakeholder confidence in shared objectives (confirmation bias)
  maintained stable performance across all conditions. For policy
  makers, this suggests that fostering stakeholder commitment to
  agreed-upon goals can support rather than hinder coordination efforts.
  The high mean performance (0.970 ± 0.002) with no statistical
  degradation (p=0.602) indicates that stakeholder alignment provides a
  stable foundation for collaboration.
\item
  \textbf{Status quo bias has a critical tipping point}---Resistance to
  change (status quo bias) significantly degrades coordination beyond a
  threshold intensity of approximately 0.20-0.25. Below this threshold,
  minimal performance impact occurs; above it, coordination
  effectiveness declines sharply. This finding suggests that policy
  reforms requiring stakeholder behavioral changes should be implemented
  incrementally, with each change kept small enough to remain below
  disruptive thresholds. Statistical significance (F=1.593, p=0.044)
  confirms this threshold effect is not due to random variation.
\item
  \textbf{Narrow framing consistently reduces effectiveness}---When
  stakeholders focus on local optimization rather than system-wide
  outcomes, coordination performance degrades linearly and significantly
  (F=1.985, p=0.028). The regression analysis (R²=0.204, slope=-0.00107)
  indicates that each incremental increase in narrow framing reduces
  overall system performance. For policy design, this underscores the
  importance of institutional mechanisms that maintain system-wide
  perspectives, such as integrated performance metrics, cross-sector
  coordination requirements, and information-sharing platforms.
\end{enumerate}

\textbf{Practical Recommendations}

Based on these statistical findings, policy makers should: - Foster
stakeholder confidence in shared objectives rather than attempting to
eliminate confirmation bias - Implement institutional changes
incrementally, keeping each reform below the status quo bias threshold
(\textasciitilde0.20-0.25 intensity) - Design institutional structures
that require and reward system-wide perspectives, countering natural
tendencies toward narrow framing - Monitor coordination performance
through metrics aligned with these statistically verified patterns

\subsection{5. Institutional Design Framework for Effective
Collaboration and
Co-creation}\label{institutional-design-framework-for-effective-collaboration-and-co-creation}

\subsubsection{5.1 Theoretical Foundation of Institutional
Design}\label{theoretical-foundation-of-institutional-design}

Based on empirical findings from Japan MaaS analysis and computational
modeling results, a comprehensive institutional design framework is
proposed that addresses the coordination challenges identified in both
studies. This framework utilizes institutional design theory while
incorporating insights about cognitive bias effects and democratic
participation requirements.

Key insights from the analysis indicate that effective collaboration and
co-creation require institutional arrangements that enable leveraging
the constructive potential of confirmation bias, managing the threshold
effects of status quo bias, mitigating negative impacts of narrow
framing, and ensuring democratic accountability while enabling
operational efficiency.

\subsubsection{5.2 Three-Tier Institutional
Architecture}\label{three-tier-institutional-architecture}

\pandocbounded{\includegraphics[keepaspectratio,alt={Figure 11: Three-Tier Institutional Design for Effective Collaboration}]{../../results/figures/three_tier_institutional_design_improved.png}}
\emph{Figure 11: Three-tier institutional design for effective
collaboration. Institutional design for policy coordination that
strategically leverages cognitive biases. The upper section shows the
basic framework separating quality definition from implementation, while
the lower section details specific mechanisms for managing
administrative-business relationships through environmental design.}

\paragraph{6.2.1 Tier 1: Political-Citizen Interface (Democratic Quality
Setting)}\label{tier-1-political-citizen-interface-democratic-quality-setting}

The first tier, the political-citizen interface, is responsible for
ensuring democratic legitimacy and specifying quality standards. At its
core, this tier aims to constructively leverage confirmation bias by
fostering stakeholder confidence in shared quality objectives, while
simultaneously maintaining a broad perspective to avoid narrow local
optimization. This is achieved through structured citizen participation
processes that define transport service quality objectives, as well as
political oversight of administrative implementation supported by clear
accountability mechanisms. Public deliberation on the trade-offs between
efficiency, accessibility, and sustainability is also central, alongside
the establishment of quality specification frameworks that provide clear
direction yet retain interpretive flexibility. In practice, this tier is
characterized by regular citizen assemblies or deliberative polls on
transport priorities, transparent reporting on the achievement of
quality objectives, clearly defined boundaries regarding the scope and
authority of citizen participation, and integration with broader
democratic governance structures.

\paragraph{6.2.2 Tier 2: Administrative Coordination (Environmental
Design)}\label{tier-2-administrative-coordination-environmental-design}

The second tier, administrative coordination, translates quality
objectives into operational frameworks. The guiding principle here is to
manage status quo bias through incremental change approaches that remain
below disruptive thresholds, while simultaneously developing the
expertise necessary for effective multi-stakeholder coordination. This
is accomplished by building specialized administrative capacity for
stakeholder coordination and conflict resolution, implementing technical
evaluation systems that incorporate both business and public value
metrics, designing regulatory frameworks that foster competitive
environments within quality constraints, and establishing performance
monitoring and adjustment systems with effective feedback loops. The
organizational structure supporting this tier includes specialized
departments for collaborative governance staffed with appropriate
expertise, buffer mechanisms to mediate between political and business
pressures, structured processes for translating democratic input into
operational guidance, and innovation incentives that are closely aligned
with public objectives.

\paragraph{6.2.3 Tier 3: Business-Operations Interface (Autonomous
Implementation)}\label{tier-3-business-operations-interface-autonomous-implementation}

The third tier, the business-operations interface, is tasked with the
efficient provision of services within established quality frameworks.
Here, the approach allows confirmation bias to support business
confidence in quality-aligned operations, minimizes status quo bias
through competitive pressure, and maintains a system-wide perspective
via integrated performance measurement. This is realized through
competitive processes for service provision that operate within quality
constraints, quality-adjusted performance contracts that reward the
creation of public value alongside efficiency, collaborative
problem-solving processes to address operational challenges, and
innovation incentives that promote public value creation. The framework
is further supported by performance-based funding that rewards
innovation, requirements for cross-sectoral coordination, the use of
system-wide performance metrics, and the establishment of
information-sharing platforms among stakeholder groups.

\subsubsection{5.3 Interface Management
Mechanisms}\label{interface-management-mechanisms}

The management of interfaces between tiers is crucial for effective
institutional design. At the political-administrative interface, the
primary challenge is to translate democratic input into operational
guidance without undermining administrative expertise or introducing
excessive political interference. This is addressed by implementing
structured quality specification processes that provide clear direction
while maintaining interpretive flexibility. Key mechanisms include the
use of quality frameworks that specify outcomes rather than processes,
regular review cycles with sunset clauses to manage status quo bias, the
preservation of professional administrative autonomy within
democratically defined parameters, and transparent accountability
mechanisms that enable political oversight without resorting to
micromanagement.

At the administrative-business interface, the challenge lies in
balancing public objectives with operational efficiency, while also
managing the risk that business interests may capture collaborative
processes. The solution involves the adoption of competitive frameworks
that reward quality achievement alongside efficiency, thereby aligning
private interests with public objectives. This is operationalized
through environmental design that makes quality-aligned behavior
profitable, performance measurement systems that incorporate multiple
dimensions of value, competitive pressures that prevent the entrenchment
of status quo bias in service provision, and collaborative governance
structures that facilitate joint problem-solving while maintaining
robust accountability.

\subsubsection{5.4 Comparative Institutional
Analysis}\label{comparative-institutional-analysis}

To contextualize the proposed three-tier institutional design within the
broader landscape of institutional arrangements, this section presents a
systematic comparison of alternative institutional design types.
Understanding the comparative advantages and limitations of different
approaches is essential for selecting appropriate governance structures
that can effectively address the coordination challenges identified in
this research.

\textbf{Table 2: Comparative Analysis of Institutional Design Types}

{\def\LTcaptype{none} % do not increment counter
\begin{longtable}[]{@{}
  >{\raggedright\arraybackslash}p{(\linewidth - 6\tabcolsep) * \real{0.2083}}
  >{\raggedright\arraybackslash}p{(\linewidth - 6\tabcolsep) * \real{0.2361}}
  >{\raggedright\arraybackslash}p{(\linewidth - 6\tabcolsep) * \real{0.2917}}
  >{\raggedright\arraybackslash}p{(\linewidth - 6\tabcolsep) * \real{0.2639}}@{}}
\toprule\noalign{}
\begin{minipage}[b]{\linewidth}\raggedright
制度設計タイプ
\end{minipage} & \begin{minipage}[b]{\linewidth}\raggedright
協調メカニズム
\end{minipage} & \begin{minipage}[b]{\linewidth}\raggedright
予想されるバイアス影響
\end{minipage} & \begin{minipage}[b]{\linewidth}\raggedright
日本への適用可能性
\end{minipage} \\
\midrule\noalign{}
\endhead
\bottomrule\noalign{}
\endlastfoot
Centralized type & Directive control & High confirmation bias, Low
status quo & Low (decentralized tradition) \\
Competitive market type & Price mechanism & High narrow framing, Low
coordination & Medium (partial application) \\
Network type & Collaborative governance & All biases have major impact &
High (current direction) \\
Proposed 3-tier type & Hybrid & Managed bias utilization & High \\
\end{longtable}
}

\textbf{Centralized Type}: The centralized institutional design relies
on directive control mechanisms where authority is concentrated in a
single decision-making entity. This approach tends to generate high
levels of confirmation bias due to limited perspective diversity and low
status quo bias because centralized authority can implement changes
without requiring multi-stakeholder consensus. However, Japan's strong
tradition of decentralized governance and multi-stakeholder
collaboration in regional public transport policy makes this approach
largely incompatible with existing institutional arrangements. The
administrative culture emphasizing consensus-based decision-making
further limits the feasibility of centralized directive control.

\textbf{Competitive Market Type}: Market-based institutional designs
utilize price mechanisms and competitive processes to coordinate
stakeholder behavior. This approach tends to produce high levels of
narrow framing bias as individual actors focus on optimizing their own
objectives rather than system-wide performance, while generating
relatively low status quo bias due to continuous competitive pressure
for innovation. Japan has implemented partial competitive mechanisms,
particularly through the 2002 deregulation and the 2010 Life Transport
Survival Project, which created competitive selection processes for
transport service provision. However, pure market approaches face
limitations in addressing regional transport challenges where commercial
viability cannot be the sole criterion.

\textbf{Network Type}: Network-based collaborative governance emphasizes
horizontal coordination among diverse stakeholders through deliberative
processes and consensus building. This approach is susceptible to all
forms of cognitive bias identified in this research, as stakeholder
interactions create multiple opportunities for confirmation bias through
information filtering, status quo bias through consensus requirements,
and narrow framing bias through organizational position constraints.
Japan's current collaborative governance direction aligns closely with
the network type, as evidenced by regional public transport conferences
and co-creation frameworks. The high applicability reflects existing
institutional patterns, but the identified implementation gaps in Japan
MaaS projects demonstrate the vulnerabilities of this approach to
bias-induced coordination failures.

\textbf{Proposed 3-Tier Type}: The proposed three-tier hybrid design
strategically combines elements of centralized authority (quality
specification in Tier 1), competitive mechanisms (service provision in
Tier 3), and collaborative processes (administrative coordination in
Tier 2) while actively managing cognitive bias effects through
purposeful institutional design. This approach enables managed bias
utilization by leveraging the constructive aspects of confirmation bias
for stakeholder confidence, implementing threshold management for status
quo bias, and maintaining system-wide perspectives to counteract narrow
framing. The high applicability to Japan stems from its alignment with
existing institutional patterns while addressing identified
vulnerabilities through structured interface management and
bias-conscious design principles.

\textbf{Why the Three-Tier Design is Most Appropriate for Japan}

The proposed three-tier institutional design demonstrates particular
suitability for Japan's regional public transport policy context for
several reasons. First, Japan's existing collaborative governance
infrastructure, including regional public transport conferences and
co-creation frameworks, provides a foundation that can be adapted rather
than replaced. The three-tier design builds upon these existing
structures while introducing the systematic interface management
mechanisms identified as critical through the computational modeling
analysis.

Second, Japan's administrative tradition of emphasizing consensus-based
decision-making and gradual reform is accommodated through the tiered
separation of democratic quality setting from operational
implementation. This structure allows for constructive stakeholder
engagement in quality specification without subjecting operational
details to potentially paralyzing consensus requirements. The
administrative coordination tier serves as a crucial buffer, translating
democratic input into operational guidance while maintaining the
stakeholder engagement that characterizes Japanese policy processes.

Third, Japan's resource constraints at the municipal level, as
documented by the Council of Experts on Decentralization Reform (2023),
require institutional designs that maximize efficiency while maintaining
democratic accountability. The three-tier design addresses this
challenge by concentrating specialized expertise in the administrative
coordination tier rather than requiring all stakeholders to possess both
technical and democratic capacity. This specialization reduces
implementation costs while maintaining stakeholder participation through
structured processes appropriate to each tier's function.

Fourth, the computational modeling findings regarding bias effects
provide specific design guidance that is particularly relevant to
Japan's institutional context. The threshold effect of status quo bias
around intensity 0.25 indicates the importance of avoiding dramatic
institutional changes that would trigger resistance. The three-tier
design can be implemented incrementally, with existing structures
gradually adapted to new roles rather than abolished entirely.
Similarly, the constructive role of confirmation bias suggests that
Japan's existing emphasis on stakeholder relationships can be leveraged
rather than treated as an obstacle to reform.

Finally, Japan's experience with the Japan MaaS initiative and the
identified implementation gaps demonstrate the limitations of pure
network-based collaborative governance. The three-tier design addresses
these limitations by introducing clear accountability mechanisms,
structured citizen participation pathways, and performance measurement
systems that incorporate both business and public value metrics. This
balanced approach addresses the empirical finding that 92\% of Japan
MaaS projects focused primarily on business indicators while only 5\%
established meaningful citizen participation mechanisms.

The comparative institutional analysis demonstrates that the proposed
three-tier design offers superior alignment with Japan's institutional
context, cognitive bias realities, and policy implementation challenges
compared to alternative institutional types. By combining the strengths
of centralized authority, competitive efficiency, and collaborative
engagement while actively managing cognitive bias effects, this design
provides a theoretically grounded and empirically informed approach to
improving collaboration and co-creation in regional public transport
policy.

\subsection{6. Discussion}\label{discussion}

\subsubsection{6.1 Theoretical
Implications}\label{theoretical-implications}

The theoretical implications of this study span multiple dimensions.
First, a reevaluation of cognitive biases was achieved by revealing
constructive aspects of cognitive biases previously viewed negatively.
Second, quantitative analysis methods were developed by enabling
quantitative analysis through mathematical modeling of policy
collaboration. Third, contributions to institutional design theory were
made by presenting new theoretical frameworks for strategically
leveraging biases in governance arrangements.

\subsubsection{6.2 Practical Implications}\label{practical-implications}

The practical implications encompass several important areas. Regarding
application to policy formulation, the findings highlight the importance
of policy design that considers cognitive biases as strategic resources
rather than obstacles. For stakeholder management, the research
demonstrates the value of relationship management methods tailored to
specific bias characteristics of different stakeholder groups.
Concerning institutional reform, the study indicates the need for
gradual improvement strategies for existing institutions that work with
rather than against predictable cognitive tendencies.

\subsubsection{6.3 Research Limitations}\label{research-limitations}

This study acknowledges several important limitations. The simulation
environment represents a significant constraint, as computational
modeling necessarily diverges from actual policy complexity. The bias
modeling approach involves simplification of cognitive biases, which are
more nuanced and context-dependent in real-world settings than
parametric representations suggest. The validation scope is limited to
regional public transport in Japan, which may limit generalizability to
other policy areas or institutional contexts.

\subsubsection{6.4 Threats to Validity}\label{threats-to-validity}

\textbf{Internal Validity}

Internal validity concerns the extent to which the observed
relationships between cognitive biases and coordination performance can
be attributed to the experimental manipulations rather than confounding
factors. Several limitations must be acknowledged. First, the simulation
environment necessarily simplifies the actual complexity of policy
processes. Real-world collaborative transport policy implementation
involves numerous contextual factors including political dynamics,
organizational cultures, informal power relationships, and temporal
constraints that are not fully captured in the computational model.
Second, the analysis focuses on three specific cognitive biases
(confirmation bias, status quo bias, and narrow framing bias) while
excluding other potentially relevant factors such as political dynamics,
resource constraints, and organizational incentives that may also
significantly affect coordination outcomes. While this focused approach
enables rigorous analysis of specific mechanisms, it may overstate the
independent effects of these biases relative to other contextual
factors.

\textbf{External Validity}

External validity refers to the generalizability of findings beyond the
specific study context. This study's external validity is subject to
several constraints. The analysis is limited to regional public
transport in Japan, which has specific institutional characteristics
including regulatory frameworks, governance structures, and cultural
contexts that may differ substantially from other countries or policy
domains. The Japanese context features particular relationships between
government agencies, private operators, and citizens that may not
transfer directly to other institutional settings. Additionally, the
stakeholder count is fixed at eight agents in the simulation, which
represents a simplification of real-world policy networks that may
involve dozens or hundreds of stakeholders with varying degrees of
influence and interaction patterns. The fixed network topology does not
capture the dynamic formation and dissolution of stakeholder
relationships that characterizes many collaborative governance
processes.

\textbf{Construct Validity}

Construct validity concerns the degree to which the theoretical
constructs are adequately measured and represented in the study. Several
limitations should be noted. The parameter-based representation of
cognitive biases approximates complex cognitive processes that in
reality involve subtle psychological mechanisms, contextual triggers,
and individual differences. The mathematical formulations used to
represent confirmation bias (as coordination coefficient modification),
status quo bias (as resistance to change), and narrow framing bias (as
neglect of global optimization) are simplified abstractions that may not
capture the full richness of these cognitive phenomena as they manifest
in actual stakeholder decision-making. Additionally, the robot arm to
policy network mapping involves conceptual abstraction where physical
joint movements represent stakeholder behaviors, target positions
represent policy objectives, and coordination coefficients represent
collaborative tendencies. While this mapping enables quantitative
analysis, it remains a metaphorical representation that may not fully
capture the qualitative dimensions of human collaboration,
communication, and negotiation that characterize real policy processes.

\textbf{Reliability}

Reliability refers to the consistency and reproducibility of the
research findings. Several measures enhance the reliability of this
study. The experimental design includes 1,820 randomized experiments
with multiple randomization parameters (initial joint angles varied
within ±0.1 radians, target positions varied by ±0.01m, control gains
varied by ±0.5, switching thresholds varied by ±0.002) to ensure
statistical validity and minimize the impact of specific initial
conditions on results. This comprehensive experimental design enables
reproducibility through systematic documentation of simulation
parameters, bias formulations, and evaluation metrics. All performance
metrics are quantified using standardized formulas for accuracy, energy
efficiency, joint activity, and joint smoothness. Statistical analysis
employs rigorous methods including Analysis of Variance (ANOVA),
confidence interval calculations, and regression analysis with
appropriate significance testing. The study reports 95\% confidence
intervals for all key findings, enabling assessment of the precision of
effect size estimates. However, the reliability of the computational
model depends on accurate implementation of the cooperative control
algorithms, and while the simulation code has been systematically
validated, replication would require access to the specific
computational environment and parameter specifications used in this
study.

\subsection{7. Conclusion}\label{conclusion}

\subsubsection{7.1 Research Achievements}\label{research-achievements}

This study analyzed collaborative mechanisms in regional public
transport policy using robot arm coordination control simulation and
produced several key findings. Quantitative evaluation of cognitive
biases was accomplished by statistically verifying the effects of three
cognitive biases on coordination performance through 1,820 controlled
experiments. The constructive effects of certain biases were
demonstrated through the coordination performance-maintaining effect of
confirmation bias, challenging conventional assumptions about bias
uniformly hindering policy implementation. A three-tier institutional
design framework was developed that strategically leverages biases
rather than simply attempting to eliminate them.

\subsubsection{7.2 Future Research Agenda}\label{future-research-agenda}

The future research agenda encompasses several critical directions.
Empirical research represents a priority, requiring validation of the
theoretical framework with actual policy cases to test the practical
applicability of the computational findings. Model extension offers
opportunities to explore the combined effects of more complex cognitive
biases and their interactions in multi-stakeholder environments.
International comparison research could examine the applicability of the
institutional design framework in different cultural and institutional
environments beyond Japan.

\subsubsection{7.3 Policy Implications}\label{policy-implications}

The policy implications of this study offer significant insights for
governance practice. Strategic leverage of cognitive biases represents a
paradigm shift toward utilizing rather than eliminating biases in
institutional design, recognizing that certain biases can enhance rather
than hinder coordination effectiveness. Gradual institutional reform
emerges as a preferred approach, emphasizing incremental improvement
strategies that avoid triggering counterproductive status quo bias
responses. Continuous monitoring becomes essential, requiring regular
evaluation and adjustment of institutional effects to maintain optimal
bias management and stakeholder coordination over time.

\subsection{Acknowledgments}\label{acknowledgments}

The authors would like to express their deep gratitude to all those who
provided support and cooperation in conducting this research.

\subsection{References}\label{references}

MLIT (Ministry of Land, Infrastructure, Transport and Tourism). (2024).
Regional Transport Re-design. Retrieved from
https://www.mlit.go.jp/redesign/assets/pdf/top/redesign\_240426.pdf

UN (United Nations). (2015). Transforming our world: The 2030 Agenda for
Sustainable Development. Retrieved from https://sdgs.un.org/2030agenda

Abdoos, M. (2020). A Cooperative Multiagent System for Traffic Signal
Control Using Game Theory and Reinforcement Learning. \emph{IEEE
Intelligent Transportation Systems Magazine}, 13, 6-16. Acciarini, C.,
Brunetta, F., \& Boccardelli, P. (2020). Cognitive biases and
decision-making strategies in times of change: a systematic literature
review. \emph{Management Decision}. Adji, I., Sumaryadi, I., Djohan, D.,
\& Rowa, H. (2023). Collaborative Governance in the Management of
Transportation Modes in DKI Jakarta Province. \emph{Jurnal Ilmiah Ilmu
Administrasi Publik}. Anand, A., Guha, D., \& Purwar, S. (2024).
Cooperative Formation Control of the Multi-Agent System. 2024 IEEE 4th
International Conference on Sustainable Energy and Future Electric
Transportation (SEFET), 1-5. Arehart, E., Moser, D., \& Barfuss, W.
(2025). Biased Transmission Drives an Accuracy-Consensus Tradeoff in
Collective Learning. \emph{Proceedings of the National Academy of
Sciences}, 122(23), e241762012. Banisch, S., Lima, R., \& Araújo, T.
(2024). Mean-field analysis for cognitively-grounded opinion dynamics
with confirmation bias. \emph{Physical Review E}, 109(2), 024304.
Bergerot, C., Barfuss, W., \& Donner, R. V. (2024). Moderate
confirmation bias enhances decision-making in groups of
reinforcement-learning agents. \emph{PLOS Computational Biology}, 20(4),
e1011896. Bohren, J. A., \& Imbens, G. W. (2017). Bounded Rationality
and Learning: A Framework and a Robustness Result. \emph{Research Papers
in Economics}, 2017(16), 1-46. Bouraima, M., Oyaro, J., Ayyıldız, E.,
Erdogan, M., \& Maraka, N. (2023). An integrated decision support model
for effective institutional coordination framework in planning for
public transportation. \emph{Soft Computing}, 1-27. Briñón-Arranz, L.,
Seuret, A., \& Canudas-De-Wit, C. (2014). Cooperative Control Design for
Time-Varying Formations of Multi-Agent Systems. \emph{IEEE Transactions
on Automatic Control}, 59, 2283-2288. Council of Experts on
Decentralization Reform. (2023). 地方分権改革有識者会議 報告書. {[}In
Japanese{]}. Dong, E., McCarty, K., McCauley, J., Motter, A. E., \&
Moser, D. (2024). Cognitive biases can move opinion dynamics from
consensus to signatures of transient chaos. \emph{Physical Review
Letters}, 132(5), 057401. Fortunato, A. (2013). Critical Thresholds in
Social Interactions. \emph{Physical Review E}, 85, 281-311. Fossheim,
K., \& Andersen, J. (2022). The consequences of institutional design on
collaborative arrangements' power to influence urban freight
policymaking. \emph{Case Studies on Transport Policy}. Grimmelikhuijsen,
S. G., Jilke, S., Olsen, A. L., \& Tummers, L. (2017). Behavioral public
administration: Combining insights from public administration and
psychology. \emph{Public Administration Review}, 77(1), 45-56. Gulzar,
M., Rizvi, S., Javed, M., Munir, U., \& Asif, H. (2018). Multi-Agent
Cooperative Control Consensus: A Comparative Review. \emph{Electronics},
7, 22. He, F., Li, X., \& Liu, Z. (2023). A Review of Multi-Agent
Collaborative Control. 2023 2nd International Conference on Artificial
Intelligence, Human-Computer Interaction and Robotics (AIHCIR), 506-510.
Hengster-Movrić, K., \& Lewis, F. (2014). Cooperative Optimal Control
for Multi-Agent Systems on Directed Graph Topologies. \emph{IEEE
Transactions on Automatic Control}, 59, 769-774. Hu, Y., Xu, Q., Field,
B., Xia, B., \& Wu, G. (2024). Governing collaborative networks in mega
transport projects development: Integrative findings from 34 cases
worldwide. \emph{Transport Policy}. Kahneman, D. (2011). \emph{Thinking,
fast and slow}. Farrar, Straus and Giroux. Kahneman, D., \& Lovallo, D.
(1993). Timid choices and bold forecasts: A cognitive perspective on
risk taking. \emph{Management Science}, 39(1), 17-31. Kato, H.,
Taniguchi, A., \& Osuga, K. (2009). Empirical analysis of feasibility
and sustainability of community-participatory regional public transport
service supply: A case study of ``Seikatsu Bus Yokkaichi''.
\emph{Journal of Japan Society of Civil Engineers}, 65(4), 568-578.
{[}In Japanese{]}. Kita, H. (2006). \emph{Development of regional public
transport policy}. Seizando Shoten. {[}In Japanese{]}. Klijn, E. H., \&
Koppenjan, J. F. M. (2012). Governance network theory: Past, present and
future. \emph{Policy \& Politics}, 40(4), 587-606. Knoppen, D.,
Janjevic, M., \& Winkenbach, M. (2021). Prioritizing urban freight
logistics policies: Pursuing cognitive consensus across multiple
stakeholders. \emph{Environmental Science \& Policy}. Kyushu Regional
Transport Bureau. (2023).
アフターコロナ時代に向けた地域交通の共創に関する研究会 中間とりまとめ.
{[}In Japanese{]}. Lan, X., Yan, J., He, S., Zhao, Z., \& Zou, T.
(2023). Distributed cooperative reinforcement learning for multi-agent
system with collision avoidance. \emph{International Journal of Robust
and Nonlinear Control}, 34, 567-585. Liu, Y., Zhao, L., \& Zhao, M.
(2022). A Review on Cooperative Control Problems of Multi-agent Systems.
2022 41st Chinese Control Conference (CCC), 4831-4836. Mao, Y., \&
Hovakimyan, N. (2021). Dynamics of Public Opinion Evolution with
Asymmetric Cognitive Bias. \emph{arXiv preprint} arXiv:2105.14726. Mao,
Y., Hovakimyan, N., \& Zhang, J. (2022). Cost Function Learning in
Memorized Social Networks with Cognitive Behavioral Asymmetry.
\emph{IEEE Transactions on Computational Social Systems}, 9(3), 688-699.
Ministry of Land, Infrastructure, Transport and Tourism. (2022).
地域公共交通計画と乗合バス等の補助制度の連動化に関する解説パンフレット.
{[}In Japanese{]}. Ministry of Land, Infrastructure, Transport and
Tourism. (2025). 地域公共交通計画等の作成と運用の手引き 第4版 理念編.
{[}In Japanese{]}. Nagata, U. (2024). Can ``Japan MaaS'' be understood
as a type of ``Transport Town Planning''? A consideration of goals and
governance. \emph{Journal of Japan Society of Civil Engineers}, 80(20),
24-20140. https://doi.org/10.2208/jscejj.24-20140 {[}In Japanese{]}.
Nomura, Y. (2023). Efforts to secure regional transport using limited
liability partnerships in Ichinohe Town. \emph{Transport Policy
Studies}, 26(1), 45-52. {[}In Japanese{]}. Ostrom, E. (2005).
\emph{Understanding institutional diversity}. Princeton University
Press. Paz, J. P., Tobón, L., \& Cedeno, W. (2024). Consensus in Models
for Opinion Dynamics with Generalized-Bias. \emph{IEEE Transactions on
Network Science and Engineering}, 11(3), 2656-2671. Pierson, P. (2000).
Increasing returns, path dependence, and the study of politics.
\emph{American Political Science Review}, 94(2), 251-267. Rastogi, C.,
Zhang, Y., Wei, D., Varshney, K., Dhurandhar, A., \& Tomsett, R. (2020).
Deciding Fast and Slow: The Role of Cognitive Biases in AI-assisted
Decision-making. \emph{Proceedings of the ACM on Human-Computer
Interaction}, 6, 1-22. Paulsson, A., Isaksson, K., Sørensen, C., Hrelja,
R., Rye, T., \& Scholten, C. (2018). Collaboration in public transport
planning -- Why, how and what?. \emph{Research in Transportation
Economics}. Russo, J. E., \& Schoemaker, P. J. H. (1992). Managing
overconfidence. \emph{Sloan Management Review}, 33(2), 7-17. Samuelson,
W., \& Zeckhauser, R. (1988). Status quo bias in decision making.
\emph{Journal of Risk and Uncertainty}, 1(1), 7-59. Shi, P., \& Shen, Q.
(2015). Cooperative Control of Multi-Agent Systems With Unknown
State-Dependent Controlling Effects. \emph{IEEE Transactions on
Automation Science and Engineering}, 12, 827-834. Steiglechner, P.,
Keijzer, M. A., \& Moser, D. J. (2024). Noise and opinion dynamics: How
ambiguity promotes pro-majority consensus in the presence of
confirmation bias. \emph{Royal Society Open Science}, 11(6), 231587.
Sterman, J. D. (2000). \emph{Business dynamics: Systems thinking and
modeling for a complex world}. Irwin/McGraw-Hill. Yang, Z., \& Yao, Y.
(2023). A Survey on Cooperative Control of Multi-Agent Systems. 2023 2nd
International Conference on Artificial Intelligence, Human-Computer
Interaction and Robotics (AIHCIR), 511-515. Yoshida, T. (2021). Effect
analysis of bus route reorganization through inter-operator
collaboration in Hachinohe City. \emph{Transport Studies}, 64, 123-130.
{[}In Japanese{]}. Yoshihara, Y., Makino, H., Tomita, N., \& Yano, K.
(2009). Does real-time optimization of joint mobility generate globally
optimal arm movements? \emph{Transactions of the Society of Instrument
and Control Engineers}, 45(11), 570-577. {[}In Japanese{]}. ---

\emph{This paper was prepared with the aim of contributing to a new
understanding of collaboration and co-creation in regional public
transport policy.}
