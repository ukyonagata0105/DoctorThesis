% -----------------------------------------------------------------------------------------------------------------------------------------
% 博士論文テンプレート(日本語対応版 - LuaLaTeX用)
% Doctoral Thesis Template (Japanese Version - for LuaLaTeX)
% 
% 博士論文:政策形成における生成AIと人間の関係性
%      公共交通政策を事例として
% 
% コンパイル方法: lualatex main.tex && biber main && lualatex main.tex && lualatex main.tex
% -----------------------------------------------------------------------------------------------------------------------------------------
\documentclass[11pt,a4paper,twoside,openany]{report}

% LuaLaTeX用日本語サポート
\usepackage{luatexja}
\usepackage{luatexja-fontspec}
% フォント設定(システムに合わせて変更してください)
% \setmainjfont{IPAexMincho}
% \setsansjfont{IPAexGothic}

% 参考文献(各章ごと)
\usepackage[style=numeric,sorting=none,backend=biber]{biblatex}
\addbibresource{./refs/ref_ch1.bib}
\addbibresource{./refs/ref_ch2.bib}
\addbibresource{./refs/ref_ch3.bib}
\addbibresource{./refs/ref_ch4.bib}
\addbibresource{./refs/ref_ch5.bib}
\addbibresource{./refs/ref_ch6.bib}
\addbibresource{./refs/ref_ch7.bib}

% 基本パッケージ
\usepackage{amsmath,amssymb}
\usepackage{amsfonts}
\usepackage{bm}
\usepackage{fancyhdr}
\usepackage{sectsty}
\usepackage{indentfirst}
\usepackage[centering,margin=25mm]{geometry}
\usepackage{here}
\usepackage{setspace}
\usepackage{tabularx}
\usepackage{booktabs}
\usepackage{multirow}
\newcolumntype{Y}{>{\centering\arraybackslash}X}
\usepackage{graphicx}
\usepackage{xcolor}
\usepackage[colorlinks=true,linkcolor=blue,anchorcolor=magenta,citecolor=blue,urlcolor=blue]{hyperref}
\usepackage{tikz}
\usetikzlibrary{shapes,arrows,positioning,calc}

% 図表
\usepackage{float}
\usepackage{caption}
\usepackage{subcaption}

% リスト
\usepackage{enumitem}

% ユーザー定義コマンド
\newcommand{\red}[1]{\textcolor{red}{#1}}
\newcommand{\blue}[1]{\textcolor{blue}{#1}}
\newcommand{\green}[1]{\textcolor{green}{#1}}
\newcommand{\highlight}[1]{\colorbox{yellow}{#1}}

% グラフィックパス
\graphicspath{{./figure/}{./figure/ch1/}{./figure/ch2/}{./figure/ch3/}{./figure/ch4/}{./figure/ch5/}{./figure/ch6/}{./figure/ch7/}}

% 日本語インデント設定
\parindent=1zw

% 章番号と節番号の間のドット
\renewcommand{\thesection}{\thechapter.\arabic{section}}
\renewcommand{\thesubsection}{\thesection.\arabic{subsection}}

% 行間設定
\setlength{\baselineskip}{20pt}

% -----------------------------------------------------------------------------------------------------------------------------------------
\begin{document}

\pagenumbering{roman}

% -----------------------------------------------------------------------------------------------------------------------------------------
% 表紙
% -----------------------------------------------------------------------------------------------------------------------------------------
% -----------------------------------------------------------------------------------------------------------------------------------------
% 表紙
% -----------------------------------------------------------------------------------------------------------------------------------------

\begin{titlepage}
\begin{center}

\vspace*{3cm}

{\Huge 博士論文}

\vspace{2cm}

{\LARGE 政策形成における生成AIと人間の関係性:\\[0.5em]
公共交通政策を事例として}

\vspace{1cm}

{\large Generative AI and Human Relationships in Policy Formation: \\[0.3em]
A Case Study of Public Transport Policy}

\vspace{3cm}

{\Large 氏名:永田 右京}

\vspace{1cm}

{\Large 指導教員:〇〇 教授}

\vspace{2cm}

{\Large 〇〇大学 大学院 〇〇研究科\\
〇〇専攻 博士後期課程}

\vspace{2cm}

{\Large 2026年 〇月 〇日 提出}

\end{center}
\end{titlepage}

% -----------------------------------------------------------------------------------------------------------------------------------------


% -----------------------------------------------------------------------------------------------------------------------------------------
% 謝辞
% -----------------------------------------------------------------------------------------------------------------------------------------
% -----------------------------------------------------------------------------------------------------------------------------------------
% 謝辞
% -----------------------------------------------------------------------------------------------------------------------------------------

\chapter*{謝辞}
\addcontentsline{toc}{chapter}{謝辞}

本論文の作成にあたり、多大なるご指導とご支援を賜りました〇〇大学教授〇〇先生に、心より感謝申し上げます。先生の温かいご指導と厳しいご助言なしには、本論文を完成させることはできませんでした。

また、〇〇研究科の諸先生方には、研究全般にわたり貴重なご助言をいただきました。深く感謝申し上げます。

〇〇研究所の皆様には、研究の場を提供していただき、有意義な議論を重ねることができました。心より御礼申し上げます。

共同研究者の皆様、特に〇〇様には、多くのデータ収集や分析においてご協力いただきました。深く感謝いたします。

日々の研究生活を共にした研究室の仲間たちには、多くの刺激と励ましをいただきました。皆様との議論は、本研究の発展に不可欠でした。

最後に、私を支え続けてくれた家族に深く感謝いたします。皆様の理解と励ましがあったからこそ、本研究を完遂することができました。

なお、本研究の一部は、日本学術振興会科学研究費補助金(課題番号:〇〇〇〇〇〇〇〇)の助成を受けたものです。

\vspace{1cm}
\begin{flushright}
2026年〇月〇日\\
永田 右京
\end{flushright}

% -----------------------------------------------------------------------------------------------------------------------------------------


% -----------------------------------------------------------------------------------------------------------------------------------------
% 要旨(日本語)
% -----------------------------------------------------------------------------------------------------------------------------------------
% -----------------------------------------------------------------------------------------------------------------------------------------
% 要旨(日本語)
% -----------------------------------------------------------------------------------------------------------------------------------------

\chapter*{要旨}
\addcontentsline{toc}{chapter}{要旨}

本研究は、公共交通政策を舞台に、生成AIと人間の協調的関係性を探求したものである。

第1章では、研究の背景、問題の所在、研究目的を論じた。公共交通政策における「連携・共創」の実装ギャップが、制度的欠陥だけでなく、人間の認知バイアスに起因する可能性を指摘した。

第2章では、Human-AI Policy、協調的ガバナンス論、認知バイアスと意思決定、ZK-SNARKsと政策評価に関する先行研究をレビューし、理論的空白を特定した。

第3章では、日本の公共交通政策の変遷と制度設計の現状を整理し、Japan MaaS 38プロジェクトの実証分析を通じて実装ギャップの実態を明らかにした。

第4章では、協調ロボット制御モデルを用いた計算論的分析により、現状維持バイアスの閾値効果、確証バイアスの逆説的効果、狭い視野の一貫した負の影響を解明した。

第5章では、生成AIを人間の「執政の創造性」を補完する「杖」として位置づけ、ZK-SNARKsの概念を援用した政策評価システムを提案した。Constitutional AI、市民討議、LLM as a Judgeを組み合わせた三層アーキテクチャを設計した。

第6章では、認知バイアスへの対処戦略と生成AIを組み込んだ制度設計への示唆を導出した。

第7章では、研究の総括と、都市計画への展開について論じた。

本研究は、生成AIと人間の関係性を理論的・実証的に探求し、より良い政策形成のための指針を提供した。

\vspace{1cm}
\noindent
\textbf{キーワード}:生成AI、政策形成、認知バイアス、公共交通政策、ZK-SNARKs、制度設計

% -----------------------------------------------------------------------------------------------------------------------------------------


% -----------------------------------------------------------------------------------------------------------------------------------------
% 要旨(英語)
% -----------------------------------------------------------------------------------------------------------------------------------------
% -----------------------------------------------------------------------------------------------------------------------------------------
% Abstract (English)
% -----------------------------------------------------------------------------------------------------------------------------------------

\chapter*{Abstract}
\addcontentsline{toc}{chapter}{Abstract}

This study explores the collaborative relationship between generative AI and humans, using public transport policy as an empirical context.

Chapter 1 discusses the research background, problem statement, and objectives. It points out that the implementation gap in "collaboration and co-creation" in public transport policy may be attributable not only to institutional deficiencies but also to human cognitive biases.

Chapter 2 reviews prior research on Human-AI Policy, collaborative governance theory, cognitive bias and decision-making, and ZK-SNARKs in policy evaluation, identifying theoretical gaps.

Chapter 3 examines the evolution of Japanese public transport policy and current institutional design, revealing the reality of implementation gaps through empirical analysis of 38 Japan MaaS projects.

Chapter 4 employs computational analysis using cooperative robot control models to uncover the threshold effect of status quo bias, the paradoxical effect of confirmation bias, and the consistently negative impact of narrow framing.

Chapter 5 positions generative AI as a "staff" that complements human "executive creativity," proposing a policy evaluation system utilizing ZK-SNARKs concepts. It designs a three-layer architecture combining Constitutional AI, citizen deliberation, and LLM as a Judge.

Chapter 6 derives strategies for addressing cognitive biases and implications for institutional design incorporating generative AI.

Chapter 7 summarizes the research and discusses future directions for urban planning applications.

This study provides theoretically and empirically grounded guidelines for better policy formation by exploring the relationship between generative AI and humans.

\vspace{1cm}
\noindent
\textbf{Keywords}: Generative AI, Policy Formation, Cognitive Bias, Public Transport Policy, ZK-SNARKs, Institutional Design

% -----------------------------------------------------------------------------------------------------------------------------------------


% -----------------------------------------------------------------------------------------------------------------------------------------
% 目次
% -----------------------------------------------------------------------------------------------------------------------------------------
{
  \hypersetup{linkcolor=black}
  \tableofcontents
}

% -----------------------------------------------------------------------------------------------------------------------------------------
% 図目次
% -----------------------------------------------------------------------------------------------------------------------------------------
\listoffigures

% -----------------------------------------------------------------------------------------------------------------------------------------
% 表目次
% -----------------------------------------------------------------------------------------------------------------------------------------
\listoftables

% -----------------------------------------------------------------------------------------------------------------------------------------
% 発表論文リスト
% -----------------------------------------------------------------------------------------------------------------------------------------
% -----------------------------------------------------------------------------------------------------------------------------------------
% 発表論文リスト
% -----------------------------------------------------------------------------------------------------------------------------------------

\chapter*{発表論文リスト}
\addcontentsline{toc}{chapter}{発表論文リスト}

\section*{査読付き論文}

\begin{enumerate}
    \item Nagata, U. (2025). "「日本版MaaS」は「交通まちづくり」の一類型として捉えられるか?目標とガバナンスについての一考察", 『日本評価学会誌』, Vol.XX, No.X, pp.XX-XX.
    
    \item Nagata, U. (2025). "Re-designing Collaboration and Co-creation in Regional Public Transport Policy: Integrated Approach to Cognitive Biases and Institutional Coordination", 土木学会論文集, Vol.XX, No.X, pp.XX-XX.
    
    \item Nagata, U. (2025). "ZK-SNARKs概念を援用した、ウィキッド・プロブレムに対応する政策評価の仕組み", 『公共政策学会誌』, Vol.XX, No.X, pp.XX-XX.
\end{enumerate}

\section*{学会発表}

\begin{enumerate}
    \item 永田右京 (2024). "地域公共交通計画の実装ギャップに関する分析", 日本公共政策学会2024年度大会.
    
    \item 永田右京 (2024). "認知バイアスが政策協調に与える影響の計算論的分析", 土木学会第XX回年次学術講演会.
    
    \item 永田右京 (2025). "生成AIと人間の協調的関係性の設計", 日本行政学会2025年度大会.
\end{enumerate}

% -----------------------------------------------------------------------------------------------------------------------------------------


% -----------------------------------------------------------------------------------------------------------------------------------------
% 本文開始
% -----------------------------------------------------------------------------------------------------------------------------------------
\newpage
\fancyhf{}
\pagestyle{fancy}
\fancyhead[LE,RO]{\leftmark}
\fancyfoot[CE,CO]{\rightmark}
\fancyfoot[LE,RO]{\thepage}
\pagenumbering{arabic}

\chapterfont{\centering}

% -----------------------------------------------------------------------------------------------------------------------------------------
% 第1章 序論
% -----------------------------------------------------------------------------------------------------------------------------------------
% -----------------------------------------------------------------------------------------------------------------------------------------
% 第1章 序論
% -----------------------------------------------------------------------------------------------------------------------------------------

\chapter{序論}
\label{chap:introduction}

% -----------------------------------------------------------------------------------------------------------------------------------------
\section{研究の背景}
\label{sec:intro_background}

% -----------------------------------------------------------------------------------------------------------------------------------------
\subsection{生成AIと人間の関係性という問い}

近年、生成AI(Generative AI)の急速な発展により、政策形成プロセスにおけるAI活用の可能性が広く議論されている。ChatGPTをはじめとする大規模言語モデル(Large Language Models, LLM)は、テキスト生成、要約、分析などのタスクにおいて人間と同等、あるいはそれ以上の性能を示す場面も増えている。しかし、AIが「人間を代替」するのではなく、「人間を補完」する関係性をどのように設計すべきかという問いは、依然として未解決のままである。

本研究では、生成AIを人間の「執政の創造性」\footnote{価値判断、コミュニケーション、新たな規範の創造といった人間固有の能力}を補完する「杖」として位置づけ、両者の協調的関係性のあり方を探求する。

% -----------------------------------------------------------------------------------------------------------------------------------------
\subsection{公共交通政策における「連携・共創」の潮流}

日本の公共交通政策においては、2002年の規制緩和以降、「連携」と「共創」が重要な政策概念として位置づけられてきた。特に2021年の「ポストコロナ時代の地域交通の共創に関する検討会」(国土交通省)においては、交通事業者による地域活性化、異業種との協働、コミュニティ参画という三つの次元での「共創」が提唱されている。

しかし、こうした政策的意図にもかかわらず、実装段階では多くの課題が指摘されている。

% -----------------------------------------------------------------------------------------------------------------------------------------
\section{問題の所在}
\label{sec:intro_problem}

% -----------------------------------------------------------------------------------------------------------------------------------------
\subsection{協調的ガバナンスの実装ギャップ}

公共交通政策における「連携・共創」は、制度的には整備されつつあるものの、実践レベルでは大きなギャップが存在する。例えば、Japan MaaSの38プロジェクトを分析した先行研究\footnote{第3章で詳述}によれば、92\%のプロジェクトが事業指標のみを重視し、社会的影响やアクセシビリティ改善を評価指標に組み込んだのは29\%に留まる。さらに、市民参加の仕組みを設けたプロジェクトはわずか5\%であった。

この実装ギャップは、単なる制度的欠陥だけでなく、人間の認知特性に起因する可能性がある。

% -----------------------------------------------------------------------------------------------------------------------------------------
\subsection{人間の認知限界と政策形成}

人間の意思決定は、認知バイアス(cognitive biases)の影響を強く受けることが知られている\cite{kahneman2011thinking}。特に政策形成プロセスでは、変化への抵抗や現状の維持選好を示す現状維持バイアス(Status Quo Bias)、自己の信念を確認する情報の優先的選択という確証バイアス(Confirmation Bias)、局所的最適化への固執や全体最適の見落としという狭い視野(Narrow Framing)が重要な影響を及ぼす。これらのバイアスは、ステークホルダー間の協調を阻害し、政策の実装ギャップを生む一因となっている可能性がある。

% -----------------------------------------------------------------------------------------------------------------------------------------
\subsection{生成AIの可能性と限界}

生成AIは、膨大な情報の処理、パターン認識、予測を行うことで、EBPM(証拠に基づく政策形成)を支援する強力なツールとなり得る。しかし、AIには本質的な限界も存在する。第一に、規範的判断・価値創造の不在である。AIは何が社会にとって「善い」のかを判断できない。第二に、文脈理解の困難性である。学習データの範囲外の「未知の状況」への適応には限界がある。第三に、「創造性」の源泉の欠如である。人間的な自発的な揺らぎやアナログな現実世界の機微を再現できない。

これらの限界を踏まえつつ、AIを「杖」として活用する関係性をどのように設計すべきかが問われている。

% -----------------------------------------------------------------------------------------------------------------------------------------
\section{研究目的と意義}
\label{sec:intro_purpose}

本研究の目的は、以下の三点である。第一に、公共交通政策における実装ギャップの要因として、人間の認知バイアスの影響を計算論的に解明する。第二に、生成AIと人間の協調的関係性を具体化するシステムとして、ZK-SNARKs型政策評価システムの可能性を探る。第三に、認知バイアスと生成AIを考慮した制度設計への示唆を導出する。

本研究の意義は、生成AIと人間の関係性を理論的・実証的に探求し、より良い政策形成のための指針を提供することにある。

% -----------------------------------------------------------------------------------------------------------------------------------------
\section{論文の構成}
\label{sec:intro_structure}

本論文は7章から構成される。第2章では先行研究のレビューを行い、Human-AI Policyの議論を中心に整理する。第3章では舞台としての公共交通政策の現状と課題を論じる。第4章では認知バイアスの政策協調への影響について計算論的分析を行う。第5章では生成AIと人間の関係性として、ZK-SNARKs型政策評価システムを提案する。第6章では制度設計への示唆を導出する。第7章では結論として、都市計画への展開について論じる。

% -----------------------------------------------------------------------------------------------------------------------------------------


% -----------------------------------------------------------------------------------------------------------------------------------------
% 第2章 先行研究のレビュー:Human-AI Policyの議論
% -----------------------------------------------------------------------------------------------------------------------------------------
% -----------------------------------------------------------------------------------------------------------------------------------------
% 第2章 先行研究のレビュー:Human-AI Policyの議論
% -----------------------------------------------------------------------------------------------------------------------------------------

\chapter{先行研究のレビュー:Human-AI Policyの議論}
\label{chap:literature_review}

% -----------------------------------------------------------------------------------------------------------------------------------------
\section{はじめに}
\label{sec:ch2_intro}

本章では、本研究の理論的基盤となる先行研究をレビューする。特に、「執政の創造性」とルーマン理論、生成AIと政策形成、人間-AI協調の理論、協調的ガバナンス論、認知バイアスと意思決定、ZK-SNARKsと政策評価の六つの領域を中心に整理し、理論的空白を特定する。

% -----------------------------------------------------------------------------------------------------------------------------------------
\section{「執政の創造性」とは何か}
\label{sec:ch2_executive_creativity}

% -----------------------------------------------------------------------------------------------------------------------------------------
\subsection{定義}

「執政の創造性」とは、以下のように定義される:

\begin{quote}
\textbf{執政の創造性}:社会の構成員同士のコミュニケーションを前提に、価値判断を基盤として、新たなガバナンスの範囲を生成・配分していく能力。
\end{quote}

この定義は、三つの要素から構成される。第一に、政策は、単一の主体による合理的決定ではなく、複数の主体間のコミュニケーションを通じて形成される。第二に、政策決定は統計的平均や客観的指標への還元が不可能な、主体的な価値判断を要する。第三に、政策は固定された枠組みではなく、状況に応じて新たな範囲を創造し続ける動的なプロセスである。

この概念は、カンギレム(Canguilhem)の「規範の創造」概念\cite{canguilhem1966}および井庭の創造システム理論\cite{iba2009}を統合したものである。カンギレムは、「正常」とは統計的平均ではなく、「生の独自的な規範性」を肯定することであると論じた。健康とは客観的指標ではなく、主体によって体験される価値である。同様に、政策における「正しさ」も、統計的効率性ではなく、主体による価値判断に基づく。

井庭の創造システム理論では、コミュニケーションの「わかり合えなさ」から出発し、「発見」を要素とするシステム上で創造が起こるとされる。本研究は、この創造のプロセスを政策の文脈に適用し、「執政の創造性」として定式化する。

% -----------------------------------------------------------------------------------------------------------------------------------------
\subsection{政策過程モデルにおける価値判断}

「執政の創造性」の重要性は、既存の政策過程モデルを分析することで裏付けられる。主要な政策過程モデルを見てみよう。

\subsubsection{政策の窓モデル}

キングドン\cite{kingdon1984}\cite{kusano1997}による政策の窓モデルは、政策過程に「問題」「政策代替案」「政治」という3つの独立した流れが存在し、それらが特定の時点で合流したときに政策決定が生じることを示す。

キングドンは、この3つの流れが合流する瞬間を「政策の窓」が開くと表現する。窓は一時的にしか開かず、逃したら再び開くまで長く待たなければならない。

このモデルが示すのは、政策決定は論理的な手順ではなく、問題と解決案と政治的機会の偶然の出会いによって生じるということである。この「偶然の出会い」を見極め、窓が開いた瞬間に行動するには、人間の価値判断と政治的勘が不可欠である。AIは過去のデータから傾向を分析することはできても、「今がチャンスだ」と感じ取り、その瞬間に動くという状況認識と行動のタイミングについては、人間の判断に依存せざるを得ない。

\subsubsection{唱導連携モデル}

唱導連携モデル(Advocacy Coalition Framework)は、サバティエ\cite{sabatier1988}らによって開発されたモデルであり、特定の政策分野において、共通の信念体系を共有する主体たちが連携(コアリション)を形成し、政策を唱導するプロセスを分析する。

このモデルの特徴は、政策を単なる利害調整ではなく、信念体系の競合として捉える点にある。各連携は、(1)深層核信念(基本的価値観)、(2)政策核信念(具体的政策目標)、(3)二次的側面(手段的判断)という階層的な信念体系を持つ。

ここで重要なのは、どの信念を優先すべきか、どの連携の主張を採用すべきかという判断が、常に価値判断を伴うということである。統計的データや客観的分析だけでは、信念の競合を解決できない。どの価値を重視するかという判断が必要であり、これは人間の創造的適応能力に依存する。

\subsubsection{村松モデル}

村松\cite{muramatsu1981}モデルは、日本の政策過程を「与党・官僚・利益団体」の三者関係として分析する枠組みである。このモデルでは、政策決定はこれら三者の交渉と取引を通じて行われる。

真渕による修正版では、野党勢力や新規参入者の役割も考慮される。既存の産業組織が維持しようとする価値と、新規参入者が求める変革の価値が衝突する中で、政治家はバランスを取る必要がある。

このモデルが示すのもまた、利害の調整と価値の優先順位付けが政治的本質であり、それは計算可能な最適化問題ではないということである。三者間の「妥当な落としどころ」を見出すには、人間同士の交渉と相互調整が不可欠である。

% -----------------------------------------------------------------------------------------------------------------------------------------
\subsection{ウィキッド・プロブレムとの関連}

公共政策の多くは、Rittel \& Webber\cite{rittel1973}\cite{sugitani2021}の「ウィキッド・プロブレム(Wicked Problems)」の性質を帯びている。ウィキッド・プロブレムとは、決定的な問題定義がなく、問題そのものが何であるかについて利害関係者の間で合意がない問題である。停止ルールもなく、いつ問題が「解決」されたのかを客観的に判断できない。解は「良い/悪い」ではなく「より良い/より悪い」であり、最適解は存在せず、複数の利害のバランスを取るしかない。また、解を適用した結果、予期せぬ副作用が生じる可能性があり、社会実験はやり直しが効かない。さらに、すべてのウィキッド・プロブレムは本質的にユニークであり、過去の経験から単純に適用できない。

ウィキッド・プロブレムへの対応は、終わりのない創造的プロセスである。そして、「終わりがないからこそ、政治がその不快さを受け入れる必要がある」のである。この終わりのない創造的プロセスこそが、人間による政策の固有の意味であり、AIには代替不可能な領域である。

% -----------------------------------------------------------------------------------------------------------------------------------------
\section{ルーマン理論による基礎づけ}
\label{sec:ch2_luhmann}

% -----------------------------------------------------------------------------------------------------------------------------------------
\subsection{なぜルーマンか}

コミュニケーションを理論的基軸とする場合、ユルゲン・ハーバーマスの「コミュニケーション的行為理論」\cite{habermas1981}も選択肢として存在する。しかし、本研究はルーマン\cite{luhmann1984}の立場を採用する。その理由を説明する。

ハーバーマスは、コミュニケーションを「了解志向的」な行為として捉える。コミュニケーションの理想的状態では、参加者が互いに「了解」に到達し、合意を形成することが期待される。この立場からは、コミュニケーションの「失敗」や「誤解」は、理想的言語状況からの逸脱として問題視される。

これに対し、ルーマンはコミュニケーションを「了解の否定」を内包するプロセスとして捉える。ルーマンによれば、コミュニケーションは常に「理解」と「誤解」の双方を可能性として含んでおり、「わかり合えなさ」こそがコミュニケーションの本質的な特徴である。

本研究がルーマンの立場を採用する理由は、以下の二点にある。

第一に、「わかり合えなさ」を「創造の源泉」として位置づける視点が必要である。ハーバーマス的な「理想的了解」を前提とすれば、AIも「十分に良い」情報処理を行うことで「機能的な了解」に貢献できると主張しうる。しかし、ルーマン的な視点からは、「わかり合えなさ」から生じる価値判断と意味構成のプロセスこそが創造の核心であり、これをAIは代替できない。

第二に、政策的決定における「価値競合」の不可避性も重要である。公共政策は、ハーバーマスが想定する「理想的言語状況」において理性の力だけで解決可能な問題ではなく、複数の正当な価値が競合するウィキッド・プロブレムとしての性質を持つ。このような状況では、「了解」への到達よりも、「わかり合えなさ」を前提とした創造的適応こそが求められる。

% -----------------------------------------------------------------------------------------------------------------------------------------
\subsection{コミュニケーションの3段階:情報・伝達・理解}

ルーマン\cite{luhmann1984}\cite{kneer1993}によれば、コミュニケーションは「情報(Information)・伝達(Mitteilung)・理解(Verstehen)」という3つの選択過程から構成される。

\begin{description}
    \item[情報(Information)] 多数の可能性の地平からの一つの選択であり、何を語るかの選択である。
    \item[伝達(Mitteilung)] 多数の伝達可能性からの選択であり、いかに語るかの選択である。
    \item[理解(Verstehen)] 多数の理解可能性からの選択であり、いかに受け止めるかの選択である。
\end{description}

ルーマンは、「三つの選択のはたらきのすべてが総合されるときにはじめてコミュニケーションというものが成り立つ」と強調する。この3層構造は、コミュニケーションが単純な情報伝達ではなく、各段階で選択と解釈が介在する複雑な過程であることを示している。

% -----------------------------------------------------------------------------------------------------------------------------------------
\subsection{構造的カップリング:本研究の核心概念}

ここで重要になるのが、ルーマンにおける「構造的カップリング(strukturelle Kopplung)」\cite{luhmann1984}\cite{maturana1980}の概念である。これは、作動上は完全に独立(閉鎖)している複数のシステムが、互いに不可欠な環境として影響し合う関係を指す。

ルーマン理論において、最も重要な構造的カップリングは、「心的システム(意識)」と「社会システム(コミュニケーション)」の関係である。心的システムは思考し、社会システムはコミュニケーションする——それぞれ別の作動を行う。しかし、心的システムがなければコミュニケーションは発生しない。意識はコミュニケーションに対して、刺激や誘発を与えたり、あるいは邪魔をしたりすることができる。ただし、意識がコミュニケーションを「因果的に決定」するわけではない。意識はあくまで環境として、コミュニケーション・システムに「刺激」を与え、システム側がそれを独自の論理で処理する\footnote{ここで注意すべきは、ルーマン理論において「認知」は心的システム内部に所在し、社会的システムに「分布」しているわけではないことである。構造的カップリングは、社会的コミュニケーションが心的システムを刺激することを可能にするが、認知プロセスそのものは心的システムのオートポイエーシスとして完結する。}。

% -----------------------------------------------------------------------------------------------------------------------------------------
\subsection{「理解の創造性」の定義}

以上の理論的整理を踏まえ、本研究で論じる「理解の創造性」を以下のように定義する:

\begin{quote}
\textbf{理解の創造性}:心的システムと社会システムの構造的カップリングにおいて、心的システムから社会システムへの刺激として提供される、価値判断を伴う意味構成のプロセス。
\end{quote}

この定義は、ルーマンの厳密な意味での「理解」(コミュニケーション接続)とは区別される。本研究が着目するのは、コミュニケーション接続を駆動する「価値判断の源泉」としての心的システムの役割である。

この観点から、「わかり合えなさ」は単なる誤解ではなく、構造的カップリングにおいて各心的システムが独自の価値判断に基づいて意味を構成する結果として生じる、コミュニケーションの本質的な特徴として理解される。

この価値判断を伴う意味構成のプロセスこそが、人間固有の創造性であり、本研究が「執政の創造性」として定式化する対象である。

% -----------------------------------------------------------------------------------------------------------------------------------------
\subsection{AIの原理的限界}

ルーマンの理論において、機械はオートポイエーシス・システム(生命・意識・社会)とは区別される「非ポイエティック」な存在とされている\cite{luhmann1984}。AIシステムも基本的には計算機プログラム(機械)上で動作するシステムである。AI——その最先端の形態を含めて——は心的システムに該当しない。

\textbf{オートポイエティックでない}:たとえ自己学習能力を持つAIであっても、その「学習」は人間が設計したアルゴリズムと訓練データに依存している。外部からの入力なしに自律的に自身の「思考」を産出し続ける閉鎖的なネットワークを持たない。

\textbf{自己言及的な意味処理を行わない}:AIの確率的出力は、文脈に応じて「もっともらしい」次のトークンを選択するが、この選択プロセスは自己言及的ではない。AIは「この意味を選択したこと自体」を次の処理の地平として開くことはなく、単に統計的パターンに基づいて出力を生成する。

\textbf{価値判断を伴わない}:心的システムからの刺激は「何が重要か」「何を優先すべきか」という規範的判断を前提とするが、AIの出力は統計的パターンに基づいており、独自の規範的判断を伴わない。

以上の議論から、公共政策におけるAIの原理的限界が明らかになった。AIは「情報」の選択や「伝達」の補助には機能しうるが、「理解」の創造的プロセスには原理的に参加できない。AIは心的システムを持たないため、価値判断を伴う意味構成を行えないのである。

% -----------------------------------------------------------------------------------------------------------------------------------------
\section{Human-AI Policy:政策形成におけるAIと人間の関係性}
\label{sec:ch2_human_ai_policy}

% -----------------------------------------------------------------------------------------------------------------------------------------
\subsection{AI政策論の展開}

公共政策におけるAI活用に関する議論は、2010年代後半から急速に発展してきた。初期の議論は、AIによる行政サービスの効率化や自動化に焦点が置かれていたが、近年ではAIと人間の関係性そのものが問いの中心となっている\cite{shneiderman2022}。

この転換の背景には、生成AI(ChatGPT、Claude、Gemini等)の登場がある。これらの技術は、従来のAI(分類・予測)とは異なり、創発的なテキスト生成能力を持つ。この能力は、政策文書の作成、選択肢の生成、市民との対話など、政策形成の核となるプロセスに直接関与しうる。

% -----------------------------------------------------------------------------------------------------------------------------------------
\subsection{Human-Centered AIの理念}

Shneiderman (2022) \cite{shneiderman2022} は、Human-Centered AI(HCAI)の理念として、以下の2軸マトリクスを提示している:

\begin{description}
    \item[高自動化・低制御] AIが自律的に判断し、人間は結果を受け入れるのみ
    \item[高自動化・高制御] AIが提案を行い、人間が最終判断を下す
    \item[低自動化・高制御] 人間が主導し、AIが補助的な役割を果たす
    \item[低自動化・低制御] 人間もAIも十分に機能しない状態
\end{description}

本研究が着目するのは「高自動化・高制御」の領域である。この領域では、AIの計算能力と人間の規範判断力が相互に補完し合う。

% -----------------------------------------------------------------------------------------------------------------------------------------
\subsection{AIの「杖」としての位置づけ}

生成AIを「杖(Aaron's rod)」として位置づける視点は、AIが人間を代替するのではなく、人間の能力を補完・増幅する道具として活用する考え方である。

この視点からは、以下の設計原則が導かれる:
\begin{enumerate}
    \item AIは人間の最終判断を前提とする
    \item AIの限界を明示的に理解する
    \item 人間-AI協調のプロセスを透明化する
    \item AI自体のバイアスに対処する
    \item 説明責任は常に人間が負う
\end{enumerate}

この「杖」としての位置づけは、AIを「執政の創造性」を支援する道具として捉え直す視点を提供する。

生成AIは、以下の領域では人間を補完し得る:
\begin{itemize}
    \item データの処理・分析
    \item 選択肢の生成・提示
    \item 文書作成の効率化
    \item 多様な視点の提示
    \item 認知バイアスの指摘(「悪魔の代理人」機能)
\end{itemize}

一方で、以下の領域では人間の役割が不可欠である:
\begin{itemize}
    \item 規範的判断(何が「善い」か)
    \item 文脈に応じた柔軟な対応
    \item 新たな価値の創造
    \item 政治的なアカウンタビリティ
    \item 最終的な責任の所在
\end{itemize}

% -----------------------------------------------------------------------------------------------------------------------------------------
\section{協調制御理論と社会システムへの応用}
\label{sec:ch2_cooperative_control}

% -----------------------------------------------------------------------------------------------------------------------------------------
\subsection{協調制御理論の基礎}

協調制御理論(Cooperative Control Theory)は、複数の自律エージェントが協調して共通の目標を達成するための制御手法を研究する分野である\cite{yoshihara2009cooperative}。

この理論は、以下の特徴を持つ:
\begin{itemize}
    \item 分散的な意思決定
    \item 局所的な情報に基づく協調
    \item 全体的な目標の達成
\end{itemize}

% -----------------------------------------------------------------------------------------------------------------------------------------
\subsection{社会システムへの応用可能性}

協調制御理論は、社会システムの分析にも応用可能である。特に、複数のステークホルダーが関与する政策ネットワークにおいて、各主体が自律的に行動しながら全体としての政策目標を達成するプロセスをモデル化できる。

本研究では、協調ロボット制御モデルを用いて、政策ネットワークにおけるステークホルダー間の協調を分析する(第4章で詳述)。

% -----------------------------------------------------------------------------------------------------------------------------------------
\section{協調的ガバナンス論}
\label{sec:ch2_collaborative_governance}

% -----------------------------------------------------------------------------------------------------------------------------------------
\subsection{協調的ガバナンスの定義}

Ansell and Gash (2008) \cite{ansell2008collaborative} は、協調的ガバナンスを以下のように定義している:

\begin{quote}
「一つまたは複数の公共機関が、非政府のステークホルダーを、合意形成志向で審議的な集団的意思決定プロセスに直接関与させる統治のあり方」
\end{quote}

% -----------------------------------------------------------------------------------------------------------------------------------------
\subsection{公共交通における連携・共創}

日本の公共交通政策においては、「連携」と「共創」が重要な概念として位置づけられている\cite{kato2009community}。Kato et al. (2009) は、コミュニティ参加型地域公共交通の成功条件として以下を指摘している:

\begin{enumerate}
    \item 関係ステークホルダー間での認識と責任分担の共有
    \item 各ステークホルダーが参加から利益を得られること
    \item ステークホルダーを調整するキーパーソンの存在
    \item ステークホルダーの努力が利用促進・価値向上につながること
\end{enumerate}

% -----------------------------------------------------------------------------------------------------------------------------------------
\subsection{実装ギャップの指摘}

しかし、こうした理論的条件にもかかわらず、実践レベルでは多くの課題が指摘されている。Emerson et al. (2012) \cite{emerson2012integrative} は、協調的ガバナンスが直面する課題として以下を指摘している:

\begin{itemize}
    \item 高い取引コスト
    \item 最小公約数的な解決策への収束
    \item 組織された利益による捕捉
\end{itemize}

% -----------------------------------------------------------------------------------------------------------------------------------------
\section{認知バイアスと意思決定}
\label{sec:ch2_cognitive_bias}

% -----------------------------------------------------------------------------------------------------------------------------------------
\subsection{行動経済学の基礎概念}

Kahneman (2011) \cite{kahneman2011thinking} は、人間の思考を「システム1(速い思考)」と「システム2(遅い思考)」に分類し、認知バイアスがシステム1の特性に起因することを示した。

% -----------------------------------------------------------------------------------------------------------------------------------------
\subsection{政策プロセスにおける認知バイアス}

政策形成において特に重要な認知バイアスとして、以下の三つを取り上げる:

\subsubsection{現状維持バイアス(Status Quo Bias)}
Samuelson and Zeckhauser (1988) \cite{samuelson1988status} によって提唱された概念で、変化よりも現状を維持することを好む傾向を指す。

\subsubsection{確証バイアス(Confirmation Bias)}
既存の信念や仮説を支持する情報を優先的に探し、反証する情報を無視・軽視する傾向\cite{russio2015confirmation}。

\subsubsection{狭い視野(Narrow Framing)}
問題を孤立して考え、より広い文脈や長期的な影響を考慮しない傾向\cite{kahneman2011thinking}。

% -----------------------------------------------------------------------------------------------------------------------------------------
\section{ZK-SNARKsと政策評価}
\label{sec:ch2_zksnarks}

% -----------------------------------------------------------------------------------------------------------------------------------------
\subsection{ZK-SNARKsの基本概念}

ZK-SNARKs(Zero-Knowledge Succinct Non-interactive Arguments of Knowledge)は、暗号技術の一種であり、秘密情報を公開することなく、その情報の正しさを証明する技術である\cite{ben2014succinct}。

ZK-SNARKsは以下の4つの特性を持つ:
\begin{description}
    \item[Zero-Knowledge] 証明を通して元の秘密情報が一切漏洩しない
    \item[Succinct] 証明サイズが常に数百バイト程度と一定
    \item[Non-interactive] 証明者から検証者への1回の送信で証明完了
    \item[Arguments of Knowledge] 真の知識を所有している必要があり偽造不可能
\end{description}

% -----------------------------------------------------------------------------------------------------------------------------------------
\subsection{政策評価への応用可能性}

ZK-SNARKsの概念を政策評価に応用することで、「秘密を守りながら専門性を証明する」仕組みが実現可能になる。例えば、企業が自社の技術情報を公開せずに、政策課題への貢献可能性を証明できる。

本研究では、ZK-SNARKsの概念を援用した政策評価システムをLLM as a Judgeと組み合わせて提案する(第5章で詳述)。

% -----------------------------------------------------------------------------------------------------------------------------------------
\section{小括:理論的空白の特定}
\label{sec:ch2_summary}

先行研究のレビューから、以下の理論的空白が明らかになった:

\begin{enumerate}
    \item \textbf{「執政の創造性」の理論化}:ルーマンの社会システム理論を用いて「執政の創造性」を基礎づけ、AIと人間の役割分担を明確にした研究は限定的である
    \item \textbf{認知バイアスと政策協調の接続}:協調的ガバナンスの失敗要因として認知バイアスに着目した研究は限定的である
    \item \textbf{計算論的分析手法の欠如}:政策協調プロセスを計算論的にモデル化した研究は少ない
    \item \textbf{ZK-SNARKs概念の政策評価への応用}:暗号技術の概念を政策評価に応用した試みは先駆的である
\end{enumerate}

本研究は、これらの空白を埋めることを目指す。

% -----------------------------------------------------------------------------------------------------------------------------------------


% -----------------------------------------------------------------------------------------------------------------------------------------
% 第3章 舞台としての公共交通政策:現状と課題
% -----------------------------------------------------------------------------------------------------------------------------------------
% -----------------------------------------------------------------------------------------------------------------------------------------
% 第3章 舞台としての公共交通政策:現状と課題
% -----------------------------------------------------------------------------------------------------------------------------------------

\chapter{舞台としての公共交通政策:現状と課題}
\label{chap:public_transport_policy}

% -----------------------------------------------------------------------------------------------------------------------------------------
\section{はじめに}
\label{sec:ch3_intro}

本章では、本研究の「舞台」となる公共交通政策の現状と課題を整理する。まず日本の公共交通政策の変遷を概観し、次に制度設計の現状を分析し、最後にJapan MaaSプロジェクトの実証分析を通じて実装ギャップの実態を明らかにする。

本章における分析モデルとして、政策過程論を採用する。政策過程は、目的を設定した上で現状を理解し、そのギャップを課題として認識した上でそれを解決するための方策を列挙し、実施する政策を決定するという一連の流れである。この中で、どのように目標を立て、またどのように目標を設定しているかが、本章における大まかな分析対象である。

本章でいう「交通まちづくり」の概念は、太田(2008)の述べるような「交通に関連する地域の課題への対応をベースにして、市民と行政が協働して進めるまちづくり」である。原田編(2015)の述べるところの定義である「まちづくりの目的に貢献する交通計画」とは異なり、「市民の参加と協働により発展的に進める活動プロセス」である。この中では、「交通計画における住民、市民の参加」「都市計画、都市づくりとの連携」の2点が重要視されている。

% -----------------------------------------------------------------------------------------------------------------------------------------
\section{日本の公共交通政策の変遷}
\label{sec:ch3_transition}

% -----------------------------------------------------------------------------------------------------------------------------------------
\subsection{5つの時期区分}

日本の地域公共交通政策は、2002年の規制緩和以降、大きく5つの時期に区分できる\cite{nagata2024maas}:

\begin{description}
    \item[第1期:規制緩和期(1997-2002)] 市場原理の導入、路線許可制から届出制への移行
    \item[第2期:地域協議導入期(2003-2008)] 地域公共交通会議制度の創設
    \item[第3期:民主党政権期(2009-2012)] 地方分権の最大化
    \item[第4期:網形成計画期(2013-2017)] ネットワーク再構築の重視
    \item[第5期:効率重視期(2018-現在)] 生産性向上・効率化の強調
\end{description}

% -----------------------------------------------------------------------------------------------------------------------------------------
\subsection{STOフレームワークによる分析}

Van de Velde (1999) \cite{vandevelde1999organisational} が提唱したSTOフレームワークは、公共交通政策を3つの意思決定階層で分析する手法である:

\begin{table}[htbp]
\centering
\caption{STOフレームワークによる政策分析}
\label{tab:sto_framework}
\begin{tabular}{llll}
\toprule
階層 & 主要な意思決定事項 & 想定期間 & 主な責任主体 \\
\midrule
S (Strategy) & 政策目標、長期ビジョン & 10年〜 & 国、広域自治体 \\
T1 (上位戦術) & サービス水準目標、路線網骨格 & 1-5年 & 国、自治体 \\
T2 (下位戦術) & 具体的ダイヤ・ルート & 半年-2年 & 交通事業者 \\
O (Operations) & 日常運行、維持管理 & 日々〜月 & 交通事業者 \\
\bottomrule
\end{tabular}
\end{table}

% -----------------------------------------------------------------------------------------------------------------------------------------
\section{制度設計の現状}
\label{sec:ch3_institutional_design}

% -----------------------------------------------------------------------------------------------------------------------------------------
\subsection{官民連携(PPP)と官官連携(PuP)}

公共交通政策における連携形態は、大きく分けて官民連携(Public-Private Partnership, PPP)と官官連携(Public-Public Partnership, PuP)がある。

\subsubsection{官民連携(PPP)の事例}
\begin{itemize}
    \item 空港コンセッション(北海道・関西エアポート)
    \item 駅無人化への地域対応(滝沢市とJR東日本)
    \item デマンド交通の導入(雲南市だんだんタクシー)
\end{itemize}

\subsubsection{官官連携(PuP)の事例}
\begin{itemize}
    \item 横の連携:盛岡都市圏地域公共交通計画
    \item 縦の連携:地域公共交通会議制度
    \item 階層型連携:前橋都市圏の自治体バス広域連携
\end{itemize}

% -----------------------------------------------------------------------------------------------------------------------------------------
\subsection{地域公共交通計画の義務化}

2014年の法改正により、地域公共交通網形成計画の策定が自治体に義務化された。この制度は、国がSレベルの枠組みを提供し、自治体がT1レベルで具体化する役割分担を想定している。

しかし、自治体の資源制約により、形式的な計画策定に留まるケースも指摘されている。

% -----------------------------------------------------------------------------------------------------------------------------------------
\section{実装ギャップの実証分析:Japan MaaS}
\label{sec:ch3_maaS_analysis}

% -----------------------------------------------------------------------------------------------------------------------------------------
\subsection{分析の枠組みとリサーチクエスチョン}

本章では、いわゆる「日本版MaaS」を「交通まちづくり」実践の一類型として捉え、以下の2つの問いについて検討する:

\begin{enumerate}
    \item 「日本版MaaS」の政策進捗において目標設定がどこに置かれているか
    \item 立場を問わない市民による批判回路がどのように整備されているのか
\end{enumerate}

本研究は、日本版MaaSを「MaaS」としてではなく、「交通まちづくり」の一類型として分析し、その目的意識と計画における責任体系の構築状況、すなわちガバナンス状況を探るものである。

分析手法について、「計画における計画指標(Key Performance Indicator, KPI)に事業利用状況以外の指標がある」「会議体において、交通事業者、システム供給者、学識者、自治体以外の市民参加がある」の二つの点を検討し、交通政策を地域政策として運営できているかの評価とする。

% -----------------------------------------------------------------------------------------------------------------------------------------
\subsection{分析対象}

日本版MaaSは、その事業概要を一定のフォーマットに落とし込んで申請する仕組みとなっており、国土交通省は採択のたびにそれらをまとめた文書を公開している。これらには協議会の構成員や目標数値が記載されており、これを分析対象とする。

2020年の日本版MaaS推進事業では、全国36の事業が認定されており、提出された計画については全件採択となっている。本事業によって実施された事業の特性を説明するために、日本版MaaSとして展開された各事業について、要素技術の利用状況を確認し、類型化を試みる。

日本版MaaSの主な要素技術として、サブスク・定期など企画乗車券、オンラインで交通情報が見られる、あるいは決済できるアプリケーション、病院、商業クーポンなど外部情報の提供、デマンド交通、シェアサイクル等の新モビリティ、さらにデータ連携基盤の構築や利用が認められた。特にアプリケーションの提供が31事例、外部情報との連携が32事例と多かったものの、企画乗車券の導入と新モビリティの導入比率は低く、またデータ連携基盤の利用や構築は12事例と少なかった。

% -----------------------------------------------------------------------------------------------------------------------------------------
\subsection{分析結果}

分析の結果、双方の判断を満たし、地域政策として日本版MaaSを運用している事例は存在しなかった。

第一に、交通事業やMaaS事業の指標を導入している事業は32/36であったが、それ以外の地域目標を設定している事業は10のみであった。第二に、自治体が参加している事業に限定すると、パブリックコメントや協議会への市民参加を確認できた事業は2/32にとどまった。

\begin{table}[htbp]
\centering
\caption{Japan MaaS 38プロジェクトの評価指標分析}
\label{tab:maas_analysis}
\begin{tabular}{lcc}
\toprule
指標タイプ & プロジェクト数 & 割合 \\
\midrule
事業指標のみ & 32 & 89\% \\
非事業指標を含む & 10 & 28\% \\
市民参加の仕組み & 2 & 6\% \\
\bottomrule
\end{tabular}
\end{table}

この結果は、「連携・共創」の政策的意図が、実践レベルでは事業効率性の追求に偏向していることを示している。

% -----------------------------------------------------------------------------------------------------------------------------------------
\subsection{構造的要因}

実装ギャップの背景には、以下の構造的要因が存在する:

\begin{enumerate}
    \item \textbf{自治体の資源制約}:「逆三角形」負担構造(地方分権改革推進有識者会議, 2023)
    \item \textbf{技術的複雑性}:MaaS導入に必要な専門知識の不足
    \item \textbf{ステークホルダー間の利害対立}:交通事業者、自治体、市民の間での優先順位の相違
    \item \textbf{認知的要因}:意思決定における認知バイアスの影響(第4章で詳述)
\end{enumerate}

新たなガバナンスシステムの導入へのコストが高いことも要因として挙げられる。一般に、デジタル時代の政府活動への移行には、それに即した評価・批判のシステムへの移行が必要であるが、それには取引費用がかかりすぎることが分かっている。また、自治体の持つ政策評価システムが業績評価に偏っている点も指摘されている。三重県の「さわやか運動」から始まった政策評価では、担当者が直接事業内容を説明する「事務事業評価」が核であり、その中で簡略化された評価として業績評価が利用されることが多い。これは一種の日本の評価システムの文化であり、日本版MaaSもその流れを汲んでいる可能性がある。

% -----------------------------------------------------------------------------------------------------------------------------------------
\subsection{事例分析}

分析の射程を示すため、2つの事例を取り上げる。

\subsubsection{地方版MaaSの広域連携基盤構築モデル事業(ひたち圏域)}

日立市を中心とする茨城県北部の圏域におけるMaaS事業展開について、3市1村の協力の下で行われた日本版MaaS事業である。

本事業の目標は3つに分かれている。一つ目は「利用者関連指標」であり、取組ページへのアクセス数、アプリDL数、チケット販売数などが設定されている。二つ目は「交通事業者関連指標」であり、参加する事業者数により達成される。三つ目は「MaaS事業者関連指標」であり、接続事業者数を目標と据えている。

市民参加状況については、市民団体らしき仕組みは確認できず、市民代表者の参加形跡も確認できなかった。また、市民参加、パブリックコメントなどの情報を検索したが、見当たらなかった。

\subsubsection{鞆の浦MaaS}

広島県福山市の景勝地、鞆の浦におけるMaaS事業展開である。本事業の目標は、デジタルチケット発行数、アンケート指標(満足度70\%以上など)、電動レンタサイクル利用者の回遊エリア拡大状況に分かれている。

市民参加状況については、市民団体らしき仕組みは確認できなかったが、観光関連団体の代表として「公益社団法人福山観光コンベンション協会」が出席しており、これは交通事業者外の参加と認められる。

% -----------------------------------------------------------------------------------------------------------------------------------------
\subsection{日本版MaaSは交通まちづくりの一類型として捉えられるか}

以上の議論を踏まえて、日本版MaaSを交通まちづくりとして捉えられるかについて議論する。

交通まちづくりの基本線は、初期構想段階では市民参加を含んでいた以外は一貫しており、地域の課題を解決する方針のもと交通を計画し、提供するものであった。

これを踏まえると、日本版MaaSはその構想段階では交通まちづくりの要件に合うものの、実装段階においてはその要件に合致しないと理解できる。前者については、日本版MaaSの特徴として地域課題を解決する交通サービスの提供を支援する政策である点が該当する。一方で後者については、日本版MaaSの計画において交通サービスの利用状況が評価の中心にあり、必ずしも地域課題解決の状況をモニタリングしていない点が該当する。

% -----------------------------------------------------------------------------------------------------------------------------------------
\section{小括:公共交通政策を「実験場」として位置づける理由}
\label{sec:ch3_summary}

公共交通政策は、以下の理由から本研究の「実験場」として適している:

\begin{enumerate}
    \item \textbf{複雑なステークホルダー関係}:国、自治体、交通事業者、市民など多様な主体が関与
    \item \textbf{明確な政策目標}:持続可能な公共交通の維持・発展
    \item \textbf{実装ギャップの可視化}:制度的意図と実践の乖離が観察可能
    \item \textbf{データの入手可能性}:政策文書、統計データへのアクセスが比較的容易
\end{enumerate}

本章の分析から、以下の示唆が得られた:

第一に、日本版MaaSは構想段階では「交通まちづくり」と捉えられるが、実装段階ではそうではない。政策的意図として「連携・共創」が謳われていても、実践レベルでは事業効率性の追求に偏向してしまう構造的課題が存在する。

第二に、交通政策の理念として、交通の機能について「地域の活力に貢献」するように施策を運営するためには、少なくとも地域がどのようになるべきかの指標を入れる必要がある。交通事業やMaaS事業の指標のみでは、こうした運営には不十分である。

第三に、行政機関が手がける交通サービスに対する批判回路についても、十分に構築されていない状況が見えてきた。公共交通政策は多くの場合、土木工学を主軸とした専門性を問われることから、こうした回路が機能しにくいと推察される。

「利用者視点」を導入しているとうたう交通政策が各地で展開されているが、少なくとも政策において相手にするのは市民であり、この国の主権者である。専門性が必ずしも正しさではなく、目指す方向性はイデオロギーであり批判されうることを前提として、民主主義国家の計画のあり方を交通の面から探る作業が求められる。

次章では、この実装ギャップの認知的要因を計算論的に分析する。

% -----------------------------------------------------------------------------------------------------------------------------------------


% -----------------------------------------------------------------------------------------------------------------------------------------
% 第4章 認知バイアスの政策協調への影響:計算論的分析
% -----------------------------------------------------------------------------------------------------------------------------------------
% -----------------------------------------------------------------------------------------------------------------------------------------
% 第4章 認知バイアスの計算論的分析
% -----------------------------------------------------------------------------------------------------------------------------------------

\chapter{認知バイアスの計算論的分析}
\label{chap:cognitive_bias}

% -----------------------------------------------------------------------------------------------------------------------------------------
\section{はじめに}
\label{sec:ch4_introduction}

% -----------------------------------------------------------------------------------------------------------------------------------------
\subsection{研究の背景と問題設定}

2002年の規制緩和以降、日本の地域公共交通政策は大きく変容し、従来の規制中心的アプローチから、多様なステークホルダー間の「連携」と「共創」を強調するアプローチへと移行してきた。この変化は、政府機関、交通事業者、市民、その他の事業者など、複数の主体の専門知とリソースを活用して複雑な政策課題に対処しようとする、公共ガバナンスにおけるより広範な傾向を反映している。

こうした変革の制度化は、近年の政策展開によってさらに加速している。地域公共交通計画の策定が努力義務となり、ステークホルダー間の効率的な情報交換と協議を促進するための会議体の設立が強調されている。この制度的枠組みは、連携と共創のプロセスが機能すべき正式な構造を提供している。

連携と共創に関する政策言説は、国土交通省による2021年の政策フレームワーク「ポストコロナ時代における地域交通の共創に関する研究会」において特に顕著になった。このフレームワークは、地域社会の活性化に貢献する人の流れを積極的に創出する交通事業者、交通事業者と他産業との連携、そして交通を共同責任として捉えるコミュニティの参画という3つの主要な次元を含む「共創型交通」を明示的に求めている。

しかし、連携と共創に対する政策の強調にもかかわらず、実証的証拠は政策意図と実装成果の間に大きな乖離があることを示唆している。協調的枠組みの普及は、必ずしもより効果的または持続可能な交通ソリューションに結びついておらず、公共交通ガバナンスにおいて連携と共創がどのような条件の下で効果的に機能できるかという根本的な問いを提起している。

この実装ギャップは、日本の地方自治体が直面する構造的な資源制約によってさらに悪化している。2023年の地方分権改革有識者会議の報告書は、自治体が都道府県や国のレベルと比較して不釣り合いに高い実装コストに直面する「逆三角形」の負担構造を具体的に指摘した。報告書は、単一の職員が複数の省庁業務を担当し、一つの部門で複数の計画が策定されるケースがあること、資源不足により計画関連の行政業務に十分な時間を確保できず、国のテンプレートや他自治体の事例をほぼそのまま踏襲して資金確保を図るケースがあること、計画策定自体が努力義務で手続きが自治体の裁量に委ねられていても、地域の合意を得るために審議会での審査など相当な手続きコストが発生することなどを文書化している。これらの構造的制約は、理論的魅力にもかかわらず、連携・共創アプローチが実際の実装の現実によって損なわれる可能性がある文脈を生み出している。

% -----------------------------------------------------------------------------------------------------------------------------------------
\subsection{政策の変遷}

日本の地域公共交通政策は、市場メカニズムと公的介入のバランスに関する異なるアプローチを反映して、明確な段階を経て発展してきた。

\textbf{2002年の規制緩和}では、路線運行を許可制から届出制に変更し、補助金配分を会社全体から路線別の支援に変更し、ターゲットを絞った補助金を通じて市場競争を導入しながら不可欠なサービスを維持した。

\textbf{2006年の地域公共交通会議}では、正式な協調的ガバナンスのメカニズムを確立し、コミュニティバスの運行を可能にし、交通計画へのステークホルダー参加のためのプラットフォームを提供した。

\textbf{2010年の生存交通プロジェクト}では、「頑張る地域」のみが生活交通サービスを維持することを許可される競争的アプローチを制度化し、地方のイニシアティブと自立を強調した。

\textbf{2021年の共創フレームワーク}では、リソースの動員(「交通リソースの総動員」)、共同管理(事業者との直接的な協力)、サービス最適化(「まとめて減らす」アプローチ)を含む、共創を中心的な組織原理として明確に採用した。

% -----------------------------------------------------------------------------------------------------------------------------------------
\subsection{実装ギャップの実証的証拠}

連携と共創の原則の実際の実装を評価するため、本研究では共創政策の具体的な現れを代表する日本のMaaS(Mobility as a Service)イニシアティブの包括的な分析を実施した。2020年度に承認された38のプロジェクトすべての分析により、政策の理想と実装の現実との間に大きな乖離が明らかになった。

事業パフォーマンス指標に関しては、プロジェクトの92\%(35/38)が利用率、収益生成、運営効率に焦点を当てた指標を含んでいた。対照的に、非事業指標に関しては、プロジェクトの29\%(11/38)のみが社会的影響、アクセシビリティ改善、環境成果に対処する指標を組み込んでいた。最も懸念されるのは、市民参加のメカニズムのレベルであり、プロジェクトのわずか5\%(2/38)のみがプロジェクトのガバナンスと評価における市民参加の有意義なメカニズムを確立していた。

これらの結果は、協調的枠組みがより広い公共目的に奉仕するのではなく、交通事業者の利益に捕捉されている可能性を示しており、真の民主的参加の最小限の達成が懸念される。

% -----------------------------------------------------------------------------------------------------------------------------------------
\subsection{研究目的とアプローチ}

本研究は3つの主要な研究問いに取り組む。

第一に、協調的交通政策がその表明された目的をどの程度達成しているかを検討すること。ビジネス重視の指標からより広い社会的・民主的価値を組み込むへの移行という、連携政策の実装ギャップを調査する。

第二に、認知バイアスとステークホルダーの特性が政策実装における協調的調整の効果にどのように影響するかを探求すること。計算論的モデリングを用いて、ステークホルダー間の調整メカニズムを分析する。

第三に、連携の利益を最適化しながらその限界を緩和するのに役立つ制度的配置について決定すること。理論的に基礎づけられた制度設計の原則を導出する。

本研究は、実装ギャップの既存の実証的証拠と、認知バイアスがステークホルダー調整の効果にどのように影響するかに関する理論的洞察を統合する統合的アプローチを採用する。このアプローチは、文献レビュー、協調ロボット制御理論を用いた計算論的モデリング、制度設計理論を組み合わせることで、政策科学に貢献する。

% -----------------------------------------------------------------------------------------------------------------------------------------
\section{理論的枠組みと先行研究}
\label{sec:ch4_framework}

% -----------------------------------------------------------------------------------------------------------------------------------------
\subsection{協調的ガバナンス理論}

協調的ガバナンスは公共行政において支配的なパラダイムとして登場し、Ansell and Gash (2008)によって「一つまたは複数の公共機関が、非政府のステークホルダーを、合意形成志向で審議的な集団的意思決定プロセスに直接関与させる統治のあり方」として定義されている。

交通政策の文脈では、協調的アプローチは従来の規制枠組みの限界に対する解決策として推進されてきた。Kato et al. (2009)は、成功したコミュニティ参加型公共交通のための4つの重要な条件を特定している。ステークホルダー間での認識と責任分担の共有、すべての参加者にとっての相互利益、主要な調整者の存在、そしてステークホルダーの努力とサービス改善成果との関連性である。

しかし、協調的ガバナンスは本質的な課題に直面している。Emerson et al. (2012)は、協調には大きな取引コストが必要であり、最小公約数的な解決策につながる可能性があり、組織された利益に捕捉される可能性があると指摘している。これらの課題は、技術的複雑さ、規制上の制約、商業的利益が追加的な調整の困難さを生む交通政策において特に深刻である。

% -----------------------------------------------------------------------------------------------------------------------------------------
\subsection{公共サービス提供における共創}

共創は伝統的な協調を超えた発展を表し、公共機関とステークホルダー間の共同価値創造を強調する(Voorberg et al., 2015)。交通政策において、共創はいくつかの形態で現れる。リソースの動員は、車両、運転手、インフラなど、多様な供給源からの非伝統的な交通リソースの活用を含む。共同運営は、共同ベンチャーと共有サービスを通じた交通事業者間の正式なパートナーシップを含む。サービス統合は、複数のユーザーニーズに同時に対応する包括的なモビリティソリューションを作成するための交通と他のサービスのバンドルに焦点を当てる。

日本の政策フレームワークは、共創の3つの次元を特定している。地域の活性化に積極的に貢献する交通事業者、問題解決のためのクロスセクター協働、そして交通サービスのコミュニティオーナーシップである。しかし、実証研究は混合した結果を示している。九州運輸局(2023)は、共創が以前は不可能だった路線再編を可能にし、サービス効率を改善する一方で、事業パフォーマンスの改善は限定的であり、必要な信頼関係の構築にはかなりの時間投資が必要であることを発見した。

% -----------------------------------------------------------------------------------------------------------------------------------------
\subsection{協調と共創の効果に関する実証的証拠}

\paragraph{協調の効果}

Kato et al. (2009)は、コミュニティ参加型地域公共交通を分析し、4つの存在要件を特定したが、これらの条件を実現するために必要な要素については扱っていない。これらの要件には、関係ステークホルダー間での認識と責任分担の共有、各ステークホルダーが参加から利益を得られること、ステークホルダーを調整するキーパーソンの存在、そしてステークホルダーの努力が利用促進と価値向上につながることが含まれる。

喜多(2006)は、包括的調整計画策定における協働と連携の場の提供を、技術的人材育成の観点から肯定的に評価し、「研究者や技術専門家の支援を受けながら、交通に関わる様々なステークホルダーが協働する場と機会を提供し、計画能力を向上させることこそが、国が実施すべき政策である」と指摘している。

\paragraph{共創の効果}

九州運輸局(2023)は、九州各地域の共創に関する包括的なインタビューと分析を実施し、効果と課題の両方を特定した。

共創の効果としては、以前は実施不可能だった路線再編が可能になり、利便性の維持・向上と輸送効率の改善が同時に達成されたことが挙げられる。しかし、重要な共創の課題も明らかになった。事業パフォーマンスの改善は限定的であり、単に交通事業者の困難さを訴えるだけでは自治体の支援を得ることはできない。協働に必要な信頼関係の構築にはかなりの時間投資が必要であり、交通事業者と他の民間企業・行政との間で役割分担、費用負担、まちづくりについて議論する機会は少ない。

野村(2023)は、一戸町における有限責任事業組合(LLP)を通じた地域交通確保を分析し、自治体と交通事業者が並行してLLPメンバーとして参加することで、登記負担と責任分担コストを削減し、これを共創の事例として位置づけている。

吉田(2021)は、八戸市(2007年)における事業者間協働を実装し、200便を削減しながら等間隔化を達成し、乗客数と収益性の増加をもたらした。

% -----------------------------------------------------------------------------------------------------------------------------------------
\subsection{協調制御理論と社会システム}

\textbf{導入}

協調制御理論は、複数の自律エージェントを調整して共通の目標を達成するためのフレームワークを提供し、マルチエージェントシステム(MAS)の基盤を形成している。この理論は、自動化、ロボット工学、そしてますます社会システムや協調的ガバナンスの分析における複雑な課題に対処するために発展してきた。

\textbf{核心概念とメカニズム}

協調制御理論は、マルチエージェント調整におけるいくつかの重要な問題に対処する。コンセンサスメカニズムは、すべてのエージェントが特定の変数や状態に合意することを確保し、これはMASにおけるグループ調整の基盤である(Gulzar et al., 2018; Ying et al., 2022)。フォーメーションとコンテインメント問題は、エージェントを特定のパターンで配置することまたは特定の境界内に留めることを含む(Ying et al., 2022; Anand et al., 2024; Briñón-Arranz et al., 2014)。リソース割り当てとカバレッジは、エージェント間でタスクやリソースを効率的に配分することに焦点を当て(He et al., 2023; Ying et al., 2022)、フロッキングと接続性問題は、グループの凝集性と通信リンクの維持に対処する(Ying et al., 2022)。

協調制御における最近の発展は、リーダー・フォロワー階層(Hengster-Movrić & Lewis, 2014)、分散型強化学習アプローチ(Lan et al., 2023)、衝突回避メカニズム、そして遅延や不確実性を扱うための堅牢な制御戦略の重要性を強調している。

\textbf{社会システムへの応用}

協調制御理論は、社会システムの分析、特にガバナンスと政策調整にますます適用されている。Bouraima et al. (2023)は、公共交通計画における効果的な制度的調整のための統合意思決定支援モデルを開発した。Hu et al. (2024)は、世界中の34のケースから、メガ交通プロジェクトにおける協調ネットワークのガバナンスに関する統合的知見を導出した。Knoppen et al. (2021)は、複数のステークホルダー間での都市貨物物流政策の優先順位付けにおける認知コンセンサスの追求を調査した。

協調制御理論と公共政策分析の交差は、特に認知バイアスの影響において有望である。Kahneman (2011)の「ファスト思考とスローシンキング」の枠組みは、認知バイアスがいかに意思決定を体系的に偏向させるかを示している。政策文脈において、Kahneman and Lovallo (1993)は、認知バイアスがいかに組織的意志決定に影響を与えるかを調査した。

% -----------------------------------------------------------------------------------------------------------------------------------------
\section{計算論的モデリング・フレームワーク}
\label{sec:ch4_modeling}

% -----------------------------------------------------------------------------------------------------------------------------------------
\subsection{協調制御モデルの構造}

本研究は、Yoshihara et al. (2009)の協調ロボット制御理論を適応させ、政策実装におけるステークホルダー間の相互作用をモデル化する。N関節ロボットアームは政策ネットワークを表し、各関節$i$は以下の状態変数を持つ特定の政策ステークホルダーに対応する:

% (図:ロボットアームモデルの状態変数)
\begin{itemize}
\item 位置ベクトル:$\mathbf{p}_i \in \mathbb{R}^2$
\item 速度ベクトル:$\mathbf{v}_i \in \mathbb{R}^2$
\item 関節角度:$\theta_i \in \mathbb{R}$
\item リンク長:$l_i \in \mathbb{R}$
\end{itemize}

このマッピングにより、特定の能力(リンク長)と政策位置(関節角度)を持つ自律エージェントとして政策ステークホルダーを表現し、集合的な政策目標を達成するために調整する必要がある状況をモデル化できる。

% -----------------------------------------------------------------------------------------------------------------------------------------
\subsection{運動学モデル}

各関節の位置は、前の関節の位置と累積関節角度に依存する:

$$\mathbf{p}_i = \mathbf{p}_{i-1} + l_i \begin{bmatrix} \cos(\sum_{j=0}^{i}\theta_j) \\ \sin(\sum_{j=0}^{i}\theta_j) \end{bmatrix}$$

ここで$\mathbf{p}_0$は初期位置である。これは、政策成果に対するステークホルダー決定の累積効果を表現している。

速度は位置の時間微分として計算される:

$$\mathbf{v}_i = \frac{d\mathbf{p}_i}{dt}$$

% -----------------------------------------------------------------------------------------------------------------------------------------
\subsection{制御目標と調整メカニズム}

システムの目標は、政策目標を表す目標位置$\mathbf{p}_{target}$をエンドエフェクタ(最終ステークホルダー)が追跡するように誘導することである。各関節$i$は協調制御アルゴリズムに従う。

\paragraph{目標方向ベクトルの計算}

エンドエフェクタから目標への方向ベクトル:

$$\mathbf{n} = \mathbf{p}_{target} - \mathbf{p}_{n-1}$$

正規化された垂直ベクトル:

$$\hat{\mathbf{u}}_i = \frac{\mathbf{n} \times (\mathbf{p}_i - \mathbf{p}_{i-1})}{||\mathbf{n} \times (\mathbf{p}_i - \mathbf{p}_{i-1})||}$$

\paragraph{目標速度の分解}

目標速度$\mathbf{v}_{target}$は各関節の動きの方向に分解される:

$$\mathbf{v}_i^{local} = (\mathbf{v}_{target} \cdot \hat{\mathbf{u}}_i) \hat{\mathbf{u}}_i$$

$$\mathbf{v}_{remaining} = \mathbf{v}_{target} - \mathbf{v}_i^{local}$$

\paragraph{調整係数の計算}

各関節の調整係数$a_i$は、その目標速度と後続関節の実際の速度との差に基づいて計算される:

$$a_i = \frac{\mathbf{v}_i^{target} - \sum_{j>i} \mathbf{v}_j}{||\mathbf{v}_i^{target} - \sum_{j>i} \mathbf{v}_j|| + \epsilon_1}$$

ここで$\epsilon_1, \epsilon_2$は数値的安定性のための小さな正の定数である。

% -----------------------------------------------------------------------------------------------------------------------------------------
\subsection{認知バイアスの統合}

\paragraph{現状維持バイアス}

現状維持バイアス$b_{sq}$は、関節角度の変化に対する抵抗として組み込まれる:

$$\dot{\theta}_i^{biased} = (1 - b_{sq}) \cdot \dot{\theta}_i$$

ここで$\dot{\theta}_i$はバイアスのない角速度である。

\paragraph{確証バイアス}

確証バイアス$b_{conf}$は、調整係数の修正として組み込まれる:

$$a_i^{biased} = a_i \cdot (1 + b_{conf})$$

正の確証バイアスは自己判断への過剰な自信を表し、負の値は過度の自己疑念を表す。

\paragraph{狭い視野バイアス}

狭い視野バイアス$b_{nf}$は、全体最適化の無視として組み込まれる:

$$\mathbf{v}_i^{cooperative} = \mathbf{v}_i^{local} + (1 - b_{nf}) \cdot \sum_{j \neq i} a_j \mathbf{v}_j^{remaining}$$

% -----------------------------------------------------------------------------------------------------------------------------------------
\subsection{協調速度の計算}

各関節の協調速度$\mathbf{v}_i^{cooperative}$は以下のように計算される:

$$\mathbf{v}_i^{cooperative} = \mathbf{v}_i^{local} + \sum_{j \neq i} a_j \mathbf{v}_j^{remaining}$$

% -----------------------------------------------------------------------------------------------------------------------------------------
\subsection{関節角度の更新}

最終的な関節角度の更新は、協調速度と目標速度の内積に基づく:

$$\dot{\theta}_i = k \cdot (\mathbf{v}_i^{cooperative} \cdot \mathbf{v}_{target})$$

ここで$k$は制御ゲインである。

% -----------------------------------------------------------------------------------------------------------------------------------------
\subsection{パフォーマンス指標}

システムのパフォーマンスは以下の指標を用いて評価される:

\paragraph{精度×距離}

$$M_{accuracy} = \frac{1}{||\mathbf{p}_{end} - \mathbf{p}_{target}||} \times \frac{1}{D}$$

ここで$D = \sum_i \int ||\mathbf{v}_i|| dt$は累積移動距離である。

\paragraph{エネルギー効率}

$$M_{efficiency} = \frac{1}{\sum_i \int ||\dot{\theta}_i||^2 dt}$$

\paragraph{関節活動度}

$$M_{activity} = \frac{1}{n} \sum_i \max_t |\dot{\theta}_i(t)|$$

\paragraph{関節滑らかさ}

$$M_{smoothness} = \frac{1}{n} \sum_i \frac{1}{\int |\ddot{\theta}_i(t)| dt}$$

% -----------------------------------------------------------------------------------------------------------------------------------------
\subsection{シミュレーション実装プロセス}

% (図:シミュレーションプロセスフロー)
% 図1:ステークホルダー調整をロボットアーム制御を通じて分析するための6段階の方法論を示す包括的なシミュレーションプロセスフロー。プロセスには、初期設定、バイアス適用、制御実行、パフォーマンス評価、統計分析、政策含意の導出が含まれる。このフレームワークにより、ランダム化された条件での1,820回の実験の体系的な分析が可能になる。

% -----------------------------------------------------------------------------------------------------------------------------------------
\section{実験設計}
\label{sec:ch4_experiment}

% -----------------------------------------------------------------------------------------------------------------------------------------
\subsection{実験の概要}

認知バイアスが調整パフォーマンスに与える影響を定量的に評価するため、本研究では合計1,820回の実験を実施した。

実験設計は以下の通りである。ベースライン実験はバイアスなしで100回、確証バイアス実験は11条件で各40回(合計440回、強度0.0-1.0、ステップ0.1)、現状維持バイアス実験は21条件で各40回(合計840回、強度0.0-1.0、ステップ0.05)、狭い視野実験は11条件で各40回(合計440回、強度0.0-1.0、ステップ0.1)である。

% -----------------------------------------------------------------------------------------------------------------------------------------
\subsection{バイアス実験}

決定論的結果を避けるため、本研究では初期条件をランダム化した実験設計を実装した。

% (図:4つの主要なアニメーションシナリオ)
% 図2:異なる認知バイアスがステークホルダー調整パターンにどのように影響するかを示す4つの主要なアニメーションシナリオ。各シナリオは異なる関節構成と目標到達行動を示す:(a)バイアスのない通常動作、(b)政策目標へのフォーカスを促進する確証バイアス、(c)既存のアプローチを維持する現状維持バイアス、(d)システム全体の視野を制限する狭い視野。

各実験は統計的有効性を確保するために複数のランダム化パラメータを組み込んだ。初期関節角度は±0.1ラジアンの範囲で変化させ、目標位置は中心から±0.01m、半径変動±0.005mの範囲で変化させ、制御ゲインは±0.5の範囲で変化させ、切り替え閾値は±0.002の範囲で変化させた。

% -----------------------------------------------------------------------------------------------------------------------------------------
\subsection{統計分析方法}

統計分析は、結果の堅牢な解釈を確保するために複数の相補的な方法を用いた。分散分析(ANOVA)は、異なる実験条件間でのバイアス効果の統計的有意性を検定するために使用した。信頼区間は、効果サイズを推定し、平均パフォーマンス値の周りの不確実性の尺度を提供するために計算した。回帰分析は、バイアス強度と調整パフォーマンス指標間の線形関係を検証するために実施した。

% -----------------------------------------------------------------------------------------------------------------------------------------
\section{実験結果}
\label{sec:ch4_results}

% -----------------------------------------------------------------------------------------------------------------------------------------
\subsection{統計分析結果}

本研究は、統計的有効性を確保するためにランダム化された初期条件を持つ1,820回の実験(100回ベースライン + 440回確証バイアス + 840回現状維持バイアス + 440回狭い視野)に対して包括的な統計分析を実施した。ANOVA検定は、各バイアスタイプについて異なるパターンを明らかにした:

\paragraph{確証バイアス}
調整パフォーマンスに有意な影響は見られなかった(F = 0.838, p = 0.602)。これは、確証バイアスが建設的に活用できるという仮説を支持している。パフォーマンスは全強度レベル(0.0-1.0)で安定しており、平均精度は0.970から0.970の範囲であった。

\paragraph{現状維持バイアス}
有意な負の効果が見られた(F = 1.593, p = 0.044)。強度0.25前後での閾値効果を確認し、この閾値を超えるとパフォーマンスが急激に低下した。これは、変化への抵抗の管理に関する制度設計の推奨を検証するものである。

\paragraph{狭い視野バイアス}
有意な線形劣化効果が見られた(F = 1.985, p = 0.028)。システム全体の視野を維持することの重要性を示している。線形回帰分析は、R² = 0.204、負の傾き-0.00107を示した。

すべての実験には適切な信頼区間(95\% CI)と効果サイズの計算が含まれた。ベースライン条件は0.969 ± 0.002の平均精度を示し、バイアス比較のための安定した参照点を提供した。

% -----------------------------------------------------------------------------------------------------------------------------------------
\subsection{包括的バイアス効果分析}

% (図3:確証バイアスの建設的効果)

% (図4:現状維持バイアスの閾値効果)

% (図5:狭い視野バイアスの線形劣化)

% -----------------------------------------------------------------------------------------------------------------------------------------
\subsection{関節別分析}

政策ネットワーク内の異なるステークホルダー位置がバイアスからどのような影響を受けるかを理解するため、本研究では主要な位置に焦点を当てた関節別分析を実施した。

% (図6:基部関節(基礎ステークホルダー)へのバイアス効果)
% 図6:基部関節(基礎ステークホルダー)へのバイアス効果。基本的な政策アクターが様々なバイアスタイプにどのように異なる反応を示すかを示し、確証バイアスが基礎レベルで安定性を提供することがわかる。

% (図7:エンドエフェクタ関節(最終実装ステークホルダー)へのバイアス効果)
% 図7:エンドエフェクタ関節(最終実装ステークホルダー)へのバイアス効果。実装レベルでのバイアスが全体的な政策成果に直接影響を与える方法を示し、狭い視野が最も顕著な負の効果を示している。

関節別分析により、バイアス効果は政策ネットワーク内のステークホルダー位置によって大きく異なることが明らかになった。基部関節(基礎ステークホルダー)は確証バイアスに対してより高い耐性を示し、エンドエフェクタ関節(実装ステークホルダー)は狭い視野効果に対してより敏感であった。

% -----------------------------------------------------------------------------------------------------------------------------------------
\subsection{実験結果の要約}

1,820回の実験を通じて、以下の知見が得られた。

\begin{enumerate}
\item \textbf{確証バイアス}:調整パフォーマンスを維持する建設的な効果を持つ
\item \textbf{現状維持バイアス}:閾値0.2前後で急激な変化を示す
\item \textbf{狭い視野}:線形パフォーマンス劣化を示す
\end{enumerate}

% -----------------------------------------------------------------------------------------------------------------------------------------
\section{効果的な協調と共創のための制度設計フレームワーク}
\label{sec:ch4_design}

% -----------------------------------------------------------------------------------------------------------------------------------------
\subsection{制度設計の理論的基盤}

Japan MaaS分析と計算論的モデリングの結果に基づき、両方の研究で特定された調整の課題に対処する包括的な制度的設計フレームワークを提案する。このフレームワークは、制度設計理論を活用しながら、認知バイアス効果と民主的参加要件に関する洞察を組み込んでいる。

分析からの主要な洞察は、効果的な連携と共創には以下を可能にする制度的配置が必要であることを示している。確証バイアスの建設的な可能性を活用すること、現状維持バイアスの閾値効果を管理すること、狭い視野の負の影響を緩和すること、そして運営効率を可能にしながら民主的アカウンタビリティを確保することである。

% -----------------------------------------------------------------------------------------------------------------------------------------
\subsection{三層制度アーキテクチャ}

% (図9:効果的な協調のための三層制度設計)
% 図9:効果的な協調のための三層制度設計。認知バイアスを戦略的に活用する政策調整のための制度設計。上部は品質定義と実装を分離する基本フレームワークを示し、下部は環境設計を通じて行政・事業関係を管理するための具体的なメカニズムを詳述している。

\paragraph{第一層:政治-市民インターフェース(民主的品質設定)}

第一層である政治-市民インターフェースは、民主的正統性を確保し、品質基準を規定する責任を持つ。この層の核心は、共有された品質目標に対するステークホルダーの信頼を育むことによって確証バイアスを建設的に活用しながら、同時に狭い部分最適化を避けるために広い視野を維持することである。これは、交通サービスの品質目標を定義する市民参加プロセス、明確なアカウンタビリティメカニズムによる行政実装の政治的監視、効率性、アクセシビリティ、持続可能性の間のトレードオフに関する公開審議、そして明確な方向性を提供しながら解釈の柔軟性を保持する品質仕様フレームワークの確立を通じて達成される。実践において、この層は交通の優先事項に関する定期的な市民集会または審議型世論調査、品質目標の達成に関する透明な報告、市民参加の範囲と権限に関する明確に定義された境界、そしてより広い民主的ガバナンス構造との統合によって特徴づけられる。

\paragraph{第二層:行政的調整(環境設計)}

第二層である行政的調整は、品質目標を運用フレームワークに変換する。ここでの指導原則は、破壊的な閾値を下回る漸進的変化アプローチを通じて現状維持バイアスを管理しながら、同時に効果的なマルチステークホルダー調整に必要な専門性を開発することである。これは、ステークホルダー調整と紛争解決のための専門的な行政能力の構築、事業価値と公共価値の両方の指標を組み込んだ技術的評価システムの実装、品質制約内で競争的環境を促進する規制フレームワークの設計、そして効果的なフィードバックループを持つパフォーマンス監視・調整システムの確立によって達成される。この層をサポートする組織構造には、適切な専門性を備えた職員による協調的ガバナンス専門部署、政治的圧力と事業的圧力の間を仲介するバッファメカニズム、民主的インプットを運用ガイダンスに変換する構造化されたプロセス、そして公共目標と密接に連動した革新インセンティブが含まれる。

\paragraph{第三層:事業-運用インターフェース(自律的実装)}

第三層である事業-運用インターフェースは、確立された品質フレームワーク内での効率的なサービス提供を担う。ここでは、確証バイアスが品質に沿った運営への事業信頼をサポートすることを許容し、競争圧力を通じて現状維持バイアスを最小化し、統合されたパフォーマンス測定を通じてシステム全体の視野を維持する。これは、品質制約内で運用するサービス提供のための競争的プロセス、効率とともに公共価値の創造を報酬する品質調整パフォーマンス契約、運用上の課題に対処するための協働的問題解決プロセス、そして公共価値創造を促進する革新インセンティブを通じて実現される。フレームワークは、革新を報酬するパフォーマンスベースの資金提供、クロスセクター調整の要件、システム全体のパフォーマンス指標の使用、そしてステークホルダーグループ間の情報共有プラットフォームの確立によってさらにサポートされる。

% -----------------------------------------------------------------------------------------------------------------------------------------
\subsection{インターフェース管理メカニズム}

層間のインターフェースの管理は、効果的な制度設計にとって重要である。政治-行政インターフェースにおいて、主要な課題は、行政の専門性を損なったり過度の政治的介入を導入したりすることなく、民主的インプットを運用ガイダンスに変換することである。これは、明確な方向性を提供しながら解釈の柔軟性を維持する構造化された品質仕様プロセスを実装することによって対処される。主要なメカニズムには、プロセスではなく成果を規定する品質フレームワークの使用、現状維持バイアスを管理するためのサンセット条項を持つ定期的なレビューサイクル、民主的に定義されたパラメータ内での専門的行政自律性の保持、そしてマイクロマネジメントに頼らずに政治的監視を可能にする透明なアカウンタビリティメカニズムが含まれる。

行政-事業インターフェースにおいて、課題は公共目標と運用効率のバランスを取りながら、事業利益が協調プロセスを捕捉するリスクを管理することにある。解決策は、効率とともに品質達成を報酬する競争的フレームワークを採用することにより、私的利益と公共目標を整合させることを含む。これは、品質に沿った行動を利益可能にする環境設計、価値の複数の次元を組み込むパフォーマンス測定システム、サービス提供における現状維持バイアスの定着を防ぐ競争圧力、そして堅牢なアカウンタビリティを維持しながら共同問題解決を促進する協調的ガバナンス構造を通じて運用化される。

% -----------------------------------------------------------------------------------------------------------------------------------------
\section{考察}
\label{sec:ch4_discussion}

% -----------------------------------------------------------------------------------------------------------------------------------------
\subsection{理論的含意}

本研究の理論的含意は複数の次元にわたる。第一に、認知バイアスの再評価が達成され、以前は否定的に見られていた認知バイアスの建設的な側面が明らかになった。第二に、定量分析手法が開発され、政策協調の数学的モデリングを通じた定量分析が可能になった。第三に、制度設計理論への貢献がなされ、ガバナンス配置におけるバイアスの戦略的活用のための新しい理論的枠組みが提示された。

% -----------------------------------------------------------------------------------------------------------------------------------------
\subsection{実践的含意}

実践的含意はいくつかの重要な領域を含む。政策形成への応用に関しては、認知バイアスを障害ではなく戦略的リソースとして考慮する政策設計の重要性が強調される。ステークホルダー管理に関しては、異なるステークホルダーグループの特定のバイアス特性に合わせた関係管理手法の価値が示される。制度改革に関しては、予測可能な認知傾向に対抗するのではなく、それと協働する既存制度の漸進的改善戦略の必要性が示される。

% -----------------------------------------------------------------------------------------------------------------------------------------
\subsection{研究の限界}

本研究はいくつかの重要な限界を認めている。シミュレーション環境は、計算モデリングが必然的に実際の政策の複雑さから乖離しているという重要な制約を表している。バイアスモデリングアプローチは認知バイアスの簡略化を含んでおり、現実世界の設定ではパラメトリック表現が示唆するよりも微妙で文脈依存的である。検証範囲は日本の地域公共交通に限定されており、他の政策分野や制度的文脈への一般化可能性を制限する可能性がある。

% -----------------------------------------------------------------------------------------------------------------------------------------
\section{小括}
\label{sec:ch4_summary}

本研究では、ロボットアーム協調制御シミュレーションを用いて地域公共交通政策における協調メカニズムを分析し、いくつかの主要な発見を得た。

認知バイアスの定量評価は、1,820回の制御された実験を通じて、3つの認知バイアスが調整パフォーマンスに与える効果を統計的に検証することによって達成された。特定のバイアスの建設的効果は、確証バイアスの調整パフォーマンス維持効果を通じて実証され、バイアスが政策実装を一律に阻害するという従来の仮定に挑戦した。三層制度設計フレームワークは、バイアスを単に排除しようとするのではなく、戦略的に活用するものとして開発された。

1,820回のシミュレーション実験を通じて、三つの認知バイアスが政策協調に与える影響を定量的に解明した。確証バイアスは統計的に有意な調整低下を引き起こさず、適度な範囲では逆説的に協調を促進する可能性があることを示した。現状維持バイアスは統計的に有意な負の効果を持ち、約0.25を超えると臨界点を超えて急激に協調が阻害されることを明らかにした。狭い視野は有意な線形劣化を示し、一貫して負の影響を持つことを確認した。

これらの発見に基づき、政治-市民インターフェース、行政-事業インターフェースを分離する三層制度アーキテクチャを提案した。このフレームワークは、民主的アカウンタビリティと運用効率の両方を最適化することを目指している。

次章では、本章の計算論的分析と「執政の創造性」の理論を踏まえ、生成AIと人間の協調的関係性を具体化するZK-SNARKs型政策評価システムを提案する。

% -----------------------------------------------------------------------------------------------------------------------------------------


% -----------------------------------------------------------------------------------------------------------------------------------------
% 第5章 生成AIと人間の関係性:ZK-SNARKs型政策評価システム
% -----------------------------------------------------------------------------------------------------------------------------------------
% -----------------------------------------------------------------------------------------------------------------------------------------
% 第5章 生成AIと人間の関係性:ZK-SNARKs型政策評価システム
% -----------------------------------------------------------------------------------------------------------------------------------------

\chapter{生成AIと人間の関係性:ZK-SNARKs型政策評価システム}
\label{chap:zk_snarks_system}

% -----------------------------------------------------------------------------------------------------------------------------------------
\section{生成AIの位置づけ}
\label{sec:ch5_ai_positioning}

% -----------------------------------------------------------------------------------------------------------------------------------------
\subsection{人間の「執政の創造性」を補完する「杖」}

第2章で述べた通り、生成AIは人間の「執政の創造性」を補完する「杖」として位置づけられる。この関係性は、以下の原則に基づく:

\begin{enumerate}
    \item AIは人間の\textbf{最終判断}を前提とする
    \item AIの\textbf{限界}を明示的に理解する
    \item 人間-AI協調のプロセスを\textbf{透明化}する
    \item AI自体のバイアスに\textbf{対処}する
\end{enumerate}

% -----------------------------------------------------------------------------------------------------------------------------------------
\subsection{AIができないこと}

生成AIには、以下の本質的な限界が存在する:

\begin{description}
    \item[規範判断] 何が社会にとって「善い」のかを判断できない
    \item[価値創造] 新たな価値や規範を創造できない
    \item[文脈理解] 学習データの範囲外の状況に適応できない
\end{description}

これらの領域は、人間の役割として残される。

% -----------------------------------------------------------------------------------------------------------------------------------------
\section{認知バイアスへの対話的介入}
\label{sec:ch5_cognitive_intervention}

% -----------------------------------------------------------------------------------------------------------------------------------------
\subsection{「悪魔の代理人」機能}

生成AIは、意思決定プロセスにおける「悪魔の代理人(Devil's Advocate)」として機能し得る。具体的には:

\begin{itemize}
    \item 意思決定者の視点と対立する論点の提示
    \item 見落とされがちなリスクの指摘
    \item 代替案の生成
\end{itemize}

これは、確証バイアスへの対処に特に有効である。

% -----------------------------------------------------------------------------------------------------------------------------------------
\subsection{視野拡大の支援}

狭い視野への対処として、AIは以下の機能を提供できる:

\begin{itemize}
    \item 全体的な目標との整合性の確認
    \item 他領域との関連性の提示
    \item 長期的な影響の分析
\end{itemize}

% -----------------------------------------------------------------------------------------------------------------------------------------
\subsection{メタ認知の促進}

AIは、意思決定者自身の認知バイアスへの気づきを促すことができる。例えば、「あなたの判断は現状維持バイアスの影響を受けている可能性があります」といったフィードバックを提供する。

% -----------------------------------------------------------------------------------------------------------------------------------------
\section{ZK-SNARKs概念の援用}
\label{sec:ch5_zksnarks_concept}

% -----------------------------------------------------------------------------------------------------------------------------------------
\subsection{秘密を守りながら専門性を証明する}

ZK-SNARKsの概念を政策評価に応用することで、「秘密情報を公開することなく、その情報が正しいことを証明する」仕組みが実現可能になる。

これは、以下の政策場面で有用である:
\begin{itemize}
    \item 企業が技術の詳細を公開せずに、政策課題への貢献可能性を証明
    \item 個人が個人情報を守りながら、専門性を証明
    \item 行政が内部情報を守りながら、政策判断の根拠を説明
\end{itemize}

% -----------------------------------------------------------------------------------------------------------------------------------------
\subsection{ZK-SNARKsの4つの特性の政策評価への翻訳}

\begin{table}[htbp]
\centering
\caption{ZK-SNARKs特性の政策評価への翻訳}
\label{tab:zk_snarks_translation}
\begin{tabular}{lp{8cm}}
\toprule
ZK-SNARKs特性 & 政策評価への翻訳 \\
\midrule
Zero-Knowledge & 秘密情報(企業秘密・個人情報)の非開示 \\
Succinct & 簡潔な評価結果の提示 \\
Non-interactive & 一方向の対話での評価完結 \\
Arguments of Knowledge & 専門知識に基づく証明 \\
\bottomrule
\end{tabular}
\end{table}

% -----------------------------------------------------------------------------------------------------------------------------------------
\section{LLM as a Judgeによる実装}
\label{sec:ch5_llm_judge}

% -----------------------------------------------------------------------------------------------------------------------------------------
\subsection{Constitutional AIと市民討議}

LLM as a Judgeにおいて、評価基準はConstitutional AIの原則と市民討議を通じて設計される。これにより:

\begin{itemize}
    \item AIの評価基準に人間の価値観を組み込む
    \item 民主的正当性を確保する
    \item 透明性と説明責任を満たす
\end{itemize}

% -----------------------------------------------------------------------------------------------------------------------------------------
\subsection{決定論的運用}

LLMの確率的な出力を制御するため、以下の手法を組み合わせる:

\begin{itemize}
    \item \textbf{Temperature=0}:ランダム性を排除
    \item \textbf{Self-Consistency}:複数回の出力から一貫性のある結果を選択
    \item \textbf{TEE(秘匿実行環境)}:処理の透明性を確保
\end{itemize}

% -----------------------------------------------------------------------------------------------------------------------------------------
\subsection{控訴プロセスによる人間介入}

最終的な判断は人間が行うため、控訴プロセスを組み込む:

\begin{enumerate}
    \item AIによる一次評価
    \item 評価結果に対する異議申立ての受付
    \item 人間による二次評価
    \item 最終判断の提示
\end{enumerate}

% -----------------------------------------------------------------------------------------------------------------------------------------
\section{ZK-SNARKs型政策評価のアーキテクチャ}
\label{sec:ch5_architecture}

% -----------------------------------------------------------------------------------------------------------------------------------------
\subsection{三層アーキテクチャ}

本研究が提案するZK-SNARKs型政策評価システムは、以下の三層アーキテクチャから構成される:

\begin{description}
    \item[外層:ZK-SNARKs秘匿証明層] 秘密情報の保護、秘匿化処理
    \item[中層:LLM as a Judge評価層] 自動評価、一貫性確保
    \item[内層:Constitutional AI + 市民討議] 価値統合、評価基準の設計
\end{description}

% -----------------------------------------------------------------------------------------------------------------------------------------
\subsection{限界と課題}

このシステムには以下の限界がある:

\begin{enumerate}
    \item \textbf{数学的保証と確率的期待の違い}:ZK-SNARKsの数学的完全性は、LLMでは実現できない
    \item \textbf{AI自体のバイアス}:学習データに含まれるバイアスが評価結果に影響
    \item \textbf{透明性の限界}:LLMの内部処理の完全な説明は困難
\end{enumerate}

これらの限界に対処するため、人間による最終判断を不可欠とする。

% -----------------------------------------------------------------------------------------------------------------------------------------
\section{小括}
\label{sec:ch5_summary}

本章では、生成AIと人間の協調的関係性を具体化するシステムとして、ZK-SNARKs型政策評価システムを提案した。このシステムは:

\begin{enumerate}
    \item 認知バイアスへの対話的介入を通じて、より良い意思決定を支援
    \item 秘匿性と信頼性を両立する評価プロセスを提供
    \item 人間による最終判断を前提とした、AIの「杖」としての活用を実現
\end{enumerate}

次章では、第4章の計算論的分析と本章のシステム設計を踏まえ、制度設計への示唆を導出する。

% -----------------------------------------------------------------------------------------------------------------------------------------


% -----------------------------------------------------------------------------------------------------------------------------------------
% 第6章 制度設計への示唆
% -----------------------------------------------------------------------------------------------------------------------------------------
% -----------------------------------------------------------------------------------------------------------------------------------------
% 第6章 制度設計への示唆
% -----------------------------------------------------------------------------------------------------------------------------------------

\chapter{制度設計への示唆}
\label{chap:institutional_design}

% -----------------------------------------------------------------------------------------------------------------------------------------
\section{理論的含意}
\label{sec:ch6_theoretical_implications}

% -----------------------------------------------------------------------------------------------------------------------------------------
\subsection{第4章のシミュレーション結果からの設計原則}

第4章の計算論的分析から、以下の設計原則が導かれる。現状維持バイアスへの対処としては、臨界点($b_{sq} \approx 0.25$)を超えないよう、段階的な変化導入を設計することが重要である。確証バイアスの活用としては、適度な自信は協調を促進するため、完全な中立性よりも構造的な多様性を確保することが有効である。狭い視野への対処としては、全体目標の可視化と横断的指標の導入を必須とすることが求められる。

% -----------------------------------------------------------------------------------------------------------------------------------------
\subsection{第5章のZK-SNARKsシステムからの制度的含意}

ZK-SNARKs型政策評価システムは、以下の制度的含意を持つ。第一に、秘匿性の制度化である。企業秘密や個人情報を保護しながら政策参加を可能にする制度が必要である。第二に、自動評価の境界である。AIによる評価と人間による判断の境界を明確化する必要がある。第三に、市民参加の質的向上である。市民討議を通じた評価基準の共同設計が求められる。

% -----------------------------------------------------------------------------------------------------------------------------------------
\section{各バイアスへの対処戦略}
\label{sec:ch6_bias_strategies}

% -----------------------------------------------------------------------------------------------------------------------------------------
\subsection{現状維持バイアス:小さな変化の積み重ね}

現状維持バイアスへの対処として、小さな変化の積み重ねという戦略を提案する。具体的には、パイロット導入として小規模な実験から開始し、成功事例を蓄積することが有効である。また、段階的拡大として、臨界点を超えないよう、徐々に適用範囲を拡大することが求められる。さらに、デフォルト設定の変更として、現状維持の方向に働く制度上のデフォルトを見直すことも重要である。

% -----------------------------------------------------------------------------------------------------------------------------------------
\subsection{確証バイアス:多様な視点の構造的導入}

確証バイアスへの対処として、多様な視点の構造的導入という戦略を提案する。「悪魔の代理人」の制度化として、AIまたは人間による批判的視点の提示を必須化することが有効である。また、多様なステークホルダーの参画として、異なる立場・利害を持つ主体の関与を確保することが求められる。さらに、反証可能性の確保として、仮説を反証する証拠を積極的に探索するプロセスを組み込むことも重要である。

% -----------------------------------------------------------------------------------------------------------------------------------------
\subsection{狭い視野:全体目標の可視化}

狭い視野への対処として、全体目標の可視化という戦略を提案する。横断的評価指標として、単一の指標ではなく、複数の指標による評価を導入することが有効である。また、システムマップの作成として、政策の全体像を可視化した図の共有が求められる。さらに、長期的影響の分析として、短期的な成果だけでなく、長期的な影響も評価することが重要である。

% -----------------------------------------------------------------------------------------------------------------------------------------
\section{生成AIを組み込んだ制度設計}
\label{sec:ch6_ai_institutional_design}

% -----------------------------------------------------------------------------------------------------------------------------------------
\subsection{政治的-行政的インターフェース}

政治レベルと行政レベルの間で、AIは以下のように位置づけられる。政治的判断については、人間(政治家・議会)が最終的に決定する。行政的分析については、AIが情報の整理・分析を支援する。民主的アカウンタビリティについては、人間が説明責任を負う。

% -----------------------------------------------------------------------------------------------------------------------------------------
\subsection{行政的-事業的インターフェース}

行政レベルと事業レベルの間で、AIは以下のように位置づけられる。政策の具体化については、AIが選択肢の生成を支援する。運用の効率化については、AIが日常的な判断を補助する。人間による監督については、重要な判断は人間が行う。

% -----------------------------------------------------------------------------------------------------------------------------------------
\subsection{ZK-SNARKs型システムの制度的位置づけ}

ZK-SNARKs型政策評価システムは、以下の制度的枠組みの中で運用される。第一に、評価基準の共同設計である。市民討議を通じた基準の策定が必要である。第二に、透明性の確保である。AIの評価プロセスと基準の公開が求められる。第三に、控訴の権利である。評価結果に対する異議申立ての機会を確保する必要がある。第四に、人間による最終判断である。AIの評価は参考情報として位置づけられる。

% -----------------------------------------------------------------------------------------------------------------------------------------
\section{「連携・共創」の再設計}
\label{sec:ch6_collaboration_redesign}

% -----------------------------------------------------------------------------------------------------------------------------------------
\subsection{三層制度設計(S-T1-T2-O)の再考}

STOフレームワークを踏まえ、認知バイアスと生成AIを考慮した制度設計を提案する。

\begin{table}[htbp]
\centering
\caption{認知バイアスと生成AIを考慮した三層制度設計}
\label{tab:three_tier_design}
\begin{tabular}{lp{5cm}p{5cm}}
\toprule
階層 & 主な課題 & 対処戦略 \\
\midrule
S & 現状維持バイアスによる変化抵抗 & 長期ビジョンの共有、段階的目標設定 \\
T1 & 狭い視野による部分最適化 & 横断的指標、全体目標の可視化 \\
T2 & 確証バイアスによる視点の固定化 & 多様な視点の構造的導入 \\
O & 日常的な認知バイアス & AIによる対話的介入 \\
\bottomrule
\end{tabular}
\end{table}

S層では現状維持バイアスによる変化抵抗が課題となるため、長期ビジョンの共有と段階的目標設定が有効である。T1層では狭い視野による部分最適化が課題となるため、横断的指標と全体目標の可視化が求められる。T2層では確証バイアスによる視点の固定化が課題となるため、多様な視点の構造的導入が必要である。O層では日常的な認知バイアスが課題となるため、AIによる対話的介入が有効である。

% -----------------------------------------------------------------------------------------------------------------------------------------
\subsection{生成AIを「杖」として活用するガバナンス}

生成AIを活用するガバナンスのあり方として、以下の原則を提案する。第一に、補完性の原則である。AIは人間の能力を補完し、代替しない。第二に、透明性の原則である。AIの活用方法と限界を明示する。第三に、アカウンタビリティの原則である。最終的な判断と責任は人間が負う。第四に、継続的学習の原則である。AIと人間の協調プロセスを継続的に改善する。

% -----------------------------------------------------------------------------------------------------------------------------------------
\section{小括}
\label{sec:ch6_summary}

本章では、第4章の計算論的分析と第5章のZK-SNARKs型システム設計を踏まえ、制度設計への示唆を導出した。主な貢献として、認知バイアスへの具体的な対処戦略の提示、生成AIを組み込んだ制度設計の方向性の提示、「連携・共創」の再設計に向けた枠組みの提示が挙げられる。

次章では、本研究の総括と、都市計画への展開について論じる。

% -----------------------------------------------------------------------------------------------------------------------------------------


% -----------------------------------------------------------------------------------------------------------------------------------------
% 第7章 結論:都市計画への展開
% -----------------------------------------------------------------------------------------------------------------------------------------
% -----------------------------------------------------------------------------------------------------------------------------------------
% 第7章 結論:都市計画への展開
% -----------------------------------------------------------------------------------------------------------------------------------------

\chapter{結論:都市計画への展開}
\label{chap:conclusion}

% -----------------------------------------------------------------------------------------------------------------------------------------
\section{研究の総括}
\label{sec:ch7_summary}

% -----------------------------------------------------------------------------------------------------------------------------------------
\subsection{研究目的の達成}

本研究は、以下の3つの目的を掲げた。第一に、公共交通政策における実装ギャップの要因として、人間の認知バイアスの影響を計算論的に解明すること。第二に、生成AIと人間の協調的関係性を具体化するシステムとして、ZK-SNARKs型政策評価システムの可能性を探ること。第三に、認知バイアスと生成AIを考慮した制度設計への示唆を導出すること。

\paragraph{目的1について}
第4章において、協調ロボット制御モデルを用いた計算論的分析により、三つの認知バイアスが政策協調に与える影響を定量的に解明した。特に、現状維持バイアスの閾値効果、確証バイアスの逆説的効果、狭い視野の一貫した負の影響を発見した。

\paragraph{目的2について}
第5章において、ZK-SNARKsの概念を援用した政策評価システムを提案し、LLM as a Judge、Constitutional AI、市民討議を組み合わせたアーキテクチャを設計した。

\paragraph{目的3について}
第6章において、認知バイアスへの対処戦略、生成AIを組み込んだ制度設計、「連携・共創」の再設計に向けた枠組みを提示した。

% -----------------------------------------------------------------------------------------------------------------------------------------
\subsection{核となる主張}

本研究の核となる主張は、以下の通りである。

\begin{quote}
\textbf{生成AIは人間の「執政の創造性」を補完する「杖」として、認知バイアスへの対話的介入を通じて政策形成を支援できる。ZK-SNARKsの概念を援用することで、秘匿性と信頼性を両立した政策評価システムが可能になる。}
\end{quote}

% -----------------------------------------------------------------------------------------------------------------------------------------
\section{理論的貢献}
\label{sec:ch7_theoretical_contributions}

% -----------------------------------------------------------------------------------------------------------------------------------------
\subsection{認知バイアスの計算論的分析手法の政策科学への導入}

本研究は、協調ロボット制御理論を政策ネットワーク分析に応用し、認知バイアスの影響を計算論的に解明する手法を導入した。このアプローチは、政策科学における計算論的転回(computational turn)の一環として位置づけられる。

% -----------------------------------------------------------------------------------------------------------------------------------------
\subsection{ZK-SNARKs概念の政策評価への応用}

本研究は、暗号技術の概念であるZK-SNARKsを政策評価に応用する試みとして先駆的である。この概念転用は、「秘密を守りながら専門性を証明する」という新たな政策参加のあり方を示唆している。

% -----------------------------------------------------------------------------------------------------------------------------------------
\subsection{生成AIの「杖」としての理論的位置づけ}

本研究は、生成AIと人間の関係性を「杖」として補完的に位置づける理論的枠組みを提示した。これは、AIによる代替ではなく、AIと人間の協調を前提とする視点である。

% -----------------------------------------------------------------------------------------------------------------------------------------
\section{実践的貢献}
\label{sec:ch7_practical_contributions}

% -----------------------------------------------------------------------------------------------------------------------------------------
\subsection{制度設計への提言}

本研究は、認知バイアスと生成AIを考慮した制度設計への具体的な提言を行った。現状維持バイアスへの対処として、段階的変化導入の設計を提案した。確証バイアスへの対処として、「悪魔の代理人」の制度化を提案した。狭い視野への対処として、横断的評価指標の導入を提案した。生成AIの活用として、補完性、透明性、アカウンタビリティの原則を提示した。

% -----------------------------------------------------------------------------------------------------------------------------------------
\subsection{ZK-SNARKs型政策評価システムの設計指針}

本研究は、ZK-SNARKs型政策評価システムの具体的な設計指針を提示した。三層アーキテクチャの採用、Constitutional AIと市民討議による評価基準設計、控訴プロセスによる人間介入の確保がその主要な内容である。

% -----------------------------------------------------------------------------------------------------------------------------------------
\section{今後の課題:都市計画を舞台にした実証}
\label{sec:ch7_future_work}

% -----------------------------------------------------------------------------------------------------------------------------------------
\subsection{より複雑な政策領域への適用}

公共交通政策は、本研究の「実験場」として適切であったが、より複雑な政策領域への適用が期待される。特に、都市計画は多様なステークホルダー(土地所有者、開発業者、住民、行政など)、多様な秘密情報(土地利用計画、開発権、資産価値など)、長期的影響(数十年単位での都市構造の変化)という点で興味深い研究対象となる。

% -----------------------------------------------------------------------------------------------------------------------------------------
\subsection{ZK-SNARKsシステムの社会実装}

ZK-SNARKs型政策評価システムの社会実装に向けては、いくつかの課題がある。技術的実装としては、TEE、LLM、Constitutional AIの統合が必要である。制度的設計としては、法的位置づけや運用ルールの策定が求められる。社会的受容としては、市民の理解と信頼の獲得が重要である。

% -----------------------------------------------------------------------------------------------------------------------------------------
\subsection{生成AIと人間の協調的関係性の継続的検証}

生成AI技術は急速に進化しており、人間-AI協調のあり方も変化し続ける。本研究の枠組みは、継続的な検証と改善を必要とする。

% -----------------------------------------------------------------------------------------------------------------------------------------
\section{結び}
\label{sec:ch7_conclusion}

本研究は、公共交通政策を舞台に、生成AIと人間の協調的関係性を探求した。その過程で、認知バイアスによる協調失敗のメカニズムを計算論的に解明し、ZK-SNARKs概念を援用した政策評価システムを提案し、制度設計への示唆を導出した。

公共交通政策は、本研究の「第一の舞台」であった。今後は、都市計画という「第二の舞台」での実証を通じて、生成AIと人間のより良い協調のあり方を探求していきたい。

% -----------------------------------------------------------------------------------------------------------------------------------------


% -----------------------------------------------------------------------------------------------------------------------------------------
% 参考文献
% -----------------------------------------------------------------------------------------------------------------------------------------
\printbibliography[title={参考文献}]

% -----------------------------------------------------------------------------------------------------------------------------------------
% 付録
% -----------------------------------------------------------------------------------------------------------------------------------------
\appendix
% -----------------------------------------------------------------------------------------------------------------------------------------
% 付録
% -----------------------------------------------------------------------------------------------------------------------------------------

\chapter{付録}
\label{app:appendix}

\section{協調ロボット制御モデルの数式展開}
\label{app:robot_model}

% -----------------------------------------------------------------------------------------------------------------------------------------
\subsection{運動学モデル}

N関節ロボットアームにおいて、各関節の位置は以下の式で与えられる:

\begin{equation}
\mathbf{x}_i = \mathbf{x}_{i-1} + a_i \begin{bmatrix} \cos(\sum_{j=0}^{i-1} \theta_j) \\ \sin(\sum_{j=0}^{i-1} \theta_j) \end{bmatrix}
\end{equation}

ここで、$\mathbf{x}_0 = \mathbf{0}$ である。

% -----------------------------------------------------------------------------------------------------------------------------------------
\subsection{協調係数の計算}

各関節の協調係数 $k_i$ は、以下の式で計算される:

\begin{equation}
k_i = \exp\left(-4\ln(2) \frac{||\mathbf{v}_{l,i} - \mathbf{v}_{i+1}||^2 + \epsilon_1}{||\mathbf{v}_{l,i}||^2 + \epsilon_2}\right)
\end{equation}

ここで、$\epsilon_1, \epsilon_2$ は数値安定性のための小さな正の定数である。

% -----------------------------------------------------------------------------------------------------------------------------------------
\section{Japan MaaS プロジェクト分析の詳細}
\label{app:maas_analysis}

% -----------------------------------------------------------------------------------------------------------------------------------------
\subsection{分析対象プロジェクト一覧}

2020年度に認定された38のJapan MaaSプロジェクトを分析対象とした。プロジェクトは以下のカテゴリに分類される:

\begin{itemize}
    \item 観光型MaaS:XX件
    \item 地域課題解決型MaaS:XX件
    \item 企業主導型MaaS:XX件
\end{itemize}

% -----------------------------------------------------------------------------------------------------------------------------------------
\subsection{評価指標のコーディング基準}

評価指標は以下の基準でコーディングした:

\begin{table}[htbp]
\centering
\caption{評価指標のコーディング基準}
\begin{tabular}{ll}
\toprule
カテゴリ & コーディング基準 \\
\midrule
事業指標 & 利用者数、収益、運行効率など \\
非事業指標 & 社会影响、環境効果、アクセシビリティ改善など \\
市民参加 & ワークショップ、アンケート、協議会など \\
\bottomrule
\end{tabular}
\end{table}

% -----------------------------------------------------------------------------------------------------------------------------------------
\section{ZK-SNARKs型政策評価システムの実装詳細}
\label{app:zk_implementation}

% -----------------------------------------------------------------------------------------------------------------------------------------
\subsection{システム構成}

ZK-SNARKs型政策評価システムは、以下のコンポーネントから構成される:

\begin{itemize}
    \item 秘匿化処理モジュール
    \item LLM評価エンジン
    \item Constitutional AI基準ライブラリ
    \item 控訴処理インターフェース
\end{itemize}

% -----------------------------------------------------------------------------------------------------------------------------------------
\subsection{技術的仕様}

\begin{table}[htbp]
\centering
\caption{技術的仕様}
\begin{tabular}{ll}
\toprule
項目 & 仕様 \\
\midrule
LLM & GPT-4 / Claude \\
Temperature & 0 \\
Self-Consistency & 5回 \\
TEE & AWS Nitro Enclaves \\
\bottomrule
\end{tabular}
\end{table}

% -----------------------------------------------------------------------------------------------------------------------------------------


\end{document}
% -----------------------------------------------------------------------------------------------------------------------------------------
